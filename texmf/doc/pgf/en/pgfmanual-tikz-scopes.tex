% Copyright 2003 by Till Tantau <tantau@cs.tu-berlin.de>.
%
% This program can be redistributed and/or modified under the terms
% of the LaTeX Project Public License Distributed from CTAN
% archives in directory macros/latex/base/lppl.txt.


\section[Hierarchical Structures: Package, Environments, Scopes, and Styles]
{Hierarchical Structures:\\
  Package, Environments, Scopes, and Styles}

The present section explains how your files should be structured when
you use \tikzname. On the top level, you need to include the |tikz|
package. In the main text, each graphi needs to be put in a
|{tikzpicture}| environment. Inside these environments, you can use
|{scope}| environments to create internal groups. Inside the scopes,
you use |\path| commands to actually draw something. On all levels
(except for the package level), graphic options can be given that
apply to everything within the environment.



\subsection{Loading the Package}

\begin{package}{tikz}
  This package does not have any options.
  
  This will automatically load the \pgfname\ package and several other
  stuff that \tikzname\ needs (like the |xkeyval| package).

  \pgfname\ needs to know what \TeX\ driver you are intending to use. In
  most cases, \pgfname\ is clever enough to find out the correct driver
  for you; this is true in particular if you \LaTeX. Currently, the only
  situation where \pgfname\ cannot know the driver ``by itself'' is when
  you use plain \TeX\ or Con\TeX\ together with |dvipdfm|. In this case,
  you have to write |\def\pgfsysdriver{pgfsys-dvipdfm.def}|
  \emph{before} you input |tikz.tex|. 
\end{package}


\subsection{The Main Picture Environment}

The ``outermost'' scope of \pgfname\ and \tikzname\ is the |{tikzpicture}| 
environment. You may give drawing commands only inside this
environment, giving them outside (as is possible in many other
packages) will result in chaos.

In \tikzname, most of how a graphic ``looks like'' is governed by graphic
options. For example, there is an option for setting the color used
for drawing, another for setting the color used for filling, and some
more obscure one like the option  for setting the prefix used in the
filenames of temporary files written while plotting functions using an
external program. The graphic options are nearly always specified in a
so-called key-value style. (The ``nearly always'' refers to the name
of nodes, which can also be specified differently.) All graphic
options are local to the |{tikzpicture}| to which they apply.

\begin{environment}{{tikzpicture}\opt{\oarg{options}}}
  All \tikzname\ and nearly all \pgfname\ commands should be given inside
  this environment. Unlike other packages, it is not possible to use,
  say, |\pgfpathmoveto| outside this environment and will result in
  chaos. For \tikzname, commands like |\path| are only defined inside this
  environment, so there is little chance that you will do something
  wrong here. 

  When this environment is encountered, the \meta{options} are
  parsed. All options given here will apply to the whole
  picture. Giving options makes sense only when \tikzname\ is loaded, the
  \pgfname\ core does not define any options.

  Next, the contents of the environment is processed and the graphic
  commands therein are put into a box. Non-graphic text is suppressed
  as well as possible, but non-\pgfname\ commands inside a
  |{tikzpicture}| environment should not produce any ``output'' since
  this may totally scramble the positioning system of the backend
  drivers. The suppressing of normal text, by the way, is done by
  temporarily switching the font to |\nullfont|. You can, however,
  ``escape back'' to normal \TeX\ typesetting. This happens, for
  example, when you specify a node.

  At the end of the environment, \pgfname\ tries to make a good guess
  at a good guess at the size of the bounding box of the graphic and
  then resizes the box such that the box has this size. To ``make its
  guess,'' everytime \pgfname\ encounters a coordinate, it updates the
  bound box's size such that it encompasses all these
  coordinates. This will usually give a good 
  approximation at the bounding box, but will not always be
  accurate. First, the line thickness is not taken into
  account. Second, controls points of a curve often lie far
  ``outside'' the curve and make the bounding box too large. In this
  case, you should use the |[use as bounding box]| option.

  The following option influences the baseline of the resulting
  picture:
  \begin{itemize}
    \itemoption{baseline}\opt{|=|\meta{dimension}}
    Normally, the lower end of the picture is put on the baseline of
    the surrounding text. For example, when you give the code
    |\tikz\draw(0,0)circle(.5ex);|, \pgfname\ will find out that the
    lower end of the picture is at $-.5\mathrm{ex}$ and that the upper
    end is at $.5\mathrm{ex}$. Then, the lower end will be put on the
    baseline, resulting in the following: \tikz\draw(0,0)circle(.5ex);.

    Using this option, you can specify that the picture should be
    raised or lowered such that the height \meta{dimension} is on the
    baseline. For example, |tikz[baseline=0pt]\draw(0,0)circle(.5ex);|
    yields \tikz[baseline=0pt]\draw(0,0)circle(.5ex); since, now, the
    baseline is on the height of the $x$-axis. If you omit the
    \meta{dimensions}, |0pt| is assumed as default.

    This options is often useful for ``inlined'' graphics as in
\begin{codeexample}[]
$A \mathbin{\tikz[baseline] \draw[->>] (0pt,.5ex) -- (3ex,.5ex);} B$
\end{codeexample}
  \end{itemize}
  
  All options ``end'' at the end of the picture. To set an option
  ``globally'' you can use the following style:
  \begin{itemize}
    \itemstyle{every picture}
    This style is installed at the beginning of each picture.
\begin{codeexample}[code only]
\tikzstyle{every picture}=[semithick]
\end{codeexample}
\end{itemize}
\end{environment}

In plain \TeX, you should use instead the following commands:

\begin{plainenvironment}{{tikzpicture}\opt{\oarg{options}}}
\end{plainenvironment}

The following two commands are used for ``small'' graphics.

\begin{command}{\tikz\opt{\oarg{options}}\marg{commands}}
  This little command places the \meta{commands} inside a
  |{tikzpicture}| environment and adds a semicolon at the end. This is
  just a convenience.

  The \meta{commands} may not contain a paragraph (an empty
  line). This is a precaution to ensure that users really use this
  command only for small graphics.

  \example |\tikz{\draw (0,0) rectangle (2ex,1ex)}| yields
  \tikz{\draw (0,0) rectangle (2ex,1ex);} 
\end{command}


\begin{command}{\tikz\opt{\oarg{options}}\meta{text}|;|}
  If the \meta{text} does not start with an opening brace, the end of
  the \meta{text} is the next semicolon that is encountered.

  \example |\tikz \draw (0,0) rectangle (2ex,1ex);| yields
  \tikz \draw (0,0) rectangle (2ex,1ex);
\end{command}


\subsection{Scopes}

Inside a |{tikzpicture}| environment, you can create ``subscopes''
using the |{scope}| environment. This environment is available only
inside the |{tikzpicture}| environment, so once more, there is little
chance of doing anything wrong.

\begin{environment}{{scope}\opt{\oarg{options}}}
  All \meta{options} are local to the \meta{environment
  contents}. Furthermore, the clipping path is also local to the
  environment, that is, any clipping done inside the environment
  ``ends'' at its end.

  \example
\begin{tikzpicture}
  \begin{scope}[red]
    \draw (0mm,0mm) -- (10mm,0mm);
    \draw (0mm,1mm) -- (10mm,1mm);
  \end{scope}
  \draw (0mm,2mm) -- (10mm,2mm);
  \begin{scope}[green]
    \draw (0mm,3mm) -- (10mm,3mm);
    \draw (0mm,4mm) -- (10mm,4mm);
    \draw[blue] (0mm,5mm) -- (10mm,5mm);
  \end{scope}
\end{tikzpicture}
\begin{verbatim}
\begin{tikzpicture}
  \begin{scope}[red]
    \draw (0mm,5mm) -- (10mm,5mm);
    \draw (0mm,4mm) -- (10mm,4mm);
  \end{scope}
  \draw (0mm,3mm) -- (10mm,3mm);
  \begin{scope}[green]
    \draw (0mm,2mm) -- (10mm,2mm);
    \draw (0mm,1mm) -- (10mm,1mm);
    \draw[blue] (0mm,0mm) -- (10mm,0mm);
  \end{scope}
\end{tikzpicture}
\end{verbatim}
  
  The following style influences scopes:
  \begin{itemize}
    \itemstyle{every scope}
    This style is installed at the beginning of every scope. I do not
    know really know what this might be good for, but who knows?
  \end{itemize}
\end{environment}


In plain \TeX, you use the following commands instead:

\begin{plainenvironment}{{scope}\opt{\oarg{options}}}
\end{plainenvironment}



\subsection{Path Scopes}

The |\path| command, which is described in much more detail in later
sections, also takes graphic options. These options are local to the
path. Furthermore, it is possible to create local scopes withing a
path simply by using curly braces as in
\begin{codeexample}[]
\tikz \draw (0,0) -- (1,1)
           {[rounded corners] -- (2,0) -- (3,1)}
           -- (3,0) -- (2,1);
\end{codeexample}

Note that many options apply only to the path as a whole and cannot be
scoped in this way. For example, it is not possible to scope the
|color| of the path. See the explanations in the section on paths for
more details.

Finally, certain elements that you specify in the argument to the
|\path| command also take local options. For example, a node
specification takes options. In this case, the options apply only to
the node, not to the surrounding path.


\subsection{Styles}

There is a way of organizing sets of graphic options ``orthogonally''
to the normal scoping mechanism. For example, you might wish all your
``help lines'' to be drawn in a certain way like, say, gray and thin
(do \emph{not} dash them, that distracts). For this, you can use
\emph{styles}.

A style is simply a set of graphic options that is predefined at some
point. Once a style has been defined, it can be used anywhere using
the |style| option:

\begin{itemize}
  \itemoption{style}|=|\meta{style name}
  invokes all options that are currently set in the \meta{style
    name}. An example of a style is the predefined |help lines| style,
  which you should use for lines in the background like grid lines or
  construction lines. You can easily define new styles and modify
  existing ones.
\begin{codeexample}[]
\begin{tikzpicture}
  \draw                   (0,0) grid +(2,2);
  \draw[style=help lines] (2,0) grid +(2,2);
\end{tikzpicture}
\end{codeexample}
\end{itemize}


\begin{command}{\tikzstyle\meta{style name}\opt{|+|}|=[|\meta{options}|]|}
  This command defines the style \meta{style name}. Whenever it is
  used using the |style=|\meta{style name} command, the \meta{options}
  will be invoked. It is permissible that a style invokes another
  style using the |style=| command inside the \meta{options}, which
  allows you to build hierarchies of styles. Naturally, you should
  \emph{not} create cyclic dependencies.

  If the style already has a predefined meaning, it will
  uncermimoniously be redefined without a warning.
\begin{codeexample}[]
\tikzstyle help lines=[blue!50,very thin]
\begin{tikzpicture}
  \draw                   (0,0) grid +(2,2);
  \draw[style=help lines] (2,0) grid +(2,2);
\end{tikzpicture}
\end{codeexample}

  If the optional |+| is given, the options are \emph{added} to the
  existing defintion:
\begin{codeexample}[]
\tikzstyle help lines+=[dashed] % aaarghhh!!!
\begin{tikzpicture}
  \draw                   (0,0) grid +(2,2);
  \draw[style=help lines] (2,0) grid +(2,2);
\end{tikzpicture}
\end{codeexample}
\end{command}
