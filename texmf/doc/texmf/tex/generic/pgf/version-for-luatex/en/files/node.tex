% This file has been generated from the lua sources using LuaDoc.
% To regenerate it call "make genluadoc" in
% doc/generic/pgf/version-for-luatex/en.

\paragraph{pgflibrarygraphdrawing-node.lua}


\begin{luacommand}{{Node:\textunderscore{}\textunderscore{}eq}(\meta{object})}
Compares two nodes by name.

Parameters:
\begin{itemize}
	\item[] \meta{object} \subitem The node to be compared to self
\end{itemize}


Return value:
\begin{itemize} \item[] True if self is equal to object. \end{itemize}


\end{luacommand}\begin{luacommand}{{Node:\textunderscore{}\textunderscore{}tostring}()}
Returns a formated string representation of the node.


Return value:
\begin{itemize} \item[] String represenation of the node. \end{itemize}


\end{luacommand}\begin{luacommand}{{Node:addEdge}(\meta{edge})}
Adds new Edge to the Node.

Parameters:
\begin{itemize}
	\item[] \meta{edge} \subitem The edge to be added.
\end{itemize}



\end{luacommand}\begin{luacommand}{{Node:copy}()}
Creates a shallow copy of a node.


Return value:
\begin{itemize} \item[] Copy of the node. \end{itemize}


\end{luacommand}\begin{luacommand}{{Node:degree}()}
Computes the number of neighbour nodes.


Return value:
\begin{itemize} \item[] Number of neighbours. \end{itemize}


\end{luacommand}\begin{luacommand}{{Node:getEdges}()}
Gets all Edges of the node.


Return value:
\begin{itemize} \item[] The edges of the node as a table. \end{itemize}


\end{luacommand}\begin{luacommand}{{Node:getOption}(\meta{name})}
Returns the value of option name or nil.

Parameters:
\begin{itemize}
	\item[] \meta{name} \subitem Name of the option.
\end{itemize}


Return value:
\begin{itemize} \item[] The stored value of the option or nil. \end{itemize}


\end{luacommand}\begin{luacommand}{{Node:getTexHeight}()}
Computes the Heigth of the Node.


Return value:
\begin{itemize} \item[] Height of the Node. \end{itemize}


\end{luacommand}\begin{luacommand}{{Node:getTexWidth}()}
Computes the Width of the Node.


Return value:
\begin{itemize} \item[] Width of the Node. \end{itemize}


\end{luacommand}\begin{luacommand}{{Node:mergeOptions}(\meta{options})}
Merges options.

Parameters:
\begin{itemize}
	\item[] \meta{options} \subitem The options to be merged.
\end{itemize}



See also:
\begin{itemize}
	\item[] |mergeTable|
\end{itemize}

\end{luacommand}\begin{luacommand}{{Node:new}(\meta{values})}
Creates a new node.

Parameters:
\begin{itemize}
	\item[] \meta{values} \subitem Values (e.g. position) to be merged with the default-metatable of a node
\end{itemize}


Return value:
\begin{itemize} \item[] A newly allocated node object. \end{itemize}


\end{luacommand}\begin{luacommand}{{Node:removeEdge}(\meta{edge})}
Removes an edge from the node.

Parameters:
\begin{itemize}
	\item[] \meta{edge} \subitem The edge to remove.
\end{itemize}



\end{luacommand}\begin{luacommand}{{Node:setOption}(\meta{name},\meta{value})}
Sets the option name to value.

Parameters:
\begin{itemize}
	\item[] \meta{name} \subitem Name of the option to be set.\item[] \meta{value} \subitem Value for the option defined by name.
\end{itemize}



\end{luacommand}
