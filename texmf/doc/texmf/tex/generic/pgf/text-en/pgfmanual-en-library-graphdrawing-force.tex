% Copyright 2011 by Jannis Pohlmann
%
% This file may be distributed and/or modified
%
% 1. under the LaTeX Project Public License and/or
% 2. under the GNU Free Documentation License.
%
% See the file doc/generic/pgf/licenses/LICENSE for more details.

\section{Force-Based Graph Drawing Algorithms}
\label{section-library-graphdrawing-force-based}

{\emph{by Jannis Pohlmann}}


\begin{tikzlibrary}{graphdrawing.force}
  Load this package when you wish to use force-based graph drawing
  algorithms. You should load the |graphdrawing| library first.
\end{tikzlibrary}

\ifluatex\relax\else{LuaTeX is required for setting this manual section.}\expandafter\endinput\fi


\subsection{Overview}

% TODO Jannis: Explain ideas and concepts behind force-based graph
% drawing algorithms. Briefly explain the various approaches in that
% specific area of graph drawing algorithms (e.g. spring,
% spring-electrical and multidimensional embedding). 

...

\subsubsection{Spring and Spring-Electrical Layouts}

% TODO Jannis: Explain ideas and concepts behind spring and
% spring-electrical algorithms. Describe the technical as well as visual
% differences between the two techniques (think: no attractive forces
% and no peripheral effects in spring layouts). Explain why they were
% consolidated in the common family 'spring layout'.

\begin{key}{/graph drawing/spring layout=\meta{options}}
  \keyalias{tikz}\keyalias{tikz/graphs}
  Similar to the |>| option, this ``generic'' name for a spring layout
  algorithm is not hardwired to any specific algorithm. Rather, users
  can select an algorithm somewhere at the beginning of their program
  and then just write |\graph[spring layout]| to draw a tree.

  The \meta{options} will be forwarded to the currently selected
  algorithm.
\begin{codeexample}[]
\tikz \graph [spring layout] { a -> {b,c} };    
\end{codeexample}
  
  To change the algorithm, change the following key:
  \begin{key}{/graph drawing/spring layout/default algorithm=\meta{algorithm}}
    Set this key to the tree drawing algorithm of your choice. The
    default is currently set to the algorithm
    |Walshaw2000 spring electrical|, but this will change. 
  \end{key}
\end{key}


\subsection{Common Options}

The spring and and spring-electrical drawing algorithms are very similar
in terms of their parameters and the constraints they can handle. They
thus share a number of common \tikzname\ options for fine-tuning. These
options are split up into \emph{graph options} that can be specified
once for a graph, \emph{node options} that can be specified for each
node and \emph{edge options} that can be specified for each edge.

\subsubsection{Graph Options}

% TODO Jannis: This might be worth implementing. It's not very useful in
% the Hu2006 algorithm as it uses the Barnes-Hut algorithm, but the 
% Walshaw2000 algorithm can benefit from it.
%
%\begin{key}{/tikz/influence cutoff distance=\meta{dimension} (initially
%  0pt)}
%  Specifies a distance beyond which the attractive and repulsive forces 
%  between two nodes are assumed to be virtually non-existent. If 
%  \meta{dimension} is set to |0pt|, the cutoff distance is computed 
%  automatically.
%
%  Depending on the graph drawing algorithm being used, the distance
%  between two nodes is computed either based on the graph distance
%  (spring algorithm) or based on the Euclidean distance
%  (spring-electrical algorithm).
%  \begin{codeexample}[]
%  \end{codeexample}
%\end{key}

\begin{key}{/graph drawing/spring layout/maximum 
  iterations=\meta{number} (initially 500)}
  Depending on the characteristics of the input graph and the parameters
  chosen for the spring or spring-electrical algorithm, minimizing the
  system energy may require many iterations.

  In these situations it may come in handy to limit the number of
  iterations. This feature can also be useful to draw the same graph
  after different iterations and thereby demonstrate how the spring or
  spring-electrical algorithm improves the drawing step by step.
  \begin{codeexample}[]
\tikz \graph [spring layout={maximum iterations=1}]   { a -- b -- c -- a };
\tikz \graph [spring layout={maximum iterations=10}]  { a -- b -- c -- a };
\tikz \graph [spring layout={maximum iterations=500}] { a -- b -- c -- a };
  \end{codeexample}
\end{key}

\begin{key}{/graph drawing/spring layout/random seed=\meta{number} 
  (initially 42)}
  Specifies the seed used for Lua's pseudo-random number generator. If
  set to something other than |0|, the random number sequence generated
  by the pseudo-random number generator will be the same at every run.
  The resulting graph drawings will be reproducible in consecutive runs,
  despite randomized elements used in the algorithm.
  If set to |0|, the results are not guaranteed to be reproducible.
  \begin{codeexample}[width=5.5cm]
\tikz \graph [spring layout={random seed=1}] { 
  subgraph K_n[n=4]
};
\tikz \graph [spring layout={random seed=10}] { 
  subgraph K_n[n=4]
};
  \end{codeexample}
\end{key}

\begin{key}{/graph drawing/spring layout/natural spring
  dimension=\meta{dimension} (initially 1cm)}
\end{key}

\begin{key}{/graph drawing/spring layout/spring constant=\meta{number}}
\end{key}

\begin{key}{/graph drawing/spring layout/approximate repulsive
  forces=\opt{\meta{boolean}} (initially true)}
\end{key}

\begin{key}{/graph drawing/spring layout/cooling factor=\meta{number}
  (initially 0.95)}
\end{key}

\begin{key}{/graph drawing/spring layout/convergence
  tolerance=\meta{number} (initially 0.01)}
\end{key}

\begin{key}{/graph drawing/spring layout/coarsen=\opt{\meta{boolean}}
  (initially true)}
  Defines whether or not a multilevel approach is used that
  iteratively coarsens the input graph into graphs $G_1,\dots,G_l$ with 
  a smaller and smaller number of nodes. The coarsening stops as soon as
  a minimum number of nodes is reached, as set via the 
  |minimum graph size| option or when, in the last iteration, the 
  number of nodes was not reduced by at least the ratio specified via 
  |downsize ratio|. 

  A random initial layout is computed for the coarsest graph $G_l$ first.
  Afterwards, it is laid out by computing the attractive and repulsive
  forces between its nodes. 
  
  In the subsequent steps, the previous coarse graph $G_{l-1}$ is 
  restored and its node positions are interpolated from the nodes in 
  $G_l$. $G_{l-1}$ is again laid out by computing the forces between 
  its nodes. These steps are repeated with $G_{l-2},\dots,G_1$ until 
  the original input graph $G_0$ has been restored, interpolated 
  and laid out.

  There are a number of options to fine-tune the coarsening approach.
  They are consolidated in the |/graph drawing/spring layout/coarsening|
  prefix described below.
\end{key}

\begin{key}{/graph drawing/spring layout/coarsening=\marg{options}}
  Executes the \meta{options} with the path prefix 
  |/graph drawing/spring layout/coarsening|.

  These options can be used to configure the coarsening approach
  described in the documentation of the 
  |/graph drawing/spring layout/coarsen| option.
\end{key}

\begin{key}{/graph drawing/spring layout/coarsening/minimum graph
  size=\meta{number} (initially 2)}
  Defines the number of nodes down to which the graph is coarsened
  iteratively. The first graph that has a lesser or equal number of
  nodes becomes the coarsest graph $G_l$, where $l$ is the number of
  coarsening steps. The algorithm proceeds with the steps described in
  the documentation of the |/graph drawing/spring layout/coarsen|
  option.

  In the following example the same graph is coarsened down to two
  and three nodes, respectively. The layout of the original graph is 
  interpolated from the random initial layout and is not changed
  because the forces are not computed. Thus, in the first graph, the
  nodes can have exactly two (or three) possible coordinates in the
  final drawing:
  \begin{codeexample}[]
\tikz \graph [spring layout={maximum iterations=0,coarsen,
                             coarsening={minimum graph size=2}}] { 
  1 -- 2 -- 3 -- 4 
};
\tikz \graph [spring layout={maximum iterations=0,coarsen,
                             coarsening={minimum graph size=3}}] { 
  1 -- 2 -- 3 -- 4 
};
  \end{codeexample}
\end{key}

\begin{key}{/graph drawing/spring layout/coarsening/downsize
  ratio=\meta{number} (initially 0.25)}
\end{key}

\begin{key}{/graph drawing/spring layout/coarsening/collapse independent
  edges=\opt{\meta{boolean}} (initially true)}
\end{key}

\begin{key}{/graph drawing/spring layout/coarsening/connected
  independent nodes=\opt{\meta{boolean}} (initially false)}
\end{key}

%\begin{key}{/tikz/coarsening=\marg{options}}
%  Executes the \meta{options} with the path prefix |/tikz/coarsening|.
%  
%  These options define whether a multilevel approach is used that
%  successively coarsend into graphs with smaller and smaller number
%  of nodes. These graphs are arranged first and are then interpolated
%  into the finer graphs at the previous level. How this is done exactly
%  can be configured using the |coarsening| options described below.
%\end{key}
%
%\begin{key}{/tikz/coarsening/randomized=\opt{\meta{boolean}} (default
%  true, initially false)}
%  If set to |true|, nodes will be inspected in a random order. The
%  effect on the final drawing can only be seen by experimenting with the
%  option.
%  \begin{codeexample}[]
%  \end{codeexample}
%\end{key}
%
%\begin{key}{/tikz/coarsening/minimum size=\meta{number} (default 0)}
%  Defines the minimum number of nodes in a coarsened graph. If a
%  coarsened graph has less than \meta{number} nodes, then... % TODO
%  \begin{codeexample}[] 
%% the same graph with different minimum size values
%  \end{codeexample}
%\end{key}
%
%\begin{key}{/tikz/coarsening/nodes=\opt{\meta{boolean}} (default true,
%  initially false)}
%  \begin{codeexample}[]
%  \end{codeexample}
%\end{key}
%
%\begin{key}{/tikz/coarsening/nearby nodes=\opt{\meta{boolean}} (default
%  true, initially false)}
%  \begin{codeexample}[]
%  \end{codeexample}
%\end{key}
%
%\begin{key}{/tikz/coarsening/nodes with more 
%  neighbors=\opt{\meta{boolean}} (default true, initially false)}
%  \begin{codeexample}[]
%  \end{codeexample}
%\end{key}
%
%\begin{key}{/tikz/coarsening/nearby nodes with more 
%  neighbors=\opt{\meta{boolean}} (default true, initially false)}
%  \begin{codeexample}[]
%  \end{codeexample}
%\end{key}
%
%\begin{key}{/tikz/coarsening/edges=\opt{\meta{boolean}} (default true,
%  initially false)}
%  \begin{codeexample}[]
%  \end{codeexample}
%\end{key}
%
%\begin{key}{/tikz/coarsening/heavy edges=\opt{\meta{boolean}} (default
%  true, initially false)}
%  \begin{codeexample}[]
%  \end{codeexample}
%\end{key}
%
%\begin{key}{/tikz/coarsening/edges with light nodes=\opt{\meta{boolean}}
%  (default true, initially false)}
%  \begin{codeexample}[]
%  \end{codeexample}
%\end{key}
%
%\begin{key}{/tikz/minimum energy delta=\meta{number} (default TODO)}
%  \begin{codeexample}[]
%  \end{codeexample}
%\end{key}
%
%\begin{key}{/tikz/initial step size=\meta{dimension} (default TODO)}
%  \begin{codeexample}[]
%  \end{codeexample}
%\end{key}
%
%\begin{key}{/tikz/step control=\meta{text} (default TODO)}
%  Possible values: |monotonic|, |non-monotonic|, |strictly monotonic|.
%  \begin{codeexample}[]
%  \end{codeexample}
%\end{key}

\subsubsection{Node Options}

%\begin{key}{/tikz/electric charge=\meta{number} (default 1)}
%  Defines the electric charge of the node. The stronger the electric
%  charge, the higher the repulsive force between two nodes. Set this to
%  something between |0| and |1| to reduce the charge compared to the
%  normal setup. Values larger than |1| will generate stronger repulsion
%  between the node and the others.
%  \begin{codeexample}[] 
%\tikz \graph [spring electrical layout,orient=1:90:2] {
%  1 -- 2 -- 3 -- 4 -- 1,
%  1 -- 3, 2 -- 4,
%};
%\tikz \graph [spring electrical layout,orient=1:90:2] {
%  1 [electric charge=1] -- 2 -- 3 -- 4 -- 1,
%  1 -- 3, 2 -- 4,
%};
%\tikz \graph [spring electrical layout,orient=1:90:2] {
%  1 [electric charge=1000] -- 2 [electric charge=1000] -- 3 -- 4 -- 1,
%  1 -- 3, 2 -- 4,
%};
%  \end{codeexample}
%\end{key}

%\end{document}

% TODO Jannis: Explain this one better. Also, compare it to /tikz/at,
% which will move the node after the drawing has been computed, as
% opposed to /graph drawing/desired at, which will only move the node
% while computing the layout.
%
%\begin{key}{/tikz/desired at=\meta{coordinate}}
%  Nails the node down at the specified \meta{coordinate}. It will not
%  move from there despite the repulsive and attractive forces in the
%  system. Note that, while sometimes generating a similar effect, using
%  |at| is very different from altering the orientation of a graph
%  drawing (see section~\ref{subsection-library-graphdrawing-standard-orientation}).
%  Also, if an orientation is specified, it is given priority over
%  the |at| option in that nodes are first fixated at their |at|
%  coordinates but are later moved in order to satisfy the orientation 
%  desired by the user.
%  \begin{codeexample}[width=6.0cm]
%\tikz \graph [spring layout] {
%  1 -- 2 -- 3 -- 4 -- 2
%};
%\tikz \graph [spring layout] {
%  1 [at={(0,0)}] -- 2 [at={(0,1)}] -- 3 -- 4 -- 2
%};
%  \end{codeexample}
%\end{key}


% TODO what about node groups / clusters? This works via color classes
% but how do we define their layouts (cluster, line, circle)?

\subsubsection{Edge Options}

\begin{key}{/tikz/natural length=\meta{dimension} (default 10pt)}
  Defines the natural (zero energy) length of the edge. The smaller the
  length, the stronger the attractive force of the adjacent nodes. The
  \meta{dimension} has a strong influence of how far the nodes will be
  placed from each other in the final drawing.
  \begin{codeexample}[]
% two examples with the same graph
% notably change the natural length of one of the edges
  \end{codeexample}
\end{key}

\begin{key}{/tikz/stiffness=\meta{number} (default 0.5)}
  Defines how flexible the spring associated with the edge is. The
  higher this value is, the closer the final edge length will be to its
  |natural length|.
  \begin{codeexample}[]
% two examples with the same graph
% notably change the stiffness of one of the edges
  \end{codeexample}
\end{key}

\subsection{Options for the Spring Algorithm}

\subsubsection{Graph Options}

...

\subsubsection{Node Options}

...

\subsubsection{Edge Options}

...

\subsection{Options for the Spring-Electrical Algorithm}

\subsubsection{Graph Options}

...

\subsubsection{Node Options}

...

\subsubsection{Edge Options}

...

\begin{codeexample}[]
\vbox{ \hsize=16cm \rightskip=0cm plus 1fill
  \foreach \iterations in {1,...,20,100,500}
  {
    \tikz \graph [spring layout={maximum iterations=\iterations}, orient=1-2] 
      { subgraph K_n[n=4] };
    \penalty0
  }
}
\end{codeexample}

\begin{codeexample}[]
\vbox{ \hsize=16cm \rightskip=0cm plus 1fill
  \foreach \iterations in {1,...,20,100,500}
  {
    \tikz \graph [spring layout={maximum iterations=\iterations}, orient=1-2] 
      { subgraph C_n[n=7] };
    \penalty0
  }
}
\end{codeexample}

\endinput

%% TODO
%% Explain the following concepts:
%% - separation of graph drawing options and regular TikZ options
%% - generic graph drawing options:
%%   - component packing
%%   - orientation
%% - pre-defined graph drawing styles
%% - graph drawing options for fine-tuning the different algorithms

%%% Local Variables: 
%%% mode: latex
%%% TeX-master: "pgfmanual-pdftex-version"
%%% End: 
