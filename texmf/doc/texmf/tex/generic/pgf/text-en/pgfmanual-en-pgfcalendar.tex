% Copyright 2006 by Till Tantau
%
% This file may be distributed and/or modified
%
% 1. under the LaTeX Project Public License and/or
% 2. under the GNU Free Documentation License.
%
% See the file doc/generic/pgf/licenses/LICENSE for more details.


\section{Date and Calendar Utility Macros}
\label{section-calendar}

This section describes the package |pgfcalendar|.

\begin{package}{pgfcalendar}
  This package can be used independently of \pgfname. It has two
  purposes:
  \begin{enumerate}
  \item It provides functions for working with dates. Most noticably,
    it can convert a date in ISO-standard format (like 1975-12-26) to
    a so-called Julian day number, which is defined in Wikipedia as
    follows:  ``The Julian day or Julian day number is the
    (integer) number of days that have elapsed since the initial epoch
    at noon Universal Time (UT) Monday, January 1, 4713 BC in the
    proleptic Julian calendar.'' The package also provides a function
    for converting a Julian day number to an ISO-format date.

    Julian day numbers make it very easy to work with days. For
    example, the date ten days in the future of 2008-02-20 can
    be computed by converting this date to a Julian day number, adding
    10, and then converting it back. Also, the day of week of a given
    date can be computed by taking the Julian day number modulo~7.
  \item It provides a macro for typesetting a calendar. This macro
    is highly configurable and flexible (for example, it can produce
    both plain text calendars and also complicated \tikzname-based
    calendars), but most users will not use the macro directly. It is
    the job of a frontend to provide useful configruations for
    typesetting calendars based on this command.
  \end{enumerate}
\end{package}


\subsection{Date Conversion Macros}

\begin{command}{\pgfcalendardatetojulian\marg{date}\marg{counter}}
  This macro converts a date in ISO format to the Julian day number in
  the Gregorian calendar. The \meta{date} must expand to a string of the form
  2006-11-10, that is, the year followed by a hyphen, followed by the
  month, followed by a hyphen, followed by a day number. The
  \meta{counter} must be a \TeX-counter. 
  
  The conversion method is taken from the English Wikipedia entry on
  Julian days. 

  \newcount\mycount
  \example |\pgfcalendardatetojulian{2007-01-14}{\mycount}| sets
  |\mycount| to \pgfcalendardatetojulian{2007-01-14}{\mycount}\the\mycount.
\end{command}

\begin{command}{\pgfcalendarjuliantodate\marg{Julian day}\marg{year
      macro}\marg{month macro}\marg{day macro}}
  
\end{command}

\begin{command}{\pgfcalendarjuliantoweekday\marg{Julian day}\marg{week day counter}}
  
\end{command}


\begin{command}{\pgfcalendarweekdayname\marg{week day counter}}
  
\end{command}


\begin{command}{\pgfcalendarweekdayshortname\marg{week day counter}}
  
\end{command}


\begin{command}{\pgfcalendarmonthname\marg{month counter}}
  
\end{command}


\begin{command}{\pgfcalendarmonthshortname\marg{month counter}}
  
\end{command}

\begin{command}{\pgfcalendarnoleadingzero\marg{number}}
  
\end{command}




\subsection{Calendar Typesetting Macro}

\begin{command}{\pgfcalendar\marg{prefix}\marg{start date}\marg{end
      date}\marg{restrictions}\marg{rendering code}}
  
\end{command}



\begin{command}{\pgfcalendarsuggestedname}
  
\end{command}

\subsection{Date Restriction Macros}


\begin{command}{\pgfcalendaron\marg{date}\marg{conditions}\marg{code}}
  
\end{command}


