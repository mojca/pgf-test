% This file has been generated from the lua sources using LuaDoc.
% To regenerate it call "make genluadoc" in
% doc/generic/pgf/version-for-luatex/en.

\begin{filedescription}{pgflibrarygraphdrawing-edge.lua}


\begin{luacommand}{{Edge:\textunderscore{}\textunderscore{}eq}(\meta{other})}
Returns whether or not the two edges have the same adjacent nodes. 

Parameters:
\begin{parameterdescription}
	\item[\meta{other}] Another edge to compare with. 
\end{parameterdescription}


Return value:
\begin{parameterdescription} 
  \item[] |true| if the two edges have exactly the same adjacent nodes. 
\end{parameterdescription}


\end{luacommand}
\begin{luacommand}{{Edge:\textunderscore{}\textunderscore{}tostring}()}
Returns a readable string representation of the edge. 


Return value:
\begin{parameterdescription} 
  \item[] String representation of the edge. 
\end{parameterdescription}


\end{luacommand}
\begin{luacommand}{{Edge:addNode}(\meta{node})}
If possible, adds a node to the edge. 

Parameters:
\begin{parameterdescription}
	\item[\meta{node}] The node to be added to the edge. 
\end{parameterdescription}



\end{luacommand}
\begin{luacommand}{{Edge:containsNode}(\meta{node})}
Returns whether or not a node is adjacent to the edge. 

Parameters:
\begin{parameterdescription}
	\item[\meta{node}] The node to check. 
\end{parameterdescription}


Return value:
\begin{parameterdescription} 
  \item[] |true| if the node is adjacent to the edge. |false| otherwise. 
\end{parameterdescription}


\end{luacommand}
\begin{luacommand}{{Edge:copy}()}
Copies an edge (preventing accidental use).  The nodes of the edge are not preserved and have to be added to the copy manually if necessary. 


Return value:
\begin{parameterdescription} 
  \item[] Shallow copy of the edge. 
\end{parameterdescription}


\end{luacommand}
\begin{luacommand}{{Edge:getDegree}()}
Counts the nodes on this edge. 


Return value:
\begin{parameterdescription} 
  \item[] The number of nodes on the edge. 
\end{parameterdescription}


\end{luacommand}
\begin{luacommand}{{Edge:getNeighbour}(\meta{node})}
Gets first neighbour of the node (disregarding hyperedges). 

Parameters:
\begin{parameterdescription}
	\item[\meta{node}] The node which first neighbour should be returned. 
\end{parameterdescription}


Return value:
\begin{parameterdescription} 
  \item[] The first neighbour of the node. 
\end{parameterdescription}


\end{luacommand}
\begin{luacommand}{{Edge:getNeighbours}(\meta{node})}
Returns all neighbours of a node adjacent to the edge.  The edge direction is not taken into account, so this method always returns all neighbours even if called on a directed edge. 

Parameters:
\begin{parameterdescription}
	\item[\meta{node}] A node. Typically but not necessarily adjacent to the edge. If the node is not an intermediate or end point of the edge, an empty array is returned. 
\end{parameterdescription}


Return value:
\begin{parameterdescription} 
  \item[] An array of nodes that are adjacent to the input node via the edge the method is called on. 
\end{parameterdescription}


\end{luacommand}
\begin{luacommand}{{Edge:getNodes}()}
Returns all nodes of the edge.  Instead of calling |edge:getNodes()| the nodes can alternatively be accessed directly with |edge.nodes|. 


Return value:
\begin{parameterdescription} 
  \item[] All edges of the node. 
\end{parameterdescription}


\end{luacommand}
\begin{luacommand}{{Edge:getOption}(\meta{name})}
Returns the value of the edge option \meta{name}. 

Parameters:
\begin{parameterdescription}
	\item[\meta{name}] Name of the option. 
\end{parameterdescription}


Return value:
\begin{parameterdescription} 
  \item[] The value of the edge option \meta{name} or |nil|. 
\end{parameterdescription}


\end{luacommand}
\begin{luacommand}{{Edge:isHead}(\meta{node},\meta{ignore\_reversed})}
Checks whether a node is the head of the edge. Does not work for hyperedges.  This method only works for edges with two adjacent nodes.  For undirected edges or edges that point into both directions, the result will always be true. Directed edges may be reversed internally, so their head and tail might be switched. Whether or not this internal reversal is handled by this method can be specified with the optional second \meta{ignore\_reversed} parameter which is |false| by default. 

Parameters:
\begin{parameterdescription}
	\item[\meta{node}] The node to check.\item[\meta{ignore\_reversed}] Optional parameter. Set this to true if reversed edges should not be considered reversed for this method call. 
\end{parameterdescription}


Return value:
\begin{parameterdescription} 
  \item[] True if the node is the head of the edge. 
\end{parameterdescription}


\end{luacommand}
\begin{luacommand}{{Edge:isHyperedge}()}
Returns whether or not the edge is a hyperedge.  A hyperedge is an edge with more than two adjacent nodes. 


Return value:
\begin{parameterdescription} 
  \item[] |true| if the edge is a hyperedge. |false| otherwise. 
\end{parameterdescription}


\end{luacommand}
\begin{luacommand}{{Edge:isTail}(\meta{node},\meta{ignore\_reversed})}
Checks whether a node is the tail of the edge. Does not work for hyperedges.  This method only works for edges with two adjacent nodes.  For undirected edges or edges that point into both directions, the result will always be true.  Directed edges may be reversed internally, so their head and tail might be switched. Whether or not this internal reversal is handled by this method can be specified with the optional second \meta{ignore\_reversed} parameter which is |false| by default. 

Parameters:
\begin{parameterdescription}
	\item[\meta{node}] The node to check.\item[\meta{ignore\_reversed}] Optional parameter. Set this to true if reversed edges should not be considered reversed for this method call. 
\end{parameterdescription}


Return value:
\begin{parameterdescription} 
  \item[] True if the node is the tail of the edge. 
\end{parameterdescription}


\end{luacommand}
\begin{luacommand}{{Edge:new}(\meta{values})}
Creates an edge between nodes of a graph. 

Parameters:
\begin{parameterdescription}
	\item[\meta{values}] Values to override default edge settings. The following parameters can be set:\par |nodes|: TODO \par |edge_nodes|: TODO \par |options|: TODO \par |tikz_options|: TODO \par |direction|: TODO \par |bend_points|: TODO \par |bend_nodes|: TODO \par |reversed|: TODO \par 
\end{parameterdescription}


Return value:
\begin{parameterdescription} 
  \item[] A newly-allocated edge. 
\end{parameterdescription}


\end{luacommand}
\begin{luacommand}{{Edge:setOption}(\meta{name},\meta{value})}
Sets the edge option \meta{name} to \meta{value}. 

Parameters:
\begin{parameterdescription}
	\item[\meta{name}] Name of the option to be changed.\item[\meta{value}] New value for the edge option \meta{name}. 
\end{parameterdescription}



\end{luacommand}

\end{filedescription}