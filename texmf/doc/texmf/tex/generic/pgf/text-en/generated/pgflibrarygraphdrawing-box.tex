% This file has been generated from the lua sources using LuaDoc.
% To regenerate it call "make genluadoc" in
% doc/generic/pgf/version-for-luatex/en.

\begin{filedescription}{pgflibrarygraphdrawing-box.lua}


\begin{luacommand}{{Box:addBox}(\meta{box})}
Adds new internal Box.

Parameters:
\begin{parameterdescription}
	\item[\meta{box}] The box to be added.
\end{parameterdescription}



\end{luacommand}\begin{luacommand}{{Box:getPaths}()}
Provides all Paths this box contains.


Return value:
\begin{itemize} \item[] Recursive iteration over all paths. \end{itemize}


\end{luacommand}\begin{luacommand}{{Box:getPosAt}(\meta{place},\meta{absolute})}
Calculates the coordinates of the box according to the place parameter.

Parameters:
\begin{parameterdescription}
	\item[\meta{place}] Determines of which position of the box the coordinates should be returned (e.g. the center of the box requires the param Box.CENTER).  Possible values are: \begin{itemize} \item Box.UPPERLEFT \item Box.UPPERRIGHT \item Box.CENTER \item Box.LOWERRIGHT \item Box.LOWERLEFT \end{itemize}\item[\meta{absolute}] If true the absolute coordinates of the box will be returned, otherwise its relative coordinates.
\end{parameterdescription}


Return value:
\begin{itemize} \item[] X- and y-coordinates of the box. \end{itemize}


\end{luacommand}\begin{luacommand}{{Box:new}(\meta{values})}
Creates a new box.

Parameters:
\begin{parameterdescription}
	\item[\meta{values}] Values (e.g. height) to be merged with the default-metatable of a box.
\end{parameterdescription}


Return value:
\begin{itemize} \item[] The new box. \end{itemize}


\end{luacommand}\begin{luacommand}{{Box:recalculateSize}()}
Checks internal Boxes and resets width and height.



\end{luacommand}\begin{luacommand}{{Box:removeBox}(\meta{box})}
Removes internal Box.

Parameters:
\begin{parameterdescription}
	\item[\meta{box}] The box to remove.
\end{parameterdescription}



\end{luacommand}
\end{filedescription}