% This file has been generated from the lua sources using LuaDoc.
% To regenerate it call "make genluadoc" in
% doc/generic/pgf/version-for-luatex/en.

\begin{filedescription}{pgflibrarygraphdrawing-table-helpers.lua}


\begin{luacommand}{{table.combine\textunderscore{}pairs}(\meta{table},\meta{combine\_func},\meta{initial\_value})}
Combine all key/value pairs of \meta{table} to a single value using a combine function.  This is a very powerful function. It can be used for combining the key/value pairs of a table into a single string but can also be used to compute mathematical operations on tables, such as finding the maximum value in a table etc.  The main difference to |table.combine_values| is that keys and values are used to determine the combination value and that the key/value pairs are are passed to \meta{combine\_func} in a random order. 

Parameters:
\begin{parameterdescription}
	\item[\meta{table}] Table to iterate over.\item[\meta{combine\_func}] Function to be called for each key/value pair. It takes three parameters, the current combination value and the key/value pair. It is supposed to return a new combination value.\item[\meta{initial\_value}] Initial combination value. 
\end{parameterdescription}


Return value:
\begin{parameterdescription} 
  \item[] The final combination value after all key/value pairs have been passed over to \meta{combine\_func}. 
\end{parameterdescription}


\end{luacommand}
\begin{luacommand}{{table.combine\textunderscore{}values}(\meta{input},\meta{combine\_func},\meta{initial\_value})}
Combine all values of \meta{input} to a single value using a combine function.  This is a very powerful function. It can be used for combining the values of a table into a single string but can also be used to compute mathematical operations on tables, such as finding the maximum value in a table etc.  The main difference to |table.combine_pairs| is that the keys are ignored and that the values are passed to \meta{combine\_func} in the order they appear in the table. 

Parameters:
\begin{parameterdescription}
	\item[\meta{input}] Table to iterate over.\item[\meta{combine\_func}] Function to be called for each value. It takes two parameters, the current combination value and the current value. It is supposed to return a new combination value.\item[\meta{initial\_value}] Initial combination value. 
\end{parameterdescription}


Return value:
\begin{parameterdescription} 
  \item[] The final combination value after all values of \meta{input} have been passed over to \meta{combine\_func}. 
\end{parameterdescription}


\end{luacommand}
\begin{luacommand}{{table.copy}(\meta{source},\meta{target})}
Copies a table while preserving its metatable. 

Parameters:
\begin{parameterdescription}
	\item[\meta{source}] The table to copy.\item[\meta{target}] The table to which values are to be copied or |nil| if a new table is to be allocated. 
\end{parameterdescription}


Return value:
\begin{parameterdescription} 
  \item[] The \meta{target} table or a newly allocated table containing all keys and values of the \meta{source} table. 
\end{parameterdescription}


\end{luacommand}
\begin{luacommand}{{table.count\textunderscore{}pairs}(\meta{input})}
Count the key/value pairs in the table. 

Parameters:
\begin{parameterdescription}
	\item[\meta{input}] The table whose key/value pairs to count. 
\end{parameterdescription}


Return value:
\begin{parameterdescription} 
  \item[] Number of key/value pairs in the table. 
\end{parameterdescription}


\end{luacommand}
\begin{luacommand}{{table.filter\textunderscore{}keys}(\meta{table},\meta{filter\_func})}
Copies a table and filters out all keys using a function. 

Parameters:
\begin{parameterdescription}
	\item[\meta{table}] The table whose values are to be filtered.\item[\meta{filter\_func}] The test function to be called for each key of \meta{table}. If it returns |false| or |nil| for a key, that key will not be part of the result table. 
\end{parameterdescription}


Return value:
\begin{parameterdescription} 
  \item[] Copy of \meta{table} with its keys filtered using \meta{filter\_func}. 
\end{parameterdescription}


\end{luacommand}
\begin{luacommand}{{table.filter\textunderscore{}pairs}(\meta{table},\meta{filter\_func})}
Copies a table and filters out all key/value pairs using a function. 

Parameters:
\begin{parameterdescription}
	\item[\meta{table}] The table whose values are to be filtered.\item[\meta{filter\_func}] The test function to be called for each pair of \meta{table}. If it returns |false| or |nil| for a pair, that pair will not be part of the result table. 
\end{parameterdescription}


Return value:
\begin{parameterdescription} 
  \item[] Copy of \meta{table} with its pairs filtered using \meta{filter\_func}. 
\end{parameterdescription}


\end{luacommand}
\begin{luacommand}{{table.filter\textunderscore{}values}(\meta{input},\meta{filter\_func})}
Copies a table and filters out all values using a function. 

Parameters:
\begin{parameterdescription}
	\item[\meta{input}] The table whose values are to be filtered.\item[\meta{filter\_func}] The test function to be called for each value of the input table. If it returns |false| or |nil| for a value, that value will not be part of the result table. 
\end{parameterdescription}


Return value:
\begin{parameterdescription} 
  \item[] Copy of \meta{input} with its values filtered using \meta{filter\_func}. 
\end{parameterdescription}


\end{luacommand}
\begin{luacommand}{{table.find}(\meta{table},\meta{find\_func})}
Returns the first value in \meta{table} for which \meta{find\_func} returns |true|. 

Parameters:
\begin{parameterdescription}
	\item[\meta{table}] The table to search in.\item[\meta{find\_func}] A function to test values with. It receives a single parameter (a value of \meta{table}) and is supposed to return either |true| or |false|. 
\end{parameterdescription}


Return value:
\begin{parameterdescription} 
  \item[] The first value of \meta{table} for which \meta{find\_func} returns true. Returns |nil| if the function was |false| for al of the values in \meta{table}. 
\end{parameterdescription}


\end{luacommand}
\begin{luacommand}{{table.find\textunderscore{}index}(\meta{table},\meta{find\_func})}
Returns the index of the first value in \meta{table} for which \meta{find\_func} returns |true|. 

Parameters:
\begin{parameterdescription}
	\item[\meta{table}] The table to search in.\item[\meta{find\_func}] A function to test values with. It receives a single parameter (a value of \meta{table}) and is supposed to return either |true| or |false|. 
\end{parameterdescription}


Return value:
\begin{parameterdescription} 
  \item[] Index of the first value of \meta{table} for which \meta{find\_func} returns |true|. Returns |nil| if the function was |false| for all of the values in \meta{table}. 
\end{parameterdescription}


\end{luacommand}
\begin{luacommand}{{table.key\textunderscore{}iter}(\meta{table})}
Iterate over all keys of a table in random order. 

Parameters:
\begin{parameterdescription}
	\item[\meta{table}] The table whose keys to iterate over. 
\end{parameterdescription}


Return value:
\begin{parameterdescription} 
  \item[] An iterator for the keys of the table. 
\end{parameterdescription}


\end{luacommand}
\begin{luacommand}{{table.map}(\meta{input},\meta{map\_func})}
Maps key/value pairs of an \meta{input} table to a flat table of new values. 

Parameters:
\begin{parameterdescription}
	\item[\meta{input}] Table whose key/value pairs are to be mapped to new values.\item[\meta{map\_func}] The mapping function to be called for each key/value pair of \meta{input}. The value it returns for a pair will be inserted into the result table. 
\end{parameterdescription}


Return value:
\begin{parameterdescription} 
  \item[] A new table containing all values returned by \meta{map\_func} for the key/value pairs of the \meta{input} table. 
\end{parameterdescription}


\end{luacommand}
\begin{luacommand}{{table.map\textunderscore{}keys}(\meta{table},\meta{map\_func})}
Maps keys of a table to new keys in a copy of the table. 

Parameters:
\begin{parameterdescription}
	\item[\meta{table}] The table whose keys are to be mapped to new keys.\item[\meta{map\_func}] A function to be called for each key of \meta{table} in order to generate a new key to replace the old one in the result table. 
\end{parameterdescription}


Return value:
\begin{parameterdescription} 
  \item[] A new table with all keys of \meta{table} having been replaced with the keys returned from \meta{map\_func}. The original values are preserved. 
\end{parameterdescription}


\end{luacommand}
\begin{luacommand}{{table.map\textunderscore{}pairs}(\meta{table},\meta{map\_func})}
Maps keys and values of a table to new pairs of keys and values. 

Parameters:
\begin{parameterdescription}
	\item[\meta{table}] The table whose key and value pairs are to be replaced.\item[\meta{map\_func}] A function to be called for each key and value pair of \meta{table} in order to generate a new pair to replace the old one. 
\end{parameterdescription}


Return value:
\begin{parameterdescription} 
  \item[] A new table with all key and value pairs of \meta{table} having been replaced with the pairs returned from \meta{map\_func}. 
\end{parameterdescription}


\end{luacommand}
\begin{luacommand}{{table.map\textunderscore{}values}(\meta{input},\meta{map\_func})}
Maps values of a table to new values in a new table. 

Parameters:
\begin{parameterdescription}
	\item[\meta{input}] The table whose values are to be mapped to new values.\item[\meta{map\_func}] A function to be called for each value in order to generate a new value to replace the old one in the result table. 
\end{parameterdescription}


Return value:
\begin{parameterdescription} 
  \item[] A new table with all values of the \meta{input} table having been replaced with the values returned from \meta{map\_func}. 
\end{parameterdescription}


\end{luacommand}
\begin{luacommand}{{table.merge}(\meta{table1},\meta{table2},\meta{first\_metatable})}
Merges the key/value pairs of two tables.  This function merges the key/value pairs of the two input tables.  All |nil| values of the first table are overwritten by the corresponding values of the second table.  By default the metatable of the second input table is applied to the resulting table. If \meta{first\_metatable} is set to |true| however, the metatable of the first input table will be used. 

Parameters:
\begin{parameterdescription}
	\item[\meta{table1}] First table with key/value pairs.\item[\meta{table2}] Second table with key/value pairs.\item[\meta{first\_metatable}] Whether to inherit the metatable of \meta{table1} or not. 
\end{parameterdescription}


Return value:
\begin{parameterdescription} 
  \item[] A new table with the key/value pairs of the two input tables merged together. 
\end{parameterdescription}


\end{luacommand}
\begin{luacommand}{{table.randomized\textunderscore{}pair\textunderscore{}iter}(\meta{table})}
Iterate over the key/value pairs of \meta{table} in a truely random order. 

Parameters:
\begin{parameterdescription}
	\item[\meta{table}] The table whose key/value pairs to iterate over. 
\end{parameterdescription}


Return value:
\begin{parameterdescription} 
  \item[] A randomized iterator for the values of \meta{table}. 
\end{parameterdescription}


\end{luacommand}
\begin{luacommand}{{table.randomized\textunderscore{}value\textunderscore{}iter}(\meta{table})}
Iterate over the values of \meta{table} in a truely random order. 

Parameters:
\begin{parameterdescription}
	\item[\meta{table}] The table whose values to iterate over. 
\end{parameterdescription}


Return value:
\begin{parameterdescription} 
  \item[] A randomized iterator for the values of the table. 
\end{parameterdescription}


\end{luacommand}
\begin{luacommand}{{table.remove\textunderscore{}values}(\meta{input},\meta{remove\_func})}
Removes all values from \meta{input} for which \meta{remove\_func} returns |true|.  Important note: this method does not work with dictionaries. Make sure only to process number-indexed arrays with it. 

Parameters:
\begin{parameterdescription}
	\item[\meta{input}] The table to remove values from.\item[\meta{remove\_func}] Function to be called for each value of \meta{input}. If it returns |false|, the value will be removed from the table in-place. 
\end{parameterdescription}


Return value:
\begin{parameterdescription} 
  \item[] \meta{input} which was edited in-place. 
\end{parameterdescription}


\end{luacommand}
\begin{luacommand}{{table.update\textunderscore{}values}(\meta{table},\meta{update\_func})}
Update values of \meta{table} in-place using an update function. 

Parameters:
\begin{parameterdescription}
	\item[\meta{table}] The table whose values are to be updated.\item[\meta{update\_func}] A function that takes two parameters, the key/value pairs of \meta{table} and returns a new value to replace the old one. 
\end{parameterdescription}


Return value:
\begin{parameterdescription} 
  \item[] The input \meta{table}. 
\end{parameterdescription}


\end{luacommand}
\begin{luacommand}{{table.value\textunderscore{}iter}(\meta{table})}
Iterate over all values of a table.  FIXME: The iterators stops if a key's value is nil. But we actually want to continue iterating until the end of the table. 

Parameters:
\begin{parameterdescription}
	\item[\meta{table}] The table whose values to iterate over. 
\end{parameterdescription}


Return value:
\begin{parameterdescription} 
  \item[] An iterator for the values of the table. 
\end{parameterdescription}


\end{luacommand}

\end{filedescription}