\section{Mindmap Drawing Library}

\begin{package}{pgflibrarytikzmindmap}
  This packages provides styles for drawing mindmap diagrams.
\end{package}

\subsection{Overview}

This library is intended to make the creation of mindmaps easier. A
\emph{mindmap} is a graphical representation of a concept together
with related concepts and annotations. Mindmaps are, essentially,
trees, possibly with a few extra edges added, but they are usually
drawn in a special way: The root concept is placed in the middle of
the page and is drawn as a huge circle, ellipse, or cloud. The related
concepts then ``leave'' this root concept in branch-like tendrils.

The mindmap library of \tikzname\ produces mindmaps that look a bit
different from the standard mindmaps: While the big root concept is
still a circle, related concepts are also depicted as (smaller)
circles. The related concepts are linked to the root concept via
organic-looking connections. The overall effect is visually rather
pleasing, but readers may not immediately think of a mindmap when they
see a picture created with this library.

Although it is not strictly necessary, you will usually create
mindmaps using \tikzname's tree mechanism and some of the styles and
macros of the package work best when used inside trees. However, it is
still possible and sometimes necessary to treat parts of a mindmap as
a graph with arbitrary edges and this is also possible.


\subsection{The Mindmap Style}

Every mindmap should be put in a scope or a picture where the
|mindmap| style is used. This style installs some internal settings.

\begin{itemize}
  \itemstyle{mindmap}
  Use this style with all pictures or at least scopes that contain a
  mindmap. It installs a whole bunch of settings that are useful for
  drawing mindmaps. 
\begin{codeexample}[]
\tikz[mindmap,concept color=red!50]
  \node [concept] {Root concept}
    child[grow=right] {node[concept] {Child concept}};
\end{codeexample}
  The sizes of concepts are predefined in such a way that a
  medium-size mindmap will fit on an A4 page (more or less).  
  \itemstyle{every mindmap}
  This style is included by the |mindmap| style. Change this style to
  add special settings to your mindmaps.
\begin{codeexample}[]
\tikz[large mindmap,concept color=red!50]
  \node [concept] {Root concept}
    child[grow=right] {node[concept] {Child concept}};
\end{codeexample}
  \itemstyle{large mindmap}
  This style includes the |mindmap| style, but additionally changes
  the default size of concepts and of distances so that a medium-sized
  mindmap will fit on an A3 page (A3 pages are twice as large as A4
  pages).
  \itemstyle{huge mindmap}
  This style causes conepts to be even bigger and it is best used with
  A2 paper and above.
\end{itemize}

\subsection{Concepts Nodes}

The basic entities of mindmaps are called \emph{concepts} in
\tikzname. A concept is a node of style |concept| and it must be
circular for some of the connection macros to work.


\subsubsection{Isolated Concepts}

The following styles influence how isolated concepts are rendered:

\begin{itemize}
  \itemstyle{concept}
  This style should be used with all nodes that are concepts, although
  some styles like |extra concept| install this style automatically.

  Bascially, this style makes the concept node circular and installs a
  uniform color called |concept color|, see below. Additionally, the
  style |every concept| is called.
\begin{codeexample}[]
\tikz[mindmap,concept color=red!50] \node [concept] {Some concept};
\end{codeexample}
  \itemstyle{every concept}
  In order to change the appearance of concept nodes, you should
  change this style. Note, however, that the color of a concept should
  be uniform for some of the connection bar stuff to work, so you
  should not change the color or the draw/fill state of concepts using
  this option. It is mostly useful for changing the text color and
  font.
  \itemoption{concept color}|=|\meta{color}
  This option tells \tikzname\ which color should be used for filling
  and stroking concepts. The difference between this option and just
  setting |every concept| to the desired color is that this option
  allows \tikzname\ to keep track of the colors used for
  concepts. This is important when you \emph{change} the color between
  two connected concepts. In this case, \tikzname\ can automatically
  create a shading that provides a smooth transition between the old
  and the new concept color; we will come back to this in the next
  section. 
  \itemstyle{extra concept}
  This style is intended for concepts that are not part of the
  ``mindmap tree,'' but stand beside it. Typically, they will have a
  subdued color are be smaller. In order to have these concepts appear
  in a uniform way and in order to indicate in the code that these
  concepts are extra, you can use this style.
\begin{codeexample}[]
\begin{tikzpicture}[mindmap,concept color=blue!80]
  \node [concept]                 {Root concept};
  \node [extra concept] at (10,0) {extra concept};
\end{tikzpicture}
\end{codeexample}
  \itemstyle{every extra concept}
  Change this style to change the appearance of extra concepts.
\end{itemize}


\subsubsection{Concepts in Trees}

As pointed out earlier, \tikzname\ assumes that your mindmap is build
using the |child| facilities of \tikzname. There are numerous options
that influence how concepts are rendered at the different levels of a
tree. 

\begin{itemize}
  \itemstyle{root concept}
  This style is used for the roots of mindmap trees. More precisely,
  this style is included by the |mindmap| style. Thus, by adding
  something to this, you can change how the root of a mindmap will be
  rendered.
\begin{codeexample}[]
\tikzstyle{root concept}+=[concept color=blue!80,minimum size=3.5cm]    
\tikz[mindmap] \node [concept] {Root concept};
\end{codeexample}

  Note that styles like |large mindmap| redefine these styles, so you
  should add something to this style only inside the picture.
  \itemstyle{level 1 concept}
  The |mindmap| style adds this style to the |level 1| style. This
  means that the first level children of a mindmap tree will use this
  style. 
\begin{codeexample}[]
\tikzstyle{root concept}+=[concept color=blue!80]    
\tikzstyle{level 1 concept}+=[concept color=red!50]    
\tikz[mindmap]
  \node [concept] {Root concept}
    child[grow=30] {node[concept] {child}}
    child[grow=0 ] {node[concept] {child}};
\end{codeexample}
  \itemstyle{level 2 concept}
  works like |level 1 concept|, only for second level children. 
  \itemstyle{level 3 concept}
  works like |level 1 concept|.
  \itemstyle{level 4 concept}
  works like |level 1 concept|. Note that there are not fifth and
  higher level styles, you need to modify |level 5| directly in such
  cases. 
  
  \itemoption{concept color}|=|\meta{color}
  We saw already that this option is used to change the color of
  concepts. We now have a look at its effect when used on child nodes
  of a concept. Normally, this option simply changes the color of the
  children. However, when the option is given as an option to the
  |child| operation (and not to the |node| operation and also not as
  an option to all children via the |level 1| style), \tikzname\ will
  smoothly change the concept color from the parent's color to the
  color of the child concept. 

  Here is an example:
\begin{codeexample}[]
\tikz[mindmap,concept color=blue!80]
  \node [concept] {Root concept}
    child[concept color=red,grow=30] {node[concept] {Child concept}}
    child[concept color=orange,grow=0]  {node[concept] {Child concept}};
\end{codeexample}

  In order to have all children of a certain level have a different
  concept color, a tiny bit of magic is needed:
\begin{codeexample}[]
\tikzstyle{root concept}+=[concept color=blue]    
\tikzstyle{level 1 concept}+=[set style={{every child}=[concept color=blue!50]}]    
\tikz[mindmap,text=white]
  \node [concept] {Root concept}
    child[grow=30] {node[concept] {child}}
    child[grow=0 ] {node[concept] {child}};
\end{codeexample}
\end{itemize}

\subsection{Connecting Concepts}

\subsection{Adding Annotations}

An \emph{annotation} is some text outside a mindmap that, unlike an
extra concept, simply explains something in the mindmap. The following
style is mainly intended to help readers of the code see that a node
in an annotation node.

\begin{itemize}
  \itemstyle{annotation}
  This style indicates that a node is an annotation node. It includes
  |every annotation|, which allows you to change this style in a
  convenient fashion.
\begin{codeexample}[]
\tikzstyle{every annotation}=[fill=red!20]    
\begin{tikzpicture}[mindmap,concept color=blue!80]
  \node [concept] (root)  {Root concept};

  \node [annotation,right] at (root.east)
  {The root concept is, in general, the most important concept.};
\end{tikzpicture}
\end{codeexample}
  \itemstyle{every annotation}
    This style is included by |annotation|.
\end{itemize}



%%% Local Variables: 
%%% mode: latex
%%% TeX-master: "pgfmanual-pdftex-version"
%%% End: 
