% Copyright 2010 by Till Tantau
%
% This file may be distributed and/or modified
%
% 1. under the LaTeX Project Public License and/or
% 2. under the GNU Free Documentation License.
%
% See the file doc/generic/pgf/licenses/LICENSE for more details.

\section{Style Sheets and Legends}
\label{section-dv-style-sheets}

\subsection{Overview}

In many data visualizations, different sets of data need to be
visualized in a single visualization. For instance, in a plot there
might be a line for the sine of~$x$ and another line for the cosine
of~$x$; in another visualization there might be a set of points
representing data from a first experiment and another set of points
representing data from a second experiment; and so on. In order to
indicate to which data set a data point belongs, one might plot the
curve of the sine in, say, black, and the curve of the cosine in red;
we might plot the data from the fist experiment using stars and the
data from the second experiment using circles; and so on. Finally, at
some place in the visualization -- either inside the data or in a
legend next to it -- the meaning of the colors or symbols need to be
explained.

Just as you would like \tikzname\ to map the data points automatically
onto the axes, you will also typically wish \tikzname\ to choose for
instance the coloring of the lines automatically for you. This is done
using \emph{style sheets}. There are at least two good reasons why you
should prefer style sheets over configuring the styling of each
visualizer ``by hand'' using the |style| key:
\begin{enumerate}
\item It is far more convenient to just say
  |style sheet=strong colors| than having to individually
  picking the different colors.
\item The style sheets were chosen and constructed rather
  carefully.

  For instance, the |strong colors| style sheet does not
  pick colors like pure green or pure yellow, which have very low
  contrast with respect to a white background and which often lead to
  unintelligible graphics. Instead, opposing primary colors with
  maximum contrast on a white background were picked that are visually
  quite pleasing.

  Similarly, the different dashing style sheets are
  constructed in such a way that there are only few and small gaps in
  the dashing so that no data points get lost because the dashes are
  spaced too far apart. Also dashing patterns were chosen that have a
  maximum optical difference.

  As a final example, style sheets for
  plot marks are constructed in such a way that even when two plot
  marks lie directly on top of each other, they are still easily
  distinguishable. 
\end{enumerate}
The bottom line is that whenever possible, you should use one of the
predefined style sheets rather than picking colors or dashings at
random.

\subsection{Concepts: Style Sheets}

A \emph{style sheet} is a predefined list of styles such as a list of
colors, a list of dashing pattern, a list of plot marks, or a
combinations thereof. A style sheet can be \emph{attached} to a data
point attribute. Then, the value of this attribute is used with data
points to choose which style in the list should be chosen to visualize
the data point.

In most cases, there is just one attribute to which style sheets get
attached: the |/data point/visualizer| attribute. The effect of
attaching a style sheet to this attribute is that each visualizer is
styled differently.

For the following examples, let us first define a simple data set:
\begin{codeexample}[]
\tikz \datavisualization data group {function classes} = {
  data [set=log, format=function] {
    var x : interval [0.2:2.5];
    func y = ln(\value x);
  }
  data [set=lin, format=function] {
    var x : interval [-2:2.5];
    func y = 0.5*\value x;
  }
  data [set=squared, format=function] {
    var x : interval [-1.5:1.5];
    func y = \value x*\value x;
  }
  data [set=exp, format=function] {
    var x : interval [-2.5:1];
    func y = exp(\value x);
  }
};
\end{codeexample}

\begin{codeexample}[width=6cm]
\tikz \datavisualization [
  school book axes, all axes={unit length=7.5mm},
  visualize as smooth line/.list={log, lin, squared, exp},
  style sheet=strong colors]
data group {function classes};
\end{codeexample}

\begin{codeexample}[width=6cm]
\tikz \datavisualization [
  school book axes, all axes={unit length=7.5mm},
  visualize as smooth line/.list={log, lin, squared, exp},
  style sheet=vary dashing]
data group {function classes};
\end{codeexample}



\subsection{Concepts: Legends}
\label{section-dv-labels-in}

A \emph{legend} is a box that is next to a data visualization (or
inside it at some otherwise empty position) that contains a textual
explanation of the different colors or styles used in a data
visualization.

Just as it is difficult to get colors and dashing patterns right ``by
hand,'' it is also difficult to get a legend right. For instance, when
a small line is shown in the legend that represents the actual line in
the data visualization, if the line is too short and the dashing is
too large, it may be impossible to discern which dashing is actually
meant. Similarly, when plot marks are shown on such a short line,
using a simple straight line may make it hard to read the plot marks
correctly.

The data visualization engine makes some effort to make it easy to
create high-quality legends. Additionally, it also offers ways of
easily adding labels for visualizers directly inside the data
visualization, which is even better than adding a legend, in general.

\begin{codeexample}[width=7cm]
\tikz \datavisualization [
  school book axes, all axes={unit length=7.5mm},
  x axis={label=$x$},
  visualize as smooth line/.list={log, lin, squared, exp},
  legend,
  log=    {label in legend={text=$\log x$}},
  lin=    {label in legend={text=$x/2$}},
  squared={label in legend={text=$x^2$}},
  exp=    {label in legend={text=$e^x$}},
  style sheet=vary dashing]
data group {function classes};
\end{codeexample}


\begin{codeexample}[width=6.3cm]
\tikz \datavisualization [
  school book axes,
  x axis={label=$x$},
  visualize as smooth line/.list={log, lin, squared, exp},
  every data set label/.append style={text colored},
  log=    {pin in data={text'=$\log x$, when=y is -1}},
  lin=    {pin in data={text=$x/2$, when=x is 2,
                        pin length=1ex}},
  squared={pin in data={text=$x^2$, when=x is 1.1,
                        pin angle=230}},
  exp=    {label in data={text=$e^x$, when=x is -2}},
  style sheet=vary hue]
data group {function classes};
\end{codeexample}


\subsection{Usage: Style Sheets}

\subsubsection{Picking a Style Sheet}

To use a style sheet, you need to \emph{attach} it to an
attribute. You can attach multiple style sheets to an attribute and
in this case all of these style sheets can influence the appearance of
the data points.

Most of the time, you will attach a style sheet to the |set|
attribute. This has the effect that each different data set inside the
same visualization is rendered in a different way. Since this use of
style sheets is the most common, there is a special, easy-to-remember
option for this:

\begin{key}{/tikz/data visualization/style sheet=\meta{style sheet}}
  Adds the \meta{style sheet} to the list of style sheets attached to
  the |set| attribute.
\begin{codeexample}[width=6cm]
\tikz \datavisualization [
  school book axes, all axes={unit length=7.5mm},
  visualize as smooth line/.list={log, lin, squared, exp},
  style sheet=vary thickness and dashing,
  style sheet=vary hue]
data group {function classes};
\end{codeexample}
\end{key}

While the |style sheet| key will attach a style sheet only to the
|set| attribute, the following key handler can be used to attach a
style sheet to an arbitrary attribute:


\begin{handler}{{.style sheet}=\meta{style sheet}}
  Inside a data visualization you can use this key handler together
  with an attribute, that is, with a key having the path prefix
  |/data point|. For instance, in order to attach the \meta{style
    sheet} |strong colors| to the attribute |set|, you could write
\begin{codeexample}[code only]
/data point/set/.style sheet=strong colors    
\end{codeexample}
  Indeed, the |style sheet| key is just a shorthand for the above.

  The effect of attaching a style sheet is the following:
  \begin{itemize}
  \item A new object is created that will monitor the attribute.
  \item Each time a special \emph{styling key} is emitted by the data
    visualization engine, this object will inspect the current value
    of the attribute to which it is attached.
  \item Depending on this value, one of the styles stored in the style
    sheet is chosen (how this works, exactly, will be explained in a
    moment).
  \item The chosen style is then locally applied.
  \end{itemize}
  
  In reality, things are a bit more complicated: If the attribute of
  the data point happens to have a subkey named in the same way as the
  value, then the value of is this subkey is used instead of the value
  itself. This allows you to ``rename'' a value.
  
  In a sense, a style sheet behaves much like a visualizer (see
  Section~\ref{section-dv-visualizers}): In accordance with the value
  of a certain attribute, the appearance of data points
  change. However, there are a few differences: First, the styling of
  a data point needs to be triggered explicitly and this triggering is
  not necessarily done for each data point individually, but only for
  a whole visualizer. Second, styles can be computed even when no data
  point is present. This is useful for instance in a legend since,
  here, a visual representation of a visualizer needs to be created
  independently of the actual data points.
\end{handler}

\subsubsection{Creating a New Style Sheet}

Creating a style sheet works as follows: For each
possible value that an attribute can attain we must specify a
style. This is done by creating a style key for each such possible
value with a special path prefix and setting this style key to the
desired value. The special path prefix is
|/pgf/data visualization/style sheets| followed by the name of the
style sheet.

As an example, suppose we wish to create a style sheet |test| that makes
styled data points |red| when the attribute has value |foo| and
|green| when the attribute has value |bar| and |dashed, blue| when the
attribute is |foobar|. We could then write
\begin{codeexample}[code only]
/pgf/data visualization/style sheets/test/foo/.style={red},    
/pgf/data visualization/style sheets/test/bar/.style={green},    
/pgf/data visualization/style sheets/test/foobar/.style={dashed, blue},    
\end{codeexample}

We could then attach this style sheet to the attribute |code| as
follows:
\begin{codeexample}[code only]
/data point/code/.style sheet=test
\end{codeexample}

Then, when |/data point/code=foobar| holds when the styling signal is
raised, the stying |dashed, blue| will get executed.

A natural question arises concerning the situation that the value of
the attribute is not defined as a subkey of the style sheet. In this
case, a special key gets executed:

\begin{stylekey}{/pgf/data visualization/style sheets/\meta{style
      sheet}/default style=\meta{value}}
  This key gets during styling whenever
  |/pgf/data visualization/style sheet/|\meta{style
    sheet}|/|\meta{value} is not defined. 
\end{stylekey}

Let us put all of this together in a real-life example. Suppose we
wish to create a style sheet that makes the first data set |green|, the
second |yellow| and the third one |red|. Further data sets should be,
say, |black|. The attribute that we intend to style is the |set|
attribute. For the moment, we assume that the data sets will be named
|1|, |2|, |3|, and so on (instead of, say, |experiment 1| or |sin| or
something more readable -- we will get rid of this restriction in a
minute).

We would now write:

\begin{codeexample}[]
\pgfkeys{
  /pgf/data visualization/style sheets/traffic light/.cd,
  % All these styles have the above prefix.
  1/.style={green!50!black},
  2/.style={yellow!90!black},
  3/.style={red!80!black},
  default style/.style={black}
}
\tikz \datavisualization [
  school book axes,
  visualize as line=1,
  visualize as line=2,
  visualize as line=3,
  style sheet=traffic light]
data point [x=0, y=0, set=1]
data point [x=2, y=2, set=1]
data point [x=0, y=1, set=2]
data point [x=2, y=1, set=2]
data point [x=0.5, y=1.5, set=3]
data point [x=2.25, y=1.75, set=3];
\end{codeexample}

In the above example, we have to name the visualizers |1|, |2|, |3|
and so one since the value of the |set| attribute is used both assign
data points to visualizers and also pick a style sheet. However, it
would be much nicer if we could name any way we want. To achieve this,
we use the special rule for style sheets that says that if there is a
subkey of an attribute whose name is the same name as the value, then
the value of this key is used instead. This slightly intimidating
definition is much easier to understand when we have a look at an
example:

\pgfkeys{
  /pgf/data visualization/style sheets/traffic light/.cd,
  % All these styles have the above prefix.
  1/.style={green!50!black},
  2/.style={yellow!90!black},
  3/.style={red!80!black},
  default style/.style={black}
}

\begin{codeexample}[]
% Definition of traffic light keys as above  
\begin{tikzpicture}
  \datavisualization data group {lines} = {  
    data point [x=0, y=0,       set=normal]
    data point [x=2, y=2,       set=normal]
    data point [x=0, y=1,       set=heated]
    data point [x=2, y=1,       set=heated]
    data point [x=0.5, y=1.5,   set=critical]
    data point [x=2.25, y=1.75, set=critical]
  };
  \datavisualization [
    school book axes,
    visualize as line=normal,
    visualize as line=heated,
    visualize as line=critical,
    /data point/set/normal/.initial=1,
    /data point/set/heated/.initial=2,
    /data point/set/critical/.initial=3,
    style sheet=traffic light]
  data group {lines};
\end{tikzpicture}
\end{codeexample}

Now, it is a bit bothersome that we have to set all these
|/data point/set/...| keys by hand. It turns out that this is not
necessary: Each time a visualizer is created, a subkey of
|/data point/set| with the name of the visualizer is created
automatically and a number is stored that is increased for each new
visualizer in a data visualization. This means that the three lines
starting with |/data point| are inserted automatically for you, so
they can be left out. However, you would need them for instance when
you would like several different data sets to use the same styling:


\begin{codeexample}[]
% Definition of traffic light keys as above  
\tikz \datavisualization [
  school book axes,
  visualize as line=normal,
  visualize as line=heated,
  visualize as line=critical,
  /data point/set/critical/.initial=1, % same styling as first set
  style sheet=traffic light]
data group {lines};
\end{codeexample}

We can a command that slightly simplifies the definition of style
sheets:

\begin{command}{\pgfdvdeclarestylesheet\marg{name}\marg{keys}}
  This command executes the \meta{keys} with the path prefix
  |/pgf/data visualization/style sheets/|\penalty0\meta{name}. The above
  definition of the traffic light style sheet could be rewritten as
  follows:
\begin{codeexample}[code only]
\pgfdvdeclarestylesheet{traffic light}{
  1/.style={green!50!black},
  2/.style={yellow!90!black},
  3/.style={red!80!black},
  default style/.style={black}
}
\end{codeexample}
\end{command}

As a final example, let us create a style sheet that changes the
dashing pattern according to the value of the attribute. We do not
need to define an large number of styles in this case, but can use the
|default style| key to ``calculate'' the correct dashing.

\begin{codeexample}[]
\pgfdvdeclarestylesheet{my dashings}{
  default style/.style={dash pattern={on #1pt off 1pt}}
}
\tikz \datavisualization [
  school book axes,
  visualize as line=normal,
  visualize as line=heated,
  visualize as line=critical,
  style sheet=my dashings]
data group {lines};
\end{codeexample}

\subsubsection{Creating a New Color Style Sheet}

Creating a style sheet that varies colors according to an attribute
works the same way as creating a normal style sheet: Subkeys lies |1|,
|2|, and so on use the |style| attribute to setup a color. However,
instead of using the |color| attribute to set the color, you should
use the |visualizer color| key to set the color:

\begin{key}{/tikz/visualizer color=\meta{color}}
  This key is used to set the color |visualizer color| to
  \meta{color}. This color is used by visualizers to color the data
  they visualize, rather than the current ``standard color.'' The
  reason for not using the normal current color is simply that it
  makes many internals of the data visualization engine a bit
  simpler. 
\begin{codeexample}[]
\pgfdvdeclarestylesheet{my colors}
{
  default style/.style={visualizer color=black},
  1/.style={visualizer color=black},
  2/.style={visualizer color=red!80!black},
  3/.style={visualizer color=blue!80!black},
}
\tikz \datavisualization [
  school book axes,
  visualize as line=normal,
  visualize as line=heated,
  visualize as line=critical,
  style sheet=my colors]
data group {lines};
\end{codeexample}
\end{key}

There is an additional command that makes it easy to define a style
sheet based on a \emph{color series}. Color series are a concept from
the |xcolor| package: The idea is that we start with a certain color
for the first data set and then add a certain ``color offset'' for
each next data point. Please consult the documentation of the |xcolor|
package for details.

\begin{command}{\tikzdvdeclarestylesheetcolorseries\marg{name}\marg{color
      model}\marg{initial color}\marg{step}}
  This command creates a new style sheet using
  |\pgfdvdeclarestylesheet|. This style sheet will only have a default
  style setup that maps numbers to the color in the color series
  starting with \meta{initial color} and having a stepping of
  \meta{step}. Note that when the value of the attribute is |1|, which
  it is the first data set, the \emph{second} color in the color
  series is used (since counting starts at |0| for color
  series). Thus, in general, you need to start the \meta{initial
    color} ``one early.''
\begin{codeexample}[]
\tikzdvdeclarestylesheetcolorseries{greens}{hsb}{0.3,1.3,0.8}{0,-.4,-.1}
\tikz \datavisualization [
  school book axes,
  visualize as line=normal,
  visualize as line=heated,
  visualize as line=critical,
  style sheet=greens]
data group {lines};
\end{codeexample}

\end{command}


\subsection{Usage: Labeling Data Sets}

In a visualization that contains multiple data sets, it is often
necessary to clearly point out which line or mark type corresponds to
which data set. This can be done in the main text via a sentence like
``the normal data (black) lies clearly below the critical values
(red),'' but it often a good idea to indicate data sets ideally
directly inside the data visualization or directly next to it in a
so-called legend.

The data visualization engine has direct support both for indicating
data sets directly inside the visualization and also for indicating
them in a legend.

\subsubsection{Labeling Data Sets Inside the Visualization}

The ``best'' way of indicating where a data set lies or which color is
used for it is to put a label directly inside the data
visualization. The reason this is the ``best'' way is that people do
not have to match the legend entries against the data, let alone
having to look up the meaning of line styles somewhere in the
text. However, adding a label directly inside the visualization is
also the most tricky way of indicating data sets since it is hard to
compute good positions for the labels automatically and since there
needs to be some empty space where the label can be put.

The following key is used to create a label inside the data
visualization for a data set:

\begin{key}{/tikz/data visualization/visualizer options/label in data=\meta{options}}
  This key is passed to a visualizer that has previously been created
  using keys starting |visualize as ...|. It will create a label
  inside the data visualization ``next'' to the visualizer (the
  details are explained in a moment). You can use this key multiple
  times with a visualizer to create multiple labels at different
  points with different texts.

  The \meta{options} determine which text is shown and where it is
  shown. They are executed with the following path prefix:
\begin{codeexample}[code only]
/tikz/data visualization/visualizer label options
\end{codeexample}

  In order to configure which text is shown and where, use the
  following keys inside the \meta{options}:
  
  \begin{key}{/tikz/data visualization/visualizer label options/text=\meta{text}}
    This is the text that will be displayed next to the data. It will
    be to the ``left'' of the data, see the description below.
  \end{key}
  \begin{key}{/tikz/data visualization/visualizer label options/text'=\meta{text}}
    Like |text|, only the text will be to the ``right'' of the data.
  \end{key}
  
  The following keys are used to configure where the label will be
  shown. They use different strategies to specify one data point where
  the label will be anchored. The coordinate of this data point will
  be stored in |(label| |visualizer| |coordinate)|. Independently of
  the strategy, once the data point has been chosen, the coordinate of
  the next data point is stored in |(label| |visualizer|
  |coordinate')|. Then, a (conceptual) line is created from the first
  coordinate to the second and a node is placed at the beginning of
  this line to its ``left'' or, for the |text'| option, on its ``right.'' More
  precisely, an automatic anchor is computed for a node placed
  implicitly on this line using the |auto| option or, for the
  |text'| option, using |auto,swap|.

  The node placed at the position computed in this way will have the
  \meta{text} set by the |text| or |text'| option and its styling is
  determined by the current |node style|.
  
  Let us now have a look at the different ways of determining the data
  point at which the label in anchored:
  \begin{key}{/tikz/data visualization/visualizer label
      options/when=\meta{attribute}| is|\meta{number}}
    This key causes the value of the \meta{attribute} to be monitored
    in the stream of data points. The chosen is data point is the
    first data point where the \meta{attribute} is at least
    \meta{number} (if this never happens, the last data point is used).
\begin{codeexample}[width=6.3cm]
\tikz \datavisualization [
  school book axes,
  x axis={label=$x$},
  visualize as smooth line/.list={log, lin, squared, exp},
  log=    {label in data={text'=$\log x$, when=y is -1,
                          text colored}},
  lin=    {label in data={text=$x/2$,     when=x is 2}},
  squared={label in data={text=$x^2$,     when=x is 1.1}},
  exp=    {label in data={text=$e^x$,     when=x is -2,
                          text colored}},
  style sheet=vary hue]
data group {function classes};
\end{codeexample}
  \end{key}
  \begin{key}{/tikz/data visualization/visualizer label
      options/index=\meta{number}}
    This key chooses the \meta{number}th data point belonging to the
    visualizer's data set.
\begin{codeexample}[width=6.3cm]
\tikz \datavisualization [
  school book axes,
  x axis={label=$x$},
  visualize as smooth line/.list={exp},
  exp=    {label in data={text=$5$, index=5},
           label in data={text=$10$, index=10},
           label in data={text=$20$, index=20},
           style={mark=x}},
  style sheet=vary hue]
data group {function classes};
\end{codeexample}
  \end{key}
  \begin{key}{/tikz/data visualization/visualizer label options/pos=\meta{fraction}}
    This key chooses the first data point belonging to the data set
    whose index is at least \meta{fraction} times the number of all
    data points in the data set.
\begin{codeexample}[width=6.3cm]
\tikz \datavisualization [
  school book axes,
  x axis={label=$x$},
  visualize as smooth line=exp,
  exp=    {label in data={text=$.2$, pos=0.2},
           label in data={text=$.5$, pos=0.5},
           label in data={text=$.95$, pos=0.95},
           style={mark=x}},
  style sheet=vary hue]
data group {function classes};
\end{codeexample}
  \end{key}
  \begin{key}{/tikz/data visualization/visualizer label options/auto}
    This key is executed automatically by default. It works like the
    |pos| option, where the \meta{fraction} is set to $(\meta{data set's
      index}-1/2)/\meta{number of data sets}$. For instance, when
    there are $10$ data sets, the fraction for the first one will be
    $5\%$, the fraction for the second will be $15\%$, for the third
    it will be $25\%$, ending with $95\%$ for the last one.

    The net effect of all this is that when there are several lines,
    labels will be placed at different positions along the lines with
    hopefully only little overlap.
\begin{codeexample}[width=6.3cm]
\tikz \datavisualization [
  scientific clean axes, scientific axes/height=7cm, scientific axes/width=13cm,
  visualize as smooth line/.list={linear, squared, cubed},
  linear ={label in data={text=$2x$}},
  squared={label in data={text=$x^2$}},
  cubed  ={label in data={text=$x^3$}}]
data [set=linear, format=function] {
  var x : interval [0:1.5];
  func y = 2*\value x;
}
data [set=squared, format=function] {
  var x : interval [0:1.5];
  func y = \value x * \value x;
}
data [set=cubed, format=function] {
  var x : interval [0:1.5];
  func y = \value x * \value x * \value x;
};
\end{codeexample}
  \end{key}
  
  The following keys allow you to style labels.

  \begin{key}{/tikz/data visualization/visualizer label
      options/node style=\meta{options}}
    Just passes the options to |/tikz/data visualization/node style|.
  \end{key}
  \begin{key}{/tikz/data visualization/visualizer label
      options/text colored}
    Causes the |node style| to set the text color to
    |visualizer color|. The effect of this is that the label's text
    will have the same color as the data set to which it is attached.
  \end{key}
  
  \begin{stylekey}{/tikz/data visualization/every data set label}
    This style is executed with every label that represents a
    data set. Inside this style, use |node style| to change the
    appearance of nodes. This style has a default definition, usually
    you should just append things to this style.

\begin{codeexample}[width=6.3cm]
\tikz \datavisualization [
  school book axes,
  x axis={label=$x$},
  visualize as smooth line/.list={log, lin, squared, exp},
  every data set label/.append style={text colored},
  log=    {label in data={text'=$\log x$, when=y is -1}},
  lin=    {label in data={text=$x/2$,     when=x is 2}},
  squared={label in data={text=$x^2$,     when=x is 1.1}},
  exp=    {label in data={text=$e^x$,     when=x is -2}},
  style sheet=vary hue]
data group {function classes};
\end{codeexample}
  \end{stylekey}
  
  \begin{stylekey}{/tikz/data visualization/every label in data}
    Like |every data set label|, this key is also executed with
    labels. However, this key is executed after the style sheets have
    been executed, giving you a chance to overrule their styling.
  \end{stylekey}
\end{key}


\begin{key}{/tikz/data visualization/visualizer options/pin in data=\meta{options}}
  To be explained...
\end{key}

\subsubsection{Labeling Data Sets Inside a Legend}


To be written...



\subsection{Reference: Style Sheets for Lines}

The following style sheets can be applied to visualizations that use
the |visualize as line| and related keys. For the examples, the
following style and data set are used:

\begin{codeexample}[code only]
\tikzdatavisualizationset {
  example visualization/.style={
    scientific clean axes,
    y axis={ticks={style={
          /pgf/number format/fixed,
          /pgf/number format/fixed zerofill,
          /pgf/number format/precision=2}}},
    x axis={ticks={tick suffix=${}^\circ$}},
    legend,
    1={label in legend={text=$\frac{1}{6}\sin 11x$}},
    2={label in legend={text=$\frac{1}{7}\sin 12x$}},
    3={label in legend={text=$\frac{1}{8}\sin 13x$}},
    4={label in legend={text=$\frac{1}{9}\sin 14x$}},
    5={label in legend={text=$\frac{1}{10}\sin 15x$}},
    6={label in legend={text=$\frac{1}{11}\sin 16x$}},
    7={label in legend={text=$\frac{1}{12}\sin 17x$}},
    8={label in legend={text=$\frac{1}{13}\sin 18x$}}
  }
}  
\end{codeexample}
\tikzdatavisualizationset {
  example visualization/.style={
    scientific clean axes,
    y axis={ticks={style={
          /pgf/number format/fixed,
          /pgf/number format/fixed zerofill,
          /pgf/number format/precision=2}}},
    x axis={ticks={tick suffix=${}^\circ$}},
    legend,
    1={label in legend={text=$\frac{1}{6}\sin 11x$}},
    2={label in legend={text=$\frac{1}{7}\sin 12x$}},
    3={label in legend={text=$\frac{1}{8}\sin 13x$}},
    4={label in legend={text=$\frac{1}{9}\sin 14x$}},
    5={label in legend={text=$\frac{1}{10}\sin 15x$}},
    6={label in legend={text=$\frac{1}{11}\sin 16x$}},
    7={label in legend={text=$\frac{1}{12}\sin 17x$}},
    8={label in legend={text=$\frac{1}{13}\sin 18x$}}
  }
}  

\begin{codeexample}[code only]
\tikz \datavisualization data group {sin functions} = {
  data [format=function] {
    var set : {1,...,8};
    var x : interval [0:50];
    func y = sin(\value x * (\value{set}+10))/(\value{set}+5);
  }
};  
\end{codeexample}
\tikz \datavisualization data group {sin functions} = {
  data [format=function] {
    var set : {1,...,8};
    var x : interval [0:50];
    func y = sin(\value x * (\value{set}+10))/(\value{set}+5);
  }
};  

\begin{stylesheet}{vary thickness}
  This style varies the thickness of lines. It should be used only
  when there are only two or three lines, and even then it is not
  particularly pleasing visually.
\begin{codeexample}[width=10cm]
\tikz \datavisualization [
  visualize as smooth line/.list=
    {1,2,3,4,5,6,7,8},
  example visualization,
  style sheet=vary thickness]
data group {sin functions};
\end{codeexample}
\end{stylesheet}


\begin{stylesheet}{vary dashing}
  This style varies the dashing of lines. Although it is not
  particularly pleasing visually and although visualizations using
  this style sheet tend to look ``excited'' (but not necessarily
  ``exciting''), this style sheet is often the best choice when the
  visualization is to be printed in black and white.
\begin{codeexample}[width=10cm]
\tikz \datavisualization [
  visualize as smooth line/.list=
    {1,2,3,4,5,6,7,8},
  example visualization,
  style sheet=vary dashing]
data group {sin functions};
\end{codeexample}
  As can be seen, there are only seven distinct dashing patterns. The
  eighth and further lines will use a solid line once more. You will
  then have to specify the dashing ``by hand'' using the |style|
  option together with the visualizer.
\end{stylesheet}

\begin{stylesheet}{vary dashing and thickness}
  This style alternates between varying the thickness and the dashing
  of lines. The 
  difference to just using both the |vary thickness| and
  |vary dashing| is that too thick lines are avoided. Instead, this
  style creates clearly distinguishable line styles for many lines (up
  to 14) with a minimum of visual clutter. This style is the most
  useful for visualizations when many different lines (ten or more)
  should be printed in black and white.
\begin{codeexample}[width=10cm]
\tikz \datavisualization [
  visualize as smooth line/.list=
    {1,2,3,4,5,6,7,8},
  example visualization,
  style sheet=vary thickness
              and dashing]
data group {sin functions};
\end{codeexample}
  For comparison, here is the must-less-than-satisfactory result of
  combining the two independent style sheets:
\begin{codeexample}[width=10cm]
\tikz \datavisualization [
  visualize as smooth line/.list=
    {1,2,3,4,5,6,7,8},
  example visualization,
  style sheet=vary thickness,
  style sheet=vary dashing]
data group {sin functions};
\end{codeexample}
\end{stylesheet}


\subsection{Reference: Style Sheets for Scatter Plots}

The following style sheets can be used both for scatter plots and also
with lines. In the latter case, the marks are added to the lines.

\begin{stylesheet}{cross marks}
  This style uses different crosses to distinguish between the data
  points of different data sets. The crosses were chosen in such a way
  that when two different cross marks lie at the same coordinate,
  their overall shape allows one to still uniquely determine which
  marks are on top of each other.

  This style supports only up to six different data sets.
\begin{codeexample}[width=10cm]
\tikz \datavisualization [
  visualize as scatter/.list=
    {1,2,3,4,5,6,7,8},
  example visualization,
  style sheet=cross marks]
data group {sin functions};
\end{codeexample}
\begin{codeexample}[width=10cm]
\tikz \datavisualization [
  visualize as smooth line/.list=
    {1,2,3,4,5,6,7,8},
  example visualization,
  style sheet=cross marks]
data group {sin functions};
\end{codeexample}
\end{stylesheet}


\subsection{Reference: Color Style Sheets}

Color style sheets are very useful for creating visually pleasing data
visualizations that contain multiple data sets. However, there are two
things to keep in mind:

\begin{itemize}
\item At some point, every data visualization is printed or photo
  copied in black and white by someone. In this case, data sets can
  often no longer be distinguished.
\item A few people are color blind. They will not be able to
  distinguish between red and green lines (and some people are not
  even able to distinguish colors at all).
\end{itemize}

For these reasons, if there is any chance that the data visualization
will be printed in black and white at some point, consider combining
color style sheets with style sheets like |vary dashing| to make data
sets distinguishable in all situations.


\begin{stylesheet}{strong colors}
  This style sheets uses pure primary colors that can very easily be
  distinguished. Although not as visually pleasing as the |vary hue|
  style sheet, the visualizations are easier to read when this style
  sheet is used. Up to six different data sets are supported.
\begin{codeexample}[width=10cm]
\tikz \datavisualization [
  visualize as smooth line/.list=
    {1,2,3,4,5,6,7,8},
  example visualization,
  style sheet=strong colors]
data group {sin functions};
\end{codeexample}
\begin{codeexample}[width=10cm]
\tikz \datavisualization [
  visualize as smooth line/.list=
    {1,2,3,4,5,6,7,8},
  example visualization,
  style sheet=strong colors,
  style sheet=vary dashing]
data group {sin functions};
\end{codeexample}
\end{stylesheet}


Unlike |strong colors|, the following style sheets support, in
principle, an unlimited number of data set. In practice, as always,
more than four or five data sets lead to nearly indistinguishable data
sets.

\begin{stylesheet}{vary hue}
  This style uses a different hue for each data set. 
\begin{codeexample}[width=10cm]
\tikz \datavisualization [
  visualize as smooth line/.list=
    {1,2,3,4,5,6,7,8},
  example visualization,
  style sheet=vary hue]
data group {sin functions};
\end{codeexample}
\end{stylesheet}

\begin{stylesheet}{shades of blue}
  As the name suggests, different shades of blue are used for different
  data sets.
\begin{codeexample}[width=10cm]
\tikz \datavisualization [
  visualize as smooth line/.list=
    {1,2,3,4,5,6,7,8},
  example visualization,
  style sheet=shades of blue]
data group {sin functions};
\end{codeexample}
\end{stylesheet}


\begin{stylesheet}{shades of red}
\begin{codeexample}[width=10cm]
\tikz \datavisualization [
  visualize as smooth line/.list=
    {1,2,3,4,5,6,7,8},
  example visualization,
  style sheet=shades of red]
data group {sin functions};
\end{codeexample}
\end{stylesheet}


\begin{stylesheet}{gray scale}
  For once, this style sheet can also be used when the visualization
  is printed in black and white.
\begin{codeexample}[width=10cm]
\tikz \datavisualization [
  visualize as smooth line/.list=
    {1,2,3,4,5,6,7,8},
  example visualization,
  style sheet=gray scale]
data group {sin functions};
\end{codeexample}
\end{stylesheet}

