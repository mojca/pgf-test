% Copyright 2010-2011 by Renée Ahrens
% Copyright 2010-2011 by Olof Frahm
% Copyright 2010-2011 by Jens Kluttig
% Copyright 2010-2011 by Matthias Schulz
% Copyright 2010-2011 by Stephan Schuster
% Copyright 2011 by Jannis Pohlmann
% Copyright 2011 by Till Tantau
%
% This file may be distributed and/or modified
%
% 1. under the LaTeX Project Public License and/or
% 2. under the GNU Free Documentation License.
%
% See the file doc/generic/pgf/licenses/LICENSE for more details.

\section{The Graph Drawing Engine}

\emph{by Ren\'ee Ahrens, Olof Frahm, Jens Kluttig, Matthias Schulz,
  Stephan Schuster, and Till Tantau}

\label{section-base-graphdrawing}


\ifluatex\relax\else{LuaTeX is required for setting this manual section.}\expandafter\endinput\fi

\subsection{Overview}

This chapter explains in detail how the graph drawing engine works. As
explained in Section~\ref{section-library-graphdrawing}, the graph
drawing engine provides a connection between the syntax of \tikzname\
and \pgfname for specifying graphs and code written in the Lua
programming language for computing layouts. The present section will
discuss this process in detail.

TT: To be written...


\subsection{Graph Drawing Scopes}

\subsection{Graph Parameters}

\subsection{Lua Layer: Overview}

\subsection{Lua Layer: Implementing Graph Drawing Algorithms}

\subsection{Lua Layer: The Module System}

\subsection{Lua Layer: Class Structure}

\subsection{Lua Layer: Class Reference}




TT: Rewrite the following...


The general approach is to 
intercept the immediate placement of the nodes and pass them down to
Lua, which does all the placement stuff. After the selected graph
drawing algorithm has finished, it writes the nodes back to
\tikzname\ to have the graph drawn.

This proceeding consists of a front end layer for \tikzname, an
interface to Lua and of course a set of Lua classes to represent the
graph. An algorithms can be developed independently. Only knowledge
about the Lua interface is required; specific \TeX\ programming skills
not necessary.

\subsubsection{The Front End Layer}
Let's have a look at a simple example to see what the front end looks
like:

\begin{codeexample}[]
\tikz[AhrensFKSS2011 tree,scale=2]
  \graph{root [as=Hello] -> World[fill=blue!20]};
\end{codeexample}

As you may see, the syntax is exactly the same as described in the
chapter about specifying graphs (section~\ref{section-library-graphs}).

You enable this library with the key |graph drawing|, which sets the
algorithm to use and its specific parameters. All other
\tikzname\ keys are accepted as well, like |scale| in the example
above. Each algorithm has its own keys to parametrize it. Please refer to
the appropriate sections for more information.

The keys are given within the |graph drawing| key family for graph options and per node for node specific options. Furthermore you can use any valid \tikzname\ keys as usual. 

There are some things which will not work with the graph drawing
library, like preordering the nodes. Consider for example the
|chain shift| key of the graphs library to place the nodes on a
certain grid: 

\begin{codeexample}[]
\tikzpicture
  \graph[chain shift=(45:1)] {
    a -> b -> c;
    d -> e;
  };
\endtikzpicture
\end{codeexample}

The graph drawing library does not take care of any predefined layout options by now, so the above example will be set differently:

\begin{codeexample}[]
\tikz[AhrensFKSS2011 minimize crossings, scale=2]
  \graph[chain shift=(45:1)] {
    a -> b -> c;
    d -> e;
  };
\end{codeexample}

A graph drawing algorithm will always place the nodes in its own manner. 

% what is happening in the tikz..tex file. Matthias

\subsubsection{The Interface to Lua}
The main entry point for the library to Lua is defined in the
appropriate |code| file of the library. It employs three Lua classes
to create graphs, pass down nodes and to communicate the given
options.

An overview of what happens is illustrated by the following call graph:

\begin{tikzpicture}[
    class name/.style={draw,minimum size=20pt, fill=blue!20},
    object node/.style={draw,minimum size=15pt, fill=yellow!20},
    p/.style={->,>=stealth},
    livespan/.style={thick,double},
    scale=0.9]
  % class names above
  \node (tikz) at (0,4) [class name] {\tikzname\ graph};
  \node (tex) at (5,4) [class name] {\TeX\ Interface};
  \node (interface) at (10,4) [class name] {Lua Interface};
  \node (sys) at (15,4) [class name] {Sys};
  % lines from the class names to the bottom of the picture
  \draw[livespan] (tikz) -- (0,-6.5);
  \draw[livespan] (tex) -- (5,-6.5);
  \draw[livespan] (interface) -- (10,-6.5);
  \draw[livespan] (sys) -- (15,-6.5);
  % first command: \graph{  -- generates new graph in lua interface
  \node (tikz-begin-graph) at (0,3) [object node] {|\graph{|}; %}
  \node (tex-begin-graph) at (5,3) [object node] {|\pgfgdbeginscope|};  
  \node (interface-new-graph) at (10,3) [object node] {|newGraph(|...|)|};
  \draw [p] (tikz-begin-graph.east) -- (tex-begin-graph.west);
  \draw [p] (tex-begin-graph.east) -- (interface-new-graph.west);    
  % second command: a -> b   -- generates two nodes in lua
  % and one edge
  \node (tikz-node) at (0,2) [object node] {|a -> b;|};
  \node (tex-node) at (5,2) [object node] {|\pgf@gd@positionnode@callback|};
  \node (interface-add-node-behind) at (10.1,1.9) [object node] {|addNode(|...|)|};
  \draw[p] (tikz-node.east) -- (tex-node.west);
  
  \node (interface-add-node) at (10,2) [object node] {|addNode(|...|)|};
  \draw[p] (tex-node.east) -- (interface-add-node.west);

  \node (tex-add-edge) at (5,1) [object node] {|\pgfgdaddedge|};
  \node (interface-add-edge) at (10,1) [object node] {|addEdge(|...|)|};
  \draw[p] (tikz-node.east) -- (1.5,2) -- (1.5,1) -- (tex-add-edge.west);
  \draw[p] (tex-add-edge.east) -- (interface-add-edge.west);

  % scope ends -- cloes graph, layouts it and draws it
  \node (tikz-end) at (0,0) [object node] {|};|};
  \node (tex-end) at (5,0) [object node] {|\pgfgdendscope|};
  \node (interface-draw-graph) at (10,0) [object node] {|drawGraph()|};
  \node (interface-finish-graph) at (10,-2) [object node] {|finishGraph()|};

  \node (invoke-algorithm) at (12.5,-1) [object node] {invoke algorithm};
  \draw[p] (tikz-end.east) -- (tex-end.west);
  \draw[p] (tex-end.east) -- (interface-draw-graph.west);
  \draw[p] (interface-draw-graph.east) -- (12.5,0) -- (invoke-algorithm.north);
  \draw[p] (tex-end.east) -- (7.5,0) -- (7.5,-2) -- (interface-finish-graph.west);

  % begin shipout
  \node (sys-begin-shipout) at (15,-2) [object node] {|beginShipout()|};
  \draw[p] (interface-finish-graph.east) -- (sys-begin-shipout.west);
  \node (tex-begin-shipout) at (5,-3) [object node] {|\pgfgdbeginshipout|};
  \draw[p] (sys-begin-shipout.187) -- (12,-2.2) -- (12,-3) -- (tex-begin-shipout.east);

  % put tex box
  \node (sys-puttexbox-behind) at (15.1,-4.1) [object node] {|putTeXBox(|...|)|};
  \node (sys-puttexbox) at (15,-4) [object node] {|putTeXBox(|...|)|};
  \node (tex-puttexbox) at (5,-4) [object node] {|\pgfgdinternalshipoutnode|};

  \draw[p] (12.5,-2) -- (12.5,-4) -- (sys-puttexbox.west);
  %(interface-finish-graph.east) -- (12.5,-2) -- (12.5,-4) -- (sys-puttexbox.west);
  \draw[p] (sys-puttexbox.187) -- (12,-4.2) -- (12,-4) -- (tex-puttexbox.east);

  % put edge
  \node (sys-put-edge-behind) at (15.1,-5.1) [object node] {|putEdge(|...|)|};
  \node (sys-put-edge) at (15,-5) [object node] {|putEdge(|...|)|};
  \draw[p] (12.5,-4) -- (12.5,-5) -- (sys-put-edge.west);
  %(interface-finish-graph.east) -- (12.5,-2) -- (12.5,-5) -- (sys-put-edge.west);
  % end shipout
  \node (sys-end-shipout) at (15,-6) [object node] {|endShipout()|};
  \draw[p] (12.5,-5) -- (12.5,-6) -- (sys-end-shipout.west);
  %(interface-finish-graph.east) -- (12.5,-2) -- (12.5,-6) -- (sys-end-shipout.west);
  \node (tex-end-shipout) at (5,-6) [object node] {|\pgfgdendshipout|};
  \draw[p] (sys-end-shipout.187) -- (12,-6.175) -- (12,-6) -- (tex-end-shipout.east);
\end{tikzpicture}


\paragraph{The \TeX\ side.}
\label{section-library-graphdrawing-the-tex-side}

In order to layout a graph, we need to keep \tikzname\ from placing the nodes immediately. This is done using the macro
|\pgfpositionnodelater| as described in chapter~\ref{section-shapes},
subchapter~\ref{section-shapes-deferred-node-positioning}. 

In short terms this works as follows: This macro takes another \meta{macro} as
first argument. If this is |\relax|, the behaviour is to immediately
place the node into the current picture. Any other \meta{macro} that is passed
will be executed. It works like a
callback function -- the node will be put into a box register, the
name of the node and the bounding box coordinates are stored in
separate macros and afterwards \meta{macro} will be called.

As we have to make sure, that the unplaced node will not be referenced
by \tikzname\ keys like |right of|, it is temporarily renamed to
|not yet positionedPGFGDINTERNAL|\meta{nodename}.

To finally
insert the node into the picture, we need to set the mentioned macros
and put the node into the box register. Then we can call
|\pgfpositionnodenow| with the target coordinates of the node.

The code file of the graph drawing library sets the callback function
at the beginning of a graph drawing scope, e.g.\ when a |\graph|
starts. This can also be triggered using |\pgfgdbeginscope| and
|\pgfgdendscope|, which can be used to create a sub scope in an
existing graph drawing scope. Opening a scope yields in creating a new
graph on the Lua graph stack. All subsequent operations (like adding
nodes or edges) apply to the top of the stack. 

%by now this leads to an infinite loop . when its fixed, the example
%can be uncommented :)
% \begin{codeexample}[]
% \tikzpicture[graph drawing={few intersections}, scale=2]
% \graph{
%   a->b;
% %  \graph{c->d;}; TODO: triggers an infinite loop.
%   };
% \endtikzpicture
% \end{codeexample}

The callback function gets all option keys in
|/tikz/graphs/graph drawing/|, copies the box register and passes all information down to the Lua interface class.

When the library is loaded, it initialises the Lua subsystem. This takes place by checking if \LuaTeX\ is present and then invoking the Lua loader class. 

The library code file consists mainly the following macros:

\begin{command}{\pgfgdbeginscope}
  The begin scope macro opens a new graph drawing scope. This creates a new graph object on the top of the Lua graph stack. All subsequent operations will work on this graph until |\pgfgdendscope| will be called.

It is not necessary to call it manually, because in a graph drawing environment it is executed by default at the beginning of a |\graph| statement.
\end{command}


\makeatletter
\begin{command}{\pgf@gd@positionnode@callback}
  This macro saves the keys from |/tikz/graphs/graph drawing/| into a temporary macro, sets the box register |\pgf@gd@box| to the |\pgfpositionnodelaterbox| and passes these informations down to Lua. Additionally the node name and the bounding box is passed down, too. This macro is only used internally.
\end{command}
\makeatother

\begin{command}{\pgfgdaddedge\marg{from}\marg{to}\marg{direction}}
  Adds an edge to the Lua graph object. It requires the name of the target node \meta{from}, the destination node \meta{to} with a distinct \meta{direction} like |->|.

  It is called when a |->|, |--|, |<-| or |-!-| is encountered in a graph.
\end{command}

\begin{command}{\pgfgdendscope}
  At the end of a graph drawing scope the selected algorithm runs and layouts the graph. After finishing this task the macro pops the graph from the stack.
\end{command}

\begin{command}{\pgfgdbeginshipout}
  When the layout is completed and the scope ended, this macro places a |\scope| into the output stream. The layouted graph will be placed inside an extra scope.
\end{command}

\begin{command}{\pgfgdinternalshipoutnode\marg{name}\marg{x min}\marg{x max}\marg{y min}\marg{y max}\marg{x pos}\marg{y pos}\marg{box}}
  When the algorithm finished the layout and the scope ended, the nodes have to be passed back to \tikzname. This macro takes the name of the node, the bounding box, the newly computed position and a box register number. It restores the macros set by |\pgfpositionnodelater| as mentioned above, fills the box register |\pgfpositionnodelaterbox| and then calls |\pgfpositionnodenow| with the coordinates of the node. This macro inserts the node into the current picture.
\end{command}

\begin{command}{\pgfgdendshipout}
  Issues a |\endscope| macro to close the scope opened by |\pgfgdbeginshipout|.
\end{command}

\paragraph{Lua interface class.}

The class |Interface| is the main entry point in Lua. Every communication from \TeX\ to Lua is done here.
It provides methods to create graphs, add nodes and edges to graphs and finally to invoke the selected algorithm. The |Interface| class manages the stack of graphs.

When the |newGraph()| function is called, it generates a new graph object and pushes it on the graph stack. The methods |addNode()| and |addEdge()| are called for each node and each edge, creating the actual Lua objects and adding them to the current graph.

After adding nodes and edges, when the scope ends, the interface invokes the actual algorithm to layout the graph. This is done in the |drawGraph()| function. The next step is to put the nodes back in the \TeX\ output stream. This is invoked by the |finishGraph()| method.

For a reference about the functions and their usage, please refer to section~\ref{section-library-graphdrawing-lua-documentation-interface}.

\paragraph{Lua system class.}

Communication with \TeX\ on a basic layer is done in the |Sys| class. The |beginShipout()| function opens a new scope in \tikzname\ to put all graph drawing nodes into. This prevents other graph objects outside the graph drawing scope from referencing these nodes. The |endShipout()| method closes the scope.

Nodes and edges are put in the output stream by the methods |putTeXBox()| and |putEdge()|. The first calls the |\pgfgdinternalshipoutnode| macro, which is explained in section~\ref{section-library-graphdrawing-the-tex-side}. The latter method writes the appropriate |\draw| directly to the output stream. 

For a reference about the functions and their usage, please refer to section~\ref{section-library-graphdrawing-lua-documentation-sys}.

\subsubsection{Lua Graph Representation}
Most classes in the framework (including the module objects) implement
the |__tostring| method, meaning that you can get a somewhat useful
string representation of the object via the standard |tostring|
function.

The main class which contains references to all other objects is
|Graph|.  New graphs are usually created automatically, so common ways
to get new graph objects are the |copy| method, which creates a
shallow copy (without coying nodes or edges), and the
|subGraphParent| method, which creates a deep copy of the graph, edge
and node objects starting at a designated parent node. If you need
more control by supplying your own set of already visited nodes, use
the underlying function |subGraph|.

A graph allows you to add and remove nodes and edges via |addNode|,
|addEdge|, |removeNode| and |removeEdge| respectively.  There are also
variants which remove all incident edges on a node removal and
conversely, |deleteNode| and |deleteEdge|.

Only nodes can be looked up by name with |findNode|, a
method implemented using the more generic |findNodeIf|, which supports
an arbitrary test predicate.

Lastly the |walkDepth| and |walkBreadth| methods may be used to get
iterators over all nodes and edges in a depth-first or breadth-first
order (other traversal orders may require a rewrite or extension of the
|walkAux| method).

Positions are represented using the dedicated class |Position|, the member
variables |x| and |y| contain the coordinates.  Positions can also be
relative to other positions, which can be tested using |isAbsPosition|.
The conversion to absolute coordinates is done with |getAbsCoordinates|.
The equality test method implements comparing two positions by using their
absolute positions.

For a detailed description of the mentioned classes and methods refer
to section~\ref{section-library-graphdrawing-lua-documentation-graphrep}.

\paragraph{Common graph operations.}
The following tasks are typical for manipulating the graph.
Those snippets will get you started even if you do not have any Lua
experience.

\begin{itemize}
\item Iterate over all nodes.
\begin{codeexample}[code only]
for node in table.value_iter(graph.nodes) do
   ...
end
\end{codeexample}
\item Get or set width/height of a node, e.g.\ for measuring.
\begin{codeexample}[code only]
local width, height = node.width, node.height
\end{codeexample}
\item Get or set x-y-coordinates of a node.
\begin{codeexample}[code only]
node.pos:set{x = node.pos:x() + 1}
node.pos:set{y = node.pos:y() + 1}
\end{codeexample}
\item Relate the position of node to the position of another.
\begin{codeexample}[code only]
newNode.pos:set{x = 1, y = 1}
--sets position of newNode 1 pt in y- and x-direction relative to node
newNode.pos:setOrigin(node.pos)
\end{codeexample}
\item Get absolute x-y-coordinate of node, with or without relative coordinates.
\begin{codeexample}[code only]
absX, absY = newNode:get(1), newNode:get(2)
\end{codeexample}
\item Iterate over all edges and all nodes of the current edge.
\begin{codeexample}[code only]
for edge in table.value_iter(graph.edges) do
   for node in table.value_iter(edge.nodes) do
      ...
   end
end
\end{codeexample}
\item Get the nodes connected by an edge.
\begin{codeexample}[code only]
local nodeLeft = edge:.nodes[1]
local nodeRight = edge:.nodes[2]
\end{codeexample}
\end{itemize}

A full example for a user-defined algorithm is shown in
section~\ref{section-library-graphdrawing-ownAlgorithm}.

\subsection{Registering graph drawing keys}
\label{section-base-graphdrawing-registerKeys}

Graphs and nodes in Lua have specific options, like the name of the
algorithm to use. These keys are registered on the \tikzname\ layer.


\begin{stylekey}{/tikz/graphs/graph drawing/register key}
  The argument of this style is registered as a new key for a
  graph. The name of the key and it's value will be passed down to the
  Lua graph object and should be used for algorithm-wide options. 

  An example is the |algorithm| key, which is required for each graph
  drawing context. 

  The key/value pair will be stored in |/tikz/graphs/graph drawing/@options/|.
\end{stylekey}

\begin{stylekey}{/tikz/graphs/graph drawing/register math key}
  Registering a new math key is like registering a new key, except
  that it's a parseable value. When a value is assigned to the key,
  pgf will parse the value. 

  Math keys can be used if a option holds a dimension value, like the
  |scale| option of \tikzname\. The value will be expanded and
  computed to the dimension |pt|. 

  A sample math key is introduced in the simpleexample algorithm
  (see \ref{section-library-graphdrawing-ownAlgorithm}) below.
\end{stylekey}

\begin{stylekey}{/tikz/graphs/graph drawing/register node key}
  A node key is not stored graph-wide; it is designated for a single
  node. The name/value pair is accessible from the node object in Lua;
  in \tikzname\ it will be stored in the key family |/tikz/graphs/graph drawing/@node@options/|.
\end{stylekey}

\begin{stylekey}{/tikz/graphs/graph drawing/register node math key}
  Like node key, but with parsing of it's value (see |register math key|).
\end{stylekey}


\subsection{Module System}
The package defines its own Lua module system, which is characterised by a
more dynamic view on importing symbols.  Basically, each module has a
set of imported modules and the lookup for names first happens in the local
scope, then in the current module and subsequently in all imported
modules.  Since no name is statically imported, newly assigned
variables in other modules are still visible when those were
previously imported.

Modules are accessed with the |pgf.module| call, which enables the
module for the current context, i.e. the current file. If a module
does not exist, it will be created.  Importing modules is done via
|pgf.import|.  Both functions accept a string argument for the
module name.

Modules are named hierarchically and defined modules are exported into
each parent module.  If the module name contains no period, it is
exported into the global environment.  Nevertheless, importing is only
done on request; importing a module twice doesn't do anything.
It is recommended to dedicate a single module definition file
to create it and import other modules.  For example, the package
contains a single file containing only the following two lines for
creating the |pgf.graphdrawing| module in the first place.

\begin{codeexample}[code only]
pgf.module("pgf.graphdrawing")
pgf.import("pgf")
\end{codeexample}

Symbol lookup first happens in the local namespace, then in the
current module and subsequently in all imported modules and the global
namespace.  Assignment of new variables happens in the current module
(or for variables declared |local| in the local namespace).  If you
need to assign values to the global environment use the special table
|_G| as you'd normally do in Lua.

The |pgf| module is created during the definition of the module system
and mostly contains functions for loading and debugging.  Developers
probably shouldn't touch the |pgf| namespace and instead add new
functionality to modules below this level or in new top-level
modules.

\subsubsection{Module Examples}
Let's see what consequences this module system has in praxis.  The
following code fragment starts from a clean state after rendering it
with \LuaTeX\ and then enters the |pgf.graphdrawing| module,
overwriting the global |pgf| binding and then again reverting this
change.

\begin{codeexample}[code only]
  \input tikz

  \usetikzlibrary{graphdrawing}

  \directlua{
    pgf.graphdrawing.Sys:log("1: pgf is " .. tostring(pgf))
    pgf.graphdrawing.Sys:log("1: graphdrawing is " .. tostring(graphdrawing))
    
    pgf.module("pgf.graphdrawing")

    Sys:log("2: pgf is " .. tostring(pgf))
    Sys:log("2: graphdrawing is " .. tostring(graphdrawing))

    pgf = 1

    Sys:log("3: pgf is " .. tostring(pgf))
    Sys:log("3: graphdrawing is " .. tostring(graphdrawing))

    pgf = nil

    Sys:log("4: pgf is " .. tostring(pgf))

    pgf.graphdrawing = nil

    Sys:log("5: pgf is " .. tostring(pgf))

    _G.pgf = nil

    Sys:log("6: pgf is " .. tostring(pgf))
  }
\end{codeexample}

The result will be as follows:

\begin{codeexample}[code only]
1: pgf is <module 'pgf', table: 0x7979600>
1: graphdrawing is nil

2: pgf is <module 'pgf', table: 0x7979600>
2: graphdrawing is <module 'pgf.graphdrawing', table: 0x7973c60>

3: pgf is 1
3: graphdrawing is <module 'pgf.graphdrawing', table: 0x7973c60>

4: pgf is <module 'pgf', table: 0x7979600>
5: pgf is <module 'pgf', table: 0x7979600>
6: pgf is nil
\end{codeexample}

As you can see the |pgf| table is available in the global environment
and also after using the |pgf.graphdrawing| module, although we don't
refer to it with its full name.  Assigning a new value to |pgf|
doesn't overwrite the global object, but introduces a local binding
shadowing the global one. Assigning |nil| then removes the local
binding, therefore in the next line the global variable is available
again.

Note that in all but the first case the binding to |graphdrawing|
stays the same.  Also, using these assignments, you can't accidentally
remove your access to the |pgf| or any imported modules as the last
two assignments show (the |Sys:log| method still works).


\subsection{Lua Documentation}
This sections provides a full documentation of all relevant Lua classes
used.

Every class and function in the package (except for module handling in
|pgf|) is available in the |pgf.graphdrawing| module.

\label{section-library-graphdrawing-lua-documentation}
\subsubsection{Graph Representation}
\label{section-library-graphdrawing-lua-documentation-graphrep}
% This file has been generated from the lua sources using LuaDoc.
% To regenerate it call "make genluadoc" in
% doc/generic/pgf/version-for-luatex/en.

\begin{filedescription}{pgflibrarygraphdrawing-graph.lua}


\begin{luacommand}{{Graph:\textunderscore{}\textunderscore{}tostring}()}
Returns a string representation of this graph including all nodes and edges. 


Return value:
\begin{parameterdescription} 
  \item[] Graph as string. 
\end{parameterdescription}


\end{luacommand}
\begin{luacommand}{{Graph:addEdge}(\meta{edge})}
Adds an edge to the graph. 

Parameters:
\begin{parameterdescription}
	\item[\meta{edge}] The edge to be added. 
\end{parameterdescription}



\end{luacommand}
\begin{luacommand}{{Graph:addNode}(\meta{node})}
Adds a node to the graph. 

Parameters:
\begin{parameterdescription}
	\item[\meta{node}] The node to be added. 
\end{parameterdescription}



\end{luacommand}
\begin{luacommand}{{Graph:copy}()}
Creates a shallow copy of a graph.  The nodes and edges of the original graph are not preserved in the copy. 


Return value:
\begin{parameterdescription} 
  \item[] A shallow copy of the graph. 
\end{parameterdescription}


\end{luacommand}
\begin{luacommand}{{Graph:createEdge}(\meta{first\_node},\meta{second\_node},\meta{direction},\meta{edge\_nodes},\meta{options},\meta{tikz\_options})}
Creates and adds a new edge to the graph. 

Parameters:
\begin{parameterdescription}
	\item[\meta{first\_node}] The first node of the new edge.\item[\meta{second\_node}] The second node of the new edge.\item[\meta{direction}] The direction of the new edge. Possible values are \begin{itemize} \item |Edge.UNDIRECTED|, \item |Edge.LEFT|, \item |Edge.RIGHT|, \item |Edge.BOTH| and \item |Edge.NONE| (for invisible edges). \end{itemize}\item[\meta{edge\_nodes}] A string of \tikzname\ edge nodes that needs to be passed back to the \TeX layer unmodified.\item[\meta{options}] The options of the new edge.\item[\meta{tikz\_options}] A table of \tikzname\ options to be used by graph drawing algorithms to treat the edge in special ways. 
\end{parameterdescription}


Return value:
\begin{parameterdescription} 
  \item[] The newly created edge. 
\end{parameterdescription}


\end{luacommand}
\begin{luacommand}{{Graph:deleteEdge}(\meta{edge})}
Like removeEdge, but also removes the edge from its adjacent nodes. 

Parameters:
\begin{parameterdescription}
	\item[\meta{edge}] The edge to be deleted. 
\end{parameterdescription}


Return value:
\begin{parameterdescription} 
  \item[] The removed edge or |nil| if it was not found in the graph. 
\end{parameterdescription}


\end{luacommand}
\begin{luacommand}{{Graph:deleteNode}(\meta{node})}
Like removeNode, but also deletes all adjacent edges of the removed node.  This function also removes the deleted adjacent edges from all neighbours of the removed node. 

Parameters:
\begin{parameterdescription}
	\item[\meta{node}] The node to be deleted together with its adjacent edges. 
\end{parameterdescription}


Return value:
\begin{parameterdescription} 
  \item[] The removed node or |nil| if the node was not found in the graph. 
\end{parameterdescription}


\end{luacommand}
\begin{luacommand}{{Graph:findNode}(\meta{name})}
If possible, looks up the node with the given name in the graph. 

Parameters:
\begin{parameterdescription}
	\item[\meta{name}] Name of the node to look up. 
\end{parameterdescription}


Return value:
\begin{parameterdescription} 
  \item[] The node with the given name or |nil| if it was not found in the graph. 
\end{parameterdescription}


\end{luacommand}
\begin{luacommand}{{Graph:findNodeIf}(\meta{test})}
Looks up the first node for which the function \meta{test} returns |true|. 

Parameters:
\begin{parameterdescription}
	\item[\meta{test}] A function that takes one parameter (a |Node|) and returns |true| or |false|. 
\end{parameterdescription}


Return value:
\begin{parameterdescription} 
  \item[] The first node for which \meta{test} returns |true|. 
\end{parameterdescription}


\end{luacommand}
\begin{luacommand}{{Graph:getOption}(\meta{name})}
Returns the value of the graph option \meta{name}. 

Parameters:
\begin{parameterdescription}
	\item[\meta{name}] Name of the option. 
\end{parameterdescription}


Return value:
\begin{parameterdescription} 
  \item[] The value of the graph option \meta{name} or |nil|. 
\end{parameterdescription}


\end{luacommand}
\begin{luacommand}{{Graph:mergeOptions}(\meta{options})}
Merges the given options into the options of the graph. 

Parameters:
\begin{parameterdescription}
	\item[\meta{options}] The options to be merged. 
\end{parameterdescription}



See also:
\begin{itemize}
	\item[] |mergeTable |
\end{itemize}

\end{luacommand}
\begin{luacommand}{{Graph:new}(\meta{values})}
Creates a new graph. 

Parameters:
\begin{parameterdescription}
	\item[\meta{values}] Values to override default graph settings. The following parameters can be set:\par |nodes|: The nodes of the graph.\par |edges|: The edges of the graph.\par |pos|: Initial position of the graph.\par |options|: A table of node options passed over from \tikzname. 
\end{parameterdescription}


Return value:
\begin{parameterdescription} 
  \item[] A newly-allocated graph. 
\end{parameterdescription}


\end{luacommand}
\begin{luacommand}{{Graph:removeEdge}(\meta{edge})}
If possible, removes an edge from the graph and returns it. 

Parameters:
\begin{parameterdescription}
	\item[\meta{edge}] The edge to be removed. 
\end{parameterdescription}


Return value:
\begin{parameterdescription} 
  \item[] The removed edge or |nil| if it was not found in the graph. 
\end{parameterdescription}


\end{luacommand}
\begin{luacommand}{{Graph:removeNode}(\meta{node})}
If possible, removes a node from the graph and returns it. 

Parameters:
\begin{parameterdescription}
	\item[\meta{node}] The node to remove. 
\end{parameterdescription}


Return value:
\begin{parameterdescription} 
  \item[] The removed node or |nil| if it was not found in the graph. 
\end{parameterdescription}


\end{luacommand}
\begin{luacommand}{{Graph:setOption}(\meta{name},\meta{value})}
Sets the graph option \meta{name} to \meta{value}. 

Parameters:
\begin{parameterdescription}
	\item[\meta{name}] Name of the option to be changed.\item[\meta{value}] New value for the graph option \meta{name}. 
\end{parameterdescription}



\end{luacommand}
\begin{luacommand}{{Graph:subGraph}(\meta{root},\meta{graph},\meta{visited})}
Returns a subgraph.  The resulting subgraph begins at the node root, excludes all nodes and edges that are marked as visited. 

Parameters:
\begin{parameterdescription}
	\item[\meta{root}] Root node where the operation starts.\item[\meta{graph}] Result graph object or |nil| if the original graph should be used as the parent graph.\item[\meta{visited}] Set of already visited nodes/edges or |nil|. This set will be modified so make sure not to use a table that you want to remain untouched. 
\end{parameterdescription}



\end{luacommand}
\begin{luacommand}{{Graph:subGraphParent}(\meta{root},\meta{parent},\meta{graph})}
Creates a new subgraph with \meta{parent} marked as visited.  This function can be useful if the graph is a tree structure (and \meta{parent} is the parent node of \meta{root}). 

Parameters:
\begin{parameterdescription}
	\item[\meta{root}] Root node where the operation starts.\item[\meta{parent}] Parent of the recursion step before.\item[\meta{graph}] Result graph object or |nil| if the original graph should be used as the parent graph. 
\end{parameterdescription}



See also:
\begin{itemize}
	\item[] |subGraph |
\end{itemize}

\end{luacommand}
\begin{luacommand}{{Graph:walkAux}(\meta{root},\meta{visited},\meta{remove\_index})}
Auxiliary function to walk a graph. Does nothing if no nodes exist. 

Parameters:
\begin{parameterdescription}
	\item[\meta{root}] The first node to be visited. If nil, chooses some node.\item[\meta{visited}] Set of already visited nodes and edges. |visited[v] == true| indicates that the node or edge |v| has already been visited.\item[\meta{remove\_index}] A numeric value or |nil| that defines the order in which nodes and edges are visited while traversing the graph. |nil| results in queue behavior, |1| in stack behavior. 
\end{parameterdescription}



See also:
\begin{itemize}
	\item[] |walkDepth|\item[] |walkBreadth |
\end{itemize}

\end{luacommand}
\begin{luacommand}{{Graph:walkBreadth}(\meta{root},\meta{visited})}
Returns an iterator to walk the graph in a breadth-first traversal.  The iterator returns all edges and nodes one at a time. In case only the nodes are of interest, a filter function like |iter.filter| can be used to ignore edges. 

Parameters:
\begin{parameterdescription}
	\item[\meta{root}] The first node to be visited. If nil, chooses some node.\item[\meta{visited}] Set of already visited nodes and edges. |visited[v] == true| indicates that the node or edge |v| has already been visited. 
\end{parameterdescription}



See also:
\begin{itemize}
	\item[] |iter.filter |
\end{itemize}

\end{luacommand}
\begin{luacommand}{{Graph:walkDepth}(\meta{root},\meta{visited})}
Returns an iterator to walk the graph in a depth-first traversal.  The iterator returns all edges and nodes one at a time. In case only the nodes are of interest, a filter function like |iter.filter| can be used to ignore edges. 

Parameters:
\begin{parameterdescription}
	\item[\meta{root}] The first node to be visited. If nil, chooses some node.\item[\meta{visited}] Set of already visited nodes and edges. |visited[v] == true| indicates that the node or edge |v| has already been visited. 
\end{parameterdescription}



See also:
\begin{itemize}
	\item[] |iter.filter |
\end{itemize}

\end{luacommand}

\end{filedescription}
% This file has been generated from the lua sources using LuaDoc.
% To regenerate it call "make genluadoc" in
% doc/generic/pgf/version-for-luatex/en.

\begin{filedescription}{pgflibrarygraphdrawing-node.lua}


\begin{luacommand}{{Node:\textunderscore{}\textunderscore{}eq}(\meta{object},\meta{other})}
Compares two nodes by their name. 

Parameters:
\begin{parameterdescription}
	\item[\meta{other}] Another node to compare with. 
\end{parameterdescription}


Return value:
\begin{parameterdescription} 
  \item[] |true| if both nodes have the same name. |false| otherwise. 
\end{parameterdescription}


\end{luacommand}
\begin{luacommand}{{Node:\textunderscore{}\textunderscore{}tostring}()}
Returns a formated string representation of the node. 


Return value:
\begin{parameterdescription} 
  \item[] String represenation of the node. 
\end{parameterdescription}


\end{luacommand}
\begin{luacommand}{{Node:addEdge}(\meta{edge})}
Adds new edge to the node. 

Parameters:
\begin{parameterdescription}
	\item[\meta{edge}] The edge to be added. 
\end{parameterdescription}



\end{luacommand}
\begin{luacommand}{{Node:copy}()}
Creates a shallow copy of the node.  Most notably, the edges adjacent are not preserved in the copy. 


Return value:
\begin{parameterdescription} 
  \item[] Copy of the node. 
\end{parameterdescription}


\end{luacommand}
\begin{luacommand}{{Node:getDegree}()}
Counts the adjacent edges of the node. 


Return value:
\begin{parameterdescription} 
  \item[] The number of adjacent edges of the node. 
\end{parameterdescription}


\end{luacommand}
\begin{luacommand}{{Node:getEdges}()}
Returns all edges of the node.  Instead of calling |node:getEdges()| the edges can alternatively be accessed directly with |node.edges|. 


Return value:
\begin{parameterdescription} 
  \item[] All edges of the node. 
\end{parameterdescription}


\end{luacommand}
\begin{luacommand}{{Node:getInDegree}(\meta{ignore\_reversed})}
Returns the number of incoming edges of the node. 

Parameters:
\begin{parameterdescription}
	\item[\meta{ignore\_reversed}] Optional parameter to consider reversed edges not reversed for this method call. Defaults to |false|. 
\end{parameterdescription}


Return value:
\begin{parameterdescription} 
  \item[] The number of incoming edges of the node. 
\end{parameterdescription}


See also:
\begin{itemize}
	\item[] |Node:getIncomingEdges(reversed) |
\end{itemize}

\end{luacommand}
\begin{luacommand}{{Node:getIncomingEdges}(\meta{ignore\_reversed})}
Returns the incoming edges of the node. Undefined result for hyperedges. 

Parameters:
\begin{parameterdescription}
	\item[\meta{ignore\_reversed}] Optional parameter to consider reversed edges not reversed for this method call. Defaults to |false|. 
\end{parameterdescription}


Return value:
\begin{parameterdescription} 
  \item[] Incoming edges of the node. This includes undirected edges and directed edges pointing to the node. 
\end{parameterdescription}


\end{luacommand}
\begin{luacommand}{{Node:getOption}(\meta{name})}
Returns the value of the node option \meta{name}. 

Parameters:
\begin{parameterdescription}
	\item[\meta{name}] Name of the node option. 
\end{parameterdescription}


Return value:
\begin{parameterdescription} 
  \item[] The value of the node option \meta{name} or |nil|. 
\end{parameterdescription}


\end{luacommand}
\begin{luacommand}{{Node:getOutDegree}(\meta{ignore\_reversed})}
Returns the number of edges starting at the node. 

Parameters:
\begin{parameterdescription}
	\item[\meta{ignore\_reversed}] Optional parameter to consider reversed edges not reversed for this method call. Defaults to |false|. 
\end{parameterdescription}


Return value:
\begin{parameterdescription} 
  \item[] The number of outgoing edges of the node. 
\end{parameterdescription}


See also:
\begin{itemize}
	\item[] |Node:getOutgoingEdges() |
\end{itemize}

\end{luacommand}
\begin{luacommand}{{Node:getOutgoingEdges}(\meta{ignore\_reversed})}
Returns the outgoing edges of the node. Undefined result for hyperedges. 

Parameters:
\begin{parameterdescription}
	\item[\meta{ignore\_reversed}] Optional parameter to consider reversed edges not reversed for this method call. Defaults to |false|. 
\end{parameterdescription}


Return value:
\begin{parameterdescription} 
  \item[] Outgoing edges of the node. This includes undirected edges and directed edges leaving the node. 
\end{parameterdescription}


\end{luacommand}
\begin{luacommand}{{Node:getTexHeight}()}
Computes the heigth of the node. 


Return value:
\begin{parameterdescription} 
  \item[] Height of the node. 
\end{parameterdescription}


\end{luacommand}
\begin{luacommand}{{Node:getTexWidth}()}
Computes the width of the node. 


Return value:
\begin{parameterdescription} 
  \item[] Width of the node. 
\end{parameterdescription}


\end{luacommand}
\begin{luacommand}{{Node:new}(\meta{values})}
Creates a new node. 

Parameters:
\begin{parameterdescription}
	\item[\meta{values}] Values to override default node settings. The following parameters can be set:\par |name|: The name of the node. It is obligatory to define this.\par |tex|: Information about the corresponding \TeX\ node.\par |edges|: Edges adjacent to the node.\par |pos|: Initial position of the node.\par |options|: A table of node options passed over from \tikzname. 
\end{parameterdescription}


Return value:
\begin{parameterdescription} 
  \item[] A newly allocated node. 
\end{parameterdescription}


\end{luacommand}
\begin{luacommand}{{Node:removeEdge}(\meta{edge})}
Removes an edge from the node. 

Parameters:
\begin{parameterdescription}
	\item[\meta{edge}] The edge to remove. 
\end{parameterdescription}



\end{luacommand}
\begin{luacommand}{{Node:setOption}(\meta{name},\meta{value})}
Sets the node option \meta{name} to \meta{value}. 

Parameters:
\begin{parameterdescription}
	\item[\meta{name}] Name of the node option to be changed.\item[\meta{value}] New value for the node option \meta{name}. 
\end{parameterdescription}



\end{luacommand}

\end{filedescription}
% This file has been generated from the lua sources using LuaDoc.
% To regenerate it call "make genluadoc" in
% doc/generic/pgf/version-for-luatex/en.

\begin{filedescription}{pgflibrarygraphdrawing-edge.lua}


\begin{luacommand}{{Edge:\textunderscore{}\textunderscore{}tostring}()}
Returns a readable string representation of the edge.


Return value:
\begin{itemize} \item[] String representation of the edge. \end{itemize}


\end{luacommand}\begin{luacommand}{{Edge:addNode}(\meta{node})}
Adds node to the edge.

Parameters:
\begin{parameterdescription}
	\item[\meta{node}] The node to be added to the edge.
\end{parameterdescription}



\end{luacommand}\begin{luacommand}{{Edge:containsNode}(\meta{node})}
Tests if edge contains a node.


Return value:
\begin{itemize} \item[] True if the edge contains a node. \end{itemize}


\end{luacommand}\begin{luacommand}{{Edge:copy}()}
Copies an edge (preventing accidental use).


Return value:
\begin{itemize} \item[] Shallow copy of the edge. \end{itemize}


\end{luacommand}\begin{luacommand}{{Edge:getDegree}()}
Returns number of nodes on the edge.


Return value:
\begin{itemize} \item[] Number of nodes of the edge. \end{itemize}


\end{luacommand}\begin{luacommand}{{Edge:getNeighbour}(\meta{node})}
Gets first neighbour of the node (disregarding hyperedges).

Parameters:
\begin{parameterdescription}
	\item[\meta{node}] The node which first neighbour should be returned.
\end{parameterdescription}


Return value:
\begin{itemize} \item[] The first neighbour of the node. \end{itemize}


\end{luacommand}\begin{luacommand}{{Edge:getNeighbours}(\meta{node})}
Returns all neighbours of a node.

Parameters:
\begin{parameterdescription}
	\item[\meta{node}] The node which neighbours should be returned.
\end{parameterdescription}


Return value:
\begin{itemize} \item[] Array of neighbour nodes. \end{itemize}


\end{luacommand}\begin{luacommand}{{Edge:getNodes}()}
Returns the nodes of an edge.


Return value:
\begin{itemize} \item[] Array of nodes of the edge. \end{itemize}


\end{luacommand}\begin{luacommand}{{Edge:getPath}()}
Returns the path of an edge.


Return value:
\begin{itemize} \item[] The path the edge belongs to. \end{itemize}


\end{luacommand}\begin{luacommand}{{Edge:isHyperedge}()}
Returns a boolean whether the edge is a hyperedge.


Return value:
\begin{itemize} \item[] True if the edge is a hyperedge. \end{itemize}


\end{luacommand}\begin{luacommand}{{Edge:new}(\meta{values})}
Creates an edge between nodes of a graph.

Parameters:
\begin{parameterdescription}
	\item[\meta{values}] Values (e.g. direction) to be merged with the default-metatable of an edge.
\end{parameterdescription}


Return value:
\begin{itemize} \item[] The new edge. \end{itemize}


\end{luacommand}\begin{luacommand}{{Edge:setPath}(\meta{path})}
Sets the path of an edge.

Parameters:
\begin{parameterdescription}
	\item[\meta{path}] The path the edge belongs to.
\end{parameterdescription}



\end{luacommand}
\end{filedescription}
% This file has been generated from the lua sources using LuaDoc.
% To regenerate it call "make genluadoc" in
% doc/generic/pgf/version-for-luatex/en.

\begin{filedescription}{pgflibrarygraphdrawing-position.lua}


\begin{luacommand}{{Position.calcCoordsTo}(\meta{posFrom},\meta{posTo})}
Returns a vector between two positions.

Parameters:
\begin{parameterdescription}
	\item[\meta{posFrom}] Position A.\item[\meta{posTo}] Position B.
\end{parameterdescription}


Return value:
\begin{itemize} \item[] x- and y-coordinates of the vector between posFrom and posTo. \end{itemize}


\end{luacommand}\begin{luacommand}{{Position:\textunderscore{}\textunderscore{}tostring}()}
Returns a readable string representation of the position.


Return value:
\begin{itemize} \item[] string representation of the position. \end{itemize}


\end{luacommand}\begin{luacommand}{{Position:copy}()}
Creates a copy of this position object.


Return value:
\begin{itemize} \item[] Copy of the position. \end{itemize}


\end{luacommand}\begin{luacommand}{{Position:equals}(\meta{pos})}
Returns a boolean value whether the object is equal to the given position.


Return value:
\begin{itemize} \item[] true if the position is equal to the given position pos. \end{itemize}


\end{luacommand}\begin{luacommand}{{Position:getAbsCoordinates}(\meta{x},\meta{y})}
Computes absolute coordinates of a position.

Parameters:
\begin{parameterdescription}
	\item[\meta{x}] Just used internally for recrusion.\item[\meta{y}] Just used internally for recrusion.
\end{parameterdescription}


Return value:
\begin{itemize} \item[] Absolute position. \end{itemize}


\end{luacommand}\begin{luacommand}{{Position:isAbsPosition}()}
Determines if the position is absolute.


Return value:
\begin{itemize} \item[] True if the position is absolute, else false. \end{itemize}


\end{luacommand}\begin{luacommand}{{Position:new}(\meta{values})}
Represents a relative postion.

Parameters:
\begin{parameterdescription}
	\item[\meta{values}] Values (e.g. x- and y-coordinate) to be merged with the default-metatable of a position.
\end{parameterdescription}


Return value:
\begin{itemize} \item[] A new position object. \end{itemize}


\end{luacommand}\begin{luacommand}{{Position:relateTo}(\meta{pos},\meta{keepAbsPosition})}
Relates a position to the given position.

Parameters:
\begin{parameterdescription}
	\item[\meta{pos}] The relative position.\item[\meta{keepAbsPosition}] If true, the coordinates of the position are computed in the relation to the given position pos.
\end{parameterdescription}



\end{luacommand}
\end{filedescription}
% This file has been generated from the lua sources using LuaDoc.
% To regenerate it call "make genluadoc" in
% doc/generic/pgf/version-for-luatex/en.

\begin{filedescription}{pgflibrarygraphdrawing-vector.lua}


\begin{luacommand}{{Vector:copy}()}
Creates a copy of the vector that holds the same elements as the original. 


Return value:
\begin{parameterdescription} 
  \item[] A newly-allocated copy of the vector holding exactly the same elements. 
\end{parameterdescription}


\end{luacommand}
\begin{luacommand}{{Vector:dividedBy}(\meta{other})}
Performs a vector division and returns the result in a new vector.  The possible origins of the vector operands are resolved and are dropped in the result vector. 

Parameters:
\begin{parameterdescription}
	\item[\meta{other}] Vector to divide by. 
\end{parameterdescription}


Return value:
\begin{parameterdescription} 
  \item[] A new vector with the result of the division. 
\end{parameterdescription}


\end{luacommand}
\begin{luacommand}{{Vector:dividedByScalar}(\meta{scalar})}
Divides a vector by a scalar value and returns the result in a new vector.  The possible origin of the vector is resolved and is dropped in the result vector. 

Parameters:
\begin{parameterdescription}
	\item[\meta{scalar}] Scalar value to divide the vector by. 
\end{parameterdescription}


Return value:
\begin{parameterdescription} 
  \item[] A new vector with the result of the division. 
\end{parameterdescription}


\end{luacommand}
\begin{luacommand}{{Vector:dotProduct}(\meta{other})}
Performs the dot product of two vectors and returns the result in a new vector.  The possible origins of the vector operands are resolved during the compuation. 

Parameters:
\begin{parameterdescription}
	\item[\meta{other}] Vector to perform the dot product with. 
\end{parameterdescription}


Return value:
\begin{parameterdescription} 
  \item[] A new vector with the result of the dot product. 
\end{parameterdescription}


\end{luacommand}
\begin{luacommand}{{Vector:get}(\meta{index})}
Returns the element at the given \meta{index}. 


Return value:
\begin{parameterdescription} 
  \item[] The element at the given \meta{index}. 
\end{parameterdescription}


\end{luacommand}
\begin{luacommand}{{Vector:getOrigin}()}
Gets the origin of the vector. 


Return value:
\begin{parameterdescription} 
  \item[] Origin of the vector or |nil| if none is set. 
\end{parameterdescription}


\end{luacommand}
\begin{luacommand}{{Vector:limit}(\meta{limit\_function})}
Limits all elements of the vector in-place. 

Parameters:
\begin{parameterdescription}
	\item[\meta{limit\_function}] A function that is called for each index/element pair. It is supposed to return minimum and maximum values for the element. The element is then clamped to these values. 
\end{parameterdescription}



\end{luacommand}
\begin{luacommand}{{Vector:minus}(\meta{other})}
Subtracts two vectors and returns the result in a new vector. 

Parameters:
\begin{parameterdescription}
	\item[\meta{other}] Vector to subtract. If this vector is defined relative to an origin, then that origin is resolved when computing the subtraction of the two vectors. The result becomes |self + other.origin + other|. The origin of |self| is preserved. 
\end{parameterdescription}


Return value:
\begin{parameterdescription} 
  \item[] A new vector with the result of the subtraction. 
\end{parameterdescription}


\end{luacommand}
\begin{luacommand}{{Vector:minusScalar}(\meta{scalar})}
Subtracts a scalar value from a vector and returns the result in a new vector. 

Parameters:
\begin{parameterdescription}
	\item[\meta{scalar}] Scalar value to subtract from all elements. 
\end{parameterdescription}


Return value:
\begin{parameterdescription} 
  \item[] A new vector with the result of the subtraction. 
\end{parameterdescription}


\end{luacommand}
\begin{luacommand}{{Vector:new}(\meta{n},\meta{fill\_function},\meta{origin})}
Creates a new vector with \meta{n} values using an optional \meta{fill\_function}. 

Parameters:
\begin{parameterdescription}
	\item[\meta{n}] The number of elements of the vector.\item[\meta{fill\_function}] Optional function that takes a number between 1 and \meta{n} and is expected to return a value for the corresponding element of the vector. If omitted, all elements of the vector will be initialized with 0.\item[\meta{origin}] Optional origin vector. 
\end{parameterdescription}


Return value:
\begin{parameterdescription} 
  \item[] A newly-allocated vector with \meta{n} elements. 
\end{parameterdescription}


\end{luacommand}
\begin{luacommand}{{Vector:norm}()}
Computes the Euclidean norm of the vector. 


Return value:
\begin{parameterdescription} 
  \item[] The Euclidean norm of the vector. 
\end{parameterdescription}


\end{luacommand}
\begin{luacommand}{{Vector:normalized}()}
Normalizes the vector and returns the result in a new vector.  The possible origin of the vector is resolved during the computation and is dropped in the result vector. 


Return value:
\begin{parameterdescription} 
  \item[] Normalized version of the original vector. 
\end{parameterdescription}


\end{luacommand}
\begin{luacommand}{{Vector:plus}(\meta{other})}
Performs a vector addition and returns the result in a new vector. 

Parameters:
\begin{parameterdescription}
	\item[\meta{other}] The vector to add. If this vector is defined relative to an origin, then that origin is resolved when computing the sum of the two vectors. The sum becomes |self + other.origin + other|. The origin of |self| is preserved. 
\end{parameterdescription}


Return value:
\begin{parameterdescription} 
  \item[] A new vector with the result of the addition. 
\end{parameterdescription}


\end{luacommand}
\begin{luacommand}{{Vector:plusScalar}(\meta{scalar})}
Performs an addition with a scalar value and returns the result in a new vector.  The scalar value is added to all elements of the vector. 

Parameters:
\begin{parameterdescription}
	\item[\meta{scalar}] Scalar value to add to all elements. 
\end{parameterdescription}


Return value:
\begin{parameterdescription} 
  \item[] A new vector with the result of the addition. 
\end{parameterdescription}


\end{luacommand}
\begin{luacommand}{{Vector:reset}()}
Resets all vector elements to 0 in-place.  This does not reset the origin vector. 



\end{luacommand}
\begin{luacommand}{{Vector:set}(\meta{index},\meta{value})}
Changes the element at the given \meta{index}. 

Parameters:
\begin{parameterdescription}
	\item[\meta{index}] The index of the element to change.\item[\meta{value}] New value of the element. 
\end{parameterdescription}



\end{luacommand}
\begin{luacommand}{{Vector:setOrigin}(\meta{origin},\meta{preserve\_values})}
Sets the origin of the vector. 

Parameters:
\begin{parameterdescription}
	\item[\meta{origin}] Vector to use as the origin.\item[\meta{preserve\_values}] Optional flag. If set to |true|, the origin will be set and the current elements of the vector will be changed so that the sum of the origin and the new element values is equal to the old values. 
\end{parameterdescription}



\end{luacommand}
\begin{luacommand}{{Vector:timesScalar}(\meta{scalar})}
Multiplies a vector by a scalar value and returns the result in a new vector.  The possible origin of the vector is resolved and is dropped in the result vector. 

Parameters:
\begin{parameterdescription}
	\item[\meta{scalar}] Scalar value to multiply the vector with. 
\end{parameterdescription}


Return value:
\begin{parameterdescription} 
  \item[] A new vector with the result of the multiplication. 
\end{parameterdescription}


\end{luacommand}
\begin{luacommand}{{Vector:update}(\meta{update\_function})}
Updates the values of the vector in-place. 

Parameters:
\begin{parameterdescription}
	\item[\meta{update\_function}] A function that is called for each element of the vector. The elements are replaced by the values returned from this function. 
\end{parameterdescription}



\end{luacommand}
\begin{luacommand}{{Vector:x}()}
Convenience method that returns the first element of the vector.  The origin vector is not resolved in this function call. 


Return value:
\begin{parameterdescription} 
  \item[] The first element of the vector. 
\end{parameterdescription}


\end{luacommand}
\begin{luacommand}{{Vector:y}()}
Convenience method that returns the second element of the vector.  The origin vector is not resolved in this function call. 


Return value:
\begin{parameterdescription} 
  \item[] The second element of the vector. 
\end{parameterdescription}


\end{luacommand}

\end{filedescription}
% This file has been generated from the lua sources using LuaDoc.
% To regenerate it call "make genluadoc" in
% doc/generic/pgf/version-for-luatex/en.

\begin{filedescription}{pgflibrarygraphdrawing-box.lua}


\begin{luacommand}{{Box:addBox}(\meta{box})}
Adds new internal Box.

Parameters:
\begin{parameterdescription}
	\item[\meta{box}] The box to be added.
\end{parameterdescription}



\end{luacommand}\begin{luacommand}{{Box:getPaths}()}
Provides all Paths this box contains.


Return value:
\begin{itemize} \item[] Recursive iteration over all paths. \end{itemize}


\end{luacommand}\begin{luacommand}{{Box:getPosAt}(\meta{place},\meta{absolute})}
Calculates the coordinates of the box according to the place parameter.

Parameters:
\begin{parameterdescription}
	\item[\meta{place}] Determines of which position of the box the coordinates should be returned (e.g. the center of the box requires the param Box.CENTER).  Possible values are: \begin{itemize} \item Box.UPPERLEFT \item Box.UPPERRIGHT \item Box.CENTER \item Box.LOWERRIGHT \item Box.LOWERLEFT \end{itemize}\item[\meta{absolute}] If true the absolute coordinates of the box will be returned, otherwise its relative coordinates.
\end{parameterdescription}


Return value:
\begin{itemize} \item[] X- and y-coordinates of the box. \end{itemize}


\end{luacommand}\begin{luacommand}{{Box:new}(\meta{values})}
Creates a new box.

Parameters:
\begin{parameterdescription}
	\item[\meta{values}] Values (e.g. height) to be merged with the default-metatable of a box.
\end{parameterdescription}


Return value:
\begin{itemize} \item[] The new box. \end{itemize}


\end{luacommand}\begin{luacommand}{{Box:recalculateSize}()}
Checks internal Boxes and resets width and height.



\end{luacommand}\begin{luacommand}{{Box:removeBox}(\meta{box})}
Removes internal Box.

Parameters:
\begin{parameterdescription}
	\item[\meta{box}] The box to remove.
\end{parameterdescription}



\end{luacommand}
\end{filedescription}
%% This file has been generated from the lua sources using LuaDoc.
% To regenerate it call "make genluadoc" in
% doc/generic/pgf/version-for-luatex/en.

\begin{filedescription}{pgflibrarygraphdrawing-path.lua}


\begin{luacommand}{{Path:\textunderscore{}\textunderscore{}tostring}()}
Returns a readable string representation of the path.


Return value:
\begin{parameterdescription} 
  \item[] String representation of the path
\end{parameterdescription}


\end{luacommand}
\begin{luacommand}{{Path:\textunderscore{}intersects}(\meta{a1},\meta{a2},\meta{b1},\meta{b2},\meta{allowedIntersections})}
Checks if the lines a1a2 and b1b2 intersect.

Parameters:
\begin{parameterdescription}
	\item[\meta{a1}] Start of the first line.\item[\meta{a2}] End of the first line.\item[\meta{b1}] Start of the second line.\item[\meta{b2}] End of the second line.\item[\meta{allowedIntersections}] A boolean table with the keys a1, a2, b1 and b2. If two or three of those values are true, the corresponding start and/or end points are allowed to match without being seen as intersection. If all four keys are true any matching of start and end points is allowed as long as the two lines are not coincedent. If three of the keys are true or start and end of a line are allowed to match, nill will be returned. If this optional parameter is not given, any matching points will be seen as intersections.
\end{parameterdescription}


Return value:
\begin{parameterdescription} 
  \item[] true, if lines intersect, false otherwise. If allowedIntersections contained an invalid value, nil will be returned.
\end{parameterdescription}


\end{luacommand}
\begin{luacommand}{{Path:addPoint}(\meta{point},\meta{keepAbsPosition})}
Appends new point at the end of path.

Parameters:
\begin{parameterdescription}
	\item[\meta{point}] Point to be added to the path\item[\meta{keepAbsPosition}] true if the coordinates of the point are absolute
\end{parameterdescription}



\end{luacommand}
\begin{luacommand}{{Path:createPath}(\meta{posStart},\meta{posEnd},\meta{keepAbsPosition})}
Adds a new segment to the path.

Parameters:
\begin{parameterdescription}
	\item[\meta{posStart}] Startposition of the new segment\item[\meta{posEnd}] Endposition of the new segment
\end{parameterdescription}



\end{luacommand}
\begin{luacommand}{{Path:getLastPoint}()}
Returns last point in path.


Return value:
\begin{parameterdescription} 
  \item[] last point
\end{parameterdescription}


\end{luacommand}
\begin{luacommand}{{Path:getLength}()}
Returns the length of the whole path.


Return value:
\begin{parameterdescription} 
  \item[] Length of the whole path.
\end{parameterdescription}


\end{luacommand}
\begin{luacommand}{{Path:getPoints}()}
Copies the internal points of a path.


Return value:
\begin{parameterdescription} 
  \item[] array of points
\end{parameterdescription}


\end{luacommand}
\begin{luacommand}{{Path:intersects}(\meta{path})}
Tests if the path is intersected by path.

Parameters:
\begin{parameterdescription}
	\item[\meta{path}] other path
\end{parameterdescription}



\end{luacommand}
\begin{luacommand}{{Path:move}(\meta{x},\meta{y})}
Adds new point with x,y relative to last point.

Parameters:
\begin{parameterdescription}
	\item[\meta{x}] x-coordinate of the new point\item[\meta{y}] y-coordinate of the new point
\end{parameterdescription}



\end{luacommand}
\begin{luacommand}{{Path:new}(\meta{values})}
Creates a new path.

Parameters:
\begin{parameterdescription}
	\item[\meta{values}] Values to be merged with the default-metatable of a path
\end{parameterdescription}


Return value:
\begin{parameterdescription} 
  \item[] A new path.
\end{parameterdescription}


\end{luacommand}

\end{filedescription}

\subsubsection{Base Layer}
% This file has been generated from the lua sources using LuaDoc.
% To regenerate it call "make genluadoc" in
% doc/generic/pgf/version-for-luatex/en.

\begin{filedescription}{pgflibrarygraphdrawing-interface.lua}


\begin{luacommand}{{Interface:addEdge}(\meta{from},\meta{to},\meta{direction},\meta{options})}
Adds an edge from one node to another by name.  That is, both parameters are node names and have to exist before an edge can be created between them.

Parameters:
\begin{parameterdescription}
	\item[\meta{options}] A key=value string, which is currently only passed back to the \TeX layer during shipout (deprecated, use \tikzname\ keys instead).
\end{parameterdescription}



See also:
\begin{itemize}
	\item[] |addNode|
\end{itemize}

\end{luacommand}\begin{luacommand}{{Interface:addNode}(\meta{name},\meta{xMin},\meta{yMin},\meta{xMax},\meta{yMax},\meta{options})}
Adds a new node to the graph.  The options string is parsed and assigned.

Parameters:
\begin{parameterdescription}
	\item[\meta{name}] Name of the node.\item[\meta{xMin}] Minimum x point of the bouding box.\item[\meta{yMin}] Minimum y point of the bouding box.\item[\meta{xMax}] Maximum x point of the bouding box.\item[\meta{yMax}] Maximum y point of the bouding box.\item[\meta{options}] Options to pass to the node (deprecated, use \tikzname\ keys instead).
\end{parameterdescription}



\end{luacommand}\begin{luacommand}{{Interface:drawEdge}(\meta{object})}
Helper function to put visible edges back to the TeX layer.

Parameters:
\begin{parameterdescription}
	\item[\meta{object}] Lua edge object to draw.
\end{parameterdescription}



\end{luacommand}\begin{luacommand}{{Interface:drawGraph}()}
Draws/layouts the current graph using the specified algorithm.  The algorithm is derived from the options attribute and is loaded on demand from the corresponding file, e.g. for algorithm ``simple'' it is ``pgflibrarygraphdrawing-algorithms-simple.lua'' which has to define a function named ``drawGraphAlgorithm\_simple'' in the pgf.graphdrawing module.  It is then called with the graph as single parameter.



\end{luacommand}\begin{luacommand}{{Interface:drawNode}(\meta{object})}
Helper function to actually put the node back to the TeX layer.

Parameters:
\begin{parameterdescription}
	\item[\meta{object}] The lua node object to draw.
\end{parameterdescription}



\end{luacommand}\begin{luacommand}{{Interface:finishGraph}()}
Pops the top graph from the graph stack (which is the current graph) and actually draws the nodes and edges on the canvas.



\end{luacommand}\begin{luacommand}{{Interface:getOption}(\meta{name})}
Returns the value of the graph option name.

Parameters:
\begin{parameterdescription}
	\item[\meta{name}] Name of the option.
\end{parameterdescription}


Return value:
\begin{itemize} \item[] The stored value or nil. \end{itemize}


\end{luacommand}\begin{luacommand}{{Interface:loadAlgorithm}(\meta{name})}
Loads the file with the ``pgflibrarygraphdrawing-algorithms-xyz.lua'' naming scheme.

Parameters:
\begin{parameterdescription}
	\item[\meta{name}] Name of  the algorithm, like ``xyz''.
\end{parameterdescription}


Return value:
\begin{itemize} \item[] The algorithm function or nil. \end{itemize}


\end{luacommand}\begin{luacommand}{{Interface:newGraph}(\meta{options})}
Creates a new graph and pushes it on top of the graph stack.  The options string is parsed and assigned.

Parameters:
\begin{parameterdescription}
	\item[\meta{options}] A list of options for this graph (deprecated, use \tikzname\ keys instead).
\end{parameterdescription}



See also:
\begin{itemize}
	\item[] |finishGraph|
\end{itemize}

\end{luacommand}\begin{luacommand}{{Interface:setOption}(\meta{name},\meta{value})}
Sets a graph option name to value.

Parameters:
\begin{parameterdescription}
	\item[\meta{name}] The name of the option to set.\item[\meta{value}] New value for the option.
\end{parameterdescription}



\end{luacommand}
\end{filedescription}
\label{section-library-graphdrawing-lua-documentation-interface}
% This file has been generated from the lua sources using LuaDoc.
% To regenerate it call "make genluadoc" in
% doc/generic/pgf/version-for-luatex/en.

\begin{filedescription}{pgflibrarygraphdrawing-sys.lua}


\begin{luacommand}{{Sys:beginShipout}()}
Begins the shipout of nodes by opening a scope in pgf.



\end{luacommand}\begin{luacommand}{{Sys:endShipout}()}
Ends the shipout by closing the opened scope.



See also:
\begin{itemize}
	\item[] |Sys:beginShipout()|
\end{itemize}

\end{luacommand}\begin{luacommand}{{Sys:escapeTeXNodeName}(\meta{nodename})}
Adds a ``not yet positionedPGFGDINTERNAL'' prefix to a node name. The prefix is required by pgf to place the node. Actually, when deferring the node placement, the prefix is added to avoid references to the node.

Parameters:
\begin{parameterdescription}
	\item[\meta{nodename}] Name of the node to prefix.
\end{parameterdescription}


Return value:
\begin{itemize} \item[] A newly composed string. \end{itemize}


\end{luacommand}\begin{luacommand}{{Sys:getTeXBox}()}
Retrieves a box from the transfer box register.



See also:
\begin{itemize}
	\item[] |putTeXBox|
\end{itemize}

\end{luacommand}\begin{luacommand}{{Sys:getVerboseMode}()}
Checks the verbosity of the subsystems output.


Return value:
\begin{itemize} \item[] Boolean value specifying the verbosity. \end{itemize}


\end{luacommand}\begin{luacommand}{{Sys:logMessage}(\meta{...})}
Prints objects to the TeX output, formatting them with tostring and separated by spaces.

Parameters:
\begin{parameterdescription}
	\item[\meta{...}] List of parameters.
\end{parameterdescription}



\end{luacommand}\begin{luacommand}{{Sys:putEdge}(\meta{edge},\meta{Edge})}
Assembles and outputs the TeX command to draw an edge.

Parameters:
\begin{parameterdescription}
	\item[\meta{Edge}] A lua edge object.
\end{parameterdescription}



\end{luacommand}\begin{luacommand}{{Sys:putTeXBox}(\meta{nodename},\meta{texnode},\meta{minX},\meta{minY},\meta{maxX},\meta{maxY},\meta{posX},\meta{posY},\meta{nodeName})}
Saves a box from the transfer box register.

Parameters:
\begin{parameterdescription}
	\item[\meta{texnode}] The box which contains the \TeX\ node.\item[\meta{minX}] Maximum y of the bounding box.\item[\meta{minY}] Minimal y of the bounding box.\item[\meta{posX}] X coordinate where to put the node in the output.\item[\meta{posY}] Y coordinate where to put the node in the output.\item[\meta{nodeName}] The name of the node in the box.
\end{parameterdescription}



\end{luacommand}\begin{luacommand}{{Sys:setBoxNumber}(\meta{bn})}
Init method, sets the box register number. This method is called when the \tikzname\ (pgf) library is loaded.

Parameters:
\begin{parameterdescription}
	\item[\meta{bn}] Number of the box register used for transfering boxes of the current graph.
\end{parameterdescription}



\end{luacommand}\begin{luacommand}{{Sys:setVerboseMode}(\meta{mode})}
Enables or disables verbose logging for the graph drawing library.

Parameters:
\begin{parameterdescription}
	\item[\meta{mode}] If true, enable verbose logging. Otherwise it'll be disabled.
\end{parameterdescription}



\end{luacommand}\begin{luacommand}{{Sys:unescapeTeXNodeName}(\meta{nodename})}
Removes the ``not yet positionedPGFGDINTERNAL'' prefix from a node name.

Parameters:
\begin{parameterdescription}
	\item[\meta{nodename}] Nodename without prefix.
\end{parameterdescription}


Return value:
\begin{itemize} \item[] The substring in question. \end{itemize}


See also:
\begin{itemize}
	\item[] |Sys:escapeTeXNodeName(nodename)|
\end{itemize}

\end{luacommand}
\end{filedescription}
\label{section-library-graphdrawing-lua-documentation-sys}
% This file has been generated from the lua sources using LuaDoc.
% To regenerate it call "make genluadoc" in
% doc/generic/pgf/version-for-luatex/en.

\begin{filedescription}{pgflibrarygraphdrawing-texboxregister.lua}


\begin{luacommand}{{TeXBoxRegister:getBox}(\meta{boxReference})}
Gets a box by its reference.

Parameters:
\begin{parameterdescription}
	\item[\meta{boxReference}] Reference id of the box to get.
\end{parameterdescription}



See also:
\begin{itemize}
	\item[] |TeXBoxRegister:insertBox(texbox)|
\end{itemize}

\end{luacommand}\begin{luacommand}{{TeXBoxRegister:insertBox}(\meta{texbox})}
Adds the content of a \TeX\ box to the box register class. Contents of the box will be stored. 



\end{luacommand}
\end{filedescription}

\subsubsection{Helper Classes and Functions}
% This file has been generated from the lua sources using LuaDoc.
% To regenerate it call "make genluadoc" in
% doc/generic/pgf/version-for-luatex/en.

\begin{filedescription}{pgflibrarygraphdrawing-helper.lua}


\begin{luacommand}{{parseBraces}(\meta{str},\meta{default})}
Parses a braced list of {key}{value} pairs and returns a table mapping keys to values.



\end{luacommand}

\end{filedescription}
% This file has been generated from the lua sources using LuaDoc.
% To regenerate it call "make genluadoc" in
% doc/generic/pgf/version-for-luatex/en.

\begin{filedescription}{pgflibrarygraphdrawing-table-helpers.lua}


\begin{luacommand}{{table.combine\textunderscore{}pairs}(\meta{table},\meta{combine\_func},\meta{initial\_value})}
Combine all key/value pairs of \meta{table} to a single value using a combine function.  This is a very powerful function. It can be used for combining the key/value pairs of a table into a single string but can also be used to compute mathematical operations on tables, such as finding the maximum value in a table etc.  The main difference to |table.combine_values| is that keys and values are used to determine the combination value and that the key/value pairs are are passed to \meta{combine\_func} in a random order. 

Parameters:
\begin{parameterdescription}
	\item[\meta{table}] Table to iterate over.\item[\meta{combine\_func}] Function to be called for each key/value pair. It takes three parameters, the current combination value and the key/value pair. It is supposed to return a new combination value.\item[\meta{initial\_value}] Initial combination value. 
\end{parameterdescription}


Return value:
\begin{parameterdescription} 
  \item[] The final combination value after all key/value pairs have been passed over to \meta{combine\_func}. 
\end{parameterdescription}


\end{luacommand}
\begin{luacommand}{{table.combine\textunderscore{}values}(\meta{input},\meta{combine\_func},\meta{initial\_value})}
Combine all values of \meta{input} to a single value using a combine function.  This is a very powerful function. It can be used for combining the values of a table into a single string but can also be used to compute mathematical operations on tables, such as finding the maximum value in a table etc.  The main difference to |table.combine_pairs| is that the keys are ignored and that the values are passed to \meta{combine\_func} in the order they appear in the table. 

Parameters:
\begin{parameterdescription}
	\item[\meta{input}] Table to iterate over.\item[\meta{combine\_func}] Function to be called for each value. It takes two parameters, the current combination value and the current value. It is supposed to return a new combination value.\item[\meta{initial\_value}] Initial combination value. 
\end{parameterdescription}


Return value:
\begin{parameterdescription} 
  \item[] The final combination value after all values of \meta{input} have been passed over to \meta{combine\_func}. 
\end{parameterdescription}


\end{luacommand}
\begin{luacommand}{{table.copy}(\meta{source},\meta{target})}
Copies a table while preserving its metatable. 

Parameters:
\begin{parameterdescription}
	\item[\meta{source}] The table to copy.\item[\meta{target}] The table to which values are to be copied or |nil| if a new table is to be allocated. 
\end{parameterdescription}


Return value:
\begin{parameterdescription} 
  \item[] The \meta{target} table or a newly allocated table containing all keys and values of the \meta{source} table. 
\end{parameterdescription}


\end{luacommand}
\begin{luacommand}{{table.count\textunderscore{}pairs}(\meta{input})}
Count the key/value pairs in the table. 

Parameters:
\begin{parameterdescription}
	\item[\meta{input}] The table whose key/value pairs to count. 
\end{parameterdescription}


Return value:
\begin{parameterdescription} 
  \item[] Number of key/value pairs in the table. 
\end{parameterdescription}


\end{luacommand}
\begin{luacommand}{{table.filter\textunderscore{}keys}(\meta{table},\meta{filter\_func})}
Copies a table and filters out all keys using a function. 

Parameters:
\begin{parameterdescription}
	\item[\meta{table}] The table whose values are to be filtered.\item[\meta{filter\_func}] The test function to be called for each key of \meta{table}. If it returns |false| or |nil| for a key, that key will not be part of the result table. 
\end{parameterdescription}


Return value:
\begin{parameterdescription} 
  \item[] Copy of \meta{table} with its keys filtered using \meta{filter\_func}. 
\end{parameterdescription}


\end{luacommand}
\begin{luacommand}{{table.filter\textunderscore{}pairs}(\meta{table},\meta{filter\_func})}
Copies a table and filters out all key/value pairs using a function. 

Parameters:
\begin{parameterdescription}
	\item[\meta{table}] The table whose values are to be filtered.\item[\meta{filter\_func}] The test function to be called for each pair of \meta{table}. If it returns |false| or |nil| for a pair, that pair will not be part of the result table. 
\end{parameterdescription}


Return value:
\begin{parameterdescription} 
  \item[] Copy of \meta{table} with its pairs filtered using \meta{filter\_func}. 
\end{parameterdescription}


\end{luacommand}
\begin{luacommand}{{table.filter\textunderscore{}values}(\meta{input},\meta{filter\_func})}
Copies a table and filters out all values using a function. 

Parameters:
\begin{parameterdescription}
	\item[\meta{input}] The table whose values are to be filtered.\item[\meta{filter\_func}] The test function to be called for each value of the input table. If it returns |false| or |nil| for a value, that value will not be part of the result table. 
\end{parameterdescription}


Return value:
\begin{parameterdescription} 
  \item[] Copy of \meta{input} with its values filtered using \meta{filter\_func}. 
\end{parameterdescription}


\end{luacommand}
\begin{luacommand}{{table.find}(\meta{table},\meta{find\_func})}
Returns the first value in \meta{table} for which \meta{find\_func} returns |true|. 

Parameters:
\begin{parameterdescription}
	\item[\meta{table}] The table to search in.\item[\meta{find\_func}] A function to test values with. It receives a single parameter (a value of \meta{table}) and is supposed to return either |true| or |false|. 
\end{parameterdescription}


Return value:
\begin{parameterdescription} 
  \item[] The first value of \meta{table} for which \meta{find\_func} returns true. Returns |nil| if the function was |false| for al of the values in \meta{table}. 
\end{parameterdescription}


\end{luacommand}
\begin{luacommand}{{table.find\textunderscore{}index}(\meta{table},\meta{find\_func})}
Returns the index of the first value in \meta{table} for which \meta{find\_func} returns |true|. 

Parameters:
\begin{parameterdescription}
	\item[\meta{table}] The table to search in.\item[\meta{find\_func}] A function to test values with. It receives a single parameter (a value of \meta{table}) and is supposed to return either |true| or |false|. 
\end{parameterdescription}


Return value:
\begin{parameterdescription} 
  \item[] Index of the first value of \meta{table} for which \meta{find\_func} returns |true|. Returns |nil| if the function was |false| for all of the values in \meta{table}. 
\end{parameterdescription}


\end{luacommand}
\begin{luacommand}{{table.key\textunderscore{}iter}(\meta{table})}
Iterate over all keys of a table in random order. 

Parameters:
\begin{parameterdescription}
	\item[\meta{table}] The table whose keys to iterate over. 
\end{parameterdescription}


Return value:
\begin{parameterdescription} 
  \item[] An iterator for the keys of the table. 
\end{parameterdescription}


\end{luacommand}
\begin{luacommand}{{table.map}(\meta{input},\meta{map\_func})}
Maps key/value pairs of an \meta{input} table to a flat table of new values. 

Parameters:
\begin{parameterdescription}
	\item[\meta{input}] Table whose key/value pairs are to be mapped to new values.\item[\meta{map\_func}] The mapping function to be called for each key/value pair of \meta{input}. The value it returns for a pair will be inserted into the result table. 
\end{parameterdescription}


Return value:
\begin{parameterdescription} 
  \item[] A new table containing all values returned by \meta{map\_func} for the key/value pairs of the \meta{input} table. 
\end{parameterdescription}


\end{luacommand}
\begin{luacommand}{{table.map\textunderscore{}keys}(\meta{table},\meta{map\_func})}
Maps keys of a table to new keys in a copy of the table. 

Parameters:
\begin{parameterdescription}
	\item[\meta{table}] The table whose keys are to be mapped to new keys.\item[\meta{map\_func}] A function to be called for each key of \meta{table} in order to generate a new key to replace the old one in the result table. 
\end{parameterdescription}


Return value:
\begin{parameterdescription} 
  \item[] A new table with all keys of \meta{table} having been replaced with the keys returned from \meta{map\_func}. The original values are preserved. 
\end{parameterdescription}


\end{luacommand}
\begin{luacommand}{{table.map\textunderscore{}pairs}(\meta{table},\meta{map\_func})}
Maps keys and values of a table to new pairs of keys and values. 

Parameters:
\begin{parameterdescription}
	\item[\meta{table}] The table whose key and value pairs are to be replaced.\item[\meta{map\_func}] A function to be called for each key and value pair of \meta{table} in order to generate a new pair to replace the old one. 
\end{parameterdescription}


Return value:
\begin{parameterdescription} 
  \item[] A new table with all key and value pairs of \meta{table} having been replaced with the pairs returned from \meta{map\_func}. 
\end{parameterdescription}


\end{luacommand}
\begin{luacommand}{{table.map\textunderscore{}values}(\meta{input},\meta{map\_func})}
Maps values of a table to new values in a new table. 

Parameters:
\begin{parameterdescription}
	\item[\meta{input}] The table whose values are to be mapped to new values.\item[\meta{map\_func}] A function to be called for each value in order to generate a new value to replace the old one in the result table. 
\end{parameterdescription}


Return value:
\begin{parameterdescription} 
  \item[] A new table with all values of the \meta{input} table having been replaced with the values returned from \meta{map\_func}. 
\end{parameterdescription}


\end{luacommand}
\begin{luacommand}{{table.merge}(\meta{table1},\meta{table2},\meta{first\_metatable})}
Merges the key/value pairs of two tables.  This function merges the key/value pairs of the two input tables.  All |nil| values of the first table are overwritten by the corresponding values of the second table.  By default the metatable of the second input table is applied to the resulting table. If \meta{first\_metatable} is set to |true| however, the metatable of the first input table will be used. 

Parameters:
\begin{parameterdescription}
	\item[\meta{table1}] First table with key/value pairs.\item[\meta{table2}] Second table with key/value pairs.\item[\meta{first\_metatable}] Whether to inherit the metatable of \meta{table1} or not. 
\end{parameterdescription}


Return value:
\begin{parameterdescription} 
  \item[] A new table with the key/value pairs of the two input tables merged together. 
\end{parameterdescription}


\end{luacommand}
\begin{luacommand}{{table.randomized\textunderscore{}pair\textunderscore{}iter}(\meta{table})}
Iterate over the key/value pairs of \meta{table} in a truely random order. 

Parameters:
\begin{parameterdescription}
	\item[\meta{table}] The table whose key/value pairs to iterate over. 
\end{parameterdescription}


Return value:
\begin{parameterdescription} 
  \item[] A randomized iterator for the values of \meta{table}. 
\end{parameterdescription}


\end{luacommand}
\begin{luacommand}{{table.randomized\textunderscore{}value\textunderscore{}iter}(\meta{table})}
Iterate over the values of \meta{table} in a truely random order. 

Parameters:
\begin{parameterdescription}
	\item[\meta{table}] The table whose values to iterate over. 
\end{parameterdescription}


Return value:
\begin{parameterdescription} 
  \item[] A randomized iterator for the values of the table. 
\end{parameterdescription}


\end{luacommand}
\begin{luacommand}{{table.remove\textunderscore{}values}(\meta{input},\meta{remove\_func})}
Removes all values from \meta{input} for which \meta{remove\_func} returns |true|.  Important note: this method does not work with dictionaries. Make sure only to process number-indexed arrays with it. 

Parameters:
\begin{parameterdescription}
	\item[\meta{input}] The table to remove values from.\item[\meta{remove\_func}] Function to be called for each value of \meta{input}. If it returns |false|, the value will be removed from the table in-place. 
\end{parameterdescription}


Return value:
\begin{parameterdescription} 
  \item[] \meta{input} which was edited in-place. 
\end{parameterdescription}


\end{luacommand}
\begin{luacommand}{{table.update\textunderscore{}values}(\meta{table},\meta{update\_func})}
Update values of \meta{table} in-place using an update function. 

Parameters:
\begin{parameterdescription}
	\item[\meta{table}] The table whose values are to be updated.\item[\meta{update\_func}] A function that takes two parameters, the key/value pairs of \meta{table} and returns a new value to replace the old one. 
\end{parameterdescription}


Return value:
\begin{parameterdescription} 
  \item[] The input \meta{table}. 
\end{parameterdescription}


\end{luacommand}
\begin{luacommand}{{table.value\textunderscore{}iter}(\meta{table})}
Iterate over all values of a table.  FIXME: The iterators stops if a key's value is nil. But we actually want to continue iterating until the end of the table. 

Parameters:
\begin{parameterdescription}
	\item[\meta{table}] The table whose values to iterate over. 
\end{parameterdescription}


Return value:
\begin{parameterdescription} 
  \item[] An iterator for the values of the table. 
\end{parameterdescription}


\end{luacommand}

\end{filedescription}
% This file has been generated from the lua sources using LuaDoc.
% To regenerate it call "make genluadoc" in
% doc/generic/pgf/version-for-luatex/en.

\begin{filedescription}{pgflibrarygraphdrawing-iter-helpers.lua}


\begin{luacommand}{{iter.filter}(\meta{iterator},\meta{filter\_func})}
Skips all values of an iterator for which \meta{filter\_func} returns |false|. 

Parameters:
\begin{parameterdescription}
	\item[\meta{iterator}] Original \meta{iterator} of values.\item[\meta{filter\_func}] Filter function that takes a value of the original \meta{iterator} and is expected to return |false| if the value should be skipped. 
\end{parameterdescription}


Return value:
\begin{parameterdescription} 
  \item[] A modified iterator that skips values of \meta{iterator} for which \meta{filter\_func} returns |false|. 
\end{parameterdescription}


\end{luacommand}
\begin{luacommand}{{iter.map}(\meta{iterator},\meta{map\_func})}
Maps all values of an iterator to new values.  This function will cause loops to iterate over the values of the original \meta{iterator} replaced by the values returned from \meta{map\_func}. 

Parameters:
\begin{parameterdescription}
	\item[\meta{iterator}] Original iterator whose values are to be mapped to new ones.\item[\meta{map\_func}] Mapping function that takes a value of the original \meta{iterator} and maps it to a new value that is then returned to the loop instead. 
\end{parameterdescription}


Return value:
\begin{parameterdescription} 
  \item[] A modified iterator. 
\end{parameterdescription}


\end{luacommand}
\begin{luacommand}{{iter.times}(\meta{n})}
Causes a loop to run multiple times.  Use this iterator like this to perform 100 loops: |for n in iter.times(100) do ... end|.  To iterate over the values $0, 10, 20, 30, ..., 100$ do: |for n in iter.filter(iter.times(100), function (n) return n % 10 == 0 end)| 

Parameters:
\begin{parameterdescription}
	\item[\meta{n}] Number of loops. 
\end{parameterdescription}



\end{luacommand}

\end{filedescription}
% This file has been generated from the lua sources using LuaDoc.
% To regenerate it call "make genluadoc" in
% doc/generic/pgf/version-for-luatex/en.

\begin{filedescription}{pgflibrarygraphdrawing-traversal-helpers.lua}


\begin{luacommand}{{traversal.depth\textunderscore{}first\textunderscore{}dag}(\meta{graph},\meta{initial\_nodes})}
Iterator for traversing a directed acyclic \meta{graph} in depth-first order. 

Parameters:
\begin{parameterdescription}
	\item[\meta{graph}] A directed acyclic graph. 
\end{parameterdescription}


Return value:
\begin{parameterdescription} 
  \item[] An iterator for traversing \meta{graph} in a depth-first order. 
\end{parameterdescription}


\end{luacommand}
\begin{luacommand}{{traversal.topological\textunderscore{}sorting}(\meta{graph})}
Iterator for traversing a directed \meta{graph} using a topological sorting.  A topological sorting of a directed graph is a linear ordering of its nodes such that, for every edge |(u,v)|, |u| comes before |v|.  Important note: if performed on a graph with at least one cycle a topological sorting is impossible. Thus, the nodes returned from the iterator are not guaranteed to satisfy the ``|u| comes before |v|'' criterion. The iterator may even terminate early or loop forever. 

Parameters:
\begin{parameterdescription}
	\item[\meta{graph}] A directed acyclic graph. 
\end{parameterdescription}


Return value:
\begin{parameterdescription} 
  \item[] An iterator for traversing \meta{graph} in a topological order. 
\end{parameterdescription}


\end{luacommand}

\end{filedescription}
