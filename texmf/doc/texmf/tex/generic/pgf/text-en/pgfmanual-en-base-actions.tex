% Copyright 2006 by Till Tantau
%
% This file may be distributed and/or modified
%
% 1. under the LaTeX Project Public License and/or
% 2. under the GNU Free Documentation License.
%
% See the file doc/generic/pgf/licenses/LICENSE for more details.


\section{Using Paths}

\subsection{Overview}

Once a path has been constructed, it can be \emph{used} in different
ways. For example, you can draw the path or fill it or use it for
clipping.

Numerous graph parameters influence how a path will be rendered. For
example, when you draw a path, the line width is important as well as
the dashing pattern. The options that govern how paths are rendered
can all be set with commands starting with |\pgfset|. \emph{All
  options that influence how a path is rendered always influence the
  complete path.} Thus, it is not possible to draw part of a path
using, say, a red color and drawing another part using a green
color. To achieve such an effect, you must use two paths.

In detail, paths can be used in the following ways:

\begin{enumerate}
\item
  You can \emph{stroke} (also known as \emph{draw}) a path.
\item
  You can \emph{fill} a path with a uniform color.
\item
  You can \emph{clip} subsequent renderings against the path.
\item
  You can \emph{shade} a path.
\item
  You can \emph{use the path as bounding box} for the whole picture.
\end{enumerate}
You can also perform any combination of the above, though it makes no
sense to fill and shade a path at the same time.

To perform (a combination of) the first three actions, you can use the
following command:
\begin{command}{\pgfusepath\marg{actions}}
  Applies the given \meta{actions} to the current path. Afterwards,
  the current path is (globally) empty. The following actions are
  possible:
  \begin{itemize}
  \item \declare{|fill|}
    fills the path. See Section~\ref{section-fill} for further details.
\begin{codeexample}[]
\begin{pgfpicture}
  \pgfpathmoveto{\pgfpointorigin}
  \pgfpathlineto{\pgfpoint{1cm}{1cm}}
  \pgfpathlineto{\pgfpoint{1cm}{0cm}}
  \pgfusepath{fill}
\end{pgfpicture}
\end{codeexample}
  \item \declare{|stroke|}
    strokes the path. See Section~\ref{section-stroke} for further details.
\begin{codeexample}[]
\begin{pgfpicture}
  \pgfpathmoveto{\pgfpointorigin}
  \pgfpathlineto{\pgfpoint{1cm}{1cm}}
  \pgfpathlineto{\pgfpoint{1cm}{0cm}}
  \pgfusepath{stroke}
\end{pgfpicture}
\end{codeexample}
  \item \declare{|clip|}
    clips all subsequent drawings against the path. See
    Section~\ref{section-clip} for further details.
\begin{codeexample}[]
\begin{pgfpicture}
  \pgfpathmoveto{\pgfpointorigin}
  \pgfpathlineto{\pgfpoint{1cm}{1cm}}
  \pgfpathlineto{\pgfpoint{1cm}{0cm}}
  \pgfusepath{stroke,clip}
  \pgfpathcircle{\pgfpoint{1cm}{1cm}}{0.5cm}
  \pgfusepath{fill}
\end{pgfpicture}
\end{codeexample}
  \item \declare{|discard|}
    discards the path, that is, it is not used at all. Giving this
    option (alone) has the same effect as giving an empty options
    list.
  \end{itemize}
  When more than one of the first three actions are given, they are
  applied in the above ordering, regardless of their ordering in
  \meta{actions}. Thus, |{stroke,fill}| and |{fill,stroke}| have the
  same effect. 
\end{command}

To shade a path, use the |\pgfshadepath| command, which is explained
in Section~\ref{section-shadings}.



\subsection{Stroking a Path}
\label{section-stroke}

When you use |\pgfusepath{stroke}| to stroke a path, several graphic
parameters influence how the path is drawn. The commands for setting
these parameters are explained in the following.

Note that all graphic parameters apply to the path as a whole, never
only to a part of it.

All graphic parameters are local to the current |{pgfscope}|, but they
persists past \TeX\ groups, \emph{except} for the interior rule
(even-odd or nonzero) and the arrow tip kinds. The latter graphic
parameters only persist till the end of the current \TeX\ group, but 
this may change in the future, so do not count on this.

\subsubsection{Graphic Parameter: Line Width}

\begin{command}{\pgfsetlinewidth\marg{line width}}
  This command sets the line width for subsequent strokes (in the
  current |pgfscope|). The line width is given as a normal \TeX\
  dimension like |0.4pt| or |1mm|.

\begin{codeexample}[]
\begin{pgfpicture}
  \pgfsetlinewidth{1mm}
  \pgfpathmoveto{\pgfpoint{0mm}{0mm}}
  \pgfpathlineto{\pgfpoint{2cm}{0mm}}
  \pgfusepath{stroke}
  \pgfsetlinewidth{2\pgflinewidth} % double in size
  \pgfpathmoveto{\pgfpoint{0mm}{5mm}}
  \pgfpathlineto{\pgfpoint{2cm}{5mm}}
  \pgfusepath{stroke}
\end{pgfpicture}
\end{codeexample}
\end{command}

\begin{textoken}{\pgflinewidth}
  You can access the current line width via the \TeX\ dimension
  |\pgflinewidth|. It will be set to the correct line width, that is,
  even when a \TeX\ group closed, the value will be correct since it
  is set globally, but when a |{pgfscope}| closes, the value is set to
  the correct value it had before the scope.
\end{textoken}


\subsubsection{Graphic Parameter: Caps and Joins}

\begin{command}{\pgfsetbuttcap}
  Sets the line cap to a butt cap. See Section~\ref{section-cap-joins}
  for an explanation of what this is.
\end{command}
\begin{command}{\pgfsetroundcap}
  Sets the line cap to a round cap. See again
  Section~\ref{section-cap-joins}.
\end{command}
\begin{command}{\pgfsetrectcap}
  Sets the line cap to a square cap. See again
  Section~\ref{section-cap-joins}. 
\end{command}
\begin{command}{\pgfsetroundjoin}
  Sets the line join to a round join. See again
  Section~\ref{section-cap-joins}. 
\end{command}
\begin{command}{\pgfsetbeveljoin}
  Sets the line join to a bevel join. See again
  Section~\ref{section-cap-joins}. 
\end{command}
\begin{command}{\pgfsetmiterjoin}
  Sets the line join to a miter join. See again
  Section~\ref{section-cap-joins}. 
\end{command}
\begin{command}{\pgfsetmiterlimit\marg{miter limit factor}}
  Sets the miter limit to  \meta{miter limit factor}. See again 
  Section~\ref{section-cap-joins}. 
\end{command}

\subsubsection{Graphic Parameter: Dashing}

\begin{command}{\pgfsetdash\marg{list of even length of dimensions}\marg{phase}}
  Sets the dashing of a line. The first entry in the list specifies
  the length of the first solid part of the list. The second entry
  specifies the length of the following gap. Then comes the length of
  the second solid part, following by the length of the second gap,
  and so on. The \meta{phase} specifies where the first solid part
  starts relative to the beginning of the line.

\begin{codeexample}[]
\begin{pgfpicture}
  \pgfsetdash{{0.5cm}{0.5cm}{0.1cm}{0.2cm}}{0cm}
  \pgfpathmoveto{\pgfpoint{0mm}{0mm}}
  \pgfpathlineto{\pgfpoint{2cm}{0mm}}
  \pgfusepath{stroke}
  \pgfsetdash{{0.5cm}{0.5cm}{0.1cm}{0.2cm}}{0.1cm}
  \pgfpathmoveto{\pgfpoint{0mm}{1mm}}
  \pgfpathlineto{\pgfpoint{2cm}{1mm}}
  \pgfusepath{stroke}
  \pgfsetdash{{0.5cm}{0.5cm}{0.1cm}{0.2cm}}{0.2cm}
  \pgfpathmoveto{\pgfpoint{0mm}{2mm}}
  \pgfpathlineto{\pgfpoint{2cm}{2mm}}
  \pgfusepath{stroke}
\end{pgfpicture}
\end{codeexample}

  Use |\pgfsetdash{}{0pt}| to get a solid dashing.
\end{command}

\subsubsection{Graphic Parameter: Stroke Color}

\begin{command}{\pgfsetstrokecolor\marg{color}}
  Sets the color used for stroking lines to \meta{color}, where
  \meta{color} is a \LaTeX\ color like |red| or |black!20!red|. Unlike
  the |\color| command, the effect of this command lasts till the end
  of the current |{pgfscope}| and not till the end of the current
  \TeX\ group.

  The color used for stroking may be different from the color used for
  filling. However, a |\color| command will always ``immediately
  override'' any special settings for the stroke and fill colors.

  In plain \TeX, this command will also work, but the problem of
  \emph{defining} a color arises. After all, plain \TeX\ does not
  provide \LaTeX\ colors. For this reason, \pgfname\ implements a
  minimalistic ``emulation'' of the |\definecolor|, |\colorlet|, and
  |\color| commands. Only gray-scale and rgb colors are supported. For
  most cases this turns out to be enough.

\begin{codeexample}[]
\begin{pgfpicture}
  \pgfsetlinewidth{1pt}
  \color{red}
  \pgfpathcircle{\pgfpoint{0cm}{0cm}}{3mm} \pgfusepath{fill,stroke}
  \pgfsetstrokecolor{black}
  \pgfpathcircle{\pgfpoint{1cm}{0cm}}{3mm} \pgfusepath{fill,stroke}
  \color{red}
  \pgfpathcircle{\pgfpoint{2cm}{0cm}}{3mm} \pgfusepath{fill,stroke}
\end{pgfpicture}
\end{codeexample}
\end{command}

\begin{command}{\pgfsetcolor\marg{color}}
  Sets both the stroke and fill color. The difference to the normal
  |\color| command is that the effect lasts till the end of the
  current |{pgfscope}|, not only till the end of the current \TeX\
  group. 
\end{command}


\subsubsection{Graphic Parameter: Stroke Opacity}

You can set the stroke opacity using |\pgfsetstrokeopacity|. This
command is described in Section~\ref{section-transparency}.

\subsubsection{Graphic Parameter: Arrows}

After a path has been drawn, \pgfname\ can add arrow tips at the
ends. It will only add arrows at the beginning of the first subpath
and at the end of the last subpath. For closed paths the result is not
defined, that is, it may change without notice in the future. 

\begin{command}{\pgfsetarrowsstart\marg{arrow kind}}
  Sets the arrow tip kind used at the start of a (possibly curved)
  path. When this option is used, the line will often be slightly
  shortened to ensure that the tip of the arrow will exactly ``touch''
  the ``real'' start of the line.

  To ``clear'' the start arrow, say |\pgfsetarrowsstart{}|.
\begin{codeexample}[]
\begin{pgfpicture}
  \pgfsetarrowsstart{latex}
  \pgfpathmoveto{\pgfpointorigin}
  \pgfpathlineto{\pgfpoint{1cm}{0cm}}
  \pgfusepath{stroke}
  \pgfsetarrowsstart{to}
  \pgfpathmoveto{\pgfpoint{0cm}{2mm}}
  \pgfpathlineto{\pgfpoint{1cm}{2mm}}
  \pgfusepath{stroke}
\end{pgfpicture}
\end{codeexample}

  The effect of this command persists only till the end of the current
  \TeX\ scope.

  The different possible arrow kinds are explained in
  Section~\ref{section-arrows}.  
\end{command}

\begin{command}{\pgfsetarrowsend\marg{arrow kind}}
  Sets the arrow tip kind used at the end of a path.
\begin{codeexample}[]
\begin{pgfpicture}
  \pgfsetarrowsstart{latex}
  \pgfsetarrowsend{to}
  \pgfpathmoveto{\pgfpointorigin}
  \pgfpathlineto{\pgfpoint{1cm}{0cm}}
  \pgfusepath{stroke}
\end{pgfpicture}
\end{codeexample}
\end{command}

\begin{command}{\pgfsetarrows{\texttt{\char`\{}}\meta{start kind}|-|\meta{end kind}{\texttt{\char`\}}}}
  Sets the start arrow kind to \meta{start kind} and the end kind to
  \meta{end kind}.
\begin{codeexample}[]
\begin{pgfpicture}
  \pgfsetarrows{latex-to}
  \pgfpathmoveto{\pgfpointorigin}
  \pgfpathlineto{\pgfpoint{1cm}{0cm}}
  \pgfusepath{stroke}
\end{pgfpicture}
\end{codeexample}
\end{command}

\begin{command}{\pgfsetshortenstart\marg{dimension}}
  This command will shortened the start of every stroked path by the
  given dimension. This shortening is done in addition to automatic
  shortening done by a start arrow, but it can be used even if no
  start arrow is given.

  This command is useful if you wish arrows or lines to ``stop shortly
  before'' a given point.
\begin{codeexample}[]
\begin{pgfpicture}
  \pgfpathcircle{\pgfpointorigin}{5mm}
  \pgfusepath{stroke}
  \pgfsetarrows{latex-}
  \pgfsetshortenstart{4pt}
  \pgfpathmoveto{\pgfpoint{5mm}{0cm}} % would be on the circle
  \pgfpathlineto{\pgfpoint{2cm}{0cm}}
  \pgfusepath{stroke}
\end{pgfpicture}
\end{codeexample}
\end{command}
  
\begin{command}{\pgfsetshortenend\marg{dimension}}
  Works like |\pgfsetshortenstart|.
\end{command}


\subsubsection{Inner Lines}

When a path is stroked, it is possible to request that it is stroked
twice, the second time with a different line width and a different
color. This is a useful effect for creating ``double'' lines, for
instance by setting the line width to 2pt and stroking a black line
and then setting the inner line width to 1pt and stroking a white
line on the same path as the original path. This results in what looks
like two lines, each of thickness 0.5pt, spaces 1pt apart.

You may wonder why there is direct support for this ``double
stroking'' in the basic layer. After all, this effect is easy to
achieve ``by hand''. The main reason is that arrow tips must be
treated in a special manner when such ``double lines'' are
present. First, the order of actions is important: First, the (thick)
main line should be stroked, then the (thin) inner line, and only then
should the arrow tip be drawn. Second, the way an arrow tip looks
typically depends strongly on the width of the inner line, so the
arrow tip code, which is part of the basic layer, needs access to the
inner line thickness.

Two commands are used to set the inner line width and color.

\begin{command}{\pgfsetinnerlinewidth\marg{dimension}}
  This command sets the width of the inner line. Whenever a path is
  stroked (and only then), it will be stroked normally and, afterward,
  it is stroked once more with the color set to the inner line color
  and the line width set to \meta{dimension}.

  In case arrow tips are added to a path, the path is first stroked
  normally, then the inner line is stroked, and then the arrow tip is
  added. In case the main path is shortened because of the added arrow
  tip, this shortened path is double stroked, not the original path
  (which is exactly what you want).

  When the inner line width is set to 0pt, which is the default, no
  inner line is stroked at all (not even a line of width 0pt). So, in
  order to ``switch off'' double stroking, set \meta{dimension}
  to~|0pt|.

  The setting of the inner line width is local to the current \TeX\
  group and \emph{not} to the current \pgfname\ scope.

  Note that inner lines will \emph{not} be drawn for paths that are
  also used for clipping. However, this may change in the future, so
  you should not depend on this.

\begin{codeexample}[]
\begin{pgfpicture}
  \pgfpathmoveto{\pgfpointorigin}
  \pgfpathlineto{\pgfpoint{1cm}{1cm}}
  \pgfpathlineto{\pgfpoint{1cm}{0cm}}
  \pgfsetlinewidth{2pt}
  \pgfsetinnerlinewidth{1pt}
  \pgfusepath{stroke}
\end{pgfpicture}
\end{codeexample}  
\end{command}


\begin{command}{\pgfsetinnerstrokecolor\marg{color}}
  This command sets the \meta{color} that is to be used when the inner
  line is stroked. The effect of this command is also local to the
  current \TeX\ group.

\begin{codeexample}[]
\begin{pgfpicture}
  \pgfpathmoveto{\pgfpointorigin}
  \pgfpathlineto{\pgfpoint{1cm}{1cm}}
  \pgfpathlineto{\pgfpoint{1cm}{0cm}}
  \pgfsetlinewidth{2pt}
  \pgfsetinnerlinewidth{1pt}
  \pgfsetinnerstrokecolor{red!50}
  \pgfusepath{stroke}
\end{pgfpicture}
\end{codeexample}  
\end{command}


\subsection{Filling a Path}
\label{section-fill}

Filling a path means coloring every interior point of the path with
the current fill color. It is not always obvious whether a point is
``inside'' a  path when the path is self-intersecting and/or consists
or multiple parts. In this case either the nonzero winding number rule
or the even-odd crossing number rule is used to decide, which points
lie ``inside.'' These rules are explained in
Section~\ref{section-rules}. 

\subsubsection{Graphic Parameter: Interior Rule}

You can set which rule is used using the following commands:

\begin{command}{\pgfseteorule}
  Dictates that the even-odd rule is used in subsequent fillings in
  the current \emph{\TeX\ scope}. Thus, for once, the effect of this
  command does not persist past the current \TeX\ scope.

\begin{codeexample}[]
\begin{pgfpicture}
  \pgfseteorule
  \pgfpathcircle{\pgfpoint{0mm}{0cm}}{7mm}
  \pgfpathcircle{\pgfpoint{5mm}{0cm}}{7mm}
  \pgfusepath{fill}
\end{pgfpicture}
\end{codeexample}
\end{command}

\begin{command}{\pgfsetnonzerorule}
  Dictates that the nonzero winding number rule is used in subsequent
  fillings in the current \TeX\ scope. This is the default.

\begin{codeexample}[]
\begin{pgfpicture}
  \pgfsetnonzerorule
  \pgfpathcircle{\pgfpoint{0mm}{0cm}}{7mm}
  \pgfpathcircle{\pgfpoint{5mm}{0cm}}{7mm}
  \pgfusepath{fill}
\end{pgfpicture}
\end{codeexample}
\end{command}

\subsubsection{Graphic Parameter: Filling Color}

\begin{command}{\pgfsetfillcolor\marg{color}}
  Sets the color used for filling paths to \meta{color}. Like the
  stroke color, the effect lasts only till the next use of |\color|. 
\end{command}


\subsubsection{Graphic Parameter: Fill Opacity}

You can set the fill opacity using |\pgfsetfillopacity|. This
command is described in Section~\ref{section-transparency}.

\subsection{Clipping a Path}
\label{section-clip}

When you add the |clip| option, the current path is used for
clipping subsequent drawings. The same rule as for filling is used to
decide whether a point is inside or outside the path, that is, either
the even-odd rule or the nonzero rule.

Clipping never enlarges the clipping area. Thus, when you clip against
a certain path and then clip again against another path, you clip
against the intersection of both.

The only way to enlarge the clipping path is to end the |{pgfscope}|
in which the clipping was done. At the end of a |{pgfscope}| the
clipping path that was in force at the beginning of the scope is
reinstalled. 

\subsection{Using a Path as a Bounding Box}
\label{section-using-bb}

When you add the |use as bounding box| option, the bounding box of the
picture will be enlarged such that the path in encompassed, but any
\emph{subsequent} paths of the current \TeX\ scope will not have any
effect on the size of the bounding box. Typically, you use this
command at the very beginning of a |{pgfpicture}| environment.

\begin{codeexample}[]
Left
\begin{pgfpicture}
  \pgfpathrectangle{\pgfpointorigin}{\pgfpoint{2ex}{1ex}}
  \pgfusepath{use as bounding box} % draws nothing

  \pgfpathcircle{\pgfpointorigin}{2ex}
  \pgfusepath{stroke}
\end{pgfpicture}
right.
\end{codeexample}

