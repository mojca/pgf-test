% Copyright 2010 by Renée Ahrens, Olof Frahm, Jens Kluttig, Matthias Schulz, Stephan Schuster
% Copyright 2011 by Till Tantau
% Copyright 2011 by Jannis Pohlmann
%
% This file may be distributed and/or modified
%
% 1. under the LaTeX Project Public License and/or
% 2. under the GNU Free Documentation License.
%
% See the file doc/generic/pgf/licenses/LICENSE for more details.

\section{Algorithmic Graph Drawing}

{\noindent {\emph{by Ren\'ee Ahrens, Olof-Joachim Frahm, Jens Kluttig,
  Jannis Pohlmann, Matthias Schulz, Stephan Schuster, and Till Tantau}}}

\label{section-library-graphdrawing}


\begin{tikzlibrary}{graphdrawing}
  This package provides capabilities for automatic graph drawing.

  \medskip
  \textbf{Note:} Graph drawing requires that the document is typeset
  using Lua\TeX. This package should work with \LuaTeX\ 0.4 or
  higher, which is included in all current \TeX\ distributions.
\end{tikzlibrary}

\ifluatex\relax\else{LuaTeX is required for setting this manual
  section.}\expandafter\endinput\fi 


\subsection{Overview}

\emph{Alogrithmic graph drawing} (or just \emph{graph drawing} in the
following) means that algorithms are used to decide where the nodes of
a graph are positioned on a page so that the graph ``looks nice.'' The
idea is that you, as human (or you, as a machine, if you happen to be
a machine and happen to be reading this document) just specify which
nodes are present in a graph and which edges are
present. Additionally, you may add some ``hints'' like ``this node
should be near the center'' or ``this edge is pretty important.'' You
do \emph{not} specify where, exactly, the nodes and edges should
be. This is something you leave to a \emph{graph drawing
  algorithm}. The algorithm gets your description of the graph as an
input and then decides where the nodes should go on the page.

Naturally, graph drawing is a bit of a (black?) art. There is no
``perfect'' way of drawing a graph, rather, depending on the
circumstances there are several different ways of drawing the same
graph and often it will just depend on the aesthetic sense of the
reader which layout he or she would prefer. For this reason, there is
a huge number of graph drawing algorithms ``out there'' and there are
scientific conference devoted to such algorithms, where each
year dozens of new algorithms are proposed.

Unlike the rest of \pgfname\ and \tikzname, which is implemented
purely in \TeX, the graph drawing algorithms are simply too complex to
implement them in \TeX. Instead, the programming language Lua is used
by the graph drawing library -- a programming language that has been
integrated into recent versions of \TeX. This means that (a) as a user
of the graph drawing engine you will can run \TeX\ on your documents
in the usual way, no external programs are called since Lua is already
integrated into \TeX\ and (b) it is pretty easy to implement new graph
drawing algorithms for \tikzname\ since Lua can be used and no \TeX\
programming knowledge is needed. 

The graph drawing engine of \tikzname\ provides two main features:
\begin{enumerate}
\item ``Users'' of the graph drawing engine can invoke the graph
  drawing algorithms often by just adding a single option to their
  picture. Here is a typical example, where the |spring layout| option
  invokes a so-called ``spring algorithm'' (what these are will be
  explained later):
\begin{codeexample}[]
\tikz \graph [spring layout]
  { a -> { subgraph K_n [n=4] } -> b };

\tikz \graph [spring electrical layout]
  { a -> { subgraph K_n [n=4] } -> b };
\end{codeexample}
  Here is another example:
\begin{codeexample}[]
\tikz [tree]
  \node {root}
  child { node {a} }
  child { node {a pretty long node name} };
\end{codeexample}
  An a final example:
\begin{codeexample}[]
\tikz [spring layout]
{
  \foreach \i in {1,...,6}
    \node (node \i) [fill=blue!50, text=white, circle] {\i};
    
  \draw (node 1) edge (node 2)
        (node 2) edge (node 3)
        (node 3) edge (node 4)
                 edge (node 5)
                 edge (node 6);
}
\end{codeexample}
  In all of the example, the positions of the nodes have only been
  computed \emph{after} all nodes have been created and the edges have
  been specified. For instance, in the last example without the
  |spring layout| option, all of the nodes would have been placed on
  top of each other.
\item The graph drawing engine is also intended to make is easy to
  implement new graph drawing algorithms. These algorithms can and
  must be implemented in the Lua programming language (which is
  \emph{much} easier to program than \TeX\ itself). The Lua code for a
  graph drawing algorithm gets an object-oriented model of the input
  graph as an input and must just compute the desired new positions of
  the nodes. The complete handling of passing options and
  configurations back-and-forth between the different \tikzname\ and
  \pgfname\ layers is handled by the graph drawing engine.

  The bottom line is that the graph drawing engine makes it easy
  to try out new graph drawing algorithms for medium sized graphs (up
  to a few hundred nodes).
\end{enumerate}


The documentation of the graph drawing engine is structured as
follows: The current section explains the graph drawing engine from
``the user's point of few'' and also describes the basic steps
necessary to implement a new graph drawing algorithm. The libraries
containing the different graph drawing algorithms are documented in
Sections on graph drawing
algorithms. Section~\ref{section-gd-own-algorithm} covers the
internals of how the graph drawing engine works. 




\subsection{Usage}

To use the graph drawing engine, you first need to load some
libraries. First, you should always load the |graphdrawing| library,
which will setup the basic keys. Next, you need to load another
library lie |graphdrawing.tress|. The actual graph drawing
algorithms reside in these libraries. Finally, you may also wish to
load the |graphs| library, but this is only necessary if you wish to
use the |graph| path command, which provides an easy-to-use syntax for
specifying graphs. You can also use the graph drawing engine
independently of the |graphs| library, for instance in conjunction
with the |child| or the |edge| syntax. Here is a typical setup:

\begin{codeexample}[code only]
\usetikzlibrary{graphs,graphdrawing,graphdrawing.trees}  
\end{codeexample}

Having setup things, you must then specify for which scopes the
graph drawing engine should apply an layout algorithm to the nodes in
the scope.

Before we go into the details of how these keys work, a bit of
background knowledge on how the graph drawing engine works will be
most necessary: Using a special internal key called
|graph drawing scope|, which you typically will not call directly,
the graph drawing engine can be switched on for a |{scope}|. When this
happens, a lot of things change inside \pgfname\ and \tikzname\ for
this scope: First, all nodes created inside the scope are not
immediately placed at the position where they were created. Instead,
they are ``spirited away'' to some internal table of the graph drawing
engine. Second, all edges created inside the scope using either the
|graph| command, the |edge| command, or the |child| command are also
``spirited away'' to another internal table. Then, at the end of the
scope, the graph drawing algorithm is started, which has access to
these internal tables of nodes and edge of the graph that has been
specified inside the scope. The algorithm will then compute new,
better, positions for the nodes. Finally, once the positions have been
computed, the graph drawing engine will then retrieve the nodes from
the internal table and place them at the computed positions and it
also retrieves the edges from the internal table and also adds them to
the picture.

While this theory may sound complicated, the use of the graph library
is, fortunately, pretty simple: Just add a key like |tree| or
|spring layout| to a scope and leave out any explicit positioning via
things like |at| -- the positioning will be done automatically by the
graph drawing algorithm.

The keys like |tree| or |spring layout| are explained in more detail
in the chapters on the different libraries. They all internally call
(at least) two keys: |graph drawing scope| and |algorithm|. These
keys are documented in the following, but you typically will not use
them explicitly. In addition to setting up the scope and setting the
correct algorithms, keys like |tree| and |spring layout| also take
some \meta{options} as arguments. These \meta{options} allow you setup
special graph parameters for the algorithm.

\begin{key}{/tikz/graph drawing scope}
  This key can (only) be used as an option when a \tikzname\ scope is
  started. Thus, you can pass it to |\tikz|, to |{tikzpicture}|, to
  |\scoped|, to |{scope}|, to |graph|, and to |{graph}|. For instance,
  the |tree| option (which uses |graph drawing scope| internally) can
  be used in the following ways:
\begin{codeexample}[]
\begin{tikzpicture}[tree]
  \graph {a -> {b,c}};
\end{tikzpicture}
\tikz [tree]
  \graph {a -> {b,c}};
\begin{tikzpicture}
  \draw [help lines] (0,0) grid (3,2);
  
  \scoped [tree]
    \graph {a -> {b,c}};
    
  \begin{scope}[tree, xshift=2cm, rotate=90]
    \graph {a -> {b,c}};
  \end{scope}
\end{tikzpicture}
\tikz \graph[tree] {a -> {b,c}};
\tikz \path graph[tree] {a -> {b,c}};
\end{codeexample}

  You can \emph{not} use the |graph drawing scope| key with a single
  node or on a path. In particular, to typeset a tree given in the
  |child| syntax somewhere inside a |{tikzpicture}|, you must prefix
  it with the |\scoped| command:
\begin{codeexample}[]
\begin{tikzpicture}
  \scoped [tree]
    \node {root}
    child { node {left child} }
    child { node {right child} };
\end{tikzpicture}
\end{codeexample}
  Naturally, the above could have been written more succinctly as
\begin{codeexample}[]
\tikz [tree]
  \node {root}
  child { node {left child} }
  child { node {right child} };
\end{codeexample}
  Or even more succinctly:
\begin{codeexample}[]
\tikz \graph [tree] { root -- {left child, right child} };
\end{codeexample}

  In detail, adding the |graph drawing scope| command to a scope has
  the following effects:
  \begin{itemize}
  \item The basic layer is informed, using the
    |execute at begin scope| key, that the current scope will contain
    nodes that should be positioned by a graph drawing engine. Which
    algorithm is used depends on the value of the |algorithm| key.
  \item If the |graphs| library has been loaded, the default
    positioning mechanisms of this library are switched off, leaving
    the positioning to the graph drawing engine. Also, when an edge is
    created by the |graphs| library, this is signalled to the graph
    drawing library. (To be more precise: The keys |new ->| and so on
    are redefined so that they call |\pgfgdedge| instead of creating
    an edge.
  \item The |edge| path command is modified so that it also calls
    |\pgfgdedge| instead of immediately creating any edges.
  \item The |edge from parent| path command is modified so that is
    also calls |\pgfgdedge|.
  \end{itemize}

  The bottom line of the above is the following: Inside a scope for
  which the |graph drawing scope| key has been set either explicitly
  or implicitly, you can:
  \begin{itemize}
  \item Create nodes in the usual way. The nodes will be created
    completely, but then tucked away in an internal table. This means
    that all of \tikzname's options for nodes can be applied. You can
    also name a node and reference it later.
  \item Create edges using either the syntax of the |graph| command
    (using |--|, |<-|, |->|, or |<->|), or using the |edge| command,
    or using the |child| command. These edges will, however, not be
    created immediately. Instead, the basic layer's command
    |\pgfgdedge| will be called, which stores ``all the information
    concerning the edge.'' The actual drawing of the edge will only
    happen after all nodes have been positioned.
  \item Most of the keys that can be passed to an edge will work as
    expected. In particular, you can add labels to edges using the
    usual |node| syntax for edges.
  \end{itemize}
  Here are some things that will \emph{not} work:
  \begin{itemize}
  \item When a node has a |label| or a |pin| attached to it, the label
    or the pin are also nodes  that are created immediately and, thus,
    subject to positioning by the graph drawing algorithm. However,
    this is ``typically not what we want'' and these nodes will be
    placed at the wrong positions. 

    We hope to solve this problem in the future, for the time being
    you need to use |late options| to add the labels after the nodes
    have been positioned. (Yes, we know this is not convenient, but
    neither is this trivial to fix.)

    Note that for edges, nodes positioned ``on'' these edges are not a
    problem since these nodes get only created when the edge is
    created, which is after the algorithm has already run.
  \item Only edges created using the graph syntax, the |edge| command,
    or the |child| command will correctly pass their connection
    information to the basic layer. When you write |\draw (a)--(b);|
    inside a graph drawing scope, where |a| and |b| are nodes that
    have been created inside the scope, you will get an error
    message / things will look wrong. The reason is that the usual
    |--| is not ``caught'' by the graph drawing engine and, thus,
    tries to immediately connect two nodes that do not yet exist
    (except inside some internal table).
  \item The options of edges are executed twice: Once when the edge is
    ``examined'' by the |\pgfgdedge| command and then once more when
    the edge is actually created. Fortunately, in almost all cases,
    this will not be a problem.
  \end{itemize}
\end{key}

\begin{key}{/graph drawing/algorithm=\meta{algorithm's name}}
  \label{section-gd-algorithm-key}%
  This key specifies which algorithm should be used for typesetting a
  graph. The names of these algorithm's are often a bit cryptic (like
  |AhrensFKSS2011 tree|), which is why you typically do not call this
  key directly. Instead, styles with more easy-to-remember names
  internally set this key.

  Setting this key has the following effects: When a scope with the
  |graph drawing scope| command is started, the current value of
  \meta{algorithm's name} is examined. Then, the graph drawing engine
  tries to find a Lua function called
  |drawGraphAlgorithm_|\meta{algorithm's name} inside the
  |pgf.graphdrawing| Lua mode. This function will be used to
  perform the typesetting (what this function should do is documented
  in detail in Section~\ref{section-gd-own-algorithm}). If this
  function does not exist, the engine tries to load the file 
  called |pgfgd-algorithm-|\meta{algorithm's name}|.lua| and they,
  again, tries to call the function. Thus, any \meta{algorithm's name}
  mentioned inside a document must either have already been defined in
  some Lua file loaded by some library or it must reside in a file
  with the corresponding name.
  
  In case the \meta{algorithm's name} contains characters that cannot
  be used in file names or in function names, substitutions are
  performed. In particular, all occurrences of a space are replaced by
  hyphens for file names and by underscores for function names.

  Here is an example where we switch on the graph drawing engine
  explicitly and explicitly select an algorithm:
\begin{codeexample}[]
\tikz [graph drawing scope,
       /graph drawing/algorithm=AhrensFKSS2011 tree]
  \graph { a <-> {b, c} };  
\end{codeexample}
\end{key}


\subsection{Common Graph Parameters}

Graph drawing algorithms can typically be configured in some way. For
instance, for a graph drawing algorithm that visualizes its nodes as a
tree, it will typically be useful when the user can change the
so-called \emph{level distance} and the \emph{sibling distance}. For
other algorithms, like force-based algorithms, a large number of
parameters influence the way the algorithms work.

Options that influence graph drawing algorithms will be called
\emph{graph drawing parameters} in the following. There are three kinds of
graph drawing parameters:
\begin{itemize}
\item Graph parameters,
\item node parameters, and
\item edge parameters.
\end{itemize}
A graph drawing graph parameter influences the layout of the whole
graph. A graph drawing node parameter is an option that is attached to
a single node and should only have a direct influence on this node
(like ``place this node exactly at this position, no matter what''). A
graph drawing edge parameter in important for a single edge (like
``this edge must be exactly |2cm| long'').

A graph drawing algorithm may or may not take the different graph
parameters into account. After all, these options may even outright
contradict each other, so an algorithm can only try to ``do its
best''.

While many graph parameters are very specific to a single algorithm, a
number of graph parameters will be important for many algorithms. Such
graph parameters are called \emph{common} graph parameters, the most
important of which are documented in the following. The common graph
parameters can be used like any normal \tikzname\ option. In contrast,
specific options for algorithms must be passed to the key that
installs the algorithm. For example, the orientation of a graph
is setup with the common key |orient|, which is given alongside a key
like |spring layout|:

\begin{codeexample}[]
\tikz \graph [spring layout, orient=1|2] { 1--2--3--1 };  
\end{codeexample}

In contrast, the very specific option |iterations| must be
passed to the |spring layout| key:

\begin{codeexample}[]
\tikz \graph [spring layout={iterations=3}] { 1--2--3--1 };  
\end{codeexample}



\subsubsection{Orienting a Graph}

\label{subsection-library-graphdrawing-standard-orientation}

The drawing computed for a graph may be pleasing in general but some
algorithms like the |spring| algorithm show the tendency to generate
drawings with an unexpected orientation. Also, if several graphs are 
defined in a single picture, then these will possibly overlap.
As a consequence, a postprocessing step is sometimes necessary in order
to adjust the orientation of the graph drawing. This can be achieved in
a relatively simple way using the following common options:


\begin{key}{/graph drawing/orient=\meta{orientation}}
  \keyalias{tikz}\keyalias{tikz/graphs}
  With graph drawing algorithms it makes sense for, the |graphdrawing|
  library will, in a postprocessing step, attempt to detect the 
  principal axis of the graph automatically and will adjust its 
  orientation so that it is parallel to the $x$ axis.

  This behavior may be altered by setting |orientation| using one of
  the following three notation styles for the \meta{orientation}:
  \begin{enumerate}
  \item  \meta{first axis node}|-|\meta{second axis node} 

    Specifies the axis of the graph via the names of two nodes. The
    drawing is rotated so that this axis is horizontal,
    with \meta{first axis node} being located left of \meta{second axis node}:
\begin{codeexample}[]
\tikz \graph [spring layout] { a -- b -- c -- a };
\tikz \graph [spring layout,orient=a-b] { a -- b -- c -- a };
\tikz \graph [spring layout,orient=b-a] { a -- b -- c -- a };
\tikz \graph [spring layout,orient=1-2] { subgraph K_n[n=5] };
\tikz \graph [spring layout,orient=2-1] { subgraph K_n[n=5] };
\end{codeexample}
  
  \item \meta{first axis node}\verb!|!\meta{second axis node}
    
    Similar to the previous case, only the axis is now vertical with 
    \meta{first axis node} being located above \meta{second axis node}:
\begin{codeexample}[]
\tikz \graph [spring layout,orient=a|b] { a -- b -- c -- a };
\tikz \graph [spring layout,orient=b|a] { a -- b -- c -- a };
\tikz \graph [spring layout,orient=1|4] { subgraph K_n[n=5] };
\tikz \graph [spring layout,orient=4|1] { subgraph K_n[n=5] };
\end{codeexample}
  
  \item \meta{first axis node}|:|\meta{angle}|:|\meta{second axis
      node} 

    This is the most generic variant of the three notation styles. In
    addition to specifying the axis of the graph drawing by means of two node names,
    it also defines the angle to the $x$ axis by which the final drawing 
    is to be rotated. The specification of the \meta{angle} is done in the
    same way as everywhere else in \tikzname, where positive values imply
    counter-clockwise rotation and negative values imply clockwise
    rotation:
\begin{codeexample}[]
\tikz \graph [spring layout,orient=a:90:b]  { a -- b -- c -- a };
\tikz \graph [spring layout,orient=a:45:b]  { a -- b -- c -- a };
\tikz \graph [spring layout,orient=a:0:b]   { a -- b -- c -- a };
\tikz \graph [spring layout,orient=a:-45:b] { a -- b -- c -- a };
\tikz \graph [spring layout,orient=a:-90:b] { a -- b -- c -- a };
\end{codeexample}
  \end{enumerate}
\end{key}

\begin{key}{/graph drawing/orient'=\meta{orientation}}
  \keyalias{tikz}\keyalias{tikz/graphs}
  Does the same as |orient| except that the nodes are flipped over the
  principal axis.
\begin{codeexample}[]
\tikz \graph [spring layout,orient=a-b]  { a -- b -- c -- a };
\tikz \graph [spring layout,orient'=a-b] { a -- b -- c -- a };
\end{codeexample}
\end{key}


\begin{key}{/graph drawing/desired at=\marg{coordinate}}
  \keyalias{tikz}\keyalias{tikz/graphs}
  Defines the desired position for a node. In a postprocessing step, the
  |graphdrawing| library will attempt to move the entire graph so that
  the first node with a |desired at| option is moved to its desired 
  location.
  \begin{codeexample}[]
% TODO
  \end{codeexample}
\end{key}


\subsubsection{Packing of Connected Components}

Graphs may be composed of subgraphs or \emph{components} that are not
connected to each other. In order to draw these nicely, the 
|graphdrawing| library splits them up into separate graphs, computes
their layouts with the same graph drawing algorithm independently and,
in a postprocessing step, arranges them in a non-uniform grid in the 
final drawing. This is called \emph{component packing}.

The following options can be used to configure the order and placement
strategy during component packing.

\begin{key}{/graph drawing/component packing=\marg{options}}
  \keyalias{tikz}\keyalias{tikz/graphs}
  Executes the \meta{options} with the path prefix 
  |/tikz/component packing|.
  
  Defines how to arrange the connected components of the graph after 
  their individual drawings have been computed.
\end{key}

To be written...

%\begin{key}{/tikz/component packing/layered=\opt{\meta{boolean}} (default true, initially true)}
%  |layered| arranges the different components in a non-uniform grid
%  starting in the top left corner. Components are placed in these 
%  layers in descending order of their size.
%  \begin{codeexample}[]
%\tikz \graph [experimental layout,component packing={layered}] {
%  a -- b -- c -- a,
%  d -- e -- d,
%  g
%};
%  \end{codeexample}
%\end{key}
%
%\begin{key}{/tikz/component packing/centered=\opt{\meta{boolean}} (default true, initially false)}
%  If set to |true|, arranges the different components clockwise in a 
%  non-uniform grid starting at the center. Components are placed in 
%  descending order of their size.
%  \begin{codeexample}[]
%\tikz \graph [experimental layout,component packing={centered}] {
%  a -- b -- c -- a,
%  d -- e -- d,
%  g
%};
%  \end{codeexample}
%\end{key}
%
%\begin{key}{/tikz/component packing/padding=\meta{dimension} (initially 0pt)}
%  Defines how much padding is used to separate the connected 
%  components.
%  \begin{codeexample}[]
%\tikz \graph [experimental layout,component packing={centered,padding=10pt}] {
%  a -- b -- c -- a,
%  d -- e -- d,
%  g
%};
%  \end{codeexample}
%\end{key}


\subsection{Implementing Graph Drawing Algorithms}

\label{section-gd-own-algorithm}
\label{section-library-graphdrawing-ownAlgorithm}

This section presents a simple example of how a graph drawing
algorithm can be implemented. A much more detailed explanation of how
new graph drawing algorithms can be integrated and configured in given
in Section~\ref{section-gd-implementing-algorithms}.

As explained for the |algorithm| key, the Lua function implementing
the graph drawing algorithm must be called
|drawGraphAlgorithm_|\meta{algorithm name} and be placed in the file
|pgfgd-algorithm-|\meta{algorithm name}. The function
will be called with an object representing the graph as single
argument and should modify the coordinates of the nodes of this 
graph. For example, the following code fragment implements a trivial
graph drawing algorithm that just places all nodes on a fixed-size
circle.  It is accessed with the name 
|simple demo|, or |simple-demo| so both the file- and function name 
agree on that.

\pgfgddeclareforwardedkeys{/graph drawing}{
  radius/.graph parameter=evaluate math expression,
  radius/.parameter initial=1cm,
  node radius/.node parameter=evaluate math expression
}

\begin{codeexample}[code only]
-- File pgfgd-algorithm-simple-demo.lua
  
pgf.module("pgf.graphdrawing")

--- A trivial simple node placing algorithm for demonstration purposes.
-- All nodes are positioned on a fixed-size circle.
function drawGraphAlgorithm_simple_demo(graph)
   local radius = 28.908  -- this is 1cm in points

   -- count nodes
   local nodeCount = table.count_pairs(graph.nodes)

   local alpha = (2 * math.pi) / nodeCount
   local i = 0
   for node in table.value_iter(graph.nodes) do
      -- the interesting part...
      node.pos:set{x = radius * math.cos(i * alpha)}
      node.pos:set{y = radius * math.sin(i * alpha)}
      i = i + 1
   end
end
\end{codeexample}

The algorithm computes a circular layout like in the following.

\begin{codeexample}[]
\tikz [graph drawing scope, /graph drawing/algorithm=simple demo]
  \graph { f -> c -> e -> a -> {b -> {c, d, f}, e -> b}};
\end{codeexample}

For details on how to make things like |radius| configurable and how
to setup keys so that users can just write
|\tikz[circular layout] ...|, please see Section~\ref{section-gd-implementing-algorithms}.


\endinput

