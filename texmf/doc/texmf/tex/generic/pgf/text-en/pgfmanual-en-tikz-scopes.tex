% Copyright 2003 by Till Tantau <tantau@cs.tu-berlin.de>.
%
% This program can be redistributed and/or modified under the terms
% of the LaTeX Project Public License Distributed from CTAN
% archives in directory macros/latex/base/lppl.txt.


\section[Hierarchical Structures: Package, Environments, Scopes, and Styles]
{Hierarchical Structures:\\
  Package, Environments, Scopes, and Styles}

The present section explains how your files should be structured when
you use \tikzname. On the top level, you need to include the |tikz|
package. In the main text, each graphic needs to be put in a
|{tikzpicture}| environment. Inside these environments, you can use
|{scope}| environments to create internal groups. Inside the scopes
you use |\path| commands to actually draw something. On all levels
(except for the package level), graphic options can be given that
apply to everything within the environment.



\subsection{Loading the Package and the Libraries}

\begin{package}{tikz}
  This package does not have any options.
  
  This will automatically load the \pgfname\ package and some other
  stuff that \tikzname\ needs (like the |xkeyval| package).

  \pgfname\ needs to know what \TeX\ driver you are intending to use. In
  most cases \pgfname\ is clever enough to determine the correct driver
  for you; this is true in particular if you \LaTeX. Currently, the only
  situation where \pgfname\ cannot know the driver ``by itself'' is when
  you use plain \TeX\ or Con\TeX t together with |dvipdfm|. In this case,
  you have to write |\def\pgfsysdriver{pgfsys-dvipdfm.def}|
  \emph{before} you input |tikz.tex|. 
\end{package}


\begin{command}{\usetikzlibrary\marg{list of libraries}}
  Once \tikzname\ has been loaded, you can use this command to load
  further libraries. The list of libraries should contain the names of
  libraries separated by commas. Instead of curly braces, you can also
  use square brackets, which is something Con\TeX t users will
  like. If you try to load a library a second time, nothing will
  happen. 

  \example |\usetikzlibrary{arrows}|

  The above command will load a whole bunch of extra arrow tip
  definitions.

  What this command does is to load the file
  |pgflibrarytikz|\meta{library}|.code.tex| for each \meta{library} in
  the \meta{list of libraries}. Thus, to write your own library file,
  all you need to do is to place a file of the appropriate name
  somewhere where \TeX\ can find it. \LaTeX, plain \TeX, and Con\TeX t
  users can then use your library.
\end{command}



\subsection{Creating a Picture}

\subsubsection{Creating a Picture Using an Environment}

The ``outermost'' scope of \tikzname\ is the |{tikzpicture}| 
environment. You may give drawing commands only inside this
environment, giving them outside (as is possible in many other
packages) will result in chaos.

In \tikzname, the way graphics are rendered is strongly influenced by
graphic options. For example, there is an option for setting the color used
for drawing, another for setting the color used for filling, and also
more obscure ones like the option  for setting the prefix used in the
filenames of temporary files written while plotting functions using an
external program. The graphic options are nearly always specified in a
so-called key-value style. (The ``nearly always'' refers to the name
of nodes, which can also be specified differently.) All graphic
options are local to the |{tikzpicture}| to which they apply.

\begin{environment}{{tikzpicture}\opt{\oarg{options}}}
  All \tikzname\ commands should be given inside this
  environment, except for the |\tikzstyle| command. Unlike other
  packages, it is not possible to use, say, |\pgfpathmoveto| outside
  this environment and doing so will result in chaos. For \tikzname,
  commands like |\path| are only defined inside this environment, so
  there is little chance that you will do something wrong here. 

  When this environment is encountered, the \meta{options} are
  parsed. All options given here will apply to the whole
  picture. 

  Next, the contents of the environment is processed and the graphic
  commands therein are put into a box. Non-graphic text is suppressed
  as well as possible, but non-\pgfname\ commands inside a
  |{tikzpicture}| environment should not produce any ``output'' since
  this may totally scramble the positioning system of the backend
  drivers. The suppressing of normal text, by the way, is done by
  temporarily switching the font to |\nullfont|. You can, however,
  ``escape back'' to normal \TeX\ typesetting. This happens, for
  example, when you specify a node.

  At the end of the environment, \pgfname\ tries to make a good guess
  at a good guess at the bounding box of the graphic and
  then resizes the box such that the box has this size. To ``make its
  guess,'' everytime \pgfname\ encounters a coordinate, it updates the
  bound box's size such that it encompasses all these
  coordinates. This will usually give a good 
  approximation at the bounding box, but will not always be
  accurate. First, the line thickness is not taken into
  account. Second, controls points of a curve often lie far
  ``outside'' the curve and make the bounding box too large. In this
  case, you should use the |[use as bounding box]| option.

  The following option influences the baseline of the resulting
  picture:
  \begin{itemize}
    \itemoption{baseline}\opt{|=|\meta{dimension or coordinate}}
    Normally, the lower end of the picture is put on the baseline of
    the surrounding text. For example, when you give the code
    |\tikz\draw(0,0)circle(.5ex);|, \pgfname\ will find out that the
    lower end of the picture is at $-.5\mathrm{ex}$ and that the upper
    end is at $.5\mathrm{ex}$. Then, the lower end will be put on the
    baseline, resulting in the following: \tikz\draw(0,0)circle(.5ex);.

    Using this option, you can specify that the picture should be
    raised or lowered such that the height \meta{dimension} is on the
    baseline. For example, |tikz[baseline=0pt]\draw(0,0)circle(.5ex);|
    yields \tikz[baseline=0pt]\draw(0,0)circle(.5ex); since, now, the
    baseline is on the height of the $x$-axis. If you omit the
    \meta{dimensions}, |0pt| is assumed as default.

    This options is often useful for ``inlined'' graphics as in
\begin{codeexample}[]
$A \mathbin{\tikz[baseline] \draw[->>] (0pt,.5ex) -- (3ex,.5ex);} B$
\end{codeexample}

    Instead of a \meta{dimension} you can also provide a coordinate in
    parantheses. Then the effect is to put the baseline on the
    $y$-coordinate that the give \meta{coordinate} has \emph{at the
      end of the picture}. This means that, at the end of the picture,
    the \meta{coordinate} is evaluated and then the baseline is set
    to the $y$-coordinate of the resulting point. This makes it easy
    to reference the $y$-coordinate of, say, the base line of nodes.
\begin{codeexample}[]
Hello
\tikz[baseline=(X.base)]
  \node [cross out,draw] (X) {world.};
\end{codeexample}

\begin{codeexample}[]
Top align:
\tikz[baseline=(current bounding box.north)]
  \draw (0,0) rectangle (1cm,1ex);
\end{codeexample}

    \itemoption{execute at begin picture}|=|\meta{code}
    This option can be used to install some code that will be executed
    at the beginning of the picture. This option must be
    given in the argument of the |{tikzpicture}| environment itself
    since this option will not have an effect otherwise. After all,
    the picture has already ``started'' later on.

    This option is mainly used in styles like the |every picture|
    style to execute certain code at the start  of a picture.

    \itemoption{execute at end picture}|=|\meta{code}
    This option installs some code that will be executed
    at the end of the picture. Using this option multiple times will
    cause the code to accumulate. This option must also be given in
    the optional argument of the |{tikzpicture}| environment.

\begin{codeexample}[]
\begin{tikzpicture}[execute at end picture=%
  {
    \begin{pgfonlayer}{background}
      \path[fill=yellow,rounded corners]
        (current bounding box.south west) rectangle
        (current bounding box.north east);
    \end{pgfonlayer}
  }]
  \node at (0,0) {X};
  \node at (2,1) {Y};
\end{tikzpicture}
\end{codeexample}
  \end{itemize}
  
  All options ``end'' at the end of the picture. To set an option
  ``globally'' you can use the following style:
  \begin{itemize}
    \itemstyle{every picture}
    This style is installed at the beginning of each picture.
\begin{codeexample}[code only]
\tikzstyle{every picture}=[semithick]
\end{codeexample}
  \end{itemize}
\end{environment}

In other \TeX\ format, you should use instead the following commands:

\begin{plainenvironment}{{tikzpicture}\opt{\oarg{options}}}
  This is the plain \TeX\ version of the environment.
\end{plainenvironment}

\begin{contextenvironment}{{tikzpicture}\opt{\oarg{options}}}
  This is the Con\TeX t version of the environment.
\end{contextenvironment}


\subsubsection{Creating a Picture Using a Command}

The following two commands are used for ``small'' graphics.

\begin{command}{\tikz\opt{\oarg{options}}\marg{commands}}
  This command places the \meta{commands} inside a
  |{tikzpicture}| environment and adds a semicolon at the end. This is
  just a convenience.

  The \meta{commands} may not contain a paragraph (an empty
  line). This is a precaution to ensure that users really use this
  command only for small graphics.

  \example |\tikz{\draw (0,0) rectangle (2ex,1ex)}| yields
  \tikz{\draw (0,0) rectangle (2ex,1ex);} 
\end{command}


\begin{command}{\tikz\opt{\oarg{options}}\meta{text}|;|}
  If the \meta{text} does not start with an opening brace, the end of
  the \meta{text} is the next semicolon that is encountered.

  \example |\tikz \draw (0,0) rectangle (2ex,1ex);| yields
  \tikz \draw (0,0) rectangle (2ex,1ex);
\end{command}



\subsubsection{Adding a Background}

By default, pictures do not have any background, that is, they are
``transparent'' on all parts on which you do not draw
anything. You may instead wish to have a colored background behind
your picture or a black frame around it or lines above and below it or
some other kind of decoration.

Since backgrounds are often not needed at all, the definition of
styles for adding backgrounds has been put in the library package
|pgflibrarytikzbackgrounds|. This package is documented in
Section~\ref{section-tikz-backgrounds}. 


\subsection{Using Scopes to Structure a Picture}

Inside a |{tikzpicture}| environment you can create scopes
using the |{scope}| environment. This environment is available only
inside the |{tikzpicture}| environment, so once more, there is little
chance of doing anything wrong.

\begin{environment}{{scope}\opt{\oarg{options}}}
  All \meta{options} are local to the \meta{environment
  contents}. Furthermore, the clipping path is also local to the
  environment, that is, any clipping done inside the environment
  ``ends'' at its end.

\begin{codeexample}[]
\begin{tikzpicture}
  \begin{scope}[red]
    \draw (0mm,0mm) -- (10mm,0mm);
    \draw (0mm,1mm) -- (10mm,1mm);
  \end{scope}
  \draw (0mm,2mm) -- (10mm,2mm);
  \begin{scope}[green]
    \draw (0mm,3mm) -- (10mm,3mm);
    \draw (0mm,4mm) -- (10mm,4mm);
    \draw[blue] (0mm,5mm) -- (10mm,5mm);
  \end{scope}
\end{tikzpicture}
\end{codeexample}
  
  The following style influences scopes:
  \begin{itemize}
    \itemstyle{every scope}
    This style is installed at the beginning of every scope. I do not
    know really know what this might be good for, but who knows?
  \end{itemize}

  The following options are useful for scopes:
  \begin{itemize}
    \itemoption{execute at begin scope}|=|\meta{code}
    This option install some code that will be executed
    at the beginning of the scope. This option must be
    given in the argument of the |{scope}| environment.

    The effect applies only to the current scope, not to subscopes.

    \itemoption{execute at end scope}|=|\meta{code}
    This option installs some code that will be executed
    at the end of the  current scope. Using this option multiple times
    will  cause the code to accumulate. This option must also be given
    in the optional argument of the |{scope}| environment. 

    Again, the effect applies only to the current scope, not to subscopes.
  \end{itemize}
\end{environment}

\begin{plainenvironment}{{scope}\opt{\oarg{options}}}
  Plain \TeX\ version of the environment.
\end{plainenvironment}

\begin{contextenvironment}{{scope}\opt{\oarg{options}}}
  Con\TeX t version of the environment.
\end{contextenvironment}



\subsection{Using Scopes Inside Paths}

The |\path| command, which is described in much more detail in later
sections, also takes graphic options. These options are local to the
path. Furthermore, it is possible to create local scopes within a
path simply by using curly braces as in
\begin{codeexample}[]
\tikz \draw (0,0) -- (1,1)
           {[rounded corners] -- (2,0) -- (3,1)}
           -- (3,0) -- (2,1);
\end{codeexample}

Note that many options apply only to the path as a whole and cannot be
scoped in this way. For example, it is not possible to scope the
|color| of the path. See the explanations in the section on paths for
more details.

Finally, certain elements that you specify in the argument to the
|\path| command also take local options. For example, a node
specification takes options. In this case, the options apply only to
the node, not to the surrounding path.



\subsection{Using Styles to Manage How Pictures Look}

There is a way of organizing sets of graphic options ``orthogonally''
to the normal scoping mechanism. For example, you might wish all your
``help lines'' to be drawn in a certain way like, say, gray and thin
(do \emph{not} dash them, that distracts). For this, you can use
\emph{styles}.

A style is simply a set of graphic options that is predefined at some
point. Once a style has been defined, it can be used anywhere using
the |style| option:

\begin{itemize}
  \itemoption{style}|=|\meta{style name}
  invokes all options that are currently set in the \meta{style
    name}. An example of a style is the predefined |help lines| style,
  which you should use for lines in the background like grid lines or
  construction lines. You can easily define new styles and modify
  existing ones.
\begin{codeexample}[]
\begin{tikzpicture}
  \draw                   (0,0) grid +(2,2);
  \draw[style=help lines] (2,0) grid +(2,2);
\end{tikzpicture}
\end{codeexample}
\end{itemize}


\begin{command}{\tikzstyle\meta{style name}\opt{|+|}|=[|\meta{options}|]|}
  This command defines the style \meta{style name}. Whenever it is
  used using the |style=|\meta{style name} command, the \meta{options}
  will be invoked. It is permissible that a style invokes another
  style using the |style=| command inside the \meta{options}, which
  allows you to build hierarchies of styles. Naturally, you should
  \emph{not} create cyclic dependencies.

  If the style already has a predefined meaning, it will
  unceremoniously be redefined without a warning.
\begin{codeexample}[]
\tikzstyle{help lines}=[blue!50,very thin]
\begin{tikzpicture}
  \draw                   (0,0) grid +(2,2);
  \draw[style=help lines] (2,0) grid +(2,2);
\end{tikzpicture}
\end{codeexample}

  If the optional |+| is given, the options are \emph{added} to the
  existing definition:
\begin{codeexample}[]
\tikzstyle{help lines}+=[dashed]% aaarghhh!!!
\begin{tikzpicture}
  \draw                   (0,0) grid +(2,2);
  \draw[style=help lines] (2,0) grid +(2,2);
\end{tikzpicture}
\end{codeexample}
\end{command}

It is also possible to set a style using an option:
\begin{itemize}
  \itemoption{set style}|={|\marg{style name}\opt{|+|}|=[|\meta{options}|]}|
  This option has the same effect as saying |\tikzstyle| before the
  argument of the option. 
\begin{codeexample}[]
\begin{tikzpicture}[set style={{help lines}+=[dashed]}]
  \draw                   (0,0) grid +(2,2);
  \draw[style=help lines] (2,0) grid +(2,2);
\end{tikzpicture}
\end{codeexample}
\end{itemize}


