% Copyright 2008 by Till Tantau
%
% This file may be distributed and/or modified
%
% 1. under the LaTeX Project Public License and/or
% 2. under the GNU Free Documentation License.
%
% See the file doc/generic/pgf/licenses/LICENSE for more details.


\section{Chains}

\label{section-chains}

\begin{tikzlibrary}{chains}
  This library defines options for creating chains.
\end{tikzlibrary}


\subsection{Overview}

Chains provide a convenient way of creating sequences of nodes that
are connected to each other in a ``chainwise fashion.'' More
generally, they can be used to position nodes of a branching network
in a systematic manner. An alternative way is to use matrices, see
Section~\ref{section-matrices}, but chains can also be used 
to describe the connections between nodes that have already been
connected using, say, a matrix. 

The basic idea behind a chain is the following: A chain is a
sequence of nodes where each node is placed relative to the
preceding one. Typically, each node is joined to the preceding one
using an edge.

Chains can branch and join once more, which allows you to create
sophisticated networks.


\subsection{Creating a Chain}

Typically, you construct one chain at a time, but it is
permissible to have construct multiple chains simultaneously. In this
case, the chain must be named differently and you must specify for
each node which chain it belongs to.

The first step toward creating a chain is to use the |start chain|
key.

\begin{key}{/tikz/start chain=\opt{\meta{chain name}}\opt{\meta{direction}}}
  This key should, but need not, be given as an option to a scope
  enclosing all nodes of the chain. Typically, this will be a |scope|
  or the whole |tikzpicture|, but it might just be a path on which all
  nodes are placed.

  The key starts a chain named \meta{chain name}; if the chain already
  exists, it is ``restarted.'' If no \meta{chain name} is given, the
  default value |chain| will be used instead. Chains are started
  globally, that is, it is possible to continue a chain after a scope
  has ended.

  The \meta{direction} is used to determine the placement rule for
  nodes on the chain. If it is omitted, the current value of the
  following key is used:
  \begin{key}{/tikz/chain default direction=\meta{direction}
      (initially going right)}
    This \meta{direction} is used in a |chain| option, if no other
    \meta{direction} is specified.
  \end{key}

  The \meta{direction} can have two different forms:
  \declare{|going |\meta{options}} or
  \declare{|placed |\meta{options}}. The effect of these rules will be
  explained in the description of the |on chain| option. Right now,
  just remember that the \meta{direction} you provide with the |chain|
  option applies to the whole chain.
  
  This option also sets the following key to \meta{chain name} (or to
  the default name |chain|, if no \meta{chain name} is given), but you
  can also set this key to another value later on yourself:
  \begin{key}{/tikz/chain=\meta{name}}
    This key (locally) keeps track of the current chain that is being
    constructed. This name will be used by other options like
    |on chain| if no chain name is given.

    So: If you do not provide a \meta{chain name} for the
    |start chain| option, |chain| is used as the name, while if you do not
    provide a \meta{chain name} for the |on chain| command, the value
    of the |chain| key is used.
  \end{key}
  Other than this, this key has no further effect. In particular, to
  place nodes on the chain, you must use the |on chain| option,
  described next.

\begin{codeexample}[]
\begin{tikzpicture}[start chain]
  % The chain is called just "chain"
  \node [on chain] {A};
  \node [on chain] {B};
  \node [on chain] {C};
\end{tikzpicture}
\end{codeexample}

\begin{codeexample}[]
\begin{tikzpicture}
  % Same as above, using the scope shorthand
  { [start chain]
    \node [on chain] {A};
    \node [on chain] {B};
    \node [on chain] {C};
  }
\end{tikzpicture}
\end{codeexample}

\begin{codeexample}[]
\begin{tikzpicture}[start chain=1 going right,
                    start chain=2 going below,
                    node distance=5mm,
                    every node/.style=draw]
  \node [on chain=1] {A};
  \node [on chain=1] {B};
  \node [on chain=1] {C};

  \node [on chain=2] at (0,-1) {0};
  \node [on chain=2] {1};
  \node [on chain=2] {2};
  
  \node [on chain=1] {D}; 
\end{tikzpicture}
\end{codeexample}
\end{key}


\subsection{Nodes on a Chain}

\begin{key}{/tikz/on chain=\opt{\meta{chain name}}\opt{\meta{direction}}}
  This key should be given as an option to a node. When the option is
  used, the \meta{chain name} must previously have been declared for
  the current scope using the |start chain| option. If \meta{chain name} is
  the empty string, the current value of the key |chain| is used. 

  The \meta{direction} part is optional. If present it overrides the
  \meta{direction} used for this node, otherwise the \meta{direction}
  that was given to the original |start chain| option is used.

  The effects of this option are the following:
  \begin{enumerate}
  \item An internal counter (there is one, local, counter
    for each chain) is increased. This counter reflects the current
    number of the node in the chain, where the first node is node 1,
    the second is node 2, and so on.

    This value of this internal counter is globally stored in the
    macro \declare{|\tikzchaincount|}.
  \item If the node does not yet have a name, (having been given using
    the |at| option or the at-syntax), the name of the node is set to
    \meta{chain name}|-|\meta{value of the internal chain
      counter}. For instance, if the chain is called |nums|, the first
    node would be named |nums-1|, the second |nums-2|, and so on. For
    the default chain name |chain|, the first node is named |chain-1|,
    the second |chain-2|, and so on.
  \item Independently of whether the name has been provided
    automatically or via the |at| option, the name of the node is
    globally stored in the macro \declare{|\tikzchaincurrent|}.
  \item Except for the first node, the macro
    \declare{|\tikzchainprevious|} is now globally set to the name of
    the node of the previous node on the chain. For this first node,
    this macro is globally set to the empty string.
  \item Except possibly for the first node of the chain, the placement
    rule is now executed. The placement rule is just a \tikzname\ option
    that is applied automatically to each node on the chain. Depending
    on the form of the \meta{direction} parameter (either the locally
    given one or the one given to the |chain| option), different
    things happen.

    First, it makes a difference whether the \meta{direction} starts
    with |going| or with |placed|. The difference is that in the first
    case, the placement rule is not applied to the first node of the
    chain, while in the second case the placement rule is applied also
    to this first node. The idea is that a |going|-direciton indicates
    that we are ``going somewhere relative to the previous node''
    whereas a |placed| indicates that we are ``placing nodes according
    to their number.''

    Independently of which form is used, the \meta{text} inside
    \meta{direction} that follows |going| or |placed| (separated by a
    compulsory space) can have two different effects:
    \begin{enumerate}
    \item If it contains an equal sign, then this \meta{text} is used
      as the placement rule, that is, it is simply executed.  
    \item If it does not contain an equal sign, then
      \meta{text}|=of \tikzchainprevious| is used as the placement
      rule. 
    \end{enumerate}

    Note that in the first case, inside the \meta{text} you have
    access to |\tikzchainprevious| and |\tikzchaincount| for doing
    your positioning calculations.
    complicated placings.
\begin{codeexample}[]
\begin{tikzpicture}[start chain=circle placed {at=(\tikzchaincount*30:1.5)}]
  \foreach \i in {1,...,10}
    \node [on chain] {\i};
  
  \draw (circle-1) -- (circle-10);
\end{tikzpicture}
\end{codeexample}
  \item
    The following style is executed:
    \begin{stylekey}{/tikz/every on chain}
      This key is executed for every node of a chain, including the
      first one. 
    \end{stylekey}
  \end{enumerate}

  Recall that the standard replacement rule has a form like
  |right=of (\tikzchainprevious)|. This means that each 
  new node is placed to the right of the previous one, spaced by the
  current value of |node distance|.
\begin{codeexample}[]
\begin{tikzpicture}[start chain,node distance=5mm]
  \node [draw,on chain] {};
  \node [draw,on chain] {Hallo};
  \node [draw,on chain] {Welt};
\end{tikzpicture}
\end{codeexample}

  The optional \meta{direction} allows us to temporarily change the
  direction in the middle of a chain:
\begin{codeexample}[]
\begin{tikzpicture}[start chain,node distance=5mm]
  \node [draw,on chain] {Hello};
  \node [draw,on chain] {World};
  \node [draw,on chain=going below] {,};
  \node [draw,on chain] {this};
  \node [draw,on chain] {is};
\end{tikzpicture}
\end{codeexample}

  You can also use more complicated computations in the \meta{direction}:
\begin{codeexample}[]
\begin{tikzpicture}[start chain=going {at=(\tikzchainprevious),shift=(30:1)}]
  \draw [help lines] (0,0) grid (3,2);
  \node [draw,on chain] {1};
  \node [draw,on chain] {Hello};
  \node [draw,on chain] {World};
  \node [draw,on chain] {.};
\end{tikzpicture}
\end{codeexample}
\end{key}

\begin{key}{/tikz/redirect chain=\opt{\meta{chain}}\meta{new direction}}
  This option allows you to change the \meta{direction} of the
  \meta{chain}. For all subsequent nodes on the \meta{chain}, the
  \meta{new direction} is used (except if they, in turn, locally set
  their own direction). If the \meta{chain} is omitted, the value of
  the key |current chain| is used.  
\begin{codeexample}[]
\begin{tikzpicture}[start chain=going right,node distance=5mm]
  \node [draw,on chain] {Hello};
  \node [draw,on chain] {World};
  \node [draw,redirect chain=going below,on chain] {,};
  \node [draw,on chain] {this};
  \node [draw,on chain] {is};
\end{tikzpicture}
\end{codeexample}
\end{key}

\begin{command}{\chainin}
  Description missing...
\end{command}


\subsection{Joining Nodes on a Chain}

\begin{key}{/tikz/join=\opt{|with |\meta{with} }\opt{|by |\meta{options}}}
  When this key is given to any node on a chain (except possibly for
  the first node), an |edge| command is added after the node. The
  |with| part specifies which node should be used for the start point
  of the edge; if the |with| part is omitted, the |\tikzchainprevious|
  is used. This |edge| command gets the \meta{options} as parameter
  and the current node as its target. If there is no
  previous node and no |with| is given, no |edge| command gets
  executed.  
  \begin{stylekey}{/tikz/every join}
    This style is executed each time this command is used.
  \end{stylekey}

  Note that is makes sense to call this option several times for a
  node, in order to connect it to several nodes.
\begin{codeexample}[]
\begin{tikzpicture}[start chain,node distance=5mm,
                    every join/.style={->,red}]
  \node [draw,on chain,join] {};
  \node [draw,on chain,join] {Hallo};
  \node [draw,on chain,join] {Welt};
  \node [draw,on chain=going below,
         join,join=with chain-1 by {blue,<-}] {foo};
\end{tikzpicture}
\end{codeexample}
\end{key}


\subsection{Branches}

Description missing...

\subsection{Examples}

Here is a final example of a chain ``in action.''

\begin{codeexample}[]
\begin{tikzpicture}[
    >=stealth,
    minimum size=6mm,
    terminal/.style={rectangle,rounded corners=3mm,draw,fill=white,thick},
    nonterminal/.style={rectangle,draw,fill=white,thick},
    node distance=3mm]
    
  \begin{scope}[start chain,
                every node/.style={on chain},
                terminal/.append style={join=by {->,shorten >=1pt}},
                nonterminal/.append style={join=by {->,shorten >=1pt}},
                support/.style={coordinate,join}]
    \node [support]             (start)        {};
    \node [nonterminal]                        {unsigned integer};
    \node [support]             (after ui)     {};
    \node [terminal]                           {.};
    \node [support]             (after dot)    {};
    \node [terminal]                           {digit};
    \node [support]             (after digit)  {};
    \node [support]             (skip)         {};    
    \node [support]             (before E)     {};
    \node [terminal]                           {E};
    \node [support]             (after E)      {};
    \node [support,xshift=5mm]  (between)      {};
    \node [support,xshift=5mm]  (before last)  {};
    \node [nonterminal]                        {unsigned integer};
    \node [support]             (after last)   {};
    \node [join=by ->]          (end)          {};
  \end{scope}
  \node (plus)  [terminal,above=of between] {$+$};
  \node (minus) [terminal,below=of between] {$-$};

  \begin{scope}[->,shorten >=1pt,rounded corners]
    \draw (after ui)    -- +(0,.7)  -| (skip);
    \draw (after digit) -- +(0,-.7) -| (after dot);
    \draw (before E)    -- +(0,-1) -| (after last);
    \draw (after E)     |- (plus);
    \draw (plus)        -| (before last);
    \draw (after E)     |- (minus);
    \draw (minus)       -| (before last);
  \end{scope}
\end{tikzpicture}
\end{codeexample}

