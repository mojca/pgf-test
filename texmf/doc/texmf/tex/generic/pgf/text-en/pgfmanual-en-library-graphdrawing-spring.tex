% Copyright 2011 by Jannis Pohlmann
%
% This file may be distributed and/or modified
%
% 1. under the LaTeX Project Public License and/or
% 2. under the GNU Free Documentation License.
%
% See the file doc/generic/pgf/licenses/LICENSE for more details.

\section{Force-Based Graph Drawing Algorithms}
\label{section-library-graphdrawing-force-based}

{\emph{by Jannis Pohlmann}}


\begin{tikzlibrary}{graphdrawing.spring}
  Load this package when you wish to use force-based graph drawing
  algorithms. You should load the |graphdrawing| library first.
\end{tikzlibrary}

\ifluatex\relax\else{LuaTeX is required for setting this manual section.}\expandafter\endinput\fi


\subsection{Overview}

% TODO explain the spring and spring-electrical drawing algorithms.

...

\begin{key}{/graph drawing/spring layout=\meta{options}}
  \keyalias{tikz}\keyalias{tikz/graphs}
  Similar to the |>| option, this ``generic'' name for a spring layout
  algorithm is not hardwired to any specific algorithm. Rather, users
  can select an algorithm somewhere at the beginning of their program
  and then just write |\graph[spring layout]| to draw a tree.

  The \meta{options} will be forwarded to the currently selected
  algorithm.
\begin{codeexample}[]
\tikz \graph [spring layout] { a -> {b,c} };    
\end{codeexample}
  
  To change the algorithm, change the following key:
  \begin{key}{/graph drawing/spring layout/default algorithm=\meta{algorithm}}
    Set this key to the tree drawing algorithm of your choice. The
    default is currently |standard spring electrical|, but this will change.
  \end{key}
\end{key}



\subsection{Common Options}

The spring and and spring-electrical drawing algorithms are very similar
in terms of their parameters and the constraints they can handle. They
thus share a number of common \tikzname\ options for fine-tuning. These
options are split up into \emph{graph options} that can be specified
once for a graph, \emph{node options} that can be specified for each
node and \emph{edge options} that can be specified for each edge.

\subsubsection{Graph Options}

\begin{key}{/tikz/monotonic energy minimization=\opt{\meta{boolean}} 
  (default true, initially false)}
  If set to |true|, requires a step along the force to reduce the
  system energy. If set to |false| even steps that do not lower the 
  system energy are accepted.
  \begin{codeexample}[]
  \end{codeexample}
\end{key}

\begin{key}{/tikz/influence cutoff distance=\meta{dimension} (initially
  0pt)}
  Specifies a distance beyond which the attractive and repulsive forces 
  between two nodes are assumed to be virtually non-existent. If 
  \meta{dimension} is set to |0pt|, the cutoff distance is computed 
  automatically.

  Depending on the graph drawing algorithm being used, the distance
  between two nodes is computed either based on the graph distance
  (spring algorithm) or based on the Euclidean distance
  (spring-electrical algorithm).
  \begin{codeexample}[]
  \end{codeexample}
\end{key}

\begin{key}{/tikz/maximum iterations=\meta{number} (initially 500)}
  Depending on the characteristics of the input graph and the parameters
  chosen for the spring or spring-electrical algorithm, minimizing the
  system energy may require many iterations. In rare cases and only if
  |monotonic energy minimization| and |monotonic step control| are
  turned off, the graph drawing algorithm may not even terminate.

  In these situations it may come in handy to limit the number of
  iterations. This feature can also be useful to draw the same graph
  after different iterations and thereby demonstrate how the spring or
  spring-electrical algorithm improves the drawing step by step.
  \begin{codeexample}[]
\tikz \graph [spring layout={maximum iterations=1}]   { a -- b -- c -- a };
\tikz \graph [spring layout={maximum iterations=10}]  { a -- b -- c -- a };
\tikz \graph [spring layout={maximum iterations=500}] { a -- b -- c -- a };
  \end{codeexample}
\end{key}

% \begin{key}{/tikz/random seed=\meta{number} (initially 42)}
%   Specifies the seed used for Lua's pseudo-random number generator. If
%   set to something other than |0|, the random number sequence generated
%   by the pseudo-random number generator will be the same at every run.
%   If set to |0|, the results will be different every time.
%   \begin{codeexample}[width=5.5cm]
% \tikz \graph [spring layout={random seed=1}] { 
%   subgraph K_n[n=4]
% };
% \tikz \graph [spring layout={random seed=10}] { 
%   subgraph K_n[n=4]
% };
%   \end{codeexample}
% \end{key}

\begin{key}{/tikz/coarsening=\marg{options}}
  Executes the \meta{options} with the path prefix |/tikz/coarsening|.
  
  These options define whether a multilevel approach is used that
  successively coarsend into graphs with smaller and smaller number
  of nodes. These graphs are arranged first and are then interpolated
  into the finer graphs at the previous level. How this is done exactly
  can be configured using the |coarsening| options described below.
\end{key}

\begin{key}{/tikz/coarsening/randomized=\opt{\meta{boolean}} (default
  true, initially false)}
  If set to |true|, nodes will be inspected in a random order. The
  effect on the final drawing can only be seen by experimenting with the
  option.
  \begin{codeexample}[]
  \end{codeexample}
\end{key}

\begin{key}{/tikz/coarsening/minimum size=\meta{number} (default 0)}
  Defines the minimum number of nodes in a coarsened graph. If a
  coarsened graph has less than \meta{number} nodes, then... % TODO
  \begin{codeexample}[] 
% the same graph with different minimum size values
  \end{codeexample}
\end{key}

\begin{key}{/tikz/coarsening/nodes=\opt{\meta{boolean}} (default true,
  initially false)}
  \begin{codeexample}[]
  \end{codeexample}
\end{key}

\begin{key}{/tikz/coarsening/nearby nodes=\opt{\meta{boolean}} (default
  true, initially false)}
  \begin{codeexample}[]
  \end{codeexample}
\end{key}

\begin{key}{/tikz/coarsening/nodes with more 
  neighbors=\opt{\meta{boolean}} (default true, initially false)}
  \begin{codeexample}[]
  \end{codeexample}
\end{key}

\begin{key}{/tikz/coarsening/nearby nodes with more 
  neighbors=\opt{\meta{boolean}} (default true, initially false)}
  \begin{codeexample}[]
  \end{codeexample}
\end{key}

\begin{key}{/tikz/coarsening/edges=\opt{\meta{boolean}} (default true,
  initially false)}
  \begin{codeexample}[]
  \end{codeexample}
\end{key}

\begin{key}{/tikz/coarsening/heavy edges=\opt{\meta{boolean}} (default
  true, initially false)}
  \begin{codeexample}[]
  \end{codeexample}
\end{key}

\begin{key}{/tikz/coarsening/edges with light nodes=\opt{\meta{boolean}}
  (default true, initially false)}
  \begin{codeexample}[]
  \end{codeexample}
\end{key}

\begin{key}{/tikz/minimum energy delta=\meta{number} (default TODO)}
  \begin{codeexample}[]
  \end{codeexample}
\end{key}

\begin{key}{/tikz/initial step size=\meta{dimension} (default TODO)}
  \begin{codeexample}[]
  \end{codeexample}
\end{key}

\begin{key}{/tikz/step control=\meta{text} (default TODO)}
  Possible values: |monotonic|, |non-monotonic|, |strictly monotonic|.
  \begin{codeexample}[]
  \end{codeexample}
\end{key}

\subsubsection{Node Options}

%\begin{key}{/tikz/electric charge=\meta{number} (default 1)}
%  Defines the electric charge of the node. The stronger the electric
%  charge, the higher the repulsive force between two nodes. Set this to
%  something between |0| and |1| to reduce the charge compared to the
%  normal setup. Values larger than |1| will generate stronger repulsion
%  between the node and the others.
%  \begin{codeexample}[] 
%\tikz \graph [spring electrical layout,orient=1:90:2] {
%  1 -- 2 -- 3 -- 4 -- 1,
%  1 -- 3, 2 -- 4,
%};
%\tikz \graph [spring electrical layout,orient=1:90:2] {
%  1 [electric charge=1] -- 2 -- 3 -- 4 -- 1,
%  1 -- 3, 2 -- 4,
%};
%\tikz \graph [spring electrical layout,orient=1:90:2] {
%  1 [electric charge=1000] -- 2 [electric charge=1000] -- 3 -- 4 -- 1,
%  1 -- 3, 2 -- 4,
%};
%  \end{codeexample}
%\end{key}

%\end{document}

\begin{key}{/tikz/at=\meta{coordinate}}
  Nails the node down at the specified \meta{coordinate}. It will not
  move from there despite the repulsive and attractive forces in the
  system. Note that, while sometimes generating a similar effect, using
  |at| is very different from altering the orientation of a graph
  drawing (see section~\ref{subsection-library-graphdrawing-standard-orientation}).
  Also, if an orientation is specified, it is given priority over
  the |at| option in that nodes are first fixated at their |at|
  coordinates but are later moved in order to satisfy the orientation 
  desired by the user.
  \begin{codeexample}[width=6.0cm]
\tikz \graph [spring layout] {
  1 -- 2 -- 3 -- 4 -- 2
};
\tikz \graph [spring layout] {
  1 [at={(0,0)}] -- 2 [at={(0,1)}] -- 3 -- 4 -- 2
};
  \end{codeexample}
\end{key}


% TODO what about node groups / clusters? This works via color classes
% but how do we define their layouts (cluster, line, circle)?

\subsubsection{Edge Options}

\begin{key}{/tikz/natural length=\meta{dimension} (default 10pt)}
  Defines the natural (zero energy) length of the edge. The smaller the
  length, the stronger the attractive force of the adjacent nodes. The
  \meta{dimension} has a strong influence of how far the nodes will be
  placed from each other in the final drawing.
  \begin{codeexample}[]
% two examples with the same graph
% notably change the natural length of one of the edges
  \end{codeexample}
\end{key}

\begin{key}{/tikz/stiffness=\meta{number} (default 0.5)}
  Defines how flexible the spring associated with the edge is. The
  higher this value is, the closer the final edge length will be to its
  |natural length|.
  \begin{codeexample}[]
% two examples with the same graph
% notably change the stiffness of one of the edges
  \end{codeexample}
\end{key}

\subsection{Options for the Spring Algorithm}

\subsubsection{Graph Options}

...

\subsubsection{Node Options}

...

\subsubsection{Edge Options}

...

\subsection{Options for the Spring-Electrical Algorithm}

\subsubsection{Graph Options}

...

\subsubsection{Node Options}

...

\subsubsection{Edge Options}

...

\begin{codeexample}[]
\vbox{ \hsize=16cm \rightskip=0cm plus 1fill
  \foreach \iterations in {1,...,20,100,500}
  {
    \tikz \graph [spring layout={maximum iterations=\iterations}, orient=1-2] 
      { subgraph K_n[n=4] };
    \penalty0
  }
}
\end{codeexample}

\begin{codeexample}[]
\vbox{ \hsize=16cm \rightskip=0cm plus 1fill
  \foreach \iterations in {1,...,20,100,500}
  {
    \tikz \graph [spring layout={maximum iterations=\iterations}, orient=1-2] 
      { subgraph C_n[n=7] };
    \penalty0
  }
}
\end{codeexample}

\endinput

%% TODO
%% Explain the following concepts:
%% - separation of graph drawing options and regular TikZ options
%% - generic graph drawing options:
%%   - component packing
%%   - orientation
%% - pre-defined graph drawing styles
%% - graph drawing options for fine-tuning the different algorithms

%%% Local Variables: 
%%% mode: latex
%%% TeX-master: "pgfmanual-pdftex-version"
%%% End: 
