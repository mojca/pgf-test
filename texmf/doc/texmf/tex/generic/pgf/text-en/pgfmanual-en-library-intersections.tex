% Copyright 2008 by Mark Wibrow
%
% This file may be distributed and/or modified
%
% 1. under the LaTeX Project Public License and/or
% 2. under the GNU Free Documentation License.
%
% See the file doc/generic/pgf/licenses/LICENSE for more details.


\section{Intersections Library}

{\bf\emph{This library is experimental and likely to change,
move, or disappear, without warning.}}

\begin{pgflibrary}{intersections}
  This library enables the calculation of intersections of
  two arbitrary paths. However, due to the low accuracy of
  \TeX, the paths should not be ``too complicated''.
  In particular, you should not try to intersect paths consisting 
  lots of very small segments such as plots or decorated paths.
\end{pgflibrary}

\subsection{Intesecting Two Paths in PGF}
  
  To intersect two paths in \pgfname, the following command is
  provided:
   
\begin{command}{\pgfintersectionofpaths\marg{path 1}\marg{path 2}}
  This command finds the intersection points on the paths 
  \meta{path 1} and \meta{path 2}. The number of intersection points
  (``solutions'') that are found will be stored, and each point 
  can be accessed separately. The code for \meta{path 1} and 
  \meta{path 2} is executed within a \TeX{} group and so can contain
  transformations (which will be in addition to any existing
  transformations). The code should not use the path in any way, 
  unless the path is saved first and restored afterwards.
  \pgfname{} will regard solutions as ``a bit
  special'', in that the points returned  will be ``absolute'' and 
  unaffected by any further transformations.

\begin{codeexample}[]
\begin{pgfpicture}
\pgfintersectionofpaths
{
  \pgfpathellipse{\pgfpointxy{0}{0}}{\pgfpointxy{1}{0}}{\pgfpointxy{0}{2}}
  \pgfgetpath\temppath
  \pgfusepath{stroke}
  \pgfsetpath\temppath
}
{
  \pgftransformrotate{-30}
  \pgfpathrectangle{\pgfpointorigin}{\pgfpointxy{2}{2}}
  \pgfgetpath\temppath
  \pgfusepath{stroke}
  \pgfsetpath\temppath
}
\foreach \s in {1,...,\pgfintersectionsolutions}
  {\pgfpathcircle{\pgfpointintersectionsolution{\s}}{2pt}}
\pgfusepath{stroke}
\end{pgfpicture}
\end{codeexample}

\end{command}

\begin{command}{\pgfintersectionsolutions}
  After using the |\pgfintersectionofpaths| command, this \TeX-counter
  will hold the number of solutions found.
\end{command}

\begin{command}{\pgfpointintersectionsolution\marg{number}}
  After using the |\pgfintersectionofpaths| command, this command
  will return the point for solution \meta{number}. Unfortunately
  there can be no guarantee of a ``helpful'' ordering of solutions.
\end{command}

\subsection{Intersecting Two Paths in \tikzname}

  To intersect two paths in \tikzname, they must first be
  ``named''. A ``named path'' is a path that has been named using 
  the following key:
  
\begin{key}{/tikz/path name=\meta{name}}

	The effect of this key is that, after the path has been constructed,
  just before it is used, it is associated \meta{name}. This 
  association is global, so lasts after the final semi-colon on the
  path.
  
\begin{codeexample}[code only]
\draw [path name=straight line] (0,0) -- (3,2);
\end{codeexample}

  The border path created by any nodes are also included, however,
  as this may not always be desirable, one value for \meta{name} 
  is special: the value |none|. When |path name=none| is given 
  in the options to a node on a path, the border path of that node is 
  ignored. 

\begin{codeexample}[]
\begin{tikzpicture}[every node/.style={circle, midway, minimum size=1cm, draw}]
  \draw [help lines] grid (3,2);
  \draw [path name=path 1, thick, blue] (1,0) -- (2,2) 
    node [path name=none, left=0.25cm, draw] {}
    node [right=0.25cm, draw]                {};
  \draw [path name=path 2, thick] (0,0.5) -- (3,1.5);
  \tikzintersectnamedpaths{path 1}{path 2}
  \fill [red, opacity=0.5] \foreach \s in {1,...,\solutions}
      { (solution cs:number=\s) circle (2pt) };  
\end{tikzpicture}
\end{codeexample}
\end{key}


The solution(s) for the intersection can be accessed using the
  |intersection| coodinate system, using the |/tikz/cs/solution| key
  to access different solutions. This library adds two more keys
  to specify the named paths:

\begin{key}{/tikz/cs/first path name=\meta{name}}
  Specify the first path name.
\end{key}

\begin{key}{/tikz/cs/second path name=\meta{name}}
  Specify the second path name.
\end{key}
  
  These keys can then be used as follows:
  
\begin{codeexample}[]
\begin{tikzpicture}[thick]
  \draw [help lines] grid (3,2);
  \draw [path name=circle, blue]  (1.5,1.25) circle (0.75cm);
  \draw [path name=curve] (0,0) .. controls (0,2) and (3,2) .. (3,0);
  \fill [red, opacity=0.5]
    (intersection cs:first path name=circle, second path name=curve)
     circle (2pt);
\end{tikzpicture}
\end{codeexample}

  One drawback with the |intersection| coodinate system is that
  evey time it is used, the solutions are recalculated. It would be
  easier if the solutions could be calculated ``all at once'', and
  then accessed later. It might also be nice if, say, two or
  more intersections could be calculated and making all the solutions
  available for subsequent drawing.
  This is possible by using the following command:
 
\begin{command}{\tikzintersectnamedpaths\opt{|[id=|\meta{name}|]|}\marg{path 1}\marg{path 2}}
  
  This command finds all the intersections of \meta{path 1} and 
  \meta{path 2}. By default when the optional |id| key is not 
  specified the number of solutions will be stored in the 
  \TeX-macro |\solutions|, and each solution can be accessed 
  using the |solution| coodinate system. The |id| key is
  discussed below.
  
  Solutions will only be ``remembered'' up to the end of the current
  scope. If a solution does not exist, the origin will be
  returned. 
  
\begin{coordinatesystem}{solution}
  This coordinate system returns a particluar solution from 
  a intersection (which may or may not have an identification 
  name). The following keys can be used with this coordinate
  system:

\begin{key}{/tikz/cs/solution=\meta{number}}
  This key specifies the number of the required solution. It is not
  possible to guarantee the order of the solutions.

\begin{codeexample}[]
\begin{tikzpicture}[every node/.style={opacity=1, circle, black, above left}]
  \clip (-2,-2) rectangle (2,2);
  \draw [path name=curve 1] (-2,-1) .. controls (8,-1) and (-8,1) .. (2,1);
  \draw [path name=curve 2] (-1,-2) .. controls (-1,8) and (1,-8) .. (1,2);
  \tikzintersectnamedpaths{curve 1}{curve 2}
  \fill [red, opacity=0.5] \foreach \s in {1,...,\solutions}
  { (solution cs:solution=\s) circle (2pt) node {\footnotesize\s} };
\end{tikzpicture}
\end{codeexample}
\end{key}

\begin{key}{/tikz/cs/id=\meta{name}}
  This key specifies the identification name of an intersection.
  It is used in the optional argument to the command 
  |\tikzintersectnamedpaths| to indentify the intersection as 
  \meta{name}, and in the |solution| coordinate system to obtain 
  the appropriate solutions.
  When the |id| key is used |\tikzintersectnamedpaths|, the number 
  of solutions will be stored in the \TeX-macro 
  |\|\meta{name}|solutions|, so, for example, by 
  using |id=foo|, the \TeX-macro |\foosolutions|  will expand to the 
  number of solutions. This means that, in general, \meta{name} 
  should be a single word consisting only of alphabetic characters.
\end{key}

\begin{codeexample}[]
\begin{tikzpicture}
  \draw [help lines] grid  (3,2);
  \draw [path name=line 1] (0,0) -- (1,2);
  \draw [path name=line 2] (0,1) -- (2,2);
  \draw [path name=circle] (2,1) circle (0.75cm);
  \draw [path name=square] (2,0) rectangle +(1,1);
  \tikzintersectnamedpaths[id=first]{line 1}{line 2}
  \tikzintersectnamedpaths[id=second]{circle}{square}
  \draw [red, thick]  (solution cs:solution=1,id=first) --
  	(solution cs:solution=2,id=second);
  \draw [blue, thick] (solution cs:solution=1,id=first) --
  	(solution cs:solution=1,id=second);
\end{tikzpicture}

\end{codeexample}

\end{coordinatesystem}

\end{command}

  There are keys to invoke |\tikzintersectnamedpaths| inside a path. You
  must remember, however, that solutions will only be remembered until 
  the end of the path.
  
\begin{key}{/tikz/intersect named paths=\meta{path 1}| |and| |\meta{path 2}}
  This key will calculate all the intersections of the named paths
  \meta{path 1} and \meta{path 2}.
   
\begin{codeexample}[]
\begin{tikzpicture}[every node/.style={opacity=1, circle, black, above left}]
  \draw [help lines] grid (3,2);
  \draw [path name=ellipse] (2,0.5) ellipse (0.75cm and 1cm);
  \draw [path name=rectangle, rotate=10]  (0.5,0.5) rectangle +(2,1);
  \fill [red, opacity=0.5, intersect named paths=ellipse and rectangle]
    \foreach \s in {1,...,\solutions}
      { (solution cs:solution=\s) circle (2pt) node {\footnotesize\s} };    
\end{tikzpicture}%
\end{codeexample}
   
\end{key}

\begin{key}{/tikz/intersecion id=\meta{name}}
  This key \emph{must} be given \emph{before} the |intersect| |named| 
  |paths| key is used. It will allow any intersection solutions
  to be accessed using the |id| key in the |solution| coordinate
  system.
\end{key}



