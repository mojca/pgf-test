% Copyright 2006 by Till Tantau
%
% This file may be distributed and/or modified
%
% 1. under the LaTeX Project Public License and/or
% 2. under the GNU Free Documentation License.
%
% See the file doc/generic/pgf/licenses/LICENSE for more details.


\section{Plot Mark Library}

\begin{pgflibrary}{plotmarks}
  This library defines a number of plot marks.
\end{pgflibrary}

This library defines the following plot marks in
addition to |*|, |x|, and |+| (the filling color has been set to a
dark yellow):

{
\catcode`\|=12
\medskip
\begin{tabular}{lc}
  \plotmarkentry{-}
  \index{*vbar@\protect\texttt{\protect\myvbar} plot mark}%
  \index{Plot marks!*vbar@\protect\texttt{\protect\myvbar}}
  \texttt{\char`\\pgfuseplotmark\char`\{\declare{|}\char`\}} &
  \tikz\draw[color=black!25] plot[mark=|,mark options={fill=yellow,draw=black}]
  coordinates {(0,0) (.5,0.2) (1,0) (1.5,0.2)};\\
  \plotmarkentry{o}
  \plotmarkentry{asterisk}
  \plotmarkentry{star}
  \plotmarkentry{oplus}
  \plotmarkentry{oplus*}
  \plotmarkentry{otimes}
  \plotmarkentry{otimes*}
  \plotmarkentry{square}
  \plotmarkentry{square*}
  \plotmarkentry{triangle}
  \plotmarkentry{triangle*}
  \plotmarkentry{diamond}
  \plotmarkentry{diamond*}
  \plotmarkentry{pentagon}
  \plotmarkentry{pentagon*}
  \plotmarkentry{text}
\end{tabular}
}

{
\def\showit#1{%
  \tikz\draw[color=black!25] plot[#1,mark options={fill=examplefill,draw=black}] coordinates{(0,0) (.5,0.2) (1,0) (1.5,0.2)};}%

\begin{key}{/pgf/text mark=\marg{text} (initially p)}
	Changes the text shown by |mark=text|.

	With |/pgf/text mark=m|: \pgfkeys{/pgf/text mark=m}\showit{mark=text}

	With |/pgf/text mark=A|: \pgfkeys{/pgf/text mark=A}\showit{mark=text}

	There is no limitation about the number of characters or whatever. In fact, any \TeX\ material can be inserted as \marg{text}, including images.
\end{key}
\begin{key}{/pgf/text mark style=\marg{options for \texttt{mark=text}}}
	Defines a set of options which control the appearance of |mark=text|.

	If |/pgf/text mark as node=false| (the default), \marg{options} is provided as argument to |\pgftext| -- which provides only some basic keys like |left|, |right|, |top|, |bottom|, |base| and |rotate|.

	If |/pgf/text mark as node=true|, \marg{options} is provided as argument to |\node|. This means you can provide a very powerful set of options including |anchor|, |scale|, |fill|, |draw|, |rounded corners| etc. 
\end{key}
\begin{key}{/pgf/text mark as node=\mchoice{true,false} (initially false)}
	Configures how |mark=text| will be drawn: either as |\node| or as |\pgftext|.

	The first choice is highly flexible and possibly slow, the second is very fast and usually enough.
\end{key}

}


%%% Local Variables: 
%%% mode: latex
%%% TeX-master: "pgfmanual-pdftex-version"
%%% End: 
