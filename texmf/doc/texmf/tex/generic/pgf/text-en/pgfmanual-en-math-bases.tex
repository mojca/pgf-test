% Copyright 2007 by Mark Wibrow
%
% This file may be distributed and/or modified
%
% 1. under the LaTeX Project Public License and/or
% 2. under the GNU Free Documentation License.
%
% See the file doc/generic/pgf/licenses/LICENSE for more details.

\section{Conversion Between Bases}
	
\label{pgfmath-bases}

\pgfname{} provides limited support for conversion between 
\emph{representations} of numbers. Currently the numbers must be
positive integers in the range $0$ to $2^{31}-1$, and the bases in the
range $2$ to $36$. All digits representing numbers greater than 9 (in
base ten), are alphabetic, but may be upper or lower case. Note, that
again, examples in this section, show the result of the calculation
\emph{not} what is shown on screen.

\begin{command}{\pgfmathbasetodec\marg{macro}\marg{number}\marg{base}}
	Defines \meta{macro} as the result of converting \meta{number} from
	base \meta{base} to base 10. Alphabetic digits can be upper or lower
	case.

\medskip{\def\medskip{}

\begin{codeexample}[post=\tt\footnotesize\mynumber]
\pgfmathbasetodec\mynumber{107f}{16}
\end{codeexample}


\begin{codeexample}[post=\tt\footnotesize\mynumber]
\pgfmathbasetodec\mynumber{33FC}{20}
\end{codeexample}

}\medskip

\end{command}

\begin{command}{\pgfmathdectobase\marg{macro}\marg{number}\marg{base}}
	Defines \meta{macro} as the result of converting \meta{number} from
	base 10 to base \meta{base}. Any resulting alphabetic digits are in
	\emph{lower case}.
	
\begin{codeexample}[post=\tt\footnotesize\mynumber]
\pgfmathdectobase\mynumber{65535}{16}
\end{codeexample}

\end{command}

\begin{command}{\pgfmathdectoBase\marg{macro}\marg{number}\marg{base}}
	Defines \meta{macro} as the result of converting \meta{number} from
	base 10 to base \meta{base}. Any resulting alphabetic digits are in
	\emph{upper case}.
	
\begin{codeexample}[post=\tt\footnotesize\mynumber]
\pgfmathdectoBase\mynumber{65535}{16}
\end{codeexample}

\end{command}

\begin{command}{\pgfmathbasetobase\marg{macro}\marg{number}\marg{base-1}\marg{base-2}}
	Defines \meta{macro} as the result of converting \meta{number} from
	base \meta{base-1} to base \meta{base-2}. Alphabetic digits in 
	\meta{number} can be upper or lower case, but any resulting 
	alphabetic digits are in \emph{lower case}.
	
\begin{codeexample}[post=\tt\footnotesize\mynumber]
\pgfmathbasetobase\mynumber{11011011}{2}{16}
\end{codeexample}

\end{command}

\begin{command}{\pgfmathbasetoBase\marg{macro}\marg{number}\marg{base-1}\marg{base-2}}
	Defines \meta{macro} as the result of converting \meta{number} from
	base \meta{base-1} to base \meta{base-2}. Alphabetic digits in 
	\meta{number} can be upper or lower case, but any resulting 
	alphabetic digits are in \emph{upper case}.
	
\begin{codeexample}[post=\tt\footnotesize\mynumber]
\pgfmathbasetoBase\mynumber{121212}{3}{12}
\end{codeexample}

\end{command}


\begin{command}{\pgfmathsetbasenumberlength\marg{integer}}
	Set the number of digits in the result of a base conversion to 
	\meta{integer}. If the result of a conversion has less than this 
	number it is prefixed with zeros.

\begin{codeexample}[post=\tt\footnotesize\mynumber]
\pgfmathsetbasenumberlength{8}
\pgfmathdectobase\mynumber{15}{2}
\end{codeexample}

\end{command}