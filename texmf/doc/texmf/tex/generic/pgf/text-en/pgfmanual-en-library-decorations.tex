% Copyright 2008 by Mark Wibrow
%
% This file may be distributed and/or modified
%
% 1. under the LaTeX Project Public License and/or
% 2. under the GNU Free Documentation License.
%
% See the file doc/generic/pgf/licenses/LICENSE for more details.

\section{Decoration Library}
\label{section-library-decorations}

Decorations are similar to snakes, but are not limited to moving along
straight lines. In addition they can ``do something'' with the 
path(s) that they create, such as drawing or filling them, allowing
decorations to consist of different colors, line widths, and so on.
Meta-decorations are similar to decorations execpt that, instead of
decorating a path with new paths, they decorate it with decorations.

The principles behind decorations and meta-decorations (and how to
define new ones) are described in 
Section~\ref{section-base-snakes-and-decorations}. In this section, 
two libraries are documented: |tikzlibrarydecorations|, which enables 
(meta-)decorations to be used in \tikzname, and 
|pgflibrarydecorations| which defines some basic decorations.


\subsection{Usign decorations in Ti\emph{k}Z}

\begin{tikzlibrary}{decorations}
	This library enables decorations to be used in \tikzname, and depends
	on |pgfbasedecorations| which is automatically loaded with 
	\tikzname. 
	Somewhat unhelpfully, no decorations are defined by default in 
	\tikzname, but some are defined in |pgflibrarydecorations|, 
	which is auotmatically loaded with this library.
\end{tikzlibrary}


	Using decorations in \tikzname{} is very similar to using snakes.
	There is no special path command to decorate path, but a number of
	keys:

\begin{key}{/tikz/decoration=\meta{name}}

	This option causes the decoration \meta{name} to be used
	for subsequent paths. The path can include straight lines, curves,
	rectangles, arcs, circles and ellipses. However, rectangles, circles 
	and ellipses may not be decorated succesfully with `continuous' 
	decorations (i.e., those that do not create multiple segmented 
	subpaths). In addition, due to the limits on the precision in 
	\TeX, some inaccuraces in positioning when crossing subpath 
	boundaries may occasionally	be found. 
	
\begin{codeexample}[]
\begin{tikzpicture}
  \draw [help lines] grid (3,2);
  \draw [decoration=zigzag] (0,0) .. controls (0,2) and (3,0) .. (3,2);
\end{tikzpicture}
\end{codeexample}

	The special decoration, called |none|, stops any ongoing decoration
	at most recent coordinate. Unlike snakes, you do not need to specify 
	a decoration for each section of a path, a decoration will continute 
	until (a) a new decoration is specified, (b) the decoration is 
	explicitly ended using the decoration |none|, or (c) the end of the 
	path is reached. 

	It is important to remember that the path that is specified for
	the decoration is destroyed when it is decorated, so it cannot be 
	subsequently drawn or filled. In addition, it is not possible to 
	place a node on any part of a path with an active decoration.

	Successful use of a decoration, may depend on knowing something
	about the decoration, in particular whether it draws, or 
	fills	the path it creates. A decoration which does not use its
	own path must be used with |draw| or |fill| otherwise nothing will 
	appear.
	The |zigzag| decoration, shown above, is an example of a 
	decoration which does not do anything with its own path. Thus, the
	path command |\draw| is used. Note, that some decorations do not
	create a path	at all:	
	
\begin{codeexample}[]
\begin{tikzpicture}
	\tikzset{decoration text=around and around and around and around we go}
  \draw [help lines] grid (3,2);
  \path [decoration=text] (0,1) arc (180:-180:1.5cm and 1cm);
\end{tikzpicture}
\end{codeexample}

  Considering the above example, it might be nice to have the line as
  well. Of course, the line should be behind the text, so |postaction|
  can be used for the decoration.
  
\begin{codeexample}[]
\catcode`\|12
\begin{tikzpicture}
\draw [help lines] grid (3,2);
\draw [red, dashed, postaction={decoration=text}, 
  decoration text=around and around and around and around we go] 
  (0,1) arc (180:-180:1.5cm and 1cm);
\end{tikzpicture}
\end{codeexample} 
\end{key}

\begin{key}{/tikz/meta-decoration=\marg{name}}
	This key enables a path to be decorated with the meta-decoration
	called \meta{name}. Whereas a decoration (typically) decorates a path
	with another path, a meta-decoration decorates a path with
	decoration automatons. The ultimate effect, however, is to create	a
	new path. 


\begin{codeexample}[]
\begin{tikzpicture}  
  \draw [help lines] grid (3,2);
  \draw[meta-decoration=lineto zigzag, decoration segment length=.125cm,
        meta-decoration segment length=\pgfmetadecoratedpathlength/8]
    (1,0) arc (270:90:1cm) -- (2,2) arc (90:-90:1cm)--cycle;
\end{tikzpicture}
\end{codeexample}

	The special meta-decoration, called |none|, stops any ongoing 
	meta-decoration	at most recent coordinate. Note, you cannot use
	the |decorate| key to stop a meta-decoration. Similarly you
	cannot use the |meta-decoration| key to stop a normal decoration.

	
\end{key}



\subsection{Decorations definitions}



\begin{pgflibrary}{decorations}
	This library defines some basic decorations. It depends on
	|pgfbasedecorations| which is automatically loaded with \pgfname.
	The principles behind decorations and meta-decorations (and how to
	define new ones) are described in 
	Section~\ref{section-base-snakes-and-decorations}.
	
	The decorations are influenced by the current values of keys like
  |/pgf/decoration segment amplitude|. Meta-decorations are influenced
  by the values of keys like |/pgf/meta-decoration segment amplitude|.
\end{pgflibrary}


\begin{decoration}{zigzag}
	This decoration looks like a zig-zag line. The following parameters
  influence the decoration:
  \begin{itemize}
  \item |/pgf/decoration segment amplitude|
    determines how much the zig-zag lines raises above and falls below
    a straight line to the target point.
  \item |/pgf/decoration segment length|
    determines the length of a complete ``up-down'' cycle.
  \end{itemize}
\begin{codeexample}[]
\begin{tikzpicture}
  \draw [help lines] grid (3,2);
  \draw [decoration=zigzag] (0,0) .. controls (0,2) and (3,0) .. (3,2);
\end{tikzpicture}
\end{codeexample}
\end{decoration}

\begin{decoration}{saw}
	This decoration looks like the blade of a saw. The following parameters
  influence the decoration:
  \begin{itemize}
  \item |/pgf/decoration segment amplitude|
    determines how much each spike raises above the straight line.
  \item |/pgf/decoration segment length|
    determines the length each spike.
  \end{itemize}
\begin{codeexample}[]
\begin{tikzpicture}
  \draw [help lines] grid (3,2);
  \draw [decoration=saw] (0,0) .. controls (0,2) and (3,0) .. (3,2);
\end{tikzpicture}
\end{codeexample}
\end{decoration}

\begin{decoration}{triangles}
	This decoration adds triangles to the path that point toward the
  target. The following parameters influence the decoration: 
  \begin{itemize}
  \item |/pgf/decoration segment length|
    determines the distance between consecutive triangles.
  \item |/pgf/decoration segment amplitude|
    determines half the length of the triangle side that is orthogonal
    to the path.
  \item |/pgf/decoration segment object length|
    determines the height of the triangle.
  \end{itemize}
\begin{codeexample}[]
\begin{tikzpicture}
  \draw [help lines] grid (3,2);
  \draw [decoration=triangles] (0,0) .. controls (0,2) and (3,0) .. (3,2);
\end{tikzpicture}
\end{codeexample}
\end{decoration}

\begin{decoration}{crosses}
	This decoration adds (diagonal) crosses to the path. The following
  parameters influence the decoration:  
  \begin{itemize}
  \item |/pgf/decoration segment length|
    determines the distance between consecutive crosses.
  \item |/pgf/decoration segment amplitude|
    determines half the hieght of each cross.
  \item |/pgf/decoration segment object length|
    determines width of each cross.
  \end{itemize}
\begin{codeexample}[]
\begin{tikzpicture}
  \draw [help lines] grid (3,2);
  \draw [decoration=crosses] (0,0) .. controls (0,2) and (3,0) .. (3,2);
\end{tikzpicture}
\end{codeexample}
\end{decoration}

\begin{decoration}{ticks}
  This decoration adds straight lines  the path that are orthogonal to 
  the line toward the target. The following parameters influence the 
  decoration: 
  \begin{itemize}
  \item |/pgf/decoration segment length|
    determines the distance between consecutive ticks.
  \item |/pgf/decoration segment amplitude|
    determines half the length of the ticks.
  \end{itemize}
\begin{codeexample}[]
\begin{tikzpicture}
  \draw [help lines] grid (3,2);
  \draw [decoration=ticks] (0,0) .. controls (0,2) and (3,0) .. (3,2);
\end{tikzpicture}
\end{codeexample}
\end{decoration}

\begin{decoration}{text}
	This decoration decorates the path with text.
	
\begin{codeexample}[]
\catcode`\|12
\begin{tikzpicture}
  \draw [help lines] grid (3,2);
  \draw [red, dashed, postaction={decoration=text,
    decoration text={Some text along a curve}}] 
    (0,0) .. controls (0,2) and (3,0) .. (3,2);
\end{tikzpicture}
\end{codeexample}

  \pgfname{} ``does its best'' to typeset the text, however you
	should note the following points:
	\begin{itemize}
		\item
			Each character in the text is typeset in a separate |\hbox|. This
			means that if you want fancy things like kerning or ligatures you
			will have to manually annotate the characters in the decoration 
			text within a group, for example, |W{\kern-1ptA}TER|. 
		\item
			Each character is positioned using the center of its baseline. To
			move the text vertcally (relative to the path), the additional
			transform key should be used.
		\item
			No attempt is made to ensure characters do not overlap when
			the angle between segments is considerably less than 180\textdegree{}
			(this is tricky to do in \TeX{} without a huge processing
			overhead). In general this should not be too much of a problem, 
			but, once again, kerning can be used in most cases to overcome 
			any undesirable	effects.
		\item			
			It is only possible to typeset text in math mode under considerable
			restrictions. Math mode is entered and exited using any character	
			of category code 3 (e.g., in plain \TeX{} this is |$|). 
			Math subscripts and superscripts need to be	contained within braces 
			(e.g., |{^y_i}|) as do commands like |\times| or |\cdot|. 
			However, even modestly complex mathematical	typesetting is unlikely 
			to be sucessful along a path (or even desirable).
		\item
			Some inaccuracies in positioning may be particularly apparent
			at subpath boundaries. This can (unfortunately) only be solved 
			on case by case basis	by individually kerning the offending 
			characters within a group.
	\end{itemize}
	
	The following keys are used by the |text| decoration:
	
	\begin{key}{/pgf/decoration text=\marg{text} (initially \char`\{\char`\})}
	Set the text to typeset along the curve. 
	Consecutive spaces are ignored, so |\ | (or |\space| in \LaTeX) 
	should be used to insert multiple spaces.	It is possible to
	format the text using normal formating commands, such as |\it|, |\bf|
	and |\color|, within customisable delimiters. Initially these
	delimiters are both {\tt\char`\|} (however, care will be needed 
	regarding	the category codes of delimiters --- see below). 

{\catcode`\|12
\begin{codeexample}[]
\catcode`\|12
\begin{tikzpicture}
  \draw [help lines] grid (3,2);
  \path [decoration=text,	
   decoration text={a big |\color{green}|green|| juicy apple.}] 
    (0,0) .. controls (0,2) and (3,0) .. (3,2);
\end{tikzpicture}
\end{codeexample}
}
By following the first delimiter
	with |+|, the formatting commands are added to any exisiting 
	formatting.

{\catcode`\|12
\begin{codeexample}[]
\begin{tikzpicture}
  \draw [help lines] grid (3,2);
  \path [decoration=text,	
     decoration text={a |\large|big |+\bf\color{red}|red|| juicy apple.}] 
    (0,0) .. controls (0,2) and (3,0) .. (3,2);
\end{tikzpicture}
\end{codeexample}
}
	
Internally, the text is stored in the macro |\pgfdecorationtext|. 
Any characters that have not been typeset when the end of the 
path has been reached will be stored in |\pgfdecorationrestoftext|.

\end{key}

{\catcode`\|12
\begin{key}{/pgf/decoration text format delimiters=\marg{before}\marg{after} (initially \char`\{|\char`\}\char`\{\char`\})}

	\catcode`\|13
	
	Set the characters that the text decoration will use to parse 
	formatting commands. 
	If \meta{after} is empty, then \meta{before} will be used for both
	delimiters.
	In general you should stick to characters	whose category codes are 
	|11| or |12|.
	As |+| is used to indicate that the specifed format commands 
	are added	to any exisiting ones, you should avoid using |+| as
	a delimiter. 

\begin{codeexample}[]
\begin{tikzpicture}
  \draw [help lines] grid (3,2);
  \path [decoration=text, decoration text format delimiters={[}{]}, 
  decoration text={A big [\color{red}]red[] and [\color{green}]green[] apple.}] 
    (0,0) .. controls (0,2) and (3,0) .. (3,2);
\end{tikzpicture}
\end{codeexample}
\end{key}
}

\begin{key}{/pgf/decoration text color=\meta{color} (initially black)}
	Set the color for the text.
\end{key}
\end{decoration}

\begin{meta-decoration}{lineto zigzag}
	This meta-decoration decorates the path by alternating between 
	|lineto| and |zigzag| decorations. It always finishes
	with the |lineto| decoration. The parameters for the |zigzag|
	decoration are set in the usual manner (see above).	
	This meta-decoration depends on
	the following parameter:
	
	\begin{itemize}
		\item	|/pgf/meta-decoration segment length|
     determines how far each decoration decorates the path.
  \end{itemize}

\begin{codeexample}[]
\begin{tikzpicture}
  \draw [help lines] grid (3,2);
  \draw [meta-decoration=lineto zigzag, meta-decoration segment length=0.5cm,
         decoration segment length=.125cm] 
    (0,0) .. controls (0,2) and (3,0) .. (3,2);
\end{tikzpicture}
\end{codeexample}
	
	
\end{meta-decoration}
\end{document}

