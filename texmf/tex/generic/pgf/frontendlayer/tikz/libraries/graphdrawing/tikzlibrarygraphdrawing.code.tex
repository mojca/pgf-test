% Copyright 2010 by Renée Ahrens, Olof Frahm, Jens Kluttig, Matthias Schulz, Stephan Schuster
%
% This file may be distributed and/or modified
%
% 1. under the LaTeX Project Public License and/or
% 2. under the GNU Public License.
%
% See the file doc/generic/pgf/licenses/LICENSE for more details.

\ProvidesFileRCS[v\pgfversion] $Header: /home/mojca/cron/mojca/github/cvs/pgf/pgf/generic/pgf/frontendlayer/tikz/libraries/graphdrawing/Attic/tikzlibrarygraphdrawing.code.tex,v 1.6 2011/05/02 02:38:31 jannis-pohlmann Exp $

\usepgflibrary{graphdrawing}
\usetikzlibrary{graphs}


%
% This file defines the key family for graph drawing.
%
% It makes heavy use of the graphs library.
%

%
% The definitions for the sample algorithms ``few intersections'' and
% ``standard tree'' are at the end.
%

%
% ------------------------------------------------------------------------------
%

%
% Set up graph drawing
%
\tikzgraphsset{
  graph drawing engine/.style={
    /tikz/execute at begin scope={\pgfgdbeginscope},
    /tikz/execute at end scope={\pgfgdendscope},
    new ->/.code n args={4}{\pgfgdaddedge[##3]{##1}{##2}{##4}{->}},
    new <-/.code n args={4}{\pgfgdaddedge[##3]{##1}{##2}{##4}{<-}},
    new --/.code n args={4}{\pgfgdaddedge[##3]{##1}{##2}{##4}{--}},
    new <->/.code n args={4}{\pgfgdaddedge[##3]{##1}{##2}{##4}{<->}},
    new -!-/.code n args={4}{\pgfgdaddedge[##3]{##1}{##2}{##4}{-!-}},
    no placement
  }
}

\tikzset{
  graph drawing/.code={\tikzgraphsset{#1}}
}


% set keys for node and general options for graph drawing algorithms
% and graphs
\tikzgraphsset{
  graph drawing/.cd,
  @options/.initial=,
  @node@options/.initial=,
  register key/.style={
    /tikz/graphs/#1/.style={/tikz/graphs/graph drawing/@options/.append={#1}{##1}}
  },
  register math key/.style={
    /tikz/graphs/#1/.code={
      \pgfmathsetmacro{\tikz@gd@temp}{##1}%
      \edef\tikz@gd@marshal{\noexpand\pgfkeys{/tikz/graphs/graph drawing/@options/.append={#1}{\tikz@gd@temp}}}%
      \tikz@gd@marshal%
    }
  },
  register node key/.style={
    /tikz/graphs/#1/.style={/tikz/graphs/graph drawing/@node@options/.append={#1}{##1}}
  },
  register node math key/.style={
    /tikz/graphs/#1/.code={
      \pgfmathsetmacro{\tikz@gd@temp}{##1}%
      \edef\tikz@gd@marshal{\noexpand\pgfkeys{/tikz/graphs/graph drawing/@node@options/.append={#1}{\tikz@gd@temp}}}%
      \tikz@gd@marshal%
    }
  },
  register edge style/.style={
    /tikz/#1,/pgf/graph drawing/edge options/grab/#1
  },
  register edge key/.style={
    /tikz/#1/.code={},
    /pgf/graph drawing/edge options/grab/#1/.style={
      /pgf/graph drawing/edge options/.append={#1}{##1}
    }%
  },
  register edge math key/.style={
    /tikz/#1/.code={},
    /pgf/graph drawing/edge options/grab/#1/.code={
      \pgfmathsetmacro{\tikz@gd@temp}{##1}%
      \edef\tikz@gd@marshal{\noexpand\pgfkeys{/pgf/graph drawing/edge options/.append={#1}{\tikz@gd@temp}}}%
      \tikz@gd@marshal%
    }%
  },
}

% register main key 'algorithm'
\tikzgraphsset{
  graph drawing/register key=algorithm,
}



%
% Graph drawing algorithms
%

%
% Few intersections
%
% Description:
%
% This slow algorithm will try to draw a graph such that the number of
% intersections is minimized. The parameters ``max height'' and ``max
% width'' can be used to guide the desired width and/or height of the
% tree.
%
% Example:
%
% \tikzpicture[graph drawing={few intersections}]
%   \graph{ a -> {b, c-> {d,e->b}, f->a}};
% \endtikzpicture

\tikzgraphsset{
  few intersections/.style={
    graph drawing engine,
    algorithm=localsearchgraph
  },
  graph drawing/register math key=max height,
  graph drawing/register math key=max width
}


%
% Standard tree
%
% Description:
%
% The standard tree algorithm layouts an input tree. The node parameter ``root'' is required to set the root node, whereas the optional parameter ``tree scale'' defines a local scaling of the tree.
%
% Example
%
% \tikzpicture[graph drawing={standard tree}]
%   \graph{ a [root] -> {b, c -> {d,e}}};
% \endtikzpicture
%
\tikzgraphsset{
  standard tree/.style={
    graph drawing engine,
    algorithm=arbitrarytree
  },
  graph drawing/register node key=root,
  graph drawing/register math key=level distance,
  graph drawing/register math key=sibling distance

\endinput
