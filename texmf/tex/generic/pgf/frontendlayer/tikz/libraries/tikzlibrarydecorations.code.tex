% Copyright 2008 by Mark Wibrow
%
% This file may be distributed and/or modified
%
% 1. under the LaTeX Project Public License and/or
% 2. under the GNU Public License.
%
% See the file doc/generic/pgf/licenses/LICENSE for more details.

\usepgflibrary{decorations}



\newif\iftikz@metadecoration

\def\tikz@lib@dec@addtomacro#1#2{\expandafter\def\expandafter#1\expandafter{#1#2}}

\let\tikz@lib@dec@decorationlist\tikz@nonetext


% The decorate path command:

\def\tikz@lib@decoration[#1]{%
  \begingroup%
    % Just to be on the save side...
    %
    % Now, let's parse the options:
    \tikzset{#1}%
    %
    \ifx\tikz@lib@dec@decorationlist\tikz@nonetext%
    % Ok, let's forget about this decoration stuff...
    \else
      \expandafter\tikz@lib@do@dec%
    \fi%
      % Now, we expect a brace.
      \pgfutil@ifnextchar\bgroup{%
        \begingroup%
          \aftergroup\tikz@enddecoration%
          \afterassignment\tikz@scan@next@command%
          \let\tikz@lib@next% gobble \bgroup%
      }%
      {%
        \PackageError{tikz}{A decoration must begin with a brace}{}%
        \tikz@enddecoration%
      }%
}
\def\tikz@lib@do@dec{%
      % Ok, now what?
      \iftikz@metadecoration%
        \expandafter\pgfmetadecoration\expandafter{\tikz@lib@dec@decorationlist}%
      \else%
        \expandafter\pgfdecoration\expandafter{\tikz@lib@dec@decorationlist}%
      \fi%
      \ifx\pgfdecorateexistingpath\pgfutil@empty%
        \pgfpathmoveto{\pgfqpoint{\the\tikz@lastxsaved}{\the\tikz@lastysaved}}%
      \fi%
}

\def\tikz@enddecoration{%
    \ifx\tikz@lib@dec@decorationlist\tikz@nonetext%
    % ignore this.
    \else
      % Ok, now what?
      \iftikz@metadecoration%
        \endpgfmetadecoration%
      \else%
        \endpgfdecoration%
      \fi%
    \fi
  \endgroup%
}


% The decorate path command:
\def\tikz@lib@dec@decorate@path{%
  \ifx\tikz@lib@dec@decorationlist\tikz@nonetext%
  \else%
    \pgfgetpath\tikz@lib@dec@currentpath%
    \pgfsetpath\pgfutil@empty%
    \iftikz@metadecoration%
      \expandafter\pgfmetadecoration\expandafter{\tikz@lib@dec@decorationlist}%
        \pgfsetpath\tikz@lib@dec@currentpath%
      \endpgfmetadecoration%
    \else%
      \expandafter\pgfdecoration\expandafter{\tikz@lib@dec@decorationlist}%
        \pgfsetpath\tikz@lib@dec@currentpath%
      \endpgfdecoration%
    \fi%
  \fi%
}


% /tikz/decoration=<decoration name>
%
% Decorate the path with a single decoration.
%
% Examples:
%
%	\tikz\draw [decoration=crosses]	
%		(0,0) ..controls (0,2) and (3,0) .. (3,2);
%		
%	
%	\tikz\draw (-1,0) -- (0,0) 
%		[decoration=zigzag]	.. controls (0,2) and (3,0) .. (3,2) 
%		[decoration=none]   -- (4,2);
%
%	\pgfdeclaredecoration{stars}{move}{
%		\state{move}[width=7.5pt, next state=star]{}
%	  \state{star}[width=7.5pt, next state=move]
%	  {
%	  	\pgfmathparse{round(rnd*100)}
%	  	\pgfsetfillcolor{yellow!\pgfmathresult!orange}
%	    \pgfsetstrokecolor{yellow!\pgfmathresult!red}
%	    \pgfmathparse{rnd*.75+.25}
%	    \pgftransformscale{\pgfmathresult}
%	    \pgfnode{star}{center}{}{}{\pgfusepath{stroke,fill}}
%	  }
%	  \state{final}
%	  {
%	  	\pgfpathmoveto{\pgfpointdecoratedpathlast}
%	  }
%	}
%	
%	\tikz\path[decoration=stars, rotate=45]
%		 (0,0) .. controls (0,2)  and (3,2)  .. (3,0)
%		       .. controls (3,-3) and (0,0)  .. (0,-3)
%		       .. controls (0,-5) and (3,-5) .. (3,-3); 
%
\tikzset{%
  decoration/.code={%
    \def\tikz@lib@dec@decorationlist{#1}%
    \ifx\tikz@lib@dec@decorationlist\tikz@nonetext%
    \else%
      \pgfifdecoration{#1}
      {     
        \def\tikz@lib@dec@decorationlist{{#1}{\pgfdecoratedpathlength}{}{}}%
        \tikz@metadecorationfalse%
      }
      {
        \pgfifmetadecoration{#1}
        {
          \tikz@metadecorationtrue%
        }
        {
          \PackageError{tikz}{Unknown (meta-)decoration '#1'. Perhaps
            you misspelled it?}{}
          \let\tikz@lib@dec@decorationlist=\tikz@nonetext
        }
      }
    \fi%
  }
}





% Old snakes stuff:

\tikzoption{snake}[]{%
  \def\tikz@@snake{#1}%
  \ifx\tikz@@snake\pgfutil@empty%
    \tikz@snakedtrue%
  \else%
    \ifx\tikz@@snake\tikz@nonetext%
      \tikz@snakedfalse%
    \else%
      \tikz@snakedtrue%
      \let\tikz@snake=\tikz@@snake%
    \fi%
  \fi}

\tikzoption{raise snake}{\def\pgf@snake@raise{\pgftransformyshift{#1}}}
\tikzoption{mirror snake}[true]{%
  \csname if#1\endcsname
    \def\pgf@snake@mirror{\pgftransformyscale{-1}}%
  \else%
    \let\pgf@snake@mirror=\pgfutil@empty%
  \fi
}

\tikzoption{gap before snake}{\def\tikz@presnake{{moveto}{#1}}}
\tikzoption{line before snake}{\def\tikz@presnake{{lineto}{#1}}}

\tikzoption{gap after snake}{\def\tikz@postsnake{{moveto}{#1}}\def\tikz@mainsnakelength{\pgfsnakeremainingdistance+-#1}}
\tikzoption{line after snake}{\def\tikz@postsnake{{lineto}{#1}}\def\tikz@mainsnakelength{\pgfsnakeremainingdistance+-#1}}

\tikzoption{gap around snake}{%
  \def\tikz@presnake{{moveto}{#1}}%
  \def\tikz@postsnake{{moveto}{#1}}%
  \def\tikz@mainsnakelength{\pgfsnakeremainingdistance+-#1}%
}
\tikzoption{line around snake}{%
  \def\tikz@presnake{{lineto}{#1}}%
  \def\tikz@postsnake{{lineto}{#1}}%
  \def\tikz@mainsnakelength{\pgfsnakeremainingdistance+-#1}%
}
\let\pgf@snake@mirror=\pgfutil@empty
\let\pgf@snake@raise=\pgfutil@empty

\pgfsetsnakesegmenttransformation{\pgf@snake@mirror\pgf@snake@raise}

\def\tikz@snake{zigzag}

\let\tikz@presnake=\pgfutil@empty
\let\tikz@postsnake=\pgfutil@empty
\def\tikz@mainsnakelength{\pgfsnakeremainingdistance}


\tikzset{shape snake transform/.code={%
    {%
      \pgftransformreset%
      \def\tikz@transform{}%
      \pgfkeysalso{/tikz/.cd, #1}%
      \expandafter\gdef\expandafter\tikz@g@temptransform\expandafter{\tikz@transform}%
    }%
    \expandafter\pgfsetsnakesegmenttransformation\expandafter{\tikz@g@temptransform}%
  }%
}

\tikzstyle{snake triangles 45}=      [snake=triangles,segment object length=2.41421356\pgfsnakesegmentamplitude]
\tikzstyle{snake triangles 60}=      [snake=triangles,segment object length=1.73205081\pgfsnakesegmentamplitude]
\tikzstyle{snake triangles 90}=      [snake=triangles,segment object length=\pgfsnakesegmentamplitude]



\endinput
