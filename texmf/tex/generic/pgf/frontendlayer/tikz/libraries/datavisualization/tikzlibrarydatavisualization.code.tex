% Copyright 2008 by Till Tantau
%
% This file may be distributed and/or modified
%
% 1. under the LaTeX Project Public License and/or
% 2. under the GNU Public License.
%
% See the file doc/generic/pgf/licenses/LICENSE for more details.

\ProvidesFileRCS[v\pgfversion] $Header: /home/mojca/cron/mojca/github/cvs/pgf/pgf/generic/pgf/frontendlayer/tikz/libraries/datavisualization/tikzlibrarydatavisualization.code.tex,v 1.9 2008/11/24 18:31:22 tantau Exp $

\usepgfmodule{datavisualization}


\tikzset{/tikz/data visualization/.is family,
  /tikz/data visualization/.unknown/.code={
    \let\tikz@dv@key\pgfkeyscurrentname% 
    \pgfkeys{/tikz/\tikz@dv@key/.try={#1}}%
    \ifpgfkeyssuccess%
    \else%
      \pgfkeys{/errors/unknown key={/tikz/data visualization/\tikz@dv@key}{#1}}%
    \fi%
  },
  /tikz/data visualization/data/.unknown/.code={%
    % Redirect to /pgf/data
    \let\tikz@dv@key\pgfkeyscurrentname% 
    \pgfkeys{/pgf/data/\tikz@dv@key/.try={#1}}%
    \ifpgfkeyssuccess%
    \else%
      \pgfkeys{/errors/unknown key={/pgf/data/\tikz@dv@key}{#1}}%
    \fi%
  }
}

\def\tikzdatavisualizationset{\pgfqkeys{/tikz/data visualization}}




% The main \datavisualization command
%
% This command must, as always, be given inside a tikz picture. It
% will add a data visualization to the picture; use "shift" option and
% friends to place the data visualization somewhere other than at the
% origin.
%
% The \datavisualization is followed by a sequence of blocks and ended
% with a semicolon. Each block may be one of the following:
%
% "data" blocks.
%
%   Syntax: data [options]                  % options specify an external source
%   Syntax: data [options] { inline data }  
%
%   The optional arguments may either specify an
%   external data source or the data may follow inline.
%
%   The options are executed for the path /pgf/data. The style
%   /tikz/every data is executed for each data, which can be useful to
%   generally set, say, a certain data format.
%
% "infos" blocks.
%
%   Syntax: infos [options] {code}
%
%   A block starting with "infos" may contain any code. It
%   will be executed after the visualization. In order to execute code
%   at some other time (like, say, before the visualization), use
%   options like "before visualization".
%
% Options blocks.
%
%   Syntax: [options]
%
%   The options will be executed immediately.
%
% Scope blocks.
%
%   Syntax: scope[options] { blocks }
%
%   The blocks are executed inside a scope for which the options are
%   set. This can be used to group data blocks.
%
% Examples:
%
% \datavisualization[schoolbook plot]
%   data [source=my_data_file.cvs,format=comma separated columns]
% ;
%
%  \datavisualization[schoolbook plot]
%    data [format=key value pairs]
%    {
%      x=0, y=0
%      x=1, y=1
%      x=2, y=4
%      x=3, y=9
%    };
%
%  \datavisualization[schoolbook plot]
%    data [format=function]
%    {
%      var x: interval [0:3];
%      func y = \value x*\value x;
%    }
%    [before visualization={
%      \fill [black!10] (visualization cs:x=0,y=0) -- (visualization cs:x=3,y=9);
%    }]
%    infos[red]
%    {
%      \draw (visualization cs:x=0,y=0) -- (visualization cs:x=3,y=9);
%    };
%
%  \datavisualization[schoolbook plot,
%                     every data/.style={format=comma separated columns}]
%    data [/data point/label=first experiment,source=file_1]
%    data [/data point/label=second experiment,source=file_2]
%    data [/data point/label=third experiment,source=file_3]
%    data [/data point/label=prediction,format=function]
%      { var x: interval [0,1]; func y = rand(\value x); }
%    ;

\def\tikz@lib@datavisualization{
  \begingroup%
    % Ok, first, start a new data visualization
    \pgfoonew \tikz@main@dv=new data visualization()%
    %
    \pgfset{/pgf/data/continue code=\tikz@lib@dv@parse@loop}%
    % Now enter parse loop
    \tikz@lib@dv@parse@loop
}

\def\tikz@lib@dv@parse@loop{%  
  \pgfutil@ifnextchar d\tikz@lib@dv@handle@data{%
    \pgfutil@ifnextchar ;\tikz@lib@dv@parse@end{%
      \pgfutil@ifnextchar \par\tikz@lib@dv@handle@par{%
        \pgfutil@ifnextchar s\tikz@lib@dv@handle@beginscope{%
          \pgfutil@ifnextchar i\tikz@lib@dv@handle@info{%
            \pgfutil@ifnextchar [\tikz@lib@dv@handle@options{%
              \pgfutil@ifnextchar \egroup\tikz@lib@dv@handle@endscope{%
                \PackageError{tikz}{Semicolon expected}{}%
                \endgroup%
              }%
            }%
          }%
        }%
      }%
    }%
  }%
}
\def\tikz@lib@dv@parse@end{%
    % Go!
    \tikz@main@dv.survey()%
    \tikz@main@dv.visualize()%
  \endgroup%
}
\def\tikz@lib@dv@handle@par\par{\tikz@lib@dv@parse@loop}

\def\tikz@lib@dv@handle@options[#1]{%
  \tikzdatavisualizationset{#1}%
  \tikz@lib@dv@parse@loop%
}

\def\tikz@lib@dv@handle@beginscope scope{%
  \begingroup%
    \pgfutil@ifnextchar[\tikz@lib@dv@beg@opt{\tikz@lib@dv@beg@opt[]}%}
}
\def\tikz@lib@dv@beg@opt[#1]{%
  \pgfkeys{/pgf/data/.cd,/pgf/every data/.try,#1}%
  \pgfkeysvalueof{/pgf/data visualization/obj}.add data({{\begingroup\pgfkeys{/pgf/data/.cd,/pgf/every data/.try,#1}}})%
  \pgfutil@ifnextchar\bgroup{
    \afterassignment\tikz@lib@dv@parse@loop%
    \let\tikz@dummy=%get rid of \bgroup
  }{%
    \PackageError{pgf}{Opening brace expected}{}%
  }%
}
\def\tikz@lib@dv@handle@endscope{%
    \pgfkeysvalueof{/pgf/data visualization/obj}.add data(\endgroup)%
  \endgroup%
  \afterassignment\tikz@lib@dv@parse@loop%
  \let\tikz@dummy=%get rid of \egroup
}

\def\tikz@lib@dv@handle@data data{\pgfdata}

\def\tikz@lib@dv@handle@info infos{
  \pgfutil@ifnextchar[{\tikz@lib@dv@handle@info@block@opt}{\tikz@lib@dv@handle@info@block@opt[]}}%}
\def\tikz@lib@dv@handle@info@block@opt[#1]#2{%
  \tikz@main@dv.after visualization({{\scope[#1]#2\endscope}})%
  \tikz@lib@dv@parse@loop
}


\pgfset{/pgf/every data/.style={/tikz/every data/.try,/tikz/data visualization/every data/.try}}


%
% Performing before/after stuff via keys
% 

\tikzdatavisualizationset{
  before survey/.code=\tikz@main@dv.before survey({{#1}}),
  after survey/.code=\tikz@main@dv.after survey({{#1}}),
  before visualization/.code=\tikz@main@dv.before visualization({{#1}}),
  after visualization/.code=\tikz@main@dv.after visualization({{#1}}),
  at start survey/.code=\tikz@main@dv.at start survey({{#1}}),
  at end survey/.code=\tikz@main@dv.at end survey({{#1}}),
  at start visualization/.code=\tikz@main@dv.at start visualization({{#1}}),
  at end visualization/.code=\tikz@main@dv.at end visualization({{#1}}),
}



%
% Object setup 
%

% The following key is used to create objects for the rendering
% pipeline. They cannot be created "directly" because when it is known
% that the objects needs to be created, in principle, many keys may
% not yet be known.
%
% The following options are used:
%
% store           = key that will store the object (handle)
% class           = class of the object
% arg1            = argument1 of the constructor
% ...
% arg8            = argument8 of the constructor
% when            = specifies when the object will be created. Defaults to
%                   "before survey", other possible values are "after survey" and
%                   "before/after visualization"
% before creation = code to be executed just before the object is
%                   created
% after creation  = code to be executed just after the object has been created
%
% The following styles may be useful:
%
% arg1 from key        = use the contents of the given key as arg1. Similar
%                        for other args          
% arg1 handle from key = the contents of the given key should contain
%                        an object. Then arg1 will be a handle to this
%                        object. Similar for other args          

\tikzdatavisualizationset{%
  new object/.code={%
    \def\tikz@dv@grabbed@when{before survey}%
    \let\tikz@dv@grabbed@store\pgfutil@empty
    \pgfkeys{/tikz/data visualization/new object/grab/.cd,#1}%
    \ifx\tikz@dv@grabbed@store\pgfutil@empty
      \PackageError{tikz}{Internal error: new object must be stored to
        avoid duplicated on repeated styles}{}
    \else
      \pgfkeyslet{\tikz@dv@grabbed@store}\pgfutil@empty%
    \fi
    \edef\tikz@marshal{\noexpand\tikz@main@dv.\tikz@dv@grabbed@when(\noexpand\tikz@dv@newobject{\tikz@dv@grabbed@store}}%
    \tikz@marshal{#1})%
  },
  new object/grab/store/.store in=\tikz@dv@grabbed@store,
  new object/grab/when/.store in=\tikz@dv@grabbed@when,
  new object/grab/.unknown/.code={},%ignore
}

\def\tikz@dv@newobject#1#2{%
  \pgfkeysgetvalue{#1}\tikz@dv@temp
  \ifx\tikz@dv@temp\pgfutil@empty
    \let\tikz@dv@arg@a=\tikz@lib@notused%
    \let\tikz@dv@arg@b=\tikz@lib@notused%
    \let\tikz@dv@arg@c=\tikz@lib@notused%
    \let\tikz@dv@arg@d=\tikz@lib@notused%
    \let\tikz@dv@arg@e=\tikz@lib@notused%
    \let\tikz@dv@arg@f=\tikz@lib@notused%
    \let\tikz@dv@arg@g=\tikz@lib@notused%
    \let\tikz@dv@arg@h=\tikz@lib@notused%
    \let\tikz@dv@new@after=\relax%
    \pgfkeys{/tikz/data visualization/new object/parse/.cd,#2}
    \edef\pgf@marshal{\noexpand\pgfoonew\noexpand\tikzdvobj=new \tikz@dv@new@class(}%
    \tikz@dv@add@arg{}\tikz@dv@arg@a%
    \tikz@dv@add@arg{\expandafter,}\tikz@dv@arg@b%
    \tikz@dv@add@arg{\expandafter,}\tikz@dv@arg@c%
    \tikz@dv@add@arg{\expandafter,}\tikz@dv@arg@d%
    \tikz@dv@add@arg{\expandafter,}\tikz@dv@arg@e%
    \tikz@dv@add@arg{\expandafter,}\tikz@dv@arg@f%
    \tikz@dv@add@arg{\expandafter,}\tikz@dv@arg@g%
    \tikz@dv@add@arg{\expandafter,}\tikz@dv@arg@h%
    \expandafter\def\expandafter\pgf@marshal\expandafter{\pgf@marshal)}%
    \pgf@marshal%
    \tikzdvobj.default connects()%
    \pgfkeyslet{#1}\tikzdvobj%
    \tikz@dv@new@after%
  \fi
}       
\def\tikz@lib@notused{\tikz@lib@notused}

\def\tikz@dv@add@arg#1#2{%
  \ifx#2\tikz@lib@notused%
  \else%
    \expandafter\expandafter\expandafter\def%
    \expandafter\expandafter\expandafter\pgf@marshal%
    \expandafter\expandafter\expandafter{\expandafter\pgf@marshal#1#2}%
  \fi%
}

\tikzdatavisualizationset{%
  new object/parse/.cd,
  store/.code=,% ignore
  when/.code=,% ignore
  class/.store in=\tikz@dv@new@class,
  before creation/.code=#1,
  after creation/.store in=\tikz@dv@new@after,
  arg1/.store in=\tikz@dv@arg@a,
  arg2/.store in=\tikz@dv@arg@b,
  arg3/.store in=\tikz@dv@arg@c,
  arg4/.store in=\tikz@dv@arg@d,
  arg5/.store in=\tikz@dv@arg@e,
  arg6/.store in=\tikz@dv@arg@f,
  arg7/.store in=\tikz@dv@arg@g,
  arg8/.store in=\tikz@dv@arg@h,
  arg1 from key/.code=\pgfkeysgetvalue{#1}{\tikz@dv@arg@a},
  arg2 from key/.code=\pgfkeysgetvalue{#1}{\tikz@dv@arg@b},
  arg3 from key/.code=\pgfkeysgetvalue{#1}{\tikz@dv@arg@c},
  arg4 from key/.code=\pgfkeysgetvalue{#1}{\tikz@dv@arg@d},
  arg5 from key/.code=\pgfkeysgetvalue{#1}{\tikz@dv@arg@e},
  arg6 from key/.code=\pgfkeysgetvalue{#1}{\tikz@dv@arg@f},
  arg7 from key/.code=\pgfkeysgetvalue{#1}{\tikz@dv@arg@g},
  arg8 from key/.code=\pgfkeysgetvalue{#1}{\tikz@dv@arg@h}
  arg1 handle from key/.code=\tikz@dv@handle@from@key{#1}{\tikz@dv@handle@a}{\tikz@dv@arg@a},
  arg2 handle from key/.code=\tikz@dv@handle@from@key{#1}{\tikz@dv@handle@b}{\tikz@dv@arg@b},
  arg3 handle from key/.code=\tikz@dv@handle@from@key{#1}{\tikz@dv@handle@c}{\tikz@dv@arg@c},
  arg4 handle from key/.code=\tikz@dv@handle@from@key{#1}{\tikz@dv@handle@d}{\tikz@dv@arg@d},
  arg5 handle from key/.code=\tikz@dv@handle@from@key{#1}{\tikz@dv@handle@e}{\tikz@dv@arg@e},
  arg6 handle from key/.code=\tikz@dv@handle@from@key{#1}{\tikz@dv@handle@f}{\tikz@dv@arg@f},
  arg7 handle from key/.code=\tikz@dv@handle@from@key{#1}{\tikz@dv@handle@g}{\tikz@dv@arg@g},
  arg8 handle from key/.code=\tikz@dv@handle@from@key{#1}{\tikz@dv@handle@h}{\tikz@dv@arg@h}
}

\def\tikz@dv@handle@from@key#1#2#3{%
  \pgfkeysvalueof{#1}.get handle(#2)%
  \def#3{#2}%
}




%
% Data visualization coordinate system
%
%
% This cs is used to refer to points in a datavisualization. The
% parameters are keys that are set in the /data point key
% directory. Then, a (virtual) data point is created and the
% calculated position is returned

\tikzdeclarecoordinatesystem{visualization}
{%
  \tikzset{/data point/.cd,#1}%
  \pgfcanvaspositionofvirtualdatapoint%
}




%
%
% Axes
%
%



%
% Axis base
%

\tikzdatavisualizationset{%
  new axis/.style={
    new object={
      class=scaling mapper,
      store=/tikz/data visualization/#1/scaling mapper,
      before creation={
        \pgfkeysgetvalue{/tikz/data visualization/#1/scaling}\tikz@temp
        \ifx\tikz@temp\pgfutil@empty%
          \pgfkeysgetvalue{/tikz/data visualization/#1/scaling/default}\tikz@temp
          \pgfkeyslet{/tikz/data visualization/#1/scaling}\tikz@temp
        \fi
      },
      arg1 from key=/tikz/data visualization/#1/attribute,
      arg2/.expanded=\pgfkeysvalueof{/tikz/data visualization/#1/attribute}/scaled,
      arg3 from key=/tikz/data visualization/#1/scaling,
      arg4 from key=/tikz/data visualization/#1/function
    },
    #1/attribute/.initial=#1,
    #1/function/.initial=,
    #1/scaling/.initial=,
    #1/scaling/default/.initial=0 at 0 and 1 at 1,
    #1/ticks/.initial=,
    #1/minor ticks/.initial=,
    #1/subminor ticks/.initial=,
    #1/ticks direction axis/.initial=,
    #1/.code={
      \def\tikz@dv@axis{/tikz/data visualization/#1}
      \pgfkeys{/tikz/data visualization/axis options/.cd,##1}}
  },
  % General styling
  visual/.style=,
  % General extends (both axis and ticks)
  min extend/.initial=0pt,
  max extend/.initial=0pt,
  % Special ticks keys:
  ticks direction axis/.initial=,
  at/.initial=\tikz@dv@tick@list,
  % Default ticks styles
  every ticks/.style={min extend=-2pt,max extend=2pt,visual/.style={line cap=round}},
  every minor ticks/.style={visual/.style={help lines,thin,line cap=round},min extend=-1.4pt,max extend=1.4pt},
  every subminor ticks/.style={visual/.style={help lines,line cap=round},min extend=-0.8pt,max extend=0.8pt},
}

\tikzset{
  /tikz/data visualization/axis options/.cd,
  %
  % Basic setters
  %
  attribute/.style={\tikz@dv@axis/attribute={#1}},
  function/.style={\tikz@dv@axis/function={#1}},
  scaling/.style={\tikz@dv@axis/scaling={#1}},
  log axis/.style={
    \tikz@dv@axis/function=\pgfmathparse{ln(\pgfvalue)},
    \tikz@dv@axis/scaling/default=1 at 0 and 10 at 1
  },
  %
  % Attribute setter
  %
  goto/.code={\pgfkeysvalueof{\tikz@dv@axis/scaling mapper}.set in to(#1)},
  %
  % Visualization setters
  %
  visualize/.code=\expandafter\tikz@lib@dv@av\expandafter{\tikz@dv@axis}{#1},
  %
  % Ticks settings
  %
  ticks/.style={\tikz@dv@axis/ticks={#1}},
  minor ticks/.style={\tikz@dv@axis/minor ticks={#1}},
  subminor ticks/.style={\tikz@dv@axis/subminor ticks={#1}},
  visualize ticks/.code=\expandafter\tikz@lib@dv@tv\expandafter{\tikz@dv@axis}{#1},
}

% Axis visualization
%
% #1 = name of the to-be-visualized axis name
% #2 = options for the visualization

\def\tikz@lib@dv@av#1#2{
  \tikzdatavisualizationset{
    after visualization=%
    {
      \let\tikz@dv@s@temp=\pgfutil@empty
      \scope[/tikz/data visualization/.cd,every axis/.try,#2]
        \pgfkeysvalueof{#1/scaling mapper}.set in to(min)
        \pgfpathdvmoveto%
        \pgfkeysvalueof{#1/scaling mapper}.set in to(max)
        \pgfpathdvlineto%
        \path[draw,/tikz/data visualization/visual];
      \endscope
    }
  }
}


% Ticks visualization 
%
% #1 = name of axis on which ticks should be shown
% #2 = options for the visualization

\def\tikz@lib@dv@tv#1#2{
  \tikzdatavisualizationset{
    after visualization=%
    {
      \tikz@lib@dv@show@ticks{subminor ticks}{#1}{#2}
      \tikz@lib@dv@show@ticks{minor ticks}{#1}{#2}
      \tikz@lib@dv@show@ticks{ticks}{#1}{#2}
    }
  }
}

\def\tikz@lib@dv@show@ticks#1#2#3{%
  % First, check whether there is anything to do at all:
  \pgfkeysgetvalue{#2/#1}\tikz@dv@ticks@temp%
  \ifx\tikz@dv@ticks@temp\pgfutil@empty%
    % Great, nothing to do
  \else
    {%
      \let\tikz@dv@s@temp=\pgfutil@empty
      \scope[/tikz/data visualization/.cd,every tick/.try,every #1/.try,#3]
        % Setup options
        \def\pgf@temp{\pgfqkeys{/tikz/data visualization}}
        \expandafter\pgf@temp\expandafter{\tikz@dv@ticks@temp}
        \pgfkeysgetvalue{/tikz/data visualization/at}\tikz@dv@tick@list
        \pgfkeysgetvalue{#2/scaling mapper}\pgf@dv@tick@mapper
        % Ok, now it's time to draw the ticks!
        \foreach \tikz@dv@tick@pos in \tikz@dv@tick@list 
        {
          \pgf@dv@tick@mapper.set in to(\tikz@dv@tick@pos)%
          % First, compute position of tick:
          \pgf@process{\pgfpointdvlocaldatapoint}
          \xdef\pgf@dv@tick@origin{\noexpand\pgfqpoint{\the\pgf@x}{\the\pgf@y}}
          % Ok, calculate direction vector:
          \tikzpointandanchordirection
          {\pgfkeysvalueof{/tikz/data visualization/\pgfkeysvalueof{/tikz/data visualization/ticks direction axis}/scaling mapper}.set in to(min)}
          {\pgfkeysvalueof{/tikz/data visualization/\pgfkeysvalueof{/tikz/data visualization/ticks direction axis}/scaling mapper}.set in to(max)}
          \xdef\pgf@dv@tick@dir{\noexpand\pgfqpoint{\the\pgf@x}{\the\pgf@y}}
          % Now, show something:
          \pgf@process{\pgfpointadd{\pgf@dv@tick@origin}{\pgfpointscale{\pgfkeysvalueof{/tikz/data visualization/max extend}}{\pgf@dv@tick@dir}}}
          \xdef\tikz@dv@max@tick{\the\pgf@x,\the\pgf@y}
          \pgf@process{\pgfpointadd{\pgf@dv@tick@origin}{\pgfpointscale{\pgfkeysvalueof{/tikz/data visualization/min extend}}{\pgf@dv@tick@dir}}}
          \xdef\tikz@dv@min@tick{\the\pgf@x,\the\pgf@y}
          \draw[/tikz/data visualization/visual] (\tikz@dv@min@tick) -- (\tikz@dv@max@tick) node [anchor=\tikz@dv@max@anchor] {\tikz@dv@tick@pos};
        }
      \endscope
    }
  \fi
}


% Help function
\def\tikzpointandanchordirection#1#2{%
  % This function works like pgfpointdvdirection, but also computes
  % appriate "min" and "max" anchors.
  \pgf@process{\pgfpointdvdirection{#1}{#2}}
  {
    \pgf@ya=\pgf@y
    \pgf@y=-\pgf@x
    \pgf@x=\pgf@ya
    \tikz@auto@anchor
    \xdef\tikz@dv@max@anchor{\tikz@anchor}
    \tikz@auto@anchor@prime
    \xdef\tikz@dv@min@anchor{\tikz@anchor}
  }
}


%
% Cartesian axes
%

\tikzdatavisualizationset{%
  new Cartesian axis/.style={
    new axis={#1},
    new object={
      class=linear transformer,
      store=/tikz/data visualization/#1/linear transformer,
      arg1/.expanded=\pgfkeysvalueof{/tikz/data visualization/#1/attribute}/scaled,
      arg2 from key=/tikz/data visualization/#1/unit vector,
    },
    #1/scaling/default/.initial=0 at 0 and 1 at 1cm,
    #1/unit vector/.initial=,
  }
}

\tikzset{
  /tikz/data visualization/axis options/.cd,
  unit vector/.code=\tikz@scan@one@point\tikz@lib@dv@uv#1,
  length/.style={\tikz@dv@axis/scaling=min at 0 and max at #1},
  positive length/.style={\tikz@dv@axis/scaling=0 at 0 and max at #1},
  negative length/.style={\tikz@dv@axis/scaling=min at -#1 and 0 at 0},
  unit length/.style={\tikz@dv@axis/scaling=0 at 0 and 1 at #1},
  power unit length/.style={\tikz@dv@axis/scaling=1 at 0 and 10 at #1}
}

\def\tikz@lib@dv@uv#1{%
  \pgfkeyssetvalue{\tikz@dv@axis/unit vector}{#1}
}



%
%
% Basic axis systems
%
%

\tikzdatavisualizationset{
  xy Cartesian/.style={
    new Cartesian axis=x axis,
    x axis={attribute=x,unit vector={(1pt,0cm)}},
    new Cartesian axis=y axis,
    y axis={attribute=y,unit vector={(0cm,1pt)}}
  },
  xy axes/.style={x axis={#1},y axis={#1}},
  uv Cartesian/.style={
    new Cartesian axis=u axis,
    u axis={attribute=u,unit vector={(1pt,0cm)}},
    new Cartesian axis=v axis,
    v axis={attribute=v,unit vector={(0cm,1pt)}}
  },
  uv axes/.style={u axis={#1},v axis={#1}},
}

\tikzdatavisualizationset{
  xyz Cartesian/.style={
    xy Cartesian,
    new Cartesian axis=z axis,
    z axis={attribute=z,unit vector={(-0.385pt,-0.385pt)}}
  },
  xyz axes/.style={x axis={#1},y axis={#1},z axis={#1}},
  uvw Cartesian/.style={
    uv Cartesian,
    new Cartesian axis=w axis,
    w axis={attribute=w,unit vector={(-0.385pt,-0.385pt)}}
  }
  uvw axes/.style={u axis={#1},v axis={#1},w axis={#1}},
}


%
% Basic plot visualizers
%

\tikzdatavisualizationset{
  new line plot/.style={
    new object={
      when=after survey,
      store=/tikz/data visualization/plot handler visualizer,
      class=plot handler visualizer,
      arg1=\tikz@plot@handler,
      arg2=#1
    },
    /data point/#1/.initial=line plot,
    /data point/#1/use path/.initial=\pgfusepath{stroke},
    /pgf/data/#1/label/.code=\pgfkeysalso{/data point/#1=##1},
    /pgf/data/#1/style/.code=\pgfkeysalso{/data point/#1/use path={\path[draw,every line plot/.try,##1];}},
  },
  new line plot/.default=line plot
}

\tikzdatavisualizationset{
  new rectangles/.style={
    new object={
      store=/tikz/data visualization/rectangle visualizer,
      class=rectangle visualizer,
      arg1 from key=/tikz/data visualization/#1/attribute 1,
      arg2 from key=/tikz/data visualization/#1/attribute 2,
      arg3=#1/use path
    },
    #1/attribute 1/.initial=x,
    #1/attribute 2/.initial=y,
    /data point/#1/use path/.initial=\pgfusepath{stroke},
    /data point/#1/style/.code=\pgfkeysalso{/data point/#1/use path={\path[every rectangles/.try,##1];}}
  },
  new rectangles/.default=rectangles
}




%
% Basic plots
%


% The school book plot
%
% This plot can be used to create plots that look like plots in a
% typical school book: There are two axes called "x axis" and "y axis"
% that meet at the origin, there are arrows at the ends of the axes,
% pointing in the directions of the positive axes. No scaling is done
% by default, rather one unit equals one 1cm. This ensures that the
% even when multiple plots are created, the same scaling will be
% used each time. 

\tikzdatavisualizationset{
  school book plot/.style={
    xy Cartesian,
    x axis={
      visualize={y axis={goto=0},style=->},
    },
    y axis={
      visualize={x axis={goto=0},style=->}
    },
    new line plot,
    every school book plot/.try
  },
  school book 3d plot/.style={
    xyz Cartesian,
    x axis={visualize={y axis={goto=0},z axis={goto=0},style=->}},
    y axis={visualize={x axis={goto=0},z axis={goto=0},style=->}},
    z axis={visualize={y axis={goto=0},z axis={goto=0},style=->}},
    new line plot,
    every school book plot/.try
  }
}

\tikzdatavisualizationset{
  scientific plot/width/.initial=161.8pt,% golden ratio...
  scientific plot/height/.initial=100pt,
  scientific plot/.style={
    xy Cartesian,
    x axis={
      length=\pgfkeysvalueof{/tikz/data visualization/scientific plot/width},
      visualize={y axis={goto=min}},
      visualize={y axis={goto=max}}
    },
    y axis={
      length=\pgfkeysvalueof{/tikz/data visualization/scientific plot/height},
      visualize={x axis={goto=min}},
      visualize={x axis={goto=max}}
    },
    new line plot,
  },
  scientific 3d plot/length/.initial=100pt,
  scientific 3d plot/.style={
    xyz Cartesian,
    x axis={
      length=\pgfkeysvalueof{/tikz/data visualization/scientific 3d plot/length},
      visualize={y axis={goto=min},z axis={goto=min}},
      visualize={y axis={goto=max},z axis={goto=min}},
      visualize={y axis={goto=min},z axis={goto=max}},
      visualize={y axis={goto=max},z axis={goto=max}}
    },
    y axis={
      length=\pgfkeysvalueof{/tikz/data visualization/scientific 3d plot/length},
      visualize={x axis={goto=min},z axis={goto=min}},
      visualize={x axis={goto=max},z axis={goto=min}},
      visualize={x axis={goto=min},z axis={goto=max}},
      visualize={x axis={goto=max},z axis={goto=max}}
    },
    z axis={
      length=\pgfkeysvalueof{/tikz/data visualization/scientific 3d plot/length},
      visualize={x axis={goto=min},y axis={goto=min}},
      visualize={x axis={goto=max},y axis={goto=min}},
      visualize={x axis={goto=min},y axis={goto=max}},
      visualize={x axis={goto=max},y axis={goto=max}}
    },
    new line plot
  }
}


\endinput
