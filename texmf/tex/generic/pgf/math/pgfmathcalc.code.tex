% Copyright 2007 by Mark Wibrow
%
% This file may be distributed and/or modified
%
% 1. under the LaTeX Project Public License and/or
% 2. under the GNU Public License.
%
% See the file doc/generic/pgf/licenses/LICENSE for more details.

% This file loads all the parsing, functions and operator stuff
%
% Version 0.0 08/03/2007

% Copyright 2007 Mark Wibrow
%
% but subject to the LaTeX Project Public License 
% (http://www.latex-project.org/lppl.txt)
%
% and the GNU Public License 
% (http://www.gnu.org/licenses/gpl.txt)
%

% This file defines utilities common to the \pgfmath files.
%
% Version 1.414213 29/9/2007

\def\pgfmathincluded{}

% \pgfmath@ensureregister
%
% Ensure a control sequence exists as a TeX count/dimen register.
%
% #1 - count or dimen.
% #2 - a control sequence.
%
\def\pgfmath@ensureregister#1#2{%
	\begingroup%
	\setbox0=\hbox\bgroup\begingroup% In case #2 is a defined macro. Unlikely, but you never know...
		\afterassignment\pgfmath@@ensureregister%
		\noexpand#20.0pt\let\pgfmath@temp\relax\relax\pgfmath@#2#1\pgfmath@}

\def\pgfmath@@ensureregister#1#2\pgfmath@#3#4#5\pgfmath@{%
		\endgroup\egroup%
	\endgroup%
	\ifx#4c% 
		\ifx#1.% Already a count register.
		\else%
			\csname newcount\endcsname#3% This gets round \outer in plain TeX.
		\fi%
	\else%
		\ifx#4d% 
			\ifx\let#1% Already a dimen register.
			\else%
				\csname newdimen\endcsname#3%
			\fi%
		\fi%
	\fi%
}

% Check registers exist. If not, create them. 
%
\pgfmath@ensureregister{dimen}{\pgf@x}
\pgfmath@ensureregister{dimen}{\pgf@xa}
\pgfmath@ensureregister{dimen}{\pgf@xb}
\pgfmath@ensureregister{dimen}{\pgf@xc}

\pgfmath@ensureregister{dimen}{\pgf@y}
\pgfmath@ensureregister{dimen}{\pgf@ya}
\pgfmath@ensureregister{dimen}{\pgf@yb}
\pgfmath@ensureregister{dimen}{\pgf@yc}

\pgfmath@ensureregister{dimen}{\pgfutil@tempdima}
\pgfmath@ensureregister{dimen}{\pgfutil@tempdimb}

\pgfmath@ensureregister{count}{\c@pgf@counta}
\pgfmath@ensureregister{count}{\c@pgf@countb}
\pgfmath@ensureregister{count}{\c@pgf@countc}
\pgfmath@ensureregister{count}{\c@pgf@countd}

\pgfmath@ensureregister{count}{\pgfutil@tempcnta}
\pgfmath@ensureregister{count}{\pgfutil@tempcntb}

% Alias for \pgfmath.
%
\let\pgfmath@x\pgf@x
\let\pgfmath@xa\pgf@xa
\let\pgfmath@xb\pgf@xb
\let\pgfmath@xc\pgf@xc

\let\pgfmath@y\pgf@y
\let\pgfmath@ya\pgf@ya
\let\pgfmath@yb\pgf@yb
\let\pgfmath@yc\pgf@yc

\let\pgfmath@tempdima\pgfutil@tempdima
\let\pgfmath@tempdima\pgfutil@tempdima

\let\c@pgfmath@counta\c@pgf@counta
\let\c@pgfmath@countb\c@pgf@countb
\let\c@pgfmath@countc\c@pgf@countc
\let\c@pgfmath@countd\c@pgf@countd

\let\pgfmath@tempcnta\pgfutil@tempcnta
\let\pgfmath@tempcntb\pgfutil@tempcntb

% Now check if pgfutil and pgf@sys stuff exists. If it does, fine.
% Just \let pgfmath versions. If not, define pgfmath versions (Arghhh). 
%

% \...ifnextchar
%
\ifx\pgfutil@ifnextchar\pgfmath@undefined%
	\long\def\pgfmath@ifnextchar#1#2#3{%
  \let\pgfmath@reserved@d#1%
  \def\pgfmath@reserved@a{#2}%
  \def\pgfmath@reserved@b{#3}%
  \futurelet\pgfmath@let@token\pgfmath@ifnch}
	\def\pgfmath@ifnch{%
	  \ifx\pgfmath@let@token\pgfmath@sptoken%
	    \let\pgfmath@reserved@c\pgfmath@xifnch%
	  \else%
	    \ifx\pgfmath@let@token\pgfmath@reserved@d%
	      \let\pgfmath@reserved@c\pgfmath@reserved@a%
	    \else%
	      \let\pgfmath@reserved@c\pgfmath@reserved@b%
	    \fi%
	  \fi%
	  \pgfmath@reserved@c}
	{%
	  \def\:{\global\let\pgfmath@sptoken= } \:
	  \def\:{\pgfmath@xifnch} \expandafter\gdef\: {\futurelet\pgfmath@let@token\pgfmath@ifnch}
	}%
\else%
	\let\pgfmath@ifnextchar\pgfutil@ifnextchar%
	% Still need to define \pgfmath@sptoken (cannot \let).
	{%
		\def\:{\global\let\pgfmath@sptoken= } \:
	}%
\fi%

% \...ifundefined
%
\ifx\pgfutil@ifundefined\pgfmath@undefined%
	\long\def\pgfmath@ifundefined#1{%
	  \expandafter\ifx\csname#1\endcsname\relax%
	    \expandafter\pgfmath@firstoftwo%
	  \else%
	    \expandafter\pgfmath@secondoftwo%
	  \fi}%
\else%
	\let\pgfmath@ifundefined\pgfutil@ifundefined%
\fi%

% \...selectfont
%
\ifx\pgfutil@selectfont\pgfmath@undefined%
	\ifx\selectfont\pgfmath@undefined%
		\let\pgfmath@selectfont\rm% Plain TeX and ConTeXt.
	\else%
		\let\pgfmath@selectfont\selectfont% LaTeX.
	\fi%
\else%
	\let\pgfmath@selectfont\pgfutil@selectfont%
\fi%

% New definition easier than fussing around with let.
%
\newif\ifpgfmath@in@
\def\pgfmath@in@#1#2{%
 \def\pgfmath@in@@##1#1##2##3\pgfmath@in@@{%
  \ifx\pgfmath@in@##2\pgfmath@in@false\else\pgfmath@in@true\fi}%
 \pgfmath@in@@#2#1\pgfmath@in@\pgfmath@in@@}


% \...tonumber
%
\ifx\pgf@sys@tonumber\pgfmath@undefined%
  {\catcode`\p=12\catcode`\t=12\gdef\Pgf@geT#1pt{#1}}
	\def\pgfmath@tonumber#1{\expandafter\Pgf@geT\the#1}
\else%
	\let\pgfmath@tonumber\pgf@sys@tonumber%
\fi%


% \PackageWarning and \PackageError
%
\ifx\PackageWarning\pgfmath@undefined
	\def\pgfmath@PackageWarning#1#2{\immediate\write-1{Package #1: Warning! #2.}}%
\else%
	\let\pgfmath@PackageWarning\PackageWarning%
\fi%
\ifx\PackageError\pgfmath@undefined
	\def\pgfmath@PackageError#1#2#3{\immediate\write-1{Package #1: Error! #2.}}%
\else%
	\let\pgfmath@PackageError\PackageError%
\fi%

\def\pgfmath@error#1#2{%
	\pgfmath@PackageError{PGF Math}{%
		#1%
		\pgfutil@ifundefined{pgfmath@expression}{}{ (in '\pgfmath@expression')}%
	}{#2}%
}
\def\pgfmath@warning#1{%
	\pgfmath@PackageWarning{PGF Math}{%
		#1%
		\pgfutil@ifundefined{pgfmath@expression}{}{ (in '\pgfmath@expression')}%
	}%
}

% \pgfmath@pt
%
% Needed to test for 'pt' resulting from expansion using \the.
%
{\catcode`\p=12\catcode`\t=12\gdef\PgfmaTh@PT{pt}}		
\edef\pgfmath@pt{\PgfmaTh@PT}%

% \pgfmath@empty
%
% A handy macro.
%
\def\pgfmath@empty{}

\def\pgfmath@firstoftwo#1#2{#1}
\def\pgfmath@secondoftwo#1#2{#2}

% \pgfmath@gobbletilpgfmath@
%
% Gobble stream until \pgfmath@ {which is undefined}.
% 
\def\pgfmath@gobbletilpgfmath@#1\pgfmath@{}
\def\pgfmath@gobbleone#1{}%


\def\pgfmath@namedef#1{\expandafter\def\csname#1\endcsname}
\def\pgfmath@nameedef#1{\expandafter\edef\csname#1\endcsname}

% \pgfmathloop
%
% A version of the standard TeX and LaTeX
% \loop, with an additional macro \pgfmathcounter
% (which is *not* a TeX counter) which keeps
% track of the iterations.
%
\newif\ifpgfmathcontinueloop
\def\pgfmathloop#1\repeatpgfmathloop{%
	\def\pgfmathcounter{1}%
	\def\pgfmath@iterate{%
		#1\relax%
		{% Do this inside a group, just in case...
			\c@pgfmath@counta\pgfmathcounter%
			\advance\c@pgfmath@counta1\relax%
			\xdef\pgfmathloop@temp{\the\c@pgfmath@counta}%
		}%	
		\edef\pgfmathcounter{\pgfmathloop@temp}%
		\expandafter\pgfmath@iterate\fi}%
	\pgfmath@iterate\let\pgfmath@iterate\relax}
\let\repeatpgfmathloop\fi
\def\pgfmathbreakloop{\let\pgfmath@iterate\relax}%

% \pgfmath@returnone
%
% #1 - a dimension representing a number or dimension.
%
% Changed by TT: made faster and simpler since it is used
% *heavily*. #1 must be something that can be assigned to a
% dimension. 
%
\def\pgfmath@returnone#1\endgroup{%
  \pgfmath@x#1%
  \edef\pgfmath@temp{\pgfmath@tonumber{\pgfmath@x}}%
  \expandafter\endgroup\expandafter\def\expandafter\pgfmathresult\expandafter{\pgfmath@temp}%
}

\let\pgfmathreturn=\pgfmath@returnone

% \pgfmath@smuggleone
%
% Smuggle a macro outside a group.
%
% Changed by TT: Speedup by insisting, that smuggleone is directly
% followed by \endgroup
%
\def\pgfmath@smuggleone#1\endgroup{%
  \expandafter\endgroup\expandafter\def\expandafter#1\expandafter{#1}}  

\let\pgfmathsmuggle=\pgfmath@smuggleone

% This file parses/evaluates a decimal expression.
%
% (c) 2007 Mark Wibrow
%
% but subject to the LaTeX Project Public License, 
% (http://www.latex-project.org/lppl.txt)
%
% and the GNU Public License,
% (http://www.gnu.org/licenses/gpl.txt)

% \pgfmathparse, \pgfmathqparse
%
% Evaluates a decimal expression.
%
% #1 - the expression.
%
% returns
%
% x = the result as a dimension.
%
% E.g.
% \pgfmathparse{3pt*2cm+1.5}
% \pgfmathqparse{3pt*2cm+1.5pt}
%
% Every number in \pgfmathqparse *must*
% specify a unit.
%
\newif\ifpgfmath@quickparse

\def\pgfmathparse{%
	\pgfmath@quickparsefalse%
	\pgfmathparse@}

\def\pgfmathqparse{%
	\pgfmath@quickparsetrue%
	\pgfmathparse@}
		
\def\pgfmath@calcreal#1{#1}%
\def\pgfmath@calcminof#1#2{min(#1,#2)}%
\def\pgfmath@calcmaxof#1#2{max(#1,#2)}%
\def\pgfmath@calcratio#1#2{#1/#2}%

\def\pgfmathparse@#1{%
  \begingroup%    
    % Reinstall correct font, so that dimensions like 1em are correct
    \pgfutil@selectfont%
    \let\real\pgfmath@calcreal%
    \let\minof\pgfmath@calcminof%
    \let\maxof\pgfmath@calcmaxof%
    \let\ratio\pgfmath@calcratio%
    \edef\pgfmath@temp{#1}%
    \pgfmath@resetparsingparameters%
    \global\pgfmathunitsdeclaredfalse%
    \ifpgfmath@quickparse%
      \let\pgfmath@parseoperand\pgfmath@quickparseoperand%
    \else%
      \let\pgfmath@parseoperand\pgfmath@parseoperand%
    \fi%
    \let\pgfmath@parsepostgroup\pgfmath@parseoperator%
    \expandafter\pgfmath@parse@\pgfmath@temp @@@@@@@@@@@\pgfmath@empty}


% \pgfmath@resetparsingparameters
%
% Reset the stack at the begining of the parse/group.
%
\def\pgfmath@resetparsingparameters{%
	\pgfmath@stack{\pgfmath@empty\pgfmath@empty\pgfmath@empty\pgfmath@empty}%
	\def\pgfmath@stacknextoperator{\pgfmath@empty}% Will not work with \let
}
% \pgfmath@parse@
%
% Start parsing. Expect one of
% 1) the end of the parse
% 2) the start of a group
% 3) a (possible) operand.
%
\def\pgfmath@parse@#1{%
	\def\pgfmath@token{}%
	\ifx#1@%
		\let\pgfmath@parsenext\pgfmath@endparse%
	\else%
		\ifx#1(%
			\let\pgfmath@parsenext\pgfmath@startparsegroup%
		\else%
			\edef\pgfmath@token{#1}%
			\let\pgfmath@parsenext\pgfmath@parseoperand%
	\fi\fi%
	\pgfmath@parsenext%
}

% If no TeX units are declared *at any point* in the parse
% the result is scaled by \pgfmathresultunitscale.
\newif\ifpgfmathunitsdeclared
\def\pgfmathsetresultunitscale#1{\def\pgfmathresultunitscale{#1}}
\def\pgfmathresultunitscale{1}

% \pgfmath@endparse
%
% Everything stops here.
%
\def\pgfmath@endparse#1\pgfmath@empty{%
    \pgfmath@processalloperations%
    \pgfmath@stackpop{\pgfmathresult}%
    \begingroup%
      \ifpgfmathunitsdeclared%
        \pgfmath@x1pt\relax%
      \else%
        \afterassignment\pgfmath@gobbletilpgfmath@%
        \pgfmath@x\pgfmathresultunitscale pt\relax\pgfmath@%
      \fi%
      \expandafter\pgfmath@x\pgfmathresult\pgfmath@x%
      \pgfmath@returnone\pgfmath@x%
    \endgroup%
    \pgfmath@smuggleone{\pgfmathresult}%
  \endgroup%
}

% \pgfmath@startparsegroup
%
% When opening ( is scanned start a new group.
%
\def\pgfmath@startparsegroup{%
	\begingroup%
		\let\pgfmath@parsepostgroup\pgfmath@parseoperator%
		\pgfmath@resetparsingparameters%
		\pgfmath@parse@}

% \pgfmath@endparsegroup
%
% When closing ) is scanned, processes all waiting
% operations (within the group) and close the group.
%
\def\pgfmath@endparsegroup{%
		\pgfmath@processalloperations%
		\pgfmath@stackpop{\pgfmathresult}%
		\expandafter\pgfmath@x\pgfmathresult pt\relax%
		\pgfmath@returnone\pgfmath@x%
	\endgroup%
	\pgfmath@parsepostgroup%
}

% \pgfmath@parseoperator
%
% An operator is expected here. 
% Or the end of the parse or parse group.
% 
\def\pgfmath@parseoperator#1{%
	\def\pgfmath@token{}%
	% Push the operand in \pgfmathresult on to the stack. 
	\expandafter\pgfmath@stackpushoperand\expandafter{\pgfmathresult}%
	\ifx#1@%
		\let\pgfmath@parsenext\pgfmath@endparse%
	\else%
		\ifx#1+%
			\let\pgfmath@parsenext\pgfmath@parseadd%
		\else%
			\ifx#1-%
				\let\pgfmath@parsenext\pgfmath@parsesubtract%
			\else%
				\ifx#1*%
					\let\pgfmath@parsenext\pgfmath@parsemultiply%
				\else%
					\ifx#1/%
						\let\pgfmath@parsenext\pgfmath@parsedivide%
					\else
						\ifx#1)%
							\let\pgfmath@parsenext\pgfmath@endparsegroup%
						\else%
							\ifx#1r%
								\let\pgfmath@parsenext\pgfmath@parseradians%
							\else%
								\ifx#1>%
									\let\pgfmath@parsenext\pgfmath@parsegreaterthan%
								\else%
									\ifx#1<%
										\let\pgfmath@parsenext\pgfmath@parselessthan%
									\else%
										\if#1=%
											\let\pgfmath@parsenext\pgfmath@parseequalto%
										\else%
											\if#1^%
												\let\pgfmath@parsenext\pgfmath@parsepower%
											\else%
												\pgfmath@error{Unknown operator `#1'}%
												\let\pgfmath@parsenext\relax%
	\fi\fi\fi\fi\fi\fi\fi\fi\fi\fi\fi%
	\pgfmath@parsenext%
}

% Use a \toks register as a stack.
\newtoks\pgfmath@stack

% \pgfmath@stackpushoperator
% 
% Push an operator (actually its macro e.g., \pgfmathadd@)
% on to the stack. And keep track of it using the macro
% \pgfmath@stacknextoperator.
%
\def\pgfmath@stackpushoperator#1{%
	\edef\pgfmath@temp{\noexpand#1\the\pgfmath@stack}%
	\expandafter\pgfmath@stack\expandafter{\pgfmath@temp}%
	\def\pgfmath@stacknextoperator{#1}}% <- Must \def. Cannot \let.

% \pgfmath@stackpushoperand
%
% Push an operand (i.e. a number) on the stack. It is
% put within a TeX group to make popping a lot simpler.
%
\def\pgfmath@stackpushoperand#1{%
	\edef\pgfmath@temp{{#1}\the\pgfmath@stack}%
	\expandafter\pgfmath@stack\expandafter{\pgfmath@temp}%
}

% \pgfmath@stackpeek
%
% Peek (i.e. without removal) at the top of the stack.
%
\def\pgfmath@stackpeek{\expandafter\pgfmath@stackpeek@\the\pgfmath@stack\pgfmath@}
\def\pgfmath@stackpeek@#1#2\pgfmath@{#1}%

% \pgfmath@stackpop
%
% Pop (i.e. remove) the top of the stack into #1.
%
\def\pgfmath@stackpop#1{\expandafter\pgfmath@stackpop@\expandafter#1\the\pgfmath@stack\pgfmath@}
\def\pgfmath@stackpop@#1#2#3\pgfmath@{\edef#1{#2}\pgfmath@stack{#3}}%

% \pgfmath@stackpopoperation
%
% Remove and perform an operation from the stack.
%
\def\pgfmath@stackpopoperation{%
	\expandafter\pgfmath@stackpopoperation@\the\pgfmath@stack\pgfmath@%
}
\def\pgfmath@stackpopoperation@#1#2#3#4#5\pgfmath@{%
	\ifx\pgfmath@empty#1\relax%
			\pgfmath@stack{\pgfmath@empty\pgfmath@empty\pgfmath@empty\pgfmath@empty}%
	\else%
		\ifx\pgfmath@empty#2\relax%
			\pgfmath@stack{{#1}\pgfmath@empty\pgfmath@empty\pgfmath@empty\pgfmath@empty}%
		\else%
			#2{#3}{#1}%
			\pgfmath@stack{#4#5}%
			\expandafter\pgfmath@stackpushoperand\expandafter{\pgfmathresult}%
	\fi\fi%
	\def\pgfmath@stacknextoperator{#4}}

% \pgfmath@processalloperations
%
% Process all operation in the stack. The 
% overall result is at the top of the stack.
%
\def\pgfmath@processalloperations{%
	\expandafter\pgfutil@in@\pgfmath@stacknextoperator{\pgfmath@empty}%
	\ifpgfutil@in@%
		\let\pgfmath@processnext\relax%		
	\else%
		\pgfmath@stackpopoperation%
		\let\pgfmath@processnext\pgfmath@processalloperations%
	\fi%
	\pgfmath@processnext}

% \pgfmath@processoperationsuntil
%
% Process operations in the stack, until the specified
% operation/s is/are encountered. The overall result is 
% at the top of the stack.
%
\def\pgfmath@processoperationsuntil#1{%
	\expandafter\pgfutil@in@\pgfmath@stacknextoperator{#1\pgfmath@empty}%
	\ifpgfutil@in@%
		\let\pgfmath@processnext\pgfmath@processoperationsuntil@end%		
	\else%
		\pgfmath@stackpopoperation%
		\let\pgfmath@processnext\pgfmath@processoperationsuntil%
	\fi%
	\pgfmath@processnext{#1}}
\def\pgfmath@processoperationsuntil@end#1{}


% OK. Now the operators are parsed.
% These correspond to the + - / * ^ < > = mod and r operators 
%
\def\pgfmath@parseadd{%	
	% If no operator has been assigned (i.e. + is the first operator scanned),
	% do nothing, except add addition to the stack.
	\ifx\pgfmath@stacknextoperator\pgfmath@empty%
	\else%
		% Empty the process stack up to any inequalities.
		\pgfmath@processoperationsuntil{\pgfmathequalto@\pgfmathlessthan@\pgfmathgreaterthan@}%
	\fi%
	\pgfmath@stackpushoperator{\pgfmathadd@}%
	\pgfmath@parse@}
	
\def\pgfmath@parsesubtract{%	
	% If no operator has been assigned (i.e. - is the first operator scanned),
	% do nothing, except add subtract to the stack.
	\ifx\pgfmath@stacknextoperator\pgfmath@empty%
	\else%
		% Empty the process stack up to any inequalities.
		\pgfmath@processoperationsuntil{\pgfmathequalto@\pgfmathlessthan@\pgfmathgreaterthan@}%
	\fi%
	\pgfmath@stackpushoperator{\pgfmathsubtract@}%
	\pgfmath@parse@}
	

\def\pgfmath@parsemultiply{%
	% If no operator has been assigned (i.e. * is the first operator scanned),
	% do nothing, except push multiply onto the stack.
	\ifx\pgfmath@stacknextoperator\pgfmath@empty%
	\else%
		% Process all operations up to inequalites or addition/subtraction
		\pgfmath@processoperationsuntil{\pgfmathequalto@\pgfmathlessthan@\pgfmathgreaterthan@%
			\pgfmathadd@\pgfmathsubtract@}%
	\fi%
	\pgfmath@stackpushoperator{\pgfmathmultiply@}%
	\pgfmath@parse@}
	
\def\pgfmath@parsedivide{%
	% If no operator has been assigned (i.e. / is the first operator scanned),
	% do nothing, except push divide onto the stack.
	\ifx\pgfmath@stacknextoperator\pgfmath@empty%
	\else%
		% Process all operations up to inequalites or addition/subtraction
		\pgfmath@processoperationsuntil{\pgfmathequalto@\pgfmathlessthan@\pgfmathgreaterthan@%
			\pgfmathadd@\pgfmathsubtract@}%
	\fi%
	\pgfmath@stackpushoperator{\pgfmathdivide@}%
	\pgfmath@parse@}

\def\pgfmath@parsegreaterthan{%
	% On scanning an equality/inequality operator everything up to
	% (but not including) the operator is evaluated... 
	\pgfmath@processalloperations%
	% ...and the operation pushed onto the stack.
	\pgfmath@stackpushoperator{\pgfmathgreaterthan@}%
	\pgfmath@parse@}

\def\pgfmath@parselessthan{%
	\pgfmath@processalloperations%
	\pgfmath@stackpushoperator{\pgfmathlessthan@}%
	\pgfmath@parse@}

\def\pgfmath@parseequalto={%
	\pgfmath@processalloperations%
	\pgfmath@stackpushoperator{\pgfmathequalto@}%
	\pgfmath@parse@}

\def\pgfmath@parsepower{%
	% Easy, just push power onto the stack.
	\pgfmath@stackpushoperator{\pgfmathpow@}%
	\pgfmath@parse@}

\def\pgfmath@parseradians{%
	% Actually this is a post-fix function...
	\ifx\pgfmath@primaryoperation\pgfmath@empty%
	\else%
		\pgfmath@processoperationsuntil{\pgfmathequalto@\pgfmathlessthan@\pgfmathgreaterthan@%
			\pgfmathadd@\pgfmathsubtract@}%
	\fi%	
	\pgfmath@stackpop{\pgfmath@temp}%
	\pgfmathdeg@{\pgfmath@temp}%
	% ...so processing returns to \pgfmath@parseoperator
	\pgfmath@parseoperator}
	
\newcount\c@pgfmath@parsecounta
\newcount\c@pgfmath@parsecountb
\newcount\c@pgfmath@parsecountc
\newdimen\pgfmath@parsex

% \pgfmath@quickparseoperand
%
% An operand can *only* be a dimension.
%
\def\pgfmath@quickparseoperand{%
	\afterassignment\pgfmath@quickparseoperand@%
	\pgfmath@parsex\pgfmath@token}
\def\pgfmath@quickparseoperand@{%
	\edef\pgfmathresult{\pgfmath@tonumber{\pgfmath@parsex}}
	\pgfmath@parseoperator%
}
	
% \pgfmath@parseoperand
%
% An operand can be an ASCII number (with or without dimensions, with
% or without a decimal point), a TeX register (count, dimen or skip,
% which may have expanded with \the), or a function e.g. sin(X).
%
% It is assumed that after \edef-ing, the only unexpanded tokens are registers. 
%
\def\pgfmath@parseoperand{%
	\def\pgfmath@sign{}%
	\expandafter\pgfmath@parseoperandsign\pgfmath@token}
\def\pgfmath@parseoperandsign#1{%
	\pgfmath@in@#1{-+}% 
	\ifpgfmath@in@
		% I suppose there are (silly) people who might complain if
		% they can't say 2---5, or 3+-----7. Just for them...
		\edef\pgfmath@sign{\pgfmath@sign#1}%
		\expandafter\pgfmath@parseoperandsign%
	\else%
		\expandafter\pgfmath@parsenumberorfunction\expandafter#1%
	\fi%
}


\newif\ifpgfmath@dimen@

\def\pgfmath@ifregisterdimen@#1\pgfmath@{%
	\pgfmath@in@{p}{#1}%
	\ifpgfmath@in@%
		\pgfmath@dimen@true%
	\else%
		\pgfmath@dimen@false%
	\fi} 	
		

\def\pgfmath@parsenumberorfunction#1{%
  \let\pgfmath@parsenext\pgfmath@parsenumberorfunction@simple%
  \ifx#1\wd\let\pgfmath@parsenext\pgfmath@boxdimen\fi%
  \ifx#1\ht\let\pgfmath@parsenext\pgfmath@boxdimen\fi%
  \ifx#1\dp\let\pgfmath@parsenext\pgfmath@boxdimen\fi%
  \pgfmath@parsenext#1%
}

\def\pgfmath@boxdimen#1#2{%
  \pgfmath@parsex=#1#2%
  \edef\pgfmathresult{\pgfmath@tonumber{\pgfmath@parsex}}%
  \global\pgfmathunitsdeclaredtrue% a dimension has units 
  \pgfmath@parseoperator%
}

\def\pgfmath@parsenumberorfunction@simple#1{%	
	\expandafter\ifcat#1\relax%
		% So, a TeX register.
		\afterassignment\pgfmath@ifregisterdimen@%
		\pgfmath@parsex\pgfmath@sign#1pt\relax\pgfmath@%
		\ifpgfmath@dimen@% 
			% A dimension! So stop scanning operand here.
			\edef\pgfmathresult{\pgfmath@tonumber{\pgfmath@parsex}}%
         \global\pgfmathunitsdeclaredtrue% a dimension has units
			\def\pgfmath@temp{}%
			\let\pgfmath@parsenext\pgfmath@parseoperator%
		\else%
			% A count! Expand it, but carry on as usual as it might
			% be immediately followed by a dimension.
			\ifdim\pgfmath@parsex<0pt\relax%
				\edef\pgfmath@sign{\pgfmath@sign-}%
				\pgfmath@parsex-\pgfmath@parsex%
			\fi%
			\edef\pgfmath@temp{\pgfmath@tonumber{\pgfmath@parsex}}%
			\let\pgfmath@parsenext\pgfmath@parsenumberorfunction@%
		\fi%
	\else%
		% Could be a number or a function...?
		\edef\pgfmath@temp{#1}%
		\let\pgfmath@parsenext\pgfmath@parsenumberorfunction@%
	\fi%
	\expandafter\pgfmath@parsenext\pgfmath@temp%
}%

\def\pgfmath@parsenumberorfunction@#1{%
	\pgfmath@in@{#1}{.0123456789}%
	\ifpgfmath@in@%
		\let\pgfmath@parsenext\pgfmath@parsenumber%
	\else%
		\let\pgfmath@parsenext\pgfmath@parsefunction%
	\fi%
	\pgfmath@parsenext#1%
}%

\def\pgfmath@parsenumber{% 
	% Consider the number 3.14159
	% 3 is parsed by assignment to a, then '.' is absorbed and 14159
	% parsed by assignement to b (actually b=114159, see below). 
	\c@pgfmath@parsecountc0\relax%
	\afterassignment\pgfmath@parsedecimalpoint%
	\c@pgfmath@parsecounta0}

\def\pgfmath@parsedecimalpoint#1{%
	\ifx#1.% Is there a decimal point? If not, see if there are any units.
		\let\pgfmath@parsenext\pgfmath@parsemantissa%
	\else%
		\c@pgfmath@parsecountb10\relax% The first digit of b is gobbled (see below). 
		\let\pgfmath@parsenext\pgfmath@parseunits%
	\fi%
	\pgfmath@parsenext#1}



% We would like to assign the following number (which is the mantisse)
% to a number. However, for a very long mantisse as in 3.141592653589793238462643
% this will fail since the number becomes too large.
%
% Because of this, we have to do some ``magic'': We scan 9 tokens and
% then do an assignment with a guard after 9 tokens so that the
% assigment cannot fail.
%
\def\pgfmath@parsemantissa.#1#2#3#4#5#6#7#8#9{%
  \afterassignment\pgfmath@checkforguard%    
  % Consider the number: 2.005
  % 2 is assigned to a, but b will be assigned 5, which is *not right*. 
  % So using 1 here hereresults in  b=1005. The first digit is then  
  % gobbled later, when expanded with \the.
  \c@pgfmath@parsecountb1#1#2#3#4#5#6#7#8#9\relax}%

\def\pgfmath@checkforguard{%
  \pgfutil@ifnextchar\relax%
  {%
    % Ok, this is a looong mantisse. Start gobbling all following
    % numbers
    \pgfmath@gobblemantisse%
  }%
  {%
    \pgfmath@removeguard%
  }%
}

\def\pgfmath@gobblemantisse\relax#1#2#3#4#5#6#7#8#9{%
  \afterassignment\pgfmath@checkforguard%
  \c@pgfmath@parsecountc0#1#2#3#4#5#6#7#8#9\relax% these digits are ignored
}

\def\pgfmath@removeguard#1\relax{\pgfmath@parseunits#1}



 
\def\pgfmath@parseunits#1{%
  % Here the extra first digit in b is gobbled. 
  \edef\pgfmath@resulttemp{%
    \pgfmath@sign\the\c@pgfmath@parsecounta.\expandafter\pgfmath@gobbleone\the\c@pgfmath@parsecountb}%
  % Check whether #1 is a \dp, \wd, or \ht
  \let\pgfmath@parsenext=\pgfmath@parseunits@simple%
  \ifx#1\wd\let\pgfmath@parsenext=\pgfmath@parseunits@boxdimen\fi%
  \ifx#1\ht\let\pgfmath@parsenext=\pgfmath@parseunits@boxdimen\fi%
  \ifx#1\dp\let\pgfmath@parsenext=\pgfmath@parseunits@boxdimen\fi%
  \pgfmath@parsenext#1%
}

\def\pgfmath@parseunits@boxdimen#1#2{%
  \pgfmath@parsex=#1#2%
  \pgfmath@parsex\pgfmath@resulttemp\pgfmath@parsex%
  \edef\pgfmathresult{\pgfmath@tonumber{\pgfmath@parsex}}%
  \global\pgfmathunitsdeclaredtrue% a dimension has units 
  \pgfmath@parseoperator%
}

\def\pgfmath@parseunits@simple#1{%
	% Now check if #1 is actually a register.
	\ifcat#1\relax%
		\afterassignment\pgfmath@ifregisterdimen@%
		\pgfmath@parsex#1pt\relax\pgfmath@%
		\pgfmath@parsex\pgfmath@resulttemp\pgfmath@parsex%
		\ifpgfmath@dimen@% If it's a dimen stop scanning operand here.
                        \global\pgfmathunitsdeclaredtrue% a dimension has units
			\edef\pgfmathresult{\pgfmath@tonumber{\pgfmath@parsex}}%
			\let\pgfmath@parsenext\pgfmath@parseoperator%
		\else
			\edef\pgfmath@resulttemp{\pgfmath@tonumber{\pgfmath@parsex}}%
			\let\pgfmath@parsenext\pgfmath@parseunits@%
		\fi%
		\expandafter\pgfmath@parsenext%
	\else%
		\expandafter\pgfmath@parseunits@\expandafter#1%
	\fi}

\def\pgfmath@parseunits@#1#2{%
	% Check if #1 is an operator, or the character `@' indicating the end of the parse.
	\pgfmath@in@#1{@+-*/^r<>=()}%
	\ifpgfmath@in@%
		\edef\pgfmath@tokens{#1#2}%
		\edef\pgfmathresult{\pgfmath@resulttemp}%
		\let\pgfmath@parsenext\pgfmath@parseoperator%
	\else%
		\pgfmath@in@#1{Ee}% e+0X
		\ifpgfmath@in@%
			% Oh no! It might be and em or an ex...
			\pgfmath@in@#2{mx}%
			\ifpgfmath@in@%
				\global\pgfmathunitsdeclaredtrue%
				\pgfmath@parsex\pgfmath@resulttemp#1#2\relax%
				\edef\pgfmathresult{\pgfmath@tonumber{\pgfmath@parsex}}%
				\edef\pgfmath@tokens{}%
				\let\pgfmath@parsenext\pgfmath@parseoperator%
			\else%
				\edef\pgfmath@tokens{#2}%
				\let\pgfmath@parsenext\pgfmath@parsescientific%
			\fi%
		\else
			% Anything else *should* be a TeX unit.
			\global\pgfmathunitsdeclaredtrue%
			\pgfmath@parsex\pgfmath@resulttemp#1#2\relax%
			\edef\pgfmathresult{\pgfmath@tonumber{\pgfmath@parsex}}%
			\edef\pgfmath@tokens{}%
			\let\pgfmath@parsenext\pgfmath@parseoperator%
	\fi\fi%
	\expandafter\pgfmath@parsenext\pgfmath@tokens%
}

% Parse 'Scientific' notation in the form xEyy
%
\def\pgfmath@parsescientific{%
	\afterassignment\pgfmath@parsescientific@%
	\c@pgfmath@parsecounta}

\def\pgfmath@parsescientific@{%
	\edef\pgfmath@parsesci@temp{\pgfmath@resulttemp\the\c@pgfmath@parsecountc}%
	\expandafter\pgfmathscientific\expandafter{\pgfmath@parsesci@temp}%
		{\the\c@pgfmath@parsecounta}%
	\edef\pgfmath@resulttemp{\pgfmathresult}%
	\pgfmath@parseunits@}

	
% OK. Now the fun stuff. We parse functions here. Note that 
% *no* calulations are done in the following macros. All the
% mathematics is done in pgfmathoperations.code and friends. 
%
% Functions dealt with here are:
%
% round(X)         'half-up' rounding.
% floor(X)         floor function.
% ciel(X)          ceiling function.
% abs(X)           absolute function.
%
% exp(X)           e^X (0 <= X <~= 9.7).
%
% sin(X)           sine function.
% cos(X)           cosine function.
% tan(X)           tan function.
% asin(X)          arcsine of X (in radians)    -1 <= X <= 1
% acos(X)          arccosine of X (in radians)  -1 <= X <= 1
% atan(X)          arctangent of X (in radians) -1 <= X <= 1
% veclen(X,Y)      the length Z where Z^2 = X^2 + Y^2
% mod(X,Y)         X modulo Y
% max(X,Y)         the maximum of X or Y
% min(X,Y)         the minimum of X or Y
%
% NB veclen, mod, max, and min *cannot* be nested.
%
% deg(X)           converts X to degrees (X in radians)
% rad(X)           converts X to radians (X in degrees)
%
% rnd              generate pseudo-random number X (0 <= X <= 1).
% rand             generate pseudo-random number X (1 <= X <= -1).
% sqrt(X)          square root.
% 
% pi               the constant PI.

\def\pgfmath@parsefunction{%
	\def\pgfmath@parsedfunctionname{}%
	\futurelet\pgfmath@parsetoken\pgfmath@parsefunction@@}
\def\pgfmath@parsefunction@@#1{%
	\pgfmath@in@#1{()@+-*/^<>=}% A function name ends with one of these...
	\ifpgfmath@in@%
		\let\pgfmath@parsefunctionnext\pgfmath@parsefunction@end%
	\else%
		\ifx\pgfmath@parsetoken\pgfutil@sptoken% ...or a space.
			\let\pgfmath@parsefunctionnext\pgfmath@parsefunction@end%
		\else%
			\let\pgfmath@parsefunctionnext\pgfmath@parsefunction@continue%
	\fi\fi%
	\pgfmath@parsefunctionnext#1}

\def\pgfmath@parsefunction@end#1{%
	\expandafter\ifx\csname pgfmath@parsefunction@\pgfmath@parsedfunctionname\endcsname\relax%
		\pgfmath@reportunknownfunction%
		\let\pgfmath@parsefunctionnext\relax%
	\else%
		\let\pgfmath@parsefunctionnext\pgfmath@executeparsefunction@%
	\fi%
	\pgfmath@parsefunctionnext#1}

\def\pgfmath@parsefunction@continue#1{%
	\edef\pgfmath@parsedfunctionname{\pgfmath@parsedfunctionname#1}%
	\futurelet\pgfmath@parsetoken\pgfmath@parsefunction@@}
	
\def\pgfmath@executeparsefunction@{\csname pgfmath@parsefunction@\pgfmath@parsedfunctionname\endcsname}
		
		
\def\pgfmath@reportunknownfunction{%
	\pgfmath@error{Unknown function `\pgfmath@parsedfunctionname'}{}%
}

% \pgfmath@postfunction
% 
% In scanning a function e.g. sin(40), we subvert the normal parsing 
% group mechanism by messing around with \pgfmath@parsepostgroup, so 
% that after scanning ), the parser doesn't scan for an operator, but 
% returns to the macros scanning the function. 
% Here the mechanism is restored, and the value of the function is 
% stored along with the approprate sign, which was saved earlier.
%
\def\pgfmath@postfunction{%
	\let\pgfmath@parsepostgroup\pgfmath@parseoperator%
	\edef\pgfmathresult{\pgfmath@sign\pgfmathresult}%
	\pgfmath@parseoperator}
	

% \pgfmath@parsefunction@abs
%
\def\pgfmath@parsefunction@abs{%
	\let\pgfmath@parsepostgroup\pgfmath@parsefunction@abs@%
	\expandafter\pgfmath@parse@}
\def\pgfmath@parsefunction@abs@{%	
	\expandafter\pgfmathabs@\expandafter{\pgfmathresult}%
	\pgfmath@postfunction%
}

% \pgfmath@parsefunction@sqrt
%
\def\pgfmath@parsefunction@sqrt{%
	\let\pgfmath@parsepostgroup\pgfmath@parsefunction@sqrt@%
	\expandafter\pgfmath@parse@}
\def\pgfmath@parsefunction@sqrt@{%	
	\expandafter\pgfmathsqrt@\expandafter{\pgfmathresult}%
	\pgfmath@postfunction%
}

% \pgfmath@parsefunction@round
%
\def\pgfmath@parsefunction@round{%
	\let\pgfmath@parsepostgroup\pgfmath@parsefunction@round@%
	\expandafter\pgfmath@parse@}
\def\pgfmath@parsefunction@round@{%	
	\expandafter\pgfmathround@\expandafter{\pgfmathresult}%
	\pgfmath@postfunction%
}

% \pgfmath@parsefunction@floor
%
\def\pgfmath@parsefunction@floor{%
	\let\pgfmath@parsepostgroup\pgfmath@parsefunction@floor@%
	\expandafter\pgfmath@parse@}
\def\pgfmath@parsefunction@floor@{%	
	\expandafter\pgfmathfloor@\expandafter{\pgfmathresult}%
	\pgfmath@postfunction%
}

% \pgfmath@parsefunction@ceil
%
\def\pgfmath@parsefunction@ceil{%
	\let\pgfmath@parsepostgroup\pgfmath@parsefunction@ceil@%
	\expandafter\pgfmath@parse@}
\def\pgfmath@parsefunction@ceil@{%	
	\expandafter\pgfmathceil@\expandafter{\pgfmathresult}%
	\pgfmath@postfunction%
}

% \pgfmath@parsefunction@sin
%
\def\pgfmath@parsefunction@sin{%
	\let\pgfmath@parsepostgroup\pgfmath@parsefunction@sin@%
	\expandafter\pgfmath@parse@}
\def\pgfmath@parsefunction@sin@{%	
	\expandafter\pgfmathsin@\expandafter{\pgfmathresult}%
	\pgfmath@postfunction%
}

% \pgfmath@parsefunction@cos
%
\def\pgfmath@parsefunction@cos{%
	\let\pgfmath@parsepostgroup\pgfmath@parsefunction@cos@%
	\expandafter\pgfmath@parse@}
\def\pgfmath@parsefunction@cos@{%	
	\expandafter\pgfmathcos@\expandafter{\pgfmathresult}%
	\pgfmath@postfunction%
}

% \pgfmath@parsefunction@asin
%
\def\pgfmath@parsefunction@asin{%
	\let\pgfmath@parsepostgroup\pgfmath@parsefunction@asin@%
	\expandafter\pgfmath@parse@}
\def\pgfmath@parsefunction@asin@{%	
	\expandafter\pgfmathasin@\expandafter{\pgfmathresult}%
	\pgfmath@postfunction%
}

% \pgfmath@parsefunction@acos
%
\def\pgfmath@parsefunction@acos{%
	\let\pgfmath@parsepostgroup\pgfmath@parsefunction@acos@%
	\expandafter\pgfmath@parse@}
\def\pgfmath@parsefunction@acos@{%	
	\expandafter\pgfmathacos@\expandafter{\pgfmathresult}%
	\pgfmath@postfunction}

% \pgfmath@parsefunction@atan
%
\def\pgfmath@parsefunction@atan{%
	\let\pgfmath@parsepostgroup\pgfmath@parsefunction@atan@%
	\expandafter\pgfmath@parse@}
\def\pgfmath@parsefunction@atan@{%	
	\expandafter\pgfmathatan@\expandafter{\pgfmathresult}%
	\pgfmath@postfunction%
}

% \pgfmath@parsefunction@tan
%
\def\pgfmath@parsefunction@tan{%
	\let\pgfmath@parsepostgroup\pgfmath@parsefunction@tan@%
	\expandafter\pgfmath@parse@}
\def\pgfmath@parsefunction@tan@{%	
	\expandafter\pgfmathtan@\expandafter{\pgfmathresult}%
	\pgfmath@postfunction%
}

% \pgfmath@parsefunction@cot
%
\def\pgfmath@parsefunction@cot{%
	\let\pgfmath@parsepostgroup\pgfmath@parsefunction@cot@%
	\expandafter\pgfmath@parse@}
\def\pgfmath@parsefunction@cot@{%	
	\expandafter\pgfmathcot@\expandafter{\pgfmathresult}%
	\pgfmath@postfunction%
}

% \pgfmath@parsefunction@sec
%
\def\pgfmath@parsefunction@sec{%
	\let\pgfmath@parsepostgroup\pgfmath@parsefunction@sec@%
	\expandafter\pgfmath@parse@}
\def\pgfmath@parsefunction@sec@{%	
	\expandafter\pgfmathsec@\expandafter{\pgfmathresult}%
	\pgfmath@postfunction%
}

% \pgfmath@parsefunction@cosec
%
\def\pgfmath@parsefunction@cosec{%
	\let\pgfmath@parsepostgroup\pgfmath@parsefunction@cosec@%
	\expandafter\pgfmath@parse@}
\def\pgfmath@parsefunction@cosec@{%	
	\expandafter\pgfmathcosec@\expandafter{\pgfmathresult}%
	\pgfmath@postfunction%
}

% \pgfmath@parsefunction@rad
%
\def\pgfmath@parsefunction@rad{%
	\let\pgfmath@parsepostgroup\pgfmath@parsefunction@rad@%
	\pgfmath@parse@}
\def\pgfmath@parsefunction@rad@{%
	\expandafter\pgfmathrad@\expandafter{\pgfmathresult}%
	\pgfmath@postfunction}%
	
% \pgfmath@parsefunction@rad
%
\def\pgfmath@parsefunction@deg{%
	\let\pgfmath@parsepostgroup\pgfmath@parsefunction@deg@%
	\expandafter\pgfmath@parse@}
\def\pgfmath@parsefunction@deg@{%
	\expandafter\pgfmathdeg@\expandafter{\pgfmathresult}%
	\pgfmath@postfunction}%
	
% \pgfmath@parsefunction@rnd
%
\def\pgfmath@parsefunction@rnd{%
	\expandafter\pgfmathrnd%
	\expandafter\pgfmath@postfunction}
			
% \pgfmath@parsefunction@rand
%
\def\pgfmath@parsefunction@rand{%
	\expandafter\pgfmathrand%
	\expandafter\pgfmath@postfunction}%
			
% \pgfmath@parsefunction@exp
%
\def\pgfmath@parsefunction@exp{%
	\let\pgfmath@parsepostgroup\pgfmath@parsefunction@exp@%
	\expandafter\pgfmath@parse@}
\def\pgfmath@parsefunction@exp@{%	
	\expandafter\pgfmathexp@\expandafter{\pgfmathresult}%
	\pgfmath@postfunction%
}

% \pgfmath@parsefunction@pi
%
\def\pgfmath@parsefunction@pi{%
	\pgfmathpi%
	\pgfmath@postfunction%
}


% \pgfmath@parsefunction@veclen
%
\def\pgfmath@parsefunction@veclen{%
	\expandafter\pgfmath@parsefunction@veclen@}
\def\pgfmath@parsefunction@veclen@(#1,{%
	\pgfmathparse@{#1}\edef\pgfmath@firstoperand{\pgfmathresult}%
	\let\pgfmath@parsepostgroup\pgfmath@parsefunction@veclen@@
	\pgfmath@startparsegroup}
\def\pgfmath@parsefunction@veclen@@{%
	\edef\pgfmath@secondoperand{\pgfmathresult}%
	\pgfmathveclen@{\pgfmath@firstoperand}{\pgfmath@secondoperand}%
	\pgfmath@postfunction}

% \pgfmath@parsefunction@mod 
%
\def\pgfmath@parsefunction@mod{%
	\expandafter\pgfmath@parsefunction@mod@}
\def\pgfmath@parsefunction@mod@(#1,{%
	\pgfmathparse@{#1}\edef\pgfmath@firstoperand{\pgfmathresult}%
	\let\pgfmath@parsepostgroup\pgfmath@parsefunction@mod@@
	\pgfmath@startparsegroup}
\def\pgfmath@parsefunction@mod@@{%
	\edef\pgfmath@secondoperand{\pgfmathresult}%
	\pgfmathmod@{\pgfmath@firstoperand}{\pgfmath@secondoperand}%
	\pgfmath@postfunction}

% \pgfmath@parsefunction@max
%
\def\pgfmath@parsefunction@max{%
	\expandafter\pgfmath@parsefunction@max@}
\def\pgfmath@parsefunction@max@(#1,{%
	\pgfmathparse@{#1}\edef\pgfmath@firstoperand{\pgfmathresult}%
	\let\pgfmath@parsepostgroup\pgfmath@parsefunction@max@@
	\pgfmath@startparsegroup}
\def\pgfmath@parsefunction@max@@{%
	\edef\pgfmath@secondoperand{\pgfmathresult}%
	\pgfmathmax@{\pgfmath@firstoperand}{\pgfmath@secondoperand}%
	\pgfmath@postfunction}	

% \pgfmath@parsefunction@min 
%
\def\pgfmath@parsefunction@min{%
	\expandafter\pgfmath@parsefunction@min@}
\def\pgfmath@parsefunction@min@(#1,{%
	\pgfmathparse@{#1}\edef\pgfmath@firstoperand{\pgfmathresult}%
	\let\pgfmath@parsepostgroup\pgfmath@parsefunction@min@@
	\pgfmath@startparsegroup}
\def\pgfmath@parsefunction@min@@{%
	\edef\pgfmath@secondoperand{\pgfmathresult}%
	\pgfmathmin@{\pgfmath@firstoperand}{\pgfmath@secondoperand}%
	\pgfmath@postfunction}
	
% \pgfmath@parsefunction@pow 
%
\def\pgfmath@parsefunction@pow{%
	\expandafter\pgfmath@parsefunction@pow@}
\def\pgfmath@parsefunction@pow@(#1,{%
	\pgfmathparse@{#1}\edef\pgfmath@firstoperand{\pgfmathresult}%
	\let\pgfmath@parsepostgroup\pgfmath@parsefunction@pow@@
	\pgfmath@startparsegroup}
\def\pgfmath@parsefunction@pow@@{%
	\edef\pgfmath@secondoperand{\pgfmathresult}%
	\pgfmathpow@{\pgfmath@firstoperand}{\pgfmath@secondoperand}%
	\pgfmath@postfunction}
% Copyright 2007 by Mark Wibrow
%
% This file may be distributed and/or modified
%
% 1. under the LaTeX Project Public License and/or
% 2. under the GNU Public License.
%
% See the file doc/generic/pgf/licenses/LICENSE for more details.

% This file defines the mathematical functions and operators.
%
% Version 0.0 08/03/2007

% This file defines the mathematical functions and operators.
%
% Adding/redefining extra operators/functions:
%
% Each operator/function XXX has two forms:
%
%
% \pgfmathXXX#1...   a public version which evaluates any
%                    arguments passed to it and passes the
%                    results on to...
%
% \pgfmathXXX@#1...  a non-public version which performs 
%                    required calculation on arguments which
%                    must have already been evaluated (i.e.
%                    *without* dimensions).
% 
% If a function XXX is to be included in the parser, it is 
% recommended, for consistency, that where possible, the 
% pgfmathparser file should define the macro \pgfmath@parseXXX.
% The parser should (ideally) then call \pgfmathXXX@.
%
% It is recommend that the pgfmath versions of the pgf dimension
% and count registers be used, i.e., \pgfmath@x for \pgfmath@x, 
% \c@pgfmath@counta for c@pgfmath@counta, and so on. These are currently
% \let to their pgf equivalents, but it may be necessary to change 
% this.
%
% It is also recommened that all calculations (where necessary)
% take place within a TeX group. \pgfmath@returnone#1 makes and
% expanded version of #1 global and stores this in \pgfmathresult 
% after the group is ended.
%

\input pgfmathtrig.code.tex% Load the trig. stuff.
\input pgfmathrnd.code.tex%  Load the random stuff.


% \pgfmathadd
%
% Add #1 and #2.
%
\def\pgfmathadd#1#2{%
	\pgfmathparse{#1}\let\pgfmath@adda\pgfmathresult%
	\pgfmathparse{#2}\let\pgfmath@addb\pgfmathresult%
	\pgfmathadd@{\pgfmath@adda}{\pgfmath@addb}}
\def\pgfmathadd@#1#2{%
	\begingroup%
		\edef\pgfmath@a{#1}%
		\edef\pgfmath@b{#2}%
		\pgfmath@x\pgfmath@a pt\relax%
		\pgfmath@y\pgfmath@b pt\relax%
		\advance\pgfmath@x by\pgfmath@y%
		\pgfmath@returnone\pgfmath@x%
	\endgroup%
}

% \pgfmathsubtract
%
% Subtract #2 from #1.
%
\def\pgfmathsubtract#1#2{%
	\pgfmathparse{#1}\let\pgfmath@subtracta\pgfmathresult%
	\pgfmathparse{#2}\let\pgfmath@subtractb\pgfmathresult%
	\pgfmathsubtract@{\pgfmath@subtracta}{\pgfmath@subtractb}}

\def\pgfmathsubtract@#1#2{%
	\begingroup%
		\edef\pgfmath@a{#1}%
		\edef\pgfmath@b{#2}%
		\pgfmath@x\pgfmath@a pt\relax%
		\pgfmath@y\pgfmath@b pt\relax%
		\advance\pgfmath@x by-\pgfmath@y%
		\pgfmath@returnone\pgfmath@x%
	\endgroup%
}

% \pgfmathmultiply
%
% Multiply #1 by #2.
%
\def\pgfmathmultiply#1#2{%
	\pgfmathparse{#1}\let\pgfmath@multiplya\pgfmathresult%
	\pgfmathparse{#2}\let\pgfmath@multiplyb\pgfmathresult%
	\pgfmathmultiply@{\pgfmath@multiplya}{\pgfmath@multiplyb}}
\def\pgfmathmultiply@#1#2{%
	\begingroup%
		\edef\pgfmath@a{#1}%
		\edef\pgfmath@b{#2}%
		\pgfmath@x\pgfmath@a pt\relax%
		\pgfmath@x\pgfmath@b\pgfmath@x%
		\pgfmath@returnone\pgfmath@x%
	\endgroup%
}

% \pgfmathdivide
%
% Divide #1 by #2.
%
\def\pgfmathdivide#1#2{%
	\pgfmathparse{#1}\let\pgfmath@a\pgfmathresult%
	\pgfmathparse{#2}\let\pgfmath@b\pgfmathresult%
	\pgfmathdivide@{\pgfmath@a}{\pgfmath@b}}
\def\pgfmathdivide@#1#2{%
	\begingroup
		% Expand for safety.
		\edef\pgfmath@a{#1}%
		\edef\pgfmath@b{#2}%
		\pgfmath@x\pgfmath@b pt\relax%
		\ifdim\pgfmath@x=0pt\relax%
			\pgfmath@error{You asked me to calculate `\pgfmath@a/\pgfmath@b', %
				but I cannot divide any number by zero}{}%
		\else%
			\afterassignment\pgfmath@x%
			\expandafter\c@pgfmath@counta\the\pgfmath@x\relax%
			\ifdim\pgfmath@x=0pt\relax%
				\expandafter\pgfmath@x\pgfmath@a pt\relax%
				\divide\pgfmath@x\c@pgfmath@counta%
			\else%
				% If #1=1 use the quicker reciprocal operation%
				\pgfmath@x\pgfmath@a pt\relax%
				\ifdim\pgfmath@x=1pt\relax%
					\pgfmathreciprocal@{\pgfmath@b}%
					\pgfmath@x\pgfmath@a pt\relax%
					\pgfmath@x\pgfmathresult\pgfmath@x%
				\else%
					% Use slower ln method.
					\c@pgfmath@counta1\relax%
					\ifdim\pgfmath@a pt<0pt\relax%
						\edef\pgfmath@a{-\pgfmath@a}%
						\c@pgfmath@counta-\c@pgfmath@counta%
					\fi%
					\ifdim\pgfmath@b pt<0pt\relax%
						\edef\pgfmath@b{-\pgfmath@b}%
						\c@pgfmath@counta-\c@pgfmath@counta%
					\fi%
					\ifdim\pgfmath@a pt=0pt\relax%
						\def\pgfmath@a{0}%
					\else%
						\pgfmathln@{\pgfmath@a}%
						\let\pgfmath@a\pgfmathresult%
					\fi%
					\pgfmathln@{\pgfmath@b}%
					\let\pgfmath@b\pgfmathresult%
					\pgfmathsubtract@{\pgfmath@a}{\pgfmath@b}%
					\pgfmathexp@{\pgfmathresult}%
					\pgfmath@x\pgfmathresult pt\relax%
					\multiply\pgfmath@x\c@pgfmath@counta%
				\fi%
			\fi%
		\fi%
		\pgfmath@returnone\pgfmath@x%
	\endgroup%
}

% \pgfmathgreaterthan
%
% 1.0 if #1 > #2. Otherwise 0.0
%
\def\pgfmathgreaterthan#1#2{%
	\pgfmathparse{#1}\let\pgfmath@a\pgfmathresult%
	\pgfmathparse{#2}\let\pgfmath@b\pgfmathresult%
	\pgfmathgreaterthan@{\pgfmath@a}{\pgfmath@b}}
\def\pgfmathgreaterthan@#1#2{%
	\begingroup%
		\edef\pgfmath@a{#1}%
		\edef\pgfmath@b{#2}%
		\pgfmath@x\pgfmath@a pt\relax%
		\pgfmath@y\pgfmath@b pt\relax%
		\advance\pgfmath@x-\pgfmath@y%
		\ifdim\pgfmath@x>0pt\relax%
			\pgfmath@x1pt\relax%
		\else%
			\pgfmath@x0pt\relax%
		\fi%
		\pgfmath@returnone\pgfmath@x%
	\endgroup%
}

% \pgfmathlessthan
%
% 1.0 if #1< #2. Otherwise 0.0
%
\def\pgfmathlessthan#1#2{%
	\pgfmathparse{#1}\let\pgfmath@a\pgfmathresult%
	\pgfmathparse{#2}\let\pgfmath@b\pgfmathresult%
	\pgfmathlessthan@{\pgfmath@a}{\pgfmath@b}}
\def\pgfmathlessthan@#1#2{%
	\begingroup%
		\edef\pgfmath@a{#1}%
		\edef\pgfmath@b{#2}%
		\pgfmath@x\pgfmath@a pt\relax%
		\pgfmath@y\pgfmath@b pt\relax%
		\advance\pgfmath@x-\pgfmath@y\relax%
		\ifdim\pgfmath@x<0pt\relax%
			\pgfmath@x1pt\relax%
		\else%
			\pgfmath@x0pt\relax%
		\fi%
		\pgfmath@returnone\pgfmath@x%
	\endgroup%
}

% \pgfmathequalto
%
% 1.0 if #1 = #2. Otherwise 0.0
%
\def\pgfmathequalto#1#2{%
	\pgfmathparse{#1}\let\pgfmath@a\pgfmathresult%
	\pgfmathparse{#2}\let\pgfmath@b\pgfmathresult%
	\pgfmathadd@{\pgfmath@a}{\pgfmath@b}}
\def\pgfmathequalto@#1#2{%
	\begingroup%
		\edef\pgfmath@a{#1}%
		\edef\pgfmath@b{#2}%
		\pgfmath@x\pgfmath@a pt\relax%
		\pgfmath@y\pgfmath@b pt\relax%
		\advance\pgfmath@x-\pgfmath@y%
		\ifdim\pgfmath@x=0pt\relax%
			\pgfmath@x1pt\relax%
		\else%
			\pgfmath@x0pt\relax%
		\fi%
		\pgfmath@returnone\pgfmath@x%
	\endgroup%
}

% \pgfmathreciprocal
%
% 1 / #1
%
\def\pgfmathreciprocal#1{%
	\pgfmathparse{#1}%
	\pgfmathreciprocal@{\pgfmathresult}}
\def\pgfmathreciprocal@#1{%
	\begingroup%
		\expandafter\pgfmath@x#1pt\relax%
		\ifdim\pgfmath@x=0pt\relax%
			\pgfmath@error{You asked me to calculate `1/#1', but I cannot divide any number by zero}{}%
		\fi%
		\edef\pgfmath@reciprocaltemp{\pgfmath@tonumber{\pgfmath@x}}%
		\expandafter\pgfmathreciprocal@@\pgfmath@reciprocaltemp0000000\pgfmath@}
\def\pgfmathreciprocal@@#1.#2#3#4#5#6#7\pgfmath@{%
		\c@pgfmath@counta#2#3#4#5#6\relax%
		% If the number is an integer, use TeX arithmatic.
		\ifnum\c@pgfmath@counta=0\relax%
			\pgfmath@x1pt\relax%
			\divide\pgfmath@x#1\relax%
		\else%
			\ifnum#1>100\relax%
				\c@pgfmath@counta#1#2#3#4\relax%
				\c@pgfmath@countb1000000000\relax%
				\divide\c@pgfmath@countb\c@pgfmath@counta%
				\c@pgfmath@counta\c@pgfmath@countb%
				\divide\c@pgfmath@counta10000\relax%
				\pgfmath@x\c@pgfmath@counta pt\relax%
				\multiply\c@pgfmath@counta-10000\relax%
				\advance\c@pgfmath@countb\c@pgfmath@counta%
				\pgfmath@y\c@pgfmath@countb pt\relax%
				\divide\pgfmath@y1000000\relax%		
				\advance\pgfmath@x\pgfmath@y%
			\else%
				\c@pgfmath@counta#1#2#3#4#5#6\relax%
				\c@pgfmath@countb1000000000\relax%
				\divide\c@pgfmath@countb\c@pgfmath@counta%
				\c@pgfmath@counta\c@pgfmath@countb%
				\divide\c@pgfmath@counta10000\relax%
				\pgfmath@x\c@pgfmath@counta pt\relax%
				\multiply\c@pgfmath@counta-10000\relax%
				\advance\c@pgfmath@countb\c@pgfmath@counta%
				\pgfmath@y\c@pgfmath@countb pt\relax%
				\pgfmath@y.1\pgfmath@y% Yes! This way is more accurate. Go figure...
				\pgfmath@y.1\pgfmath@y%	
				\pgfmath@y.1\pgfmath@y%	
				\pgfmath@y.1\pgfmath@y%		
				\advance\pgfmath@x\pgfmath@y%
			\fi%
		\fi%
		\pgfmath@returnone\pgfmath@x%
	\endgroup
}



% \pgfmathln
%
% Calculate ln(#1}
%
% This is based on an algorithm due to Rouben Rostamian and
% uses coefficients contributed by Alain Matthes.
%
\def\pgfmathln#1{\pgfmathparse{#1}\pgfmathln@{\pgfmathresult}}
\def\pgfmathln@#1{%
	\begingroup%
		\expandafter\pgfmath@x#1pt\relax%
		\ifdim\pgfmath@x>0pt\else%
			\pgfmath@error{I cannot calculate the logarithm of `#1'}{}%
		\fi%		
		\c@pgfmath@counta0\relax%
		\ifdim\pgfmath@x>2pt\relax%
			\ifdim\pgfmath@x<128pt\relax%
				\ifdim\pgfmath@x<8pt\relax%
					\ifdim\pgfmath@x<4pt\relax%
						\divide\pgfmath@x2\relax%
						\c@pgfmath@counta1\relax%
					\else%
						\divide\pgfmath@x4\relax%
						\c@pgfmath@counta2\relax%
					\fi%
				\else%
					\ifdim\pgfmath@x<32pt\relax%
						\ifdim\pgfmath@x<16pt\relax%
							\divide\pgfmath@x8\relax%
							\c@pgfmath@counta3\relax%
						\else%
							\divide\pgfmath@x16\relax%
							\c@pgfmath@counta4\relax%
						\fi%
					\else%
						\ifdim\pgfmath@x<64pt\relax%
							\divide\pgfmath@x32\relax%
							\c@pgfmath@counta5\relax%
						\else%
							\divide\pgfmath@x64\relax%
							\c@pgfmath@counta6\relax%
						\fi%
					\fi%
				\fi%
			\else%
				\ifdim\pgfmath@x<2048pt\relax%
					\ifdim\pgfmath@x<512pt\relax%
						\ifdim\pgfmath@x<256pt\relax%
							\divide\pgfmath@x128\relax%
							\c@pgfmath@counta7\relax%
						\else%
							\divide\pgfmath@x256\relax%
							\c@pgfmath@counta8\relax%
						\fi%
					\else%
						\ifdim\pgfmath@x<1024pt\relax%
							\divide\pgfmath@x512\relax%
							\c@pgfmath@counta9\relax%
						\else%
							\divide\pgfmath@x1024\relax%
							\c@pgfmath@counta10\relax%
						\fi%
					\fi%
				\else%
					\ifdim\pgfmath@x<8192pt\relax%
						\ifdim\pgfmath@x<4096pt\relax%
							\divide\pgfmath@x2048\relax%
							\c@pgfmath@counta11\relax%
						\else%
							\divide\pgfmath@x4096\relax%
							\c@pgfmath@counta12\relax%
						\fi%
					\else%
						\divide\pgfmath@x8192\relax%
						\c@pgfmath@counta13\relax%
					\fi%
				\fi%
			\fi%	
		\else%
			\ifdim\pgfmath@x<1pt\relax%
				\ifdim\pgfmath@x>0.0078125pt\relax% 2^-7
					\ifdim\pgfmath@x>0.125pt\relax% 2^-3
						\ifdim\pgfmath@x>0.5pt\relax% 2^-1
							\multiply\pgfmath@x2\relax%
							\c@pgfmath@counta-1\relax%
						\else%
							\multiply\pgfmath@x4\relax%
							\c@pgfmath@counta-2\relax%
						\fi%
					\else%
						\ifdim\pgfmath@x>0.03125pt\relax% 2^-5
							\ifdim\pgfmath@x>0.0625pt\relax%
								\multiply\pgfmath@x8\relax%
								\c@pgfmath@counta-3\relax%
							\else%
								\multiply\pgfmath@x16\relax%
								\c@pgfmath@counta-4\relax%
							\fi%
						\else%
							\ifdim\pgfmath@x>0.015625pt\relax% 2^-6
								\multiply\pgfmath@x32\relax%
								\c@pgfmath@counta-5\relax%
							\else%
								\multiply\pgfmath@x64\relax%
								\c@pgfmath@counta-6\relax%
							\fi%
						\fi%
					\fi%
				\else%
					\ifdim\pgfmath@x>0.000244140625pt\relax% 2^-11
						\ifdim\pgfmath@x>0.001953125pt\relax% 2^-9
							\ifdim\pgfmath@x>0.00390625pt\relax% 2^-8
								\multiply\pgfmath@x128\relax%
								\c@pgfmath@counta-7\relax%
							\else%
								\multiply\pgfmath@x256\relax%
								\c@pgfmath@counta-8\relax%
							\fi%
						\else%
							\ifdim\pgfmath@x>0.0009765625pt\relax% 2^-10
								\multiply\pgfmath@x512\relax%
								\c@pgfmath@counta-9\relax%
							\else%
								\multiply\pgfmath@x1024\relax%
								\c@pgfmath@counta-10\relax%
							\fi%
						\fi%
					\else%
						\ifdim\pgfmath@x>0.0001220703125pt\relax% 2^13
							\ifdim\pgfmath@x>0.0002441406256pt\relax% 2^12
								\multiply\pgfmath@x2048\relax%
								\c@pgfmath@counta-11\relax%
							\else%
								\multiply\pgfmath@x4096\relax%
								\c@pgfmath@counta-12\relax%
							\fi%
						\else%
							\multiply\pgfmath@x8192\relax%
							\c@pgfmath@counta-13\relax%
						\fi%
					\fi%
				\fi%
			\fi%	
		\fi%
		%
		% Use A+(B+(C+(D+(E+F*x)*x)*x)*x)*x
		%
		% where:
		% A = -2.787927935
		% B =  5.047861502
		% C = -3.489886985
		% D =  1.589480044
		% E = -.4025153233
		% F =  0.04300521312
		%
		\edef\pgfmath@temp{\pgfmath@tonumber{\pgfmath@x}}%
		\pgfmath@x0.04300521312\pgfmath@x%
		\advance\pgfmath@x-.4025153233pt\relax%
		\pgfmath@x\pgfmath@temp\pgfmath@x%
		\advance\pgfmath@x1.589480044pt\relax%
		\pgfmath@x\pgfmath@temp\pgfmath@x%
		\advance\pgfmath@x-3.489886985pt\relax%
		\pgfmath@x\pgfmath@temp\pgfmath@x%
		\advance\pgfmath@x5.047861502pt\relax%
		\pgfmath@x\pgfmath@temp\pgfmath@x%
		\advance\pgfmath@x-2.787927935pt\relax%
		\advance\pgfmath@x\c@pgfmath@counta pt\relax%
		\pgfmath@x0.6931471806\pgfmath@x%	
		\pgfmath@returnone\pgfmath@x%
	\endgroup%
}
	
% \pgfmathabs
%
% Calculate |#1|
%
\def\pgfmathabs#1{%
	\pgfmathparse{#1}%
	\pgfmathabsolute@{\pgfmathresult}}
\def\pgfmathabs@#1{%
	\begingroup%
		\expandafter\pgfmath@x#1pt\relax%
		\ifdim\pgfmath@x<0pt\relax%
			\pgfmath@x=-\pgfmath@x%
		\fi%
	\pgfmath@returnone\pgfmath@x%
	\endgroup%
}

% \pgfmathmod
%
% Calculate #1 mod #2.
%
\def\pgfmathmod#1#2{%
	\pgfmathparse{#1}\let\pgfmath@a\pgfmathresult%
	\pgfmathparse{#2}\let\pgfmath@b\pgfmathresult%
	\pgfmathmod@{\pgfmath@a}{\pgfmath@b}%
}
\def\pgfmathmod@#1#2{%
	\begingroup%
		\edef\pgfmath@a{#1}%
		\edef\pgfmath@b{#2}%
		\pgfmath@x\pgfmath@a pt\relax%
		\pgfmath@xa\pgfmath@x%
		\pgfmath@xb\pgfmath@b pt\relax%
		\c@pgfmath@counta=\pgfmath@xa%
		\c@pgfmath@countb=\pgfmath@xb%
		\divide\c@pgfmath@counta\c@pgfmath@countb%
		\multiply\pgfmath@xb\c@pgfmath@counta%
		\advance\pgfmath@x-\pgfmath@xb%
		\pgfmath@returnone\pgfmath@x%
	\endgroup%
}

% \pgfmathsqrt
%
% Square-root of #1.
%
%
\def\pgfmathsqrt#1{\pgfmathparse{#1}\pgfmathsqrt@{\pgfmathresult}}
\def\pgfmathsqrt@#1{%
	\begingroup%
		\expandafter\pgfmath@x#1pt\relax%
		\ifdim\pgfmath@x>9999pt\relax%
			\def\pgfmath@zeros{0}%
			\def\pgfmath@targetiterations{3}%
		\else%
			\ifdim\pgfmath@x>999pt\relax%
				\def\pgfmath@zeros{}%
				\def\pgfmath@targetiterations{2}%
			\else%
				\ifdim\pgfmath@x>99pt\relax%
					\def\pgfmath@zeros{0}%
					\def\pgfmath@targetiterations{2}%
				\else%
					\ifdim\pgfmath@x>9pt\relax%
						\def\pgfmath@zeros{}%
						\def\pgfmath@targetiterations{1}%
					\else%
						\ifdim\pgfmath@x<0pt\relax%
							\pgfmath@error{I cannot calculate the square-root of the negative number `#1'}{}%
						\else%
							\def\pgfmath@zeros{0}%
							\def\pgfmath@targetiterations{1}%
		\fi\fi\fi\fi\fi%
		\edef\pgfmath@temp{\pgfmath@zeros\pgfmath@tonumber{\pgfmath@x}}%
		\expandafter\pgfmath@sqrt@\pgfmath@temp\pgfmath@%
}
\def\pgfmath@sqrt@#1.#2\pgfmath@{\pgfmath@@sqrt@#1#2\pgfmath@empty\pgfmath@empty\pgfmath@}

\def\pgfmath@@sqrt@#1#2{%
		\c@pgfmath@countb#1#2\relax%
		\ifnum\c@pgfmath@countb>35\relax%
			\ifnum\c@pgfmath@countb>63\relax%
				\ifnum\c@pgfmath@countb>80\relax%
					\c@pgfmath@counta9\relax%
				\else%
					\c@pgfmath@counta8\relax%
				\fi%
			\else%
				\ifnum\c@pgfmath@countb>48\relax%
					\c@pgfmath@counta7\relax%
				\else%
					\c@pgfmath@counta6\relax%
				\fi%
			\fi%
		\else%
			\ifnum\c@pgfmath@countb>15\relax%
				\ifnum\c@pgfmath@countb>24\relax%
					\c@pgfmath@counta5\relax%
				\else%
					\c@pgfmath@counta4\relax%
				\fi%
			\else%
				\ifnum\c@pgfmath@countb>3\relax%
					\ifnum\c@pgfmath@countb>8\relax%
						\c@pgfmath@counta3\relax%
					\else%
						\c@pgfmath@counta2\relax%
					\fi%
				\else%
					\ifnum\c@pgfmath@countb>0\relax%
						\c@pgfmath@counta1\relax%
					\else%
						\c@pgfmath@counta0\relax%
					\fi%
				\fi%
			\fi%
		\fi%
		\edef\pgfmath@root{\the\c@pgfmath@counta}%
		\edef\pgfmath@rootspecial{\the\c@pgfmath@counta}%
		\multiply\c@pgfmath@counta-\c@pgfmath@counta\relax%
		\advance\c@pgfmath@counta#1#2\relax%
		\edef\pgfmath@remainder{\the\c@pgfmath@counta}%
		\pgfmath@@@sqrt@%
}

\def\pgfmath@@@sqrt@#1#2{%
		\ifx\pgfmath@empty#1%
			\edef\pgfmath@remainder{\pgfmath@remainder00}%
			\def\pgfmath@tokens{\pgfmath@empty\pgfmath@empty}%
		\else%
			\ifx\pgfmath@empty#2%
				\edef\pgfmath@remainder{\pgfmath@remainder#10}%
				\def\pgfmath@tokens{\pgfmath@empty\pgfmath@empty}%
			\else%
				\edef\pgfmath@remainder{\pgfmath@remainder#1#2}%
				\def\pgfmath@tokens{}%
		\fi\fi%
		\c@pgfmath@counta\pgfmath@rootspecial\relax%
		\multiply\c@pgfmath@counta20\relax%
		\c@pgfmath@countb\c@pgfmath@counta%
		\multiply\c@pgfmath@countb6\relax%
		\advance\c@pgfmath@countb36\relax%
		\c@pgfmath@countc\c@pgfmath@counta\relax%
		\multiply\c@pgfmath@countc2\relax%
		\ifnum\c@pgfmath@countb>\pgfmath@remainder\relax% 
			\advance\c@pgfmath@countb-\c@pgfmath@countc%
			\advance\c@pgfmath@countb-20\relax%
			\ifnum\c@pgfmath@countb>\pgfmath@remainder\relax%
				\advance\c@pgfmath@countb-\c@pgfmath@countc%
				\advance\c@pgfmath@countb-12\relax%
				\ifnum\c@pgfmath@countb>\pgfmath@remainder\relax%
					\advance\c@pgfmath@countb-\c@pgfmath@counta%
					\advance\c@pgfmath@countb-3\relax%
					\ifnum\c@pgfmath@countb>\pgfmath@remainder\relax%
						\def\pgfmath@digit{0}%
					\else%
						\def\pgfmath@digit{1}%
					\fi%
				\else%
					\advance\c@pgfmath@countb\c@pgfmath@counta%
					\advance\c@pgfmath@countb5\relax%
					\ifnum\c@pgfmath@countb>\pgfmath@remainder\relax%
						\def\pgfmath@digit{2}%
					\else%
						\def\pgfmath@digit{3}%
					\fi%
				\fi%
			\else%
				\advance\c@pgfmath@countb\c@pgfmath@counta%
				\advance\c@pgfmath@countb9\relax%
				\ifnum\c@pgfmath@countb>\pgfmath@remainder\relax%
					\def\pgfmath@digit{4}%
				\else%
					\def\pgfmath@digit{5}%
				\fi%
			\fi%
		\else%
			\advance\c@pgfmath@countb\c@pgfmath@countc%
			\advance\c@pgfmath@countb28\relax%
			\ifnum\c@pgfmath@countb>\pgfmath@remainder\relax%
				\advance\c@pgfmath@countb-\c@pgfmath@counta%
				\advance\c@pgfmath@countb-15\relax%
				\ifnum\c@pgfmath@countb>\pgfmath@remainder\relax%
					\def\pgfmath@digit{6}%
				\else%
					\def\pgfmath@digit{7}%
				\fi%
			\else%
				\advance\c@pgfmath@countb\c@pgfmath@counta%
				\advance\c@pgfmath@countb17\relax%
				\ifnum\c@pgfmath@countb>\pgfmath@remainder\relax%
					\def\pgfmath@digit{8}%
				\else%
					\def\pgfmath@digit{9}%
				\fi%
			\fi%
		\fi%
		\edef\pgfmath@rootspecial{\pgfmath@rootspecial\pgfmath@digit}%
		\advance\c@pgfmath@counta\pgfmath@digit\relax%
		\multiply\c@pgfmath@counta-\pgfmath@digit\relax%
		\advance\c@pgfmath@counta\pgfmath@remainder\relax%
		\edef\pgfmath@remainder{\the\c@pgfmath@counta}%
		\c@pgfmath@counta\pgfmath@targetiterations\relax%
		\advance\c@pgfmath@counta-1\relax%
		\edef\pgfmath@targetiterations{\the\c@pgfmath@counta}%
		\ifnum\c@pgfmath@counta=0\relax%
			\edef\pgfmath@root{\pgfmath@root.\pgfmath@digit}%
		\else%
			\edef\pgfmath@root{\pgfmath@root\pgfmath@digit}%
		\fi%
		\ifnum\c@pgfmath@counta=-4\relax%
			\let\pgfmath@next\pgfmath@sqrt@end%
		\else%
			\let\pgfmath@next\pgfmath@@@sqrt@%
		\fi%
		\expandafter\pgfmath@next\pgfmath@tokens%
}

\def\pgfmath@sqrt@end#1\pgfmath@{%
		\edef\pgfmathresult{\pgfmath@root}%
		\pgfmath@smuggleone\pgfmathresult%
	\endgroup}

% \pgfmathpow
%
% Calculates #1 ^ #2
%
% #2 is expected to be an integer.
%
\def\pgfmathpow#1#2{%
	\pgfmathparse{#1}\let\pgfmath@powera\pgfmathresult%
	\pgfmathparse{#2}\let\pgfmath@powerb\pgfmathresult%
	\pgfmathpow@{\pgfmath@powera}{\pgfmath@powerb}}
\def\pgfmathpow@#1#2{%
	\begingroup%
		\edef\pgfmath@a{#1}%
		\edef\pgfmath@b{#2}%
		\pgfmath@xa\pgfmath@a pt\relax%
		\afterassignment\pgfmath@gobbletilpgfmath@%
		\expandafter\c@pgfmath@counta\pgfmath@b\relax\pgfmath@
		% If #2 is negative, take the reciprocal of #1
		% and the absolute value of #2, and carry on.
		%
		\ifnum\c@pgfmath@counta<0\relax%
			\c@pgfmath@counta-\c@pgfmath@counta%
			\pgfmathreciprocal@{#1}%
			\pgfmath@xa\pgfmathresult pt\relax%
		\fi%
		\pgfmath@x=1pt\relax%
		\pgfmathloop%
			\ifnum\c@pgfmath@counta>0\relax%
				\ifodd\c@pgfmath@counta%
					\pgfmath@x\pgfmath@tonumber{\pgfmath@x}\pgfmath@xa%
				\fi
				\ifnum\c@pgfmath@counta>1\relax%
					\pgfmath@xa=\pgfmath@tonumber{\pgfmath@xa}\pgfmath@xa%
				\fi%
				\divide\c@pgfmath@counta by 2\relax%
		\repeatpgfmathloop%
		\pgfmath@returnone\pgfmath@x%
	\endgroup%
}	


% \pgfmathround
% 
% Half-up rounding.
%
\def\pgfmathround#1{\pgfmathparse{#1}\pgfmathround@{\pgfmathresult}}
\def\pgfmathround@#1{%
	\begingroup%
		\expandafter\pgfmath@x#1pt\relax%
		\afterassignment\pgfmath@xa%
		\expandafter\c@pgfmath@counta\the\pgfmath@x\relax%
		\pgfmath@xb\pgfmath@x%
		\ifdim\pgfmath@xb<0pt\relax%
			\ifdim\pgfmath@xa<0.5pt\relax%
			\else%
				\advance\c@pgfmath@counta-1\relax%
			\fi%
		\else%
			\ifdim\pgfmath@xa<0.5pt\relax%
			\else%
				\advance\c@pgfmath@counta1\relax%
			\fi%
		\fi%
		\pgfmath@returnone\c@pgfmath@counta%
	\endgroup%
}%

% \pgfmathfloor
% 
% Floor function.
%
\def\pgfmathfloor#1{\pgfmathparse{#1}\pgfmathfloor@{\pgfmathresult}}
\def\pgfmathfloor@#1{%
	\begingroup%
		\expandafter\pgfmath@x#1pt\relax%
		\afterassignment\pgfmath@gobbletilpgfmath@%
		\expandafter\c@pgfmath@counta\the\pgfmath@x\relax\pgfmath@%
		\pgfmath@x\c@pgfmath@counta pt\relax%
		\pgfmath@returnone\pgfmath@x%
	\endgroup%
}%

% \pgfmathceil
% 
% Ceiling function.
%
\def\pgfmathceil#1{\pgfmathparse{#1}\pgfmathceil@{\pgfmathresult}}
\def\pgfmathceil@#1{%
	\begingroup%
		\expandafter\pgfmath@x#1pt\relax%
		\afterassignment\pgfmath@gobbletilpgfmath@%
		\expandafter\c@pgfmath@counta\the\pgfmath@x\relax\pgfmath@%
		\pgfmath@y\pgfmath@x%
		\advance\pgfmath@y-\c@pgfmath@counta pt\relax%
		\pgfmath@x\c@pgfmath@counta pt\relax%
		\ifdim\pgfmath@y>0pt\relax%
			\advance\pgfmath@x1pt\relax%
		\fi%
	\pgfmath@returnone\pgfmath@x%
	\endgroup%
}%

% \pgfmathexp
%
% A Maclaurens expansion for e^#1.
% 0 <= #1 < ln(16384).
%
\def\pgfmathexp#1{\pgfmathparse{#1}\pgfmathexp@{\pgfmathresult}}
\def\pgfmathexp@#1{%
	\begingroup%
		\pgfmath@x1pt\relax%
		\pgfmath@xa1pt\relax%
		\pgfmath@xb\pgfmath@x%
		\pgfmathloop%
			\pgfmath@xc\pgfmathcounter pt\relax%
			\c@pgfmath@counta\pgfmath@xc%
			\divide\c@pgfmath@counta65536\relax%
			\pgfmath@xc1pt\relax%
			\divide\pgfmath@xc\c@pgfmath@counta%
			\pgfmath@xa\pgfmath@tonumber{\pgfmath@xc}\pgfmath@xa%
			\expandafter\pgfmath@xa#1\pgfmath@xa%
			\advance\pgfmath@x\pgfmath@xa%
			\ifdim\pgfmath@x=\pgfmath@xb%
			\else%
				\pgfmath@xb\pgfmath@x%
		\repeatpgfmathloop%
	\pgfmath@returnone\pgfmath@x%
	\endgroup%
}



% \pgfmathvectorlength
%
% Calcluate the Eulidean length of a 2D vector.
%
% This based on polynomial approximation co-efficents
% contributed by Rouben Rostamian.
%
% #1 - the x component of the vector.
% #2 - the y component of the vector.
%
\def\pgfmathveclen#1#2{%
	\pgfmathparse{#1}\let\pgfmath@vecx\pgfmathresult%
	\pgfmathparse{#2}\let\pgfmath@vecy\pgfmathresult%
	\pgfmathveclen@{\pgfmath@vecx}{\pgfmath@vecy}%
}
\def\pgfmathveclen@#1#2{%
	\begingroup%
		\edef\pgfmath@a{#1}%
		\edef\pgfmath@b{#2}%
		\pgfmath@x\pgfmath@a pt\relax%
		\pgfmath@y\pgfmath@b pt\relax%
		\ifdim\pgfmath@x<0pt\relax%
			\pgfmath@x-\pgfmath@x%
		\fi%
		\ifdim\pgfmath@y<0pt\relax%
			\pgfmath@y-\pgfmath@y%
		\fi%
		\ifdim\pgfmath@x=0pt\relax%
			\pgfmath@x\pgfmath@y%
		\else%
			\ifdim\pgfmath@y=0pt\relax%
			\else%
				\ifdim\pgfmath@x>\pgfmath@y%
					\pgfmath@xa\pgfmath@x%
					\pgfmath@x\pgfmath@y%
					\pgfmath@y\pgfmath@xa%
				\fi%
				% We use a scaling factor to reduce errors.
				\ifdim\pgfmath@y>10000pt\relax%
					\c@pgfmath@counta1500\relax%
				\else%
					\ifdim\pgfmath@y>1000pt\relax%
						\c@pgfmath@counta150\relax%
					\else%
						\ifdim\pgfmath@y>100pt\relax%
							\c@pgfmath@counta50\relax%
						\else%
							\c@pgfmath@counta1\relax%
						\fi%
					\fi%
				\fi%
				\divide\pgfmath@x by\c@pgfmath@counta\relax%
				\divide\pgfmath@y by\c@pgfmath@counta\relax%
				\pgfmathreciprocal@{\pgfmath@tonumber{\pgfmath@y}}%
				\pgfmath@x\pgfmathresult\pgfmath@x%
				\pgfmath@xa\pgfmath@tonumber{\pgfmath@x}\pgfmath@x%
				\edef\pgfmath@temp{\pgfmath@tonumber{\pgfmath@xa}}%
				%
				% Use A+x^2*(B+x^2*(C+x^2*(D+E*x^2))) 
				% where
				% A = +1.000012594
				% B = +0.4993615349 
				% C = -0.1195159052
				% D = +0.04453994279
				% E = -0.01019210944
				%
				\pgfmath@x-0.01019210944\pgfmath@xa%
				\advance\pgfmath@x0.04453994279pt\relax%
				\pgfmath@x\pgfmath@temp\pgfmath@x%
				\advance\pgfmath@x-0.1195159052pt\relax%
				\pgfmath@x\pgfmath@temp\pgfmath@x%
				\advance\pgfmath@x0.4993615349pt\relax%
				\pgfmath@x\pgfmath@temp\pgfmath@x%
				\advance\pgfmath@x1.000012594pt\relax%
				\ifdim\pgfmath@y<0pt\relax%
					\pgfmath@y-\pgfmath@y%
				\fi%
				\pgfmath@x\pgfmath@tonumber{\pgfmath@y}\pgfmath@x%
				% Invert the scaling factor.
				\multiply\pgfmath@x by\c@pgfmath@counta\relax%
			\fi%
		\fi%
		\pgfmath@returnone\pgfmath@x%
	\endgroup%
}

% \pgfmathmax
%
% Return the maximum of #1 or #2
%
\def\pgfmathmax#1#2{%
	\pgfmathparse@{#1}\let\pgfmath@a\pgfmathresult%
	\pgfmathparse@{#2}\let\pgfmath@b\pgfmathresult%
	\pgfmathmax@{\pgfmath@a}{\pgfmath@b}}
\def\pgfmathmax@#1#2{%
	\begingroup
		\edef\pgfmath@a{#1}%
		\edef\pgfmath@b{#2}%
		\pgfmath@x\pgfmath@a pt\relax%
		\pgfmath@y\pgfmath@b pt\relax%
		\ifdim\pgfmath@x>\pgfmath@y%
			\pgfmath@returnone\pgfmath@x%
		\else%
			\pgfmath@returnone\pgfmath@y%
		\fi%
	\endgroup}

% \pgfmathmax
%
% Return the minimim of #1 or #2
%
\def\pgfmathmin#1#2{%
	\pgfmathparse@{#1}\let\pgfmath@a\pgfmathresult%
	\pgfmathparse@{#2}\let\pgfmath@b\pgfmathresult%
	\pgfmathmin@{\pgfmath@a}{\pgfmath@b}}
\def\pgfmathmin@#1#2{%
	\begingroup
		\edef\pgfmath@a{#1}%
		\edef\pgfmath@b{#2}%
		\pgfmath@x\pgfmath@a pt\relax%
		\pgfmath@y\pgfmath@b pt\relax%
		\ifdim\pgfmath@x<\pgfmath@y%
			\pgfmath@returnone\pgfmath@x%
		\else%
			\pgfmath@returnone\pgfmath@y%
		\fi%
	\endgroup%
}

% \pgfmathscientific
%
% Return the value of #1e#2
%
% e.g. \pgfmathscientific{1.23456789123}{4}
%
% defines \pgfmathresult as 12345.67891
%
% NB This arguments *are not parsed*, as the long mantissa would be lost.
%
\def\pgfmathscientific#1#2{%
	\begingroup%
		\edef\pgfmath@sci@exponent{#2}%
		\pgfmath@x#1pt\relax%
		\ifdim\pgfmath@x<0pt\relax%
			\pgfmath@x-\pgfmath@x%
			\edef\pgfmath@sci@sign{-}%
			\edef\pgfmath@temp{\pgfmath@gobbleone#1}%
		\else%
			\edef\pgfmath@sci@sign{+}%
			\edef\pgfmath@temp{#1}%
		\fi%
		\expandafter\pgfmath@scientific@@\pgfmath@temp00000000000\pgfmath@}

\def\pgfmath@scientific@@#1.#2#3#4#5#6{%
		\edef\pgfmath@sci@int{#1}%
		\edef\pgfmath@sci@mantissaA{#2#3#4#5#6}%
		\pgfmath@scientific@@@}
	
\def\pgfmath@scientific@@@#1#2#3#4#5#6\pgfmath@{%
		\edef\pgfmath@sci@mantissaB{#1#2#3#4#5}%
		\c@pgfmath@counta\pgfmath@sci@exponent\relax%
		\c@pgfmath@countb\c@pgfmath@counta%
		\ifnum\c@pgfmath@counta<0\relax%
			\c@pgfmath@counta-\c@pgfmath@counta%
		\fi%
		\pgfmathpow@{10}{\the\c@pgfmath@counta}%
		\afterassignment\pgfmath@gobbletilpgfmath@
		\c@pgfmath@countc\pgfmathresult\relax\pgfmath@
		\edef\pgfmath@sci@factor{\the\c@pgfmath@countc}%
		\ifnum\c@pgfmath@countb<0\relax%
			% xE-y: easy...
			\pgfmath@x\pgfmath@sci@int.\pgfmath@sci@mantissaA pt\relax%
			\divide\pgfmath@x\pgfmath@sci@factor\relax%
		\else%
			% xE+y: 
			% Must do this way so as not lose digits in a long mantissa. Sigh...
			\c@pgfmath@counta\pgfmath@sci@int%
			\c@pgfmath@countb\pgfmath@sci@mantissaA%
			\multiply\c@pgfmath@counta\pgfmath@sci@factor\relax%
			\multiply\c@pgfmath@countb\pgfmath@sci@factor\relax%
			\c@pgfmath@countc\c@pgfmath@countb%
			\divide\c@pgfmath@countb100000\relax%
			\advance\c@pgfmath@counta\c@pgfmath@countb%
			\multiply\c@pgfmath@countb100000\relax%
			\advance\c@pgfmath@countc-\c@pgfmath@countb%
			\c@pgfmath@countb\pgfmath@sci@mantissaB\relax%
			\multiply\c@pgfmath@countb\pgfmath@sci@factor\relax%
			\divide\c@pgfmath@countb100000\relax%
			\advance\c@pgfmath@countc\c@pgfmath@countb%
			\advance\c@pgfmath@countc100000\relax%
			\edef\pgfmath@sci@result{\pgfmath@sci@sign\the\c@pgfmath@counta.\expandafter\pgfmath@gobbleone\the\c@pgfmath@countc pt}%
			\pgfmath@x\pgfmath@sci@result\relax%
		\fi%
		\pgfmath@returnone\pgfmath@x%
	\endgroup}
	
% \pgfmathanglebetweenpoints
%
% Define \pgfmathresult as the angle between points #1 and #2
% Should get the quadrants right as well.
%
\def\pgfmathanglebetweenpoints#1#2{%
	\begingroup%
		\pgf@process{\pgfpointdiff{#1}{#2}}%
		%
		% First approximate the angle of the external point...
		%
		\pgf@xa\pgf@x%
		\pgf@ya\pgf@y%
		\pgf@xb\pgf@x%
		\pgf@yb\pgf@y%
		\ifdim\pgf@xa<0pt\relax%
			\pgf@xa-\pgf@xa%
		\fi%
		\ifdim\pgf@ya<0pt\relax%
			\pgf@ya-\pgf@ya%
		\fi%
		\ifdim\pgf@ya>\pgf@xa%
			\pgf@x\pgf@xa%
			\pgf@y\pgf@ya%
		\else%
			\pgf@x\pgf@ya%
			\pgf@y\pgf@xa%
		\fi%
		\ifdim\pgf@y=0pt\relax%
			\pgf@x0pt%
		\else%
			\pgfmathreciprocal@{\pgf@sys@tonumber{\pgf@y}}%
			\pgf@x\pgfmathresult\pgf@x%
		\fi%
		\multiply\pgf@x1000\relax%
		\afterassignment\pgfmath@gobbletilpgfmath@%
		\expandafter\c@pgf@counta\the\pgf@x\relax\pgfmath@%
		\expandafter\pgf@x\csname pgfmath@atan@\the\c@pgf@counta\endcsname pt\relax%
		\ifdim\pgfmath@ya>\pgfmath@xa\relax%
			\pgf@x-\pgf@x%
			\advance\pgf@x90pt%
		\fi%
		\ifdim\pgf@xb<0pt%
			\ifdim\pgf@yb>0pt%
				\pgf@x-\pgf@x%
			\fi%
			\advance\pgf@x180pt\relax%
		\else%
			\ifdim\pgf@yb<0pt%
			\pgf@x-\pgf@x%
			\advance\pgf@x360pt\relax%
		\fi\fi%
		\pgfmath@returnone\pgf@x%
	\endgroup%
}

% \pgfmathrotatepointaround
%
% Rotate point #1 about point #2 by #3 degrees.
%
\def\pgfmathrotatepointaround#1#2#3{%
	\begingroup%
		\pgf@process{#2}%
		\pgf@xa\pgf@x%
		\pgf@ya\pgf@y%
		\pgf@process{#1}%
		\advance\pgf@x-\pgf@xa%
		\advance\pgf@y-\pgf@ya%
		\pgfmathsetmacro\angle{#3}%
		\pgfmathsin@{\angle}%
		\let\sineangle\pgfmathresult%
		\pgfmathcos@{\angle}%
		\let\cosineangle\pgfmathresult%
		\pgf@xa\cosineangle\pgf@x%
		\advance\pgf@xa-\sineangle\pgf@y%
		\pgf@ya\sineangle\pgf@x%
		\advance\pgf@ya\cosineangle\pgf@y%
		\pgf@process{#2}%
		\pgf@process{%
			\advance\pgf@x\pgf@xa%
			\advance\pgf@y\pgf@ya%
		}%
	\endgroup%
}

% \pgfmathanglebetweenlines
%
% Calculate the clockwise angle between a line from point #1
% to point #2 and a line from #3 to point #4.
%
\def\pgfmathanglebetweenlines#1#2#3#4{%
	\begingroup%
		\pgfmathanglebetweenpoints{#1}{#2}%
		\let\firstangle\pgfmathresult%
		\pgfmathanglebetweenpoints{#3}{#4}%
		\let\secondangle\pgfmathresult%
		\ifdim\firstangle pt>\secondangle pt\relax%
			\pgfmathadd@{\secondangle}{360}%
			\let\secondangle\pgfmathresult%
		\fi%
		\pgfmathsubtract@{\secondangle}{\firstangle}%
		\pgfmath@smuggleone\pgfmathresult%
	\endgroup%
}


% \pgfmathsetlength, \pgfmathaddtolength
%
% #1 = dimension register
% #2 = expression
%
% Description:
%
% These functions work similar to \setlength and \addtolength. Only,
% they allow #2 to contain an expression, which is evaluated before
% assignment. Furthermore, the font is setup before the assignment is
% done, so that dimensions like 1em are evaluated correctly.
%
% If #2 starts with "+", then a simple assignment is done (but the
% font is still setup). This is orders of magnitude faster than a
% parsed assignment.

\newdimen\mydim
\def\pgfmathsetlength#1#2{%
  \expandafter\pgfmath@onquick#2@@\pgfmath@%
  {%
    % Ok, quick version:
    #1#2\relax%
  }%
  {\pgfmathparse{#2}#1\pgfmathresult pt\relax}%
}
\def\pgfmathaddtolength#1#2{%
  \expandafter\pgfmath@onquick#2@@\pgfmath@%
  {%
    % Ok, quick version:
    \advance#1by#2\relax%
  }%
  {\pgfmathparse{#2}\advance#1\pgfmathresult pt\relax}%
}

% \pgfmathsetcounter, \pgfmathaddtocounter
%
% Results of parsing are truncated.
%
\def\pgfmathsetcounter#1#2{%
  \expandafter\pgfmath@onquick#2@@\pgfmath@%
  {%
    \csname c@#1\endcsname=#2\relax%
  }%
  {%
    \pgfmath@ifundefined{c@#1}{\pgfmath@error{No counter named '#1' is known}{}}{%
      \pgfmathparse{#2}%
      \afterassignment\pgfmath@gobbletilpgfmath@%
      \csname c@#1\endcsname\pgfmathresult\relax\pgfmath@%
    }%
  }%
}

\def\pgfmathaddtocounter#1#2{%
  \expandafter\pgfmath@onquick#2@@\pgfmath@%
  {%
    \advance\csname c@#1\endcsname by#2\relax%
  }%
  {%
    \pgfmath@ifundefined{c@#1}{\pgfmath@error{No counter named '#1' is known}{}}{%
      \edef\pgfmath@addtocountertemp{\expandafter\the\csname c@#1\endcsname}%
      \pgfmathparse{#2}%
      \afterassignment\pgfmath@gobbletilpgfmath@%
      \csname c@#1\endcsname\pgfmathresult\relax\pgfmath@%
      \expandafter\advance\csname c@#1\endcsname\pgfmath@addtocountertemp%
    }%
  }%
}


% Check whether a given parameter starts with quick.
%
% The command should be followed by nonempty text, ending with
% \pgfmath@ as a stop-token. Then should follow
%
% #1 = code to execute if text starts with +
% #2 = code to execute if text does not
%
% Example:
%
% \pgfmath@onquick+0pt\pgfmath@{is quick}{is slow}

\def\pgfmath@onquick{%
  \afterassignment\pgfmath@afterquick%
  \let\pgfmath@next=%
}

\def\pgfmath@afterquick#1\pgfmath@{%
  \ifx\pgfmath@next+%
    \expandafter\pgfmath@firstoftwo%
  \else%
    \expandafter\pgfmath@secondoftwo%
  \fi%
}




% \pgfmathdeclarefunction
%
% Declare a function to be used with \pgfmathusefunction
%
% #1 -> the name of the function.
% #2 -> a list of variables used by the function. When the
%       function is called with a list of arguments, these
%       variables will be instantiated to the arguments.
% #3 -> the function definition.
%
% E.g.
%
% \pgfmathdeclarefunction{my function}{\t}{sin{\t r}*60}
%
\def\pgfmathdeclarefunction#1#2#3{%	
	\expandafter\def\csname pgfmath@function@#1\endcsname{#3}%
	\expandafter\def\csname pgfmath@function@#1@arguments@\endcsname{#2}%
}%

% \pgfmathusefunction
%
% Call a declared function with appropriate arguments to
% calculate the value of the function. 
%
% Assume the function f has been declared using
%
% \pgfmathdeclarefunction{f}{\t}{3*\t+4}
%
% We say:
%
% \pgfmathusefunction{\y}{f}{\x}
%
% where \x is a macro representing a number, which forms
% the argument to the function. This will calculate f(x)
% and place the value (*without* dimension) in the macro 
% \y. The \t in the function definition is equated with
% the value of the argument \x.
%
% #1 -> the macro to store the evaluated function.
% #1 -> the function name.
% #2 -> a list of arguments.
%
\def\pgfmathusefunction#1#2#3{%
	\def\pgfmath@functionname{#2}%
	\pgfmath@ifundefined{pgfmath@function@#2}{%
		\pgfmath@error{Unknown function `#2'.}}{%
			\begingroup%
				\expandafter\ifx\csname pgfmath@function@#2@arguments@\endcsname\pgfmath@empty
				\else%
					\expandafter%
					\pgfmath@equatevariables\expandafter{\csname pgfmath@function@#2@arguments@\endcsname}{#3}%
				\fi%
				\edef\pgfmath@temp{\csname pgfmath@function@#2\endcsname}%
				\expandafter\pgfmathparse\expandafter{\pgfmath@temp}%
				\pgfmath@x=\pgfmathresult pt\relax%
				\pgfmath@returnone\pgfmath@x
			\endgroup%
			\edef#1{\pgfmathresult}%
		}%
}

% \pgfmath@equatevariables
%
% Internal macro for assigning a list of values to
% a list of variables. 
%
% \pgfmath@equatevariables{\a,\b,\c}{1.2,4,6.73}
%
% will resulit in \a <- 1.2, \b <- 4 and \c <- 6.73
%
% NB Assumes \pgfmath@functionname is defined.
%
% #1 -> list of variables.
% #2 -> list of arguments.
%
\def\pgfmath@equatevariables#1#2{%
		\expandafter\pgfmath@equatevariables@\expandafter{#2}{#1}}%
\def\pgfmath@equatevariables@#1#2{%
	\expandafter\pgfmath@equatevariables@@#2,\pgfmath@empty,\pgfmath@empty,\pgfmath@%
		#1,\pgfmath@empty,\pgfmath@empty,\pgfmath@}%
\def\pgfmath@equatevariables@@#1,#2,\pgfmath@#3,#4,\pgfmath@{%
	\ifx\pgfmath@empty#1\relax%
		\ifx\pgfmath@empty#3\relax%
		\else%
			\pgfmath@reporterror{Function `\pgfmath@functionname' called with too many variables}{}%
		\fi%
		\let\pgfmath@next\pgfmath@endequatevariables%
	\else%
		\ifx\pgfmath@empty#3\relax%
			\pgfmath@reporterror{Function `\pgfmath@functionname' called with too few variables}{}%
			\let\pgfmath@next\pgfmath@endequatevariables%
		\else%
			\expandafter\pgfmathparse\expandafter{#3}%
			\edef#1{\pgfmathresult}%
			\let\pgfmath@next\pgfmath@equatevariables@@%
		\fi%
	\fi%
	\pgfmath@next#2,\pgfmath@#4,\pgfmath@}

\def\pgfmath@endequatevariables#1\pgfmath@#2\pgfmath@{}


% \pgfmathlinearspace
%
% Partition a linear space.
%
% #1 -> the macro to store the linear space.
% #2 -> the first term.
% #3 -> the last term.
% #4 -> the number of partitions.
%
% e.g.
% \pgfmathlinearspace{\x}{0}{20}{4}
% 
% Results in \x -> 0.0,5.0,10.0,15.0,20.0
%
\def\pgfmathlinearspace#1#2#3#4{%
	\begingroup%
		\def\pgfmath@temppartitions{#4}%
		\ifx\pgfmath@temppartitions\pgfmath@empty%
			\def\pgfmath@temppartitions{100}%
		\fi%
		\pgfmathsetlength{\pgfmath@xa}{#2}%
		\pgfmathsetlength{\pgfmath@xb}{#3}%
		\pgfmath@xc=\pgfmath@xb%
		\advance\pgfmath@xb by-\pgfmath@xa%
		\pgfmath@y=\pgfmath@xb%
		\divide\pgfmath@y by\pgfmath@temppartitions%
		\xdef\pgfmathlinearspace@tempa{\pgfmath@tonumber{\pgfmath@y}}%
		\xdef\pgfmathlinearspace@tempb{\pgfmath@tonumber{\pgfmath@xa}}%
		\pgfmathloop%
			\ifnum\pgfmathcounter<#4\relax%
				\pgfmath@x=\pgfmath@xb%
				\divide\pgfmath@x by#4\relax%
				\multiply\pgfmath@x by\pgfmathcounter\relax%
				\advance\pgfmath@x by\pgfmath@xa%
				\xdef\pgfmathlinearspace@tempb{\pgfmathlinearspace@tempb,\pgfmath@tonumber{\pgfmath@x}}%
		\repeatpgfmathloop%
		\xdef\pgfmathlinearspace@tempb{\pgfmathlinearspace@tempb,\pgfmath@tonumber{\pgfmath@xc}}%
	\endgroup%
	\edef\pgfmathlinearspacepartitionwidth{\pgfmathlinearspace@tempa}%
	\edef#1{\pgfmathlinearspace@tempb}%	
}

% \pgfmathlinearseries
%
% Create a linear series.
%
% #1 -> the macro to store the linear space.
% #2 -> the first term.
% #3 -> the common difference.
% #4 -> the number of terms.
%
% e.g.
% \pgfmathlinearseries{\x}{0}{3.5}{5}
% 
% Results in \x -> 0.0,3.5,7.0,10.5,14.0
%
\def\pgfmathlinearseries#1#2#3#4{%
	\begingroup
		\pgfmathsetlength{\pgfmath@x}{#2}%
		\pgfmathsetlength{\pgfmath@xa}{#3}%
		\pgfmathsetcounter{pgfmath@counta}{#4}%
		\edef\pgfmathlinearseries@temp{\pgfmath@tonumber{\pgfmath@x}}%
		\pgfmathloop
			\ifnum\pgfmathcounter<\c@pgfmath@counta
				\advance\pgfmath@x by\pgfmath@xa%
				\xdef\pgfmathlinearseries@temp{\pgfmathlinearseries@temp,\pgfmath@tonumber{\pgfmath@x}}%
		\repeatpgfmathloop
	\endgroup
	\edef#1{\pgfmathlinearseries@temp}%	
}

% \foreach
%
% Extended to support linear series and linear spaces.
% Now it is possible to say:
%
% \foreach \x in linear space [from=0, to=2*pi, partitions=100]...
% or...
% \foreach \x in linear space [0:2*pi:100]...
%
% and...
%
% \foreach \x in linear series [first term=4, common difference=2, terms=10]...
% or...
% \foreach \x in linear series [4:2:10]...
%
%
\def\foreach#1in{%
	\def\pgffor@var{#1}%
	\pgfmath@ifnextchar l{%
		\pgfmath@collectlinearspaceorseries}{%
			\pgfmath@continueforeach}}

\def\pgfmath@collectlinearspaceorseries linear s{%
	\pgfmath@ifnextchar p{%
		\pgfmath@collectlinearspace}{%
			\pgfmath@ifnextchar e{%
				\pgfmath@collectlinearseries}{%
  					\pgfmath@reporterror{Unknown 'linear' option in \foreach.}{}%
  	}}%
}%

%
% Removed [PgfMath] since xkeyval currently is not necessarily available.
% 


\define@key{pgfmath linear space}{from}[0.0]{%
	\pgfmathparse{#1}%
	\edef\pgfmath@linearspace@from{\pgfmathresult}}%
\define@key{pgfmath linear space}{to}[100]{%
	\pgfmathparse{#1}%
	\edef\pgfmath@linearspace@to{\pgfmathresult}}%
\define@key{pgfmath linear space}{partitions}[10]{%
	\pgfmathparse{#1}%
	\edef\pgfmath@linearspace@partitions{\pgfmathresult}}%
\def\pgfmath@collectlinearspace pace#1[#2]{% #1 is a dummy.
	\pgfmath@ifin@{:}{#2}{\pgfmath@collectlinearspace@#2\pgfmath@}{%
		\setkeys{pgfmath linear space}{#2}%
		\pgfmathlinearspace{\pgfmathforeach@temp}%
			{\pgfmath@linearspace@from}%
			{\pgfmath@linearspace@to}%
			{\pgfmath@linearspace@partitions}%
		\expandafter\pgfmath@continueforeach\expandafter{\pgfmathforeach@temp}%
	}%
}
\def\pgfmath@collectlinearspace@#1:#2:#3\pgfmath@{%
	\pgfmathlinearspace{\pgfmathforeach@temp}{#1}{#2}{#3}%
	\expandafter\pgfmath@continueforeach\expandafter{\pgfmathforeach@temp}%
}

\define@key{pgfmath linear series}{first term}[0.0]{%
	\pgfmathparse{#1}%
	\edef\pgfmath@linearseries@from{\pgfmathresult}}%
\define@key{pgfmath linear series}{common difference}[100.0]{%
	\pgfmathparse{#1}%
	\edef\pgfmath@linearseries@difference{\pgfmathresult}}%
\define@key{pgfmath linear series}{terms}[10]{%
	\pgfmathparse{#1}%
	\edef\pgfmath@linearseries@terms{\pgfmathresult}}%
\def\pgfmath@collectlinearseries eries#1[#2]{% #1 is a dummy.
	\pgfmath@ifin@{:}{#2}{\pgfmath@collectlinearseries@#2\pgfmath@}{%
		\setkeys{pgfmath linear series}{#2}%
		\pgfmathlinearseries{\pgfmathforeach@temp}%
			{\pgfmath@linearseries@from}%
			{\pgfmath@linearseries@difference}%
			{\pgfmath@linearseries@terms}%
		\expandafter\pgfmath@continueforeach\expandafter{\pgfmathforeach@temp}%
	}%
}
\def\pgfmath@collectlinearseries@#1:#2:#3\pgfmath@{%
	\pgfmathlinearseries{\pgfmathforeach@temp}{#1}{#2}{#3}%
	\expandafter\pgfmath@continueforeach\expandafter{\pgfmathforeach@temp}%
}

\def\pgfmath@continueforeach#1{%
	\def\pgffor@values{#1, \pgffor@stop,}%
	\ifx\pgffor@values\pgffor@emptyvalues%
		\def\pgffor@values{\pgffor@stop,}%
	\fi%
	\let\pgffor@body\pgfutil@empty%
	\global\pgffor@continuetrue%
	\pgffor@collectbody}


\def\pgfpathfunctionto#1#2#3#4{%
	\pgfmathparse{#2}\edef\pgfmath@functionstart{\pgfmathresult}%
	\pgfmathparse{#3}\edef\pgfmath@functionend{\pgfmathresult}%
	\pgfmathparse{#2+#4}\edef\pgfmath@functionfirststep{\pgfmathresult}%
	\foreach \pgfmath@functiontemp in{\pgfmath@functionstart, \pgfmath@functionfirststep,...,\pgfmath@functionend}{%
		\pgfmathusefunction{\pgfmath@functiony}{#1}{\pgfmath@functiontemp}%
		\pgfpathlineto{\pgfpoint{\pgfmath@functiontemp pt}{\pgfmath@functiony pt}}%
	}%
}%

\def\pgfpathbetweenfunctionandxaxis#1#2#3#4{%
	\pgfmathparse{#2}\edef\pgfmath@functionstart{\pgfmathresult}%
	\pgfmathparse{#3+#4}\edef\pgfmath@functionend{\pgfmathresult}%
	\pgfmathparse{#2+#4}\edef\pgfmath@functionfirststep{\pgfmathresult}%
	\pgfpathmoveto{\pgfpoint{#2}{0}}%
	\foreach\pgfmath@functiontemp in{\pgfmath@functionstart,\pgfmath@functionfirststep,...,\pgfmath@functionend}{%
		\pgfmathusefunction{\pgfmath@functiony}{#1}{\pgfmath@functiontemp}%
		\pgfpathlineto{\pgfpoint{\pgfmath@functiontemp}{\pgfmath@functiony}}%
	}%
	\pgfmathparse{#3}%
	\pgfpathlineto{\pgfpoint{\pgfmathresult}{0}}%
	\pgfpathclose%
}%

\def\pgfpathbetweenfunctionandyaxis#1#2#3#4{%
	\pgfmathparse{#2}\edef\pgfmath@functionstart{\pgfmathresult}%
	\pgfmathparse{#3+#4}\edef\pgfmath@functionend{\pgfmathresult}%
	\pgfmathparse{#2+#4}\edef\pgfmath@functionfirststep{\pgfmathresult}%
	\pgfmathusefunction{\pgfmath@functiony}{#1}{#2}%
	\pgfpathmoveto{\pgfpoint{0pt}{\pgfmath@functiony pt}}%
	\foreach\pgfmath@functiontemp in{\pgfmath@functionstart,\pgfmath@functionfirststep,...,\pgfmath@functionend}{%
		\pgfmathusefunction{\pgfmath@functiony}{#1}{\pgfmath@functiontemp}%
		\pgfpathlineto{\pgfpoint{\pgfmath@functiontemp pt}{\pgfmath@functiony pt}}%
	}%
	\pgfmathusefunction{\pgfmath@functiony}{#1}{#3}%
	\pgfpathlineto{\pgfpoint{0pt}{\pgfmath@functiony pt}}%
	\pgfpathclose%
}%
