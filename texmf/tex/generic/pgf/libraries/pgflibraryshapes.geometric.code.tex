% Copyright 2006 by Till Tantau and Mark Wibrow
%
% This file may be distributed and/or modified
%
% 1. under the LaTeX Project Public License and/or
% 2. under the GNU Public License.
%
% See the file doc/generic/pgf/licenses/LICENSE for more details.

\ProvidesFileRCS[v\pgfversion] $Header: /home/mojca/cron/mojca/github/cvs/pgf/pgf/generic/pgf/libraries/Attic/pgflibraryshapes.geometric.code.tex,v 1.1 2007/06/08 11:24:59 tantau Exp $

\pgfdeclareshape{ellipse}
%
% Draws a circle around the text
%
{
  \savedanchor\centerpoint{%
    \pgf@x=.5\wd\pgfnodeparttextbox%
    \pgf@y=.5\ht\pgfnodeparttextbox%
    \advance\pgf@y by-.5\dp\pgfnodeparttextbox%
  }
  \savedanchor\radius{%
    % 
    % Caculate ``height radius''
    % 
    \pgf@y=.5\ht\pgfnodeparttextbox%
    \advance\pgf@y by.5\dp\pgfnodeparttextbox%
    \pgfmathsetlength\pgf@yb{\pgfshapeinnerysep}%
    \advance\pgf@y by\pgf@yb%
    % 
    % Caculate ``width radius''
    % 
    \pgf@x=.5\wd\pgfnodeparttextbox%
    \pgfmathsetlength\pgf@xb{\pgfshapeinnerxsep}%
    \advance\pgf@x by\pgf@xb%
    % 
    % Adjust
    % 
    \pgf@x=1.4142136\pgf@x%
    \pgf@y=1.4142136\pgf@y%
    % 
    % Adjust hieght, if necessary
    % 
    \pgfmathsetlength\pgf@yc{\pgfshapeminheight}%
    \ifdim\pgf@y<.5\pgf@yc%
      \pgf@y=.5\pgf@yc%
    \fi%
    % 
    % Adjust width, if necessary
    % 
    \pgfmathsetlength\pgf@xc{\pgfshapeminwidth}%
    \ifdim\pgf@x<.5\pgf@xc%
      \pgf@x=.5\pgf@xc%
    \fi%
    % 
    % Add outer sep
    % 
    \pgfmathsetlength{\pgf@xb}{\pgfshapeouterxsep}%  
    \pgfmathsetlength{\pgf@yb}{\pgfshapeouterysep}%  
    \advance\pgf@x by\pgf@xb%
    \advance\pgf@y by\pgf@yb%
  }

  %
  % Anchors
  % 
  \anchor{center}{\centerpoint}
  \anchor{mid}{\centerpoint\pgfmathsetlength\pgf@y{.5ex}}
  \anchor{base}{\centerpoint\pgf@y=0pt}
  \anchor{north}
  {
    \pgf@process{\radius}
    \pgf@ya=\pgf@y%
    \pgf@process{\centerpoint}
    \advance\pgf@y by\pgf@ya
  }
  \anchor{south}
  {
    \pgf@process{\radius}
    \pgf@ya=\pgf@y%
    \pgf@process{\centerpoint}
    \advance\pgf@y by-\pgf@ya
  }
  \anchor{west}
  {
    \pgf@process{\radius}
    \pgf@xa=\pgf@x%
    \pgf@process{\centerpoint}
    \advance\pgf@x by-\pgf@xa
  }
  \anchor{mid west}
  {%
    \pgf@process{\radius}
    \pgf@xa=\pgf@x%
    \pgf@process{\centerpoint}
    \advance\pgf@x by-\pgf@xa%
    \pgfmathsetlength\pgf@y{.5ex}
  }
  \anchor{base west}
  {%
    \pgf@process{\radius}
    \pgf@xa=\pgf@x%
    \pgf@process{\centerpoint}
    \advance\pgf@x by-\pgf@xa%
    \pgf@y=0pt
  }
  \anchor{north west}
  {
    \pgf@process{\radius}
    \pgf@xa=\pgf@x%
    \pgf@ya=\pgf@y%
    \pgf@process{\centerpoint}
    \advance\pgf@x by-0.707107\pgf@xa
    \advance\pgf@y by0.707107\pgf@ya
  }
  \anchor{south west}
  {
    \pgf@process{\radius}
    \pgf@xa=\pgf@x%
    \pgf@ya=\pgf@y%
    \pgf@process{\centerpoint}
    \advance\pgf@x by-0.707107\pgf@xa
    \advance\pgf@y by-0.707107\pgf@ya
  }
  \anchor{east}
  {%
    \pgf@process{\radius}
    \pgf@xa=\pgf@x%
    \pgf@process{\centerpoint}
    \advance\pgf@x by\pgf@xa
  }
  \anchor{mid east}
  {%
    \pgf@process{\radius}
    \pgf@xa=\pgf@x%
    \pgf@process{\centerpoint}
    \advance\pgf@x by\pgf@xa%
    \pgfmathsetlength\pgf@y{.5ex}
  }
  \anchor{base east}
  {%
    \pgf@process{\radius}
    \pgf@xa=\pgf@x%
    \pgf@process{\centerpoint}
    \advance\pgf@x by\pgf@xa%
    \pgf@y=0pt
  }
  \anchor{north east}
  {
    \pgf@process{\radius}
    \pgf@xa=\pgf@x%
    \pgf@ya=\pgf@y%
    \pgf@process{\centerpoint}
    \advance\pgf@x by0.707107\pgf@xa
    \advance\pgf@y by0.707107\pgf@ya
  }
  \anchor{south east}
  {
    \pgf@process{\radius}
    \pgf@xa=\pgf@x%
    \pgf@ya=\pgf@y%
    \pgf@process{\centerpoint}
    \advance\pgf@x by0.707107\pgf@xa
    \advance\pgf@y by-0.707107\pgf@ya
  }
  \anchorborder{
    \edef\pgf@marshal{%
      \noexpand\pgfpointborderellipse
      {\noexpand\pgfqpoint{\the\pgf@x}{\the\pgf@y}}
      {\noexpand\radius}%
    }%
    \pgf@marshal%
    \pgf@xa=\pgf@x%
    \pgf@ya=\pgf@y%
    \centerpoint%
    \advance\pgf@x by\pgf@xa%
    \advance\pgf@y by\pgf@ya%
  }

  %
  % Background path
  %
  \backgroundpath
  {
    \pgf@process{\radius}%
    \pgfutil@tempdima=\pgf@x%
    \pgfutil@tempdimb=\pgf@y%
    \pgfmathsetlength{\pgf@xb}{\pgfshapeouterxsep}%  
    \pgfmathsetlength{\pgf@yb}{\pgfshapeouterysep}%  
    \advance\pgfutil@tempdima by-\pgf@xb%
    \advance\pgfutil@tempdimb by-\pgf@yb%
    \pgfpathellipse{\centerpoint}{\pgfqpoint{\pgfutil@tempdima}{0pt}}{\pgfqpoint{0pt}{\pgfutil@tempdimb}}%
  }
}




% Set the recommended shape aspect ratio
%
% #1 = aspect ratio
%
% Example:
%
% \pgfsetshapeminwidth{1.5}

\def\pgfsetshapeaspect#1{%
  \def\pgfshapeaspect{#1}%
  % Invert
  \pgfutil@tempdima=#1pt%
  \pgfutil@tempdima=.125\pgfutil@tempdima%
  \c@pgf@counta=\pgfutil@tempdima\relax% 8192*determinant
  \pgfutil@tempdima=8192pt%
  \divide\pgfutil@tempdima by\c@pgf@counta%
  \edef\pgfshapeaspectinverse{\pgf@sys@tonumber{\pgfutil@tempdima}}
}
\pgfsetshapeaspect{1}



\pgfdeclareshape{diamond}
{
  \savedanchor\outernortheast{%
    %
    % Calculate width and height of the inner rectangle
    %
    \pgf@xa=.5\wd\pgfnodeparttextbox%
    \pgfmathsetlength\pgf@xc{\pgfshapeinnerxsep}%
    \advance\pgf@xa by\pgf@xc%
    \pgf@ya=.5\ht\pgfnodeparttextbox%
    \advance\pgf@ya by.5\dp\pgfnodeparttextbox%
    \pgfmathsetlength\pgf@yc{\pgfshapeinnerysep}%
    \advance\pgf@ya by\pgf@yc%
    %
    % Calculate width and height of diamond
    %
    \pgf@x=\pgf@xa%
    \advance\pgf@x by\pgfshapeaspect\pgf@ya%
    \pgf@y=\pgfshapeaspectinverse\pgf@xa%
    \advance\pgf@y by\pgf@ya%
    %
    % Check against minimum height/width
    %
    \pgfmathsetlength\pgf@xb{\pgfshapeminwidth}%
    \ifdim\pgf@x<\pgf@xb%
      % yes, too small. Enlarge...
      \pgf@x=\pgf@xb%
    \fi%
    \pgfmathsetlength\pgf@yb{\pgfshapeminheight}%
    \ifdim\pgf@y<\pgf@yb%
      % yes, too small. Enlarge...
      \pgf@y=\pgf@yb%
    \fi%
    %
    % Add outer border
    %
    \pgfmathsetlength\pgf@xa{\pgfshapeouterxsep}%
    \advance\pgf@x by\pgf@xa%
    \pgfmathsetlength\pgf@ya{\pgfshapeouterysep}%
    \advance\pgf@y by\pgf@ya%
  }
  \savedanchor\text{%
    \pgf@x=-.5\wd\pgfnodeparttextbox%
    \pgf@y=-.5\ht\pgfnodeparttextbox%
    \advance\pgf@y by.5\dp\pgfnodeparttextbox%
  }

  %
  % Anchors
  %
  \anchor{text}{\text}%
  \anchor{center}{\pgfpointorigin}%
  \anchor{mid}{%
    \pgf@process{\text}%
    \pgf@x=0pt%
    \pgfmathsetlength\pgf@ya{.5ex}
    \advance\pgf@y by\pgf@ya%
  }
  \anchor{base}{\pgf@process{\text}\pgf@x=0pt  }
  \anchor{north}{\pgf@process{\outernortheast}\pgf@x=0pt}
  \anchor{south}{\pgf@process{\outernortheast}\pgf@x=0pt\pgf@y=-\pgf@y}
  \anchor{west}{\pgf@process{\outernortheast}\pgf@x=-\pgf@x\pgf@y=0pt}
  \anchor{north west}{\pgf@process{\outernortheast}\pgf@x=-.5\pgf@x\pgf@y=.5\pgf@y}
  \anchor{south west}{\pgf@process{\outernortheast}\pgf@x=-.5\pgf@x\pgf@y=-.5\pgf@y}
  \anchor{east}{\pgf@process{\outernortheast}\pgf@y=0pt}
  \anchor{north east}{\pgf@process{\outernortheast}\pgf@x=.5\pgf@x\pgf@y=.5\pgf@y}
  \anchor{south east}{\pgf@process{\outernortheast}\pgf@x=.5\pgf@x\pgf@y=-.5\pgf@y}
  \anchorborder{%
    \pgf@xa=\pgf@x%
    \pgf@ya=\pgf@y%
    \pgf@process{\outernortheast}%
    \ifdim\pgf@xa>0pt%
    \else%
      \pgf@x=-\pgf@x%
    \fi%
    \ifdim\pgf@ya>0pt%
    \else%
      \pgf@y=-\pgf@y%
    \fi%
    \edef\pgf@marshal{%
      \noexpand\pgfpointintersectionoflines
      {\noexpand\pgfpointorigin}
      {\noexpand\pgfqpoint{\the\pgf@xa}{\the\pgf@ya}}
      {\noexpand\pgfqpoint{\the\pgf@x}{0pt}}
      {\noexpand\pgfqpoint{0pt}{\the\pgf@y}}%
    }%
    \pgf@process{\pgf@marshal}%
  }

  %
  % Background path
  %
  \backgroundpath{
    \pgf@process{\outernortheast}%
    \pgf@xc=\pgf@x%
    \pgf@yc=\pgf@y%
    \pgfmathsetlength{\pgf@xa}{\pgfshapeouterxsep}%
    \pgfmathsetlength{\pgf@ya}{\pgfshapeouterysep}%
    \advance\pgf@xc by-1.414213\pgf@xa%
    \advance\pgf@yc by-1.414213\pgf@ya%
    \pgfpathmoveto{\pgfqpoint{\pgf@xc}{0pt}}%
    \pgfpathlineto{\pgfqpoint{0pt}{\pgf@yc}}%
    \pgfpathlineto{\pgfqpoint{-\pgf@xc}{0pt}}%
    \pgfpathlineto{\pgfqpoint{0pt}{-\pgf@yc}}%
    \pgfpathclose%
  }
}




% \pgfsetstarpoints
%
% Set the number of points on a star.
%
\def\pgfsetstarpoints#1{%
	\pgfmathsetcounter{pgf@counta}{#1}%
	\edef\pgfstarpoints{\the\c@pgfmath@counta}}
\pgfsetstarpoints{5}

% \pgfsetstarpointheight
%
% Set the height of the points (this is the
% distance between the outer and inner point 
% radii).
%
\def\pgfsetstarpointheight#1{%
	\pgfmathparse{#1}%
	\edef\pgfstarpointheight{\pgfmathresult pt}}
\pgfsetstarpointheight{12pt}

% \pgfsetstarpointratio
%
% Set the ratio between the outer and 
% inner point radii.
%
\def\pgfsetstarpointratio#1{%
	\pgfmathparse{#1}%
	\edef\pgfstarpointratio{\pgfmathresult}%
	\def\pgfstarpointheight{-16383pt}% If negative, the ratio is used.
}
\pgfsetstarpointratio{1.75}

% \pgfsetstarrrotate
%
% Set the angle of rotation of the star
% border. This can be decimal.
%
\def\pgfsetstarrotate#1{%
	\pgfmathparse{#1}%
	\edef\pgfstarrotate{\pgfmathresult}}%
\pgfsetstarrotate{0}

% Shape star.
%
\pgfdeclareshape{star}{%
	\saveddimen{\points}{\pgf@x\pgfstarpoints pt}%
	\saveddimen{\pointratio}{\pgf@x\pgfstarpointratio pt}%
	\saveddimen{\rotate}{\pgf@x\pgfstarrotate pt}%
	\saveddimen{\pointheight}{\pgf@x\pgfstarpointheight}%
	\saveddimen{\minimumsize}{%
		\pgfmathsetlength\pgf@x{\pgfshapeminwidth}%
		\pgfmathsetlength\pgf@y{\pgfshapeminheight}%
		\ifdim\pgf@y>\pgf@x%
			\pgf@x\pgf@y%
		\fi}%	
	\saveddimen{\outersep}{%
		\pgfmathsetlength\pgf@x{\pgfshapeouterxsep}%
		\pgfmathsetlength\pgf@y{\pgfshapeouterysep}%
		\ifdim\pgf@y>\pgf@x%
			\pgf@x\pgf@y%
		\fi}%
	\savedanchor{\centerpoint}{%
		\pgf@x.5\wd\pgfnodeparttextbox%
		\pgf@y.5\ht\pgfnodeparttextbox%
		\advance\pgf@y-.5\dp\pgfnodeparttextbox%
	}%
	\saveddimen{\innerpointradius}{%
		% 
		% The innerpoint radius is the radius of the circle which 
		% can safely encompass the node textbox. 
		%
		\pgfmathsetlength\pgf@x{\pgfshapeinnerxsep}%
		\advance\pgf@x.5\wd\pgfnodeparttextbox%
		\pgfmathsetlength\pgf@y{\pgfshapeinnerysep}%
		\advance\pgf@y.5\ht\pgfnodeparttextbox%
		\ifdim\pgf@y>\pgf@x%
			\pgf@x\pgf@y%
		\fi%
		\pgfmathveclen@{\pgf@sys@tonumber{\pgf@x}}{\pgf@sys@tonumber{\pgf@x}}%
		\pgf@x\pgfmathresult pt\relax% 		
	}%
	%
	\anchor{center}{\centerpoint}%
	\anchor{mid}{\centerpoint\pgfmathsetlength\pgf@y{.5ex}}%
 	\anchor{base}{\centerpoint\pgf@y=0pt}%
 	\anchor{north}{\pgf@anchor@star@border{\pgfpoint{+0pt}{+\outerpointradius}}}%
 	\anchor{south}{\pgf@anchor@star@border{\pgfpoint{+0pt}{+-\outerpointradius}}}%
 	\anchor{east}{\pgf@anchor@star@border{\pgfpoint{+\outerpointradius}{+0pt}}}%
 	\anchor{west}{\pgf@anchor@star@border{\pgfpoint{+-\outerpointradius}{+0pt}}}%
 	\anchor{north east}{\pgf@anchor@star@border{\pgfpoint{+\outerpointradius}{+\outerpointradius}}}%
 	\anchor{north west}{\pgf@anchor@star@border{\pgfpoint{+-\outerpointradius}{+\outerpointradius}}}%
 	\anchor{south east}{\pgf@anchor@star@border{\pgfpoint{+\outerpointradius}{+-\outerpointradius}}}%
 	\anchor{south west}{\pgf@anchor@star@border{\pgfpoint{+-\outerpointradius}{+-\outerpointradius}}}%
 	%
	\backgroundpath{%
		%
		% Redefine stuff for ease of use.
		%
		\pgf@x\points%
		\c@pgf@counta\pgf@x%
		\divide\c@pgf@counta65536\relax%
		\edef\points{\the\c@pgf@counta}%
		\pgf@x\rotate%
		\edef\rotate{\pgf@sys@tonumber{\pgf@x}}%
		\pgf@x\pointratio%
		\edef\pointratio{\pgf@sys@tonumber{\pgf@x}}%
		%
		% Calculate radii.
		%
		\pgf@x\innerpointradius\relax%
		\edef\innerradius{\the\pgf@x}%
		\pgf@xa\pointheight\relax%
		\ifdim\pgf@xa<0pt\relax%
			\pgf@x\pointratio\pgf@x%
		\else%
			\advance\pgf@x\pgf@xa%
		\fi%
		\pgf@xb\pgf@x%
    	\pgf@xc\minimumsize\relax%
    	\ifdim\pgf@x<.5\pgf@xc%
    		\pgf@x.5\pgf@xc%
    	\fi%
    	\edef\outerradius{\the\pgf@x}%
    	\ifdim\pgf@x>\pgf@xb%
    		\ifdim\pgf@xa<0pt\relax%
    			\pgfmathreciprocal{\pointratio}%
    			\pgf@xc\pgfmathresult\pgf@x\relax%
    			\edef\innerradius{\the\pgf@xc}%
			\else%
				\pgf@xc\pgf@x\relax%
				\advance\pgf@xc-\pointheight%
				\edef\innerradius{\the\pgf@xc}%
			\fi%
		\fi%
		%
		% Get the total number of points (inner + outer)...
		%
		\c@pgf@counta\points%
		\advance\c@pgf@counta\c@pgf@counta%
		\edef\numpoints{\the\c@pgf@counta}%
		%
		% ...and hence the angle between points.
		%
		\pgf@x360pt\relax%
		\divide\pgf@x\c@pgf@counta%
		\edef\staranglestep{\the\pgf@x}%
		%
		% Start at 90 degrees (star always points up)...
		%
		\pgf@x90pt\relax%
		%
		% ...unless rotation is applied.
		%
		\pgf@xa\rotate pt\relax%
		\advance\pgf@x\pgf@xa%
		\edef\starangle{\the\pgf@x}%
		\let\starradius=\outerradius%
		%
		% Move to first point.
		%
		\pgfpathmoveto{%
			\pgfpointadd{\centerpoint}%
				{\pgfpointpolar{+\starangle}{+\starradius}}%
		}%
		\def\staranchorname{pgf@anchor@star@outer point}%
		\pgfmathloop%
			%
			% Create anchors. Manually \xdef as \gdef is normally used by \anchor.
			%
			\c@pgf@counta\pgfmathcounter\relax%
			\advance\c@pgf@counta1\relax%
			\divide\c@pgf@counta2\relax%
			\expandafter\xdef\csname\staranchorname\space\the\c@pgf@counta\endcsname{%
				\noexpand\pgf@lib@shapesstaranchor{\pgfmathcounter}%
			}%
			\ifnum\pgfmathcounter=\numpoints\relax% Stop.
			\else%
			\ifodd\pgfmathcounter%
				\let\starradius\innerradius%
				\def\staranchorname{pgf@anchor@star@inner point}%
			\else%
				\let\starradius\outerradius%
				\def\staranchorname{pgf@anchor@star@outer point}%
			\fi%
			\pgf@x\starangle\relax%
			\advance\pgf@x\staranglestep\relax%
			\edef\starangle{\the\pgf@x}%
			\pgfpathlineto{%
				\pgfpointadd{\centerpoint}%
					{\pgfpointpolar{+\starangle}{+\starradius}}%
			}%
			\repeatpgfmathloop%
		\pgfpathclose%
	}%
	%
	\anchorborder{%
		%
		% Save x and y.
		%
		\pgf@xa\pgf@x%
		\pgf@ya\pgf@y%
		%
		% Redefine stuff for ease of use.
		%
		\pgf@x\points%
		\c@pgf@counta\pgf@x%
		\divide\c@pgf@counta65536\relax%
		\edef\points{\the\c@pgf@counta}%
		\pgf@x\rotate%
		\edef\rotate{\pgf@sys@tonumber{\pgf@x}}%
		\pgf@x\pointratio%
		\edef\pointratio{\pgf@sys@tonumber{\pgf@x}}%
		%
		% Calculate the location of the external 
		% point relative to the node center.
		%
		\centerpoint%
		\advance\pgf@xa\pgf@x%
		\advance\pgf@ya\pgf@y%
		\edef\externalx{\the\pgf@xa}%
		\edef\externaly{\the\pgf@ya}%
		\pgf@process{\pgfpointdiff{\centerpoint}{\pgfpoint{+\externalx}{+\externaly}}}%
		%
		% First approximate the angle of the external point...
		%
		\pgf@xa\pgf@x%
		\pgf@ya\pgf@y%
		\pgf@xb\pgf@x%
		\pgf@yb\pgf@y%
		\ifdim\pgf@xa<0pt\relax%
			\pgf@xa-\pgf@xa%
		\fi%
		\ifdim\pgf@ya<0pt\relax%
			\pgf@ya-\pgf@ya%
		\fi%
		\ifdim\pgf@ya>\pgf@xa%
			\pgf@x\pgf@xa%
			\pgf@y\pgf@ya%
		\else%
			\pgf@x\pgf@ya%
			\pgf@y\pgf@xa%
		\fi%
		\ifdim\pgf@y=0pt\relax%
			\pgf@x0pt%
		\else%
			\pgfmathreciprocal@{\pgf@sys@tonumber{\pgf@y}}%
			\pgf@x\pgfmathresult\pgf@x%
		\fi%
		\multiply\pgf@x1000\relax%
		\afterassignment\pgfmath@gobbletilpgfmath@%
		\expandafter\c@pgf@counta\the\pgf@x\relax\pgfmath@%
		\expandafter\pgf@x\csname pgfmath@atan@\the\c@pgf@counta\endcsname pt\relax%
		\ifdim\pgfmath@ya>\pgfmath@xa\relax%
			\pgf@x-\pgf@x%
			\advance\pgf@x90pt%
		\fi%
		\ifdim\pgf@xb<0pt%
			\ifdim\pgf@yb>0pt%
				\pgf@x-\pgf@x%
			\fi%
			\advance\pgf@x180pt\relax%
		\else%
			\ifdim\pgf@yb<0pt%
			\pgf@x-\pgf@x%
			\advance\pgf@x360pt\relax%
		\fi\fi%		
		%
		% ...then adjust, as star points start at 90 degrees...
		%
		\advance\pgf@x-90pt\relax%
		\ifdim\pgf@x<0pt\relax%
			\advance\pgf@x360pt\relax%
		\fi%
		%
		% ...and also for rotation.
		%
		\advance\pgf@x-\rotate pt\relax%
		\ifdim\pgf@x<0pt\relax%
			\advance\pgf@x360pt\relax%
		\fi%
		%
		% Now, locate the start and end points on the star border segment...
		%
		\c@pgf@counta\points\relax%
		\pgf@y180pt\relax%
		\divide\pgf@y\c@pgf@counta\relax%
		\pgfmathreciprocal@{\pgf@sys@tonumber{\pgf@y}}%
		\pgf@x\pgfmathresult\pgf@x%
		\afterassignment\pgfmath@gobbletilpgfmath@%
		\expandafter\c@pgf@counta\the\pgf@x\relax\pgfmath@%
		%
		% ...and hence, the start and end angles of the star border segment.
		%
		\pgf@x\pgf@y%
		\multiply\pgf@x\c@pgf@counta%
		\advance\pgf@x90pt%
		\advance\pgf@x\rotate pt\relax%
		\edef\firstangle{\the\pgf@x}%
		\advance\c@pgf@counta1\relax%
		\pgf@x\pgf@y%
		\multiply\pgf@x\c@pgf@counta%
		\advance\pgf@x\rotate pt\relax%
		\advance\pgf@x90pt%
		\edef\secondangle{\the\pgf@x}%
		%
		% Get the radii and add the outer sep...
		%
		\pgf@x\innerpointradius\relax%
		\edef\innerradius{\the\pgf@x}%
		\pgf@xa\pointheight\relax%
		\ifdim\pgf@xa<0pt\relax%
			\pgf@x\pointratio\pgf@x%
		\else%
			\advance\pgf@x\pgf@xa%
		\fi%
		\pgf@xb\pgf@x%
    	\pgf@xc\minimumsize\relax%
    	\ifdim\pgf@x<.5\pgf@xc%
    		\pgf@x.5\pgf@xc%
    	\fi%
    	\edef\outerradius{\the\pgf@x}%
    	\ifdim\pgf@x>\pgf@xb%
    		\ifdim\pgf@xa<0pt\relax%
    			\pgfmathreciprocal{\pointratio}%
    			\pgf@xc\pgfmathresult\pgf@x\relax%
    			\edef\innerradius{\the\pgf@xc}%
			\else%
				\pgf@xc\pgf@x\relax%
				\advance\pgf@xc-\pgf@xb%
				\edef\innerradius{\the\pgf@xc}%
			\fi%
		\fi%
		\pgf@xa\outersep\relax%
		\pgf@x\outerradius\relax%
    	\advance\pgf@x\pgf@xa%
		\edef\outerradius{\the\pgf@x}%
		\pgf@x\innerradius\relax%
		\advance\pgf@x\pgf@xa%		
		\edef\innerradius{\the\pgf@x}%		
		\ifodd\c@pgf@counta%
			\let\firstradii\outerradius%
			\let\secondradii\innerradius%
		\else%
			\let\firstradii\innerradius%
			\let\secondradii\outerradius%
		\fi%
		%
		% ...and calculate the point on the intersection of
		% the line from the external point to \centerpoint and
		% the relevant segment of the star border.
		%
		\pgfpointintersectionoflines{\centerpoint}{\pgfpoint{+\externalx}{+\externaly}}%
			{%
				\pgfpointadd{\centerpoint}%
					{\pgfpointpolar{+\firstangle}{+\firstradii}}%
				}%	
				{%
					\pgfpointadd{\centerpoint}%
						{\pgfpointpolar{+\secondangle}{+\secondradii}}%
				}%	
	}%
}%


% \pgf@lib@shapesstaranchor
%
% Used internally to calculate inner point and  
% outer point anchor positions 'on line'.
%
\def\pgf@lib@shapesstaranchor#1{%
	%
	% Redefine stuff for ease of use.
	%
	\pgf@x\points%
	\c@pgf@counta\pgf@x%
	\divide\c@pgf@counta65536\relax%
	\edef\points{\the\c@pgf@counta}%
	\pgf@x\rotate%
	\edef\rotate{\pgf@sys@tonumber{\pgf@x}}%
	\pgf@x\pointratio%
	\edef\pointratio{\pgf@sys@tonumber{\pgf@x}}%
	%
	% Caculate radii.
	%
	\pgf@x\innerpointradius%
	\edef\innerradius{\the\pgf@x}%
	\pgf@xa\pointheight\relax%
	\ifdim\pgf@xa<0pt\relax%
		\pgf@x\pointratio\pgf@x%
	\else%
		\advance\pgf@x\pgf@xa%
	\fi%
	\pgf@xb\pgf@x%
	\pgf@xc\minimumsize\relax%
   \ifdim\pgf@x<.5\pgf@xc%
   	\pgf@x.5\pgf@xc%
   \fi%
   \edef\outerradius{\the\pgf@x}%
   \ifdim\pgf@x>\pgf@xb%
   	\ifdim\pgf@xa<0pt\relax%
   		\pgfmathreciprocal{\pointratio}%
   		\pgf@xc\pgfmathresult\pgf@x\relax%
   		\edef\innerradius{\the\pgf@xc}%
		\else%
			\pgf@xc\pgf@x\relax%
			\advance\pgf@xc-\pgf@xb%
			\edef\innerradius{\the\pgf@xc}%
		\fi%
	\fi%
	%
	% Add the outer sep.
	%
	\pgf@xa\outersep%
	\pgf@x\outerradius\relax%
   \advance\pgf@x\pgf@xa%
	\edef\outerradius{\the\pgf@x}%
	\pgf@x\innerradius\relax%
	\advance\pgf@x\pgf@xa%		
	\edef\innerradius{\the\pgf@x}%
	%
	% Calculate the angle.
	%	
	\c@pgf@counta\points%
	\pgf@x180pt\relax%
	\divide\pgf@x\c@pgf@counta%
	\c@pgf@counta#1\relax%
	\advance\c@pgf@counta-1\relax%
	\multiply\pgf@x\c@pgf@counta%
	\pgf@xa\rotate pt\relax%
	\advance\pgf@x\pgf@xa%
	\advance\pgf@x90pt\relax%
	\edef\starangle{\the\pgf@x}%
	\ifodd\c@pgf@counta%
		\let\starradius\innerradius%
	\else%
		\let\starradius\outerradius%
	\fi%
	\pgfpointadd{\centerpoint}%
			{\pgfpointpolar{\starangle}{\starradius}}%
}%


% \pgfsetpolygonsides
%
% Set the number of sides on a polygon.
%
\def\pgfsetpolygonsides#1{%
	\pgfmathsetcounter{pgf@counta}{#1}%
	\edef\pgfpolygonsides{\the\c@pgfmath@counta}}
\pgfsetpolygonsides{6}

% \pgfsetpolygonrotate
%
% Set the angle of rotation of the polygon
% border. This can be decimal.
%
\def\pgfsetpolygonrotate#1{%
	\pgfmathparse{#1}%
	\edef\pgfpolygonrotate{\pgfmathresult}}%
\pgfsetpolygonrotate{0}


% Regular polygon shape.
% 
%
\pgfdeclareshape{regular polygon}{%
	%
	% Saved dimensions.
	%
	\saveddimen{\sides}{\pgf@x\pgfpolygonsides pt}%
	\saveddimen{\rotate}{\pgf@x\pgfpolygonrotate pt}%
	\saveddimen{\minimumsize}{%
		\pgfmathsetlength\pgf@x{\pgfshapeminwidth}%
		\pgfmathsetlength\pgf@y{\pgfshapeminheight}%
		\ifdim\pgf@y>\pgf@x%
			\pgf@x\pgf@y%
		\fi}%
	\saveddimen{\outersep}{%
		\pgfmathsetlength\pgf@x{\pgfshapeouterxsep}%
		\pgfmathsetlength\pgf@y{\pgfshapeouterysep}%
		\ifdim\pgf@y>\pgf@x%
			\pgf@x\pgf@y%
		\fi}%
	\saveddimen{\radius}{%
		% 
		% The radius calculated here is the radius of the circle which 
		% can safely encompass the node textbox. This corresponds to the 
		% distance from the centre of the polygon to the mid-point of the
		% of the sides of the polygon. The desired radius for the corners 
		% of the polygon has to calculated `on-line' as the saved dimen 
		% \sides is not available here.
		%
		\pgfmathsetlength\pgf@x{\pgfshapeinnerxsep}%
		\advance\pgf@x.5\wd\pgfnodeparttextbox%
		\pgfmathsetlength\pgf@y{\pgfshapeinnerysep}%
		\advance\pgf@y.5\ht\pgfnodeparttextbox%
		\ifdim\pgf@y>\pgf@x%
			\pgf@x\pgf@y%
		\fi%
		\pgfmathveclen@{\pgf@sys@tonumber{\pgf@x}}{\pgf@sys@tonumber{\pgf@x}}%
		\pgf@x\pgfmathresult pt\relax%   
	}%
	
	%
	% Saved anchors.
	%
	\savedanchor{\centerpoint}{%
		\pgf@x.5\wd\pgfnodeparttextbox%
		\pgf@y.5\ht\pgfnodeparttextbox%
		\advance\pgf@y-.5\dp\pgfnodeparttextbox%
	}%
	
	%
	% Other anchors.
	%
	\anchor{center}{\centerpoint}%
	\anchor{mid}{\centerpoint\pgfmathsetlength\pgf@y{.5ex}}%
 	\anchor{base}{\centerpoint\pgf@y=0pt}%
 	\anchor{north}{\csname pgf@anchor@regular polygon@border\endcsname{\pgfpoint{+0pt}{+\radius}}}%
 	\anchor{south}{\csname pgf@anchor@regular polygon@border\endcsname{\pgfpoint{+0pt}{+-\radius}}}%
 	\anchor{east}{\csname pgf@anchor@regular polygon@border\endcsname{\pgfpoint{+\radius}{+0pt}}}%
 	\anchor{west}{\csname pgf@anchor@regular polygon@border\endcsname{\pgfpoint{+-\radius}{+0pt}}}%
 	\anchor{north east}{\csname pgf@anchor@regular polygon@border\endcsname{\pgfpoint{+\radius}{+\radius}}}%
 	\anchor{north west}{\csname pgf@anchor@regular polygon@border\endcsname{\pgfpoint{+-\radius}{+\radius}}}%
 	\anchor{south east}{\csname pgf@anchor@regular polygon@border\endcsname{\pgfpoint{+\radius}{+-\radius}}}%
 	\anchor{south west}{\csname pgf@anchor@regular polygon@border\endcsname{\pgfpoint{+-\radius}{+-\radius}}}%
 	
 	%
	% Background path.
	%
	\backgroundpath{%
		%
		% Redefine some stuff for ease of use.
		%
		\pgf@x\sides%
		\c@pgf@counta\pgf@x%
		\divide\c@pgf@counta65536\relax%
		\edef\sides{\the\c@pgf@counta}%
		\pgf@x\rotate%
		\edef\rotate{\pgf@sys@tonumber{\pgf@x}}%
		%
		% Get the inner angle.
		%
		\pgf@y360pt\relax%
		\divide\pgf@y\sides%
		\edef\polygonanglestep{\the\pgf@y}%
		%
		% Now recalculate the polygon *corner* radius. 
		%
		\pgf@y.5\pgf@y%
		\pgfmathcos@{\pgf@sys@tonumber{\pgf@y}}%
		\pgfmathreciprocal@{\pgfmathresult}%
		\pgf@x\radius\relax%
		\pgf@x\pgfmathresult\pgf@x%
		\pgf@xa\minimumsize\relax%
		\ifdim\pgf@x<.5\pgf@xa%
			\pgf@x.5\pgf@xa%
		\fi%
		\edef\polygonradius{\the\pgf@x}%
		%
		% Every polygon is drawn so that a side is at the bottom...
		%
		\pgf@x90pt\relax%
		\ifodd\sides%
		\else%
			\advance\pgf@x-\pgf@y%
		\fi%
		%
		% ...unless rotation is applied.
		%
		\pgf@xa\rotate pt\relax%
		\advance\pgf@x\pgf@xa%
		\edef\polygonangle{\the\pgf@x}%
		%
		% Move to first point.
		%
		\pgfpathmoveto{%
			\pgfpointadd{\centerpoint}%
				{\pgfpointpolar{+\polygonangle}{+\polygonradius}}%
		}%
		\pgfmathloop%
			%
			% Create anchors. Manually \xdef as \gdef is normally used by \anchor.
			%
			\expandafter\xdef\csname pgf@anchor@regular polygon@corner\space\pgfmathcounter\endcsname{%
				\noexpand\pgf@lib@shapescorneranchor{\pgfmathcounter}%
			}%
			\expandafter\xdef\csname pgf@anchor@regular polygon@side\space\pgfmathcounter\endcsname{%
				\noexpand\pgf@lib@shapessideanchor{\pgfmathcounter}%
			}%
			\ifnum\pgfmathcounter=\sides\relax% Stop.
			\else%
			\pgf@x\polygonangle\relax%
			\advance\pgf@x\polygonanglestep\relax%
			\edef\polygonangle{\the\pgf@x}%
			\pgfpathlineto{%
				\pgfpointadd{\centerpoint}%
					{\pgfpointpolar{+\polygonangle}{+\polygonradius}}%
			}%
			\repeatpgfmathloop%
		\pgfpathclose%
	}%
	
	%
	% Anchor border.
	%
	\anchorborder{%
		%
		% Save the external point.
		%
		\pgf@xa\pgf@x%
		\pgf@ya\pgf@y%
		\centerpoint%
		\advance\pgf@xa\pgf@x%
		\advance\pgf@ya\pgf@y%
		\edef\externalx{\the\pgf@xa}%
		\edef\externaly{\the\pgf@ya}%
		\pgf@process{\pgfpointdiff{\centerpoint}{\pgfpoint{+\externalx}{+\externaly}}}%
		%
		% Approximate the angle of the external point...
		%
		\pgf@xa\pgf@x%
		\pgf@ya\pgf@y%
		\pgf@xb\pgf@x%
		\pgf@yb\pgf@y%
		\ifdim\pgf@xa<0pt\relax%
			\pgf@xa-\pgf@xa%
		\fi%
		\ifdim\pgf@ya<0pt\relax%
			\pgf@ya-\pgf@ya%
		\fi%
		\ifdim\pgf@ya>\pgf@xa%
			\pgf@x\pgf@xa%
			\pgf@y\pgf@ya%
		\else%
			\pgf@x\pgf@ya%
			\pgf@y\pgf@xa%
		\fi%
		\ifdim\pgf@y=0pt\relax%
			\pgf@x0pt%
		\else%
			\pgfmathreciprocal@{\pgf@sys@tonumber{\pgf@y}}%
			\pgf@x\pgfmathresult\pgf@x%
		\fi%
		\multiply\pgf@x1000\relax%
		\afterassignment\pgfmath@gobbletilpgfmath@%
		\expandafter\c@pgf@counta\the\pgf@x\relax\pgfmath@%
		\expandafter\pgf@x\csname pgfmath@atan@\the\c@pgf@counta\endcsname pt\relax%
		\ifdim\pgfmath@ya>\pgfmath@xa\relax%
			\pgf@x-\pgf@x%
			\advance\pgf@x90pt%
		\fi%
		\ifdim\pgf@xb<0pt%
			\ifdim\pgf@yb>0pt%
				\pgf@x-\pgf@x%
			\fi%
			\advance\pgf@x180pt\relax%
		\else%
			\ifdim\pgf@yb<0pt%
			\pgf@x-\pgf@x%
			\advance\pgf@x360pt\relax%
		\fi\fi%
		%
		% ...(redefine stuff for ease of use)...
		%
		\pgf@y\sides%
		\c@pgf@counta\pgf@y%
		\divide\c@pgf@counta65536\relax%
		\edef\sides{\the\c@pgf@counta}%
		\pgf@y\rotate%
		\edef\rotate{\pgf@sys@tonumber{\pgf@y}}%		
		%
		% ...now adjust angle, for the number of polygon sides...
		%
		\advance\pgf@x-90pt\relax%
		\pgf@xa180pt\relax%
		\divide\pgf@xa\sides%
		%
		% ...and for if the there is an even number of sides...
		%
		\ifodd\sides%
		\else%
			\advance\pgf@x\pgf@xa%
		\fi%
		\ifdim\pgf@x<0pt\relax%
			\advance\pgf@x360pt\relax%
		\fi%
		%
		% ...and also for rotation.
		%
		\advance\pgf@x-\rotate pt\relax%
		\ifdim\pgf@x<0pt\relax%
			\advance\pgf@x360pt\relax%
		\fi%
		%
		% Now, locate the start and end points on the polygon border segment...
		%
		\c@pgf@counta\sides\relax%
		\pgf@y360pt\relax%
		\divide\pgf@y\c@pgf@counta\relax%
		\pgfmathreciprocal@{\pgf@sys@tonumber{\pgf@y}}%
		\pgf@x\pgfmathresult\pgf@x%
		\afterassignment\pgfmath@gobbletilpgfmath@%
		\expandafter\c@pgf@counta\the\pgf@x\relax\pgfmath@%
		%
		% ...and hence, the start and end angles of the polygon border segment.
		%
		\pgf@x\pgf@y%
		\multiply\pgf@x\c@pgf@counta%
		\advance\pgf@x90pt%
		\ifodd\sides%
		\else%
			\advance\pgf@x-\pgf@xa%
		\fi%
		\advance\pgf@x\rotate pt\relax%
		\edef\firstangle{\the\pgf@x}%
		\advance\c@pgf@counta1\relax%
		\pgf@x\pgf@y%
		\multiply\pgf@x\c@pgf@counta%
		\advance\pgf@x\rotate pt\relax%
		\advance\pgf@x90pt%
		\ifodd\sides%
		\else%
			\advance\pgf@x-\pgf@xa%
		\fi%
		\edef\secondangle{\the\pgf@x}%
		%
		% Get the inner angle.
		%
		\pgf@y360pt\relax%
		\divide\pgf@y\sides%
		%
		% Now recalculate the polygon *corner* radius... 
		%
		\pgf@y.5\pgf@y%
		\pgfmathcos@{\pgf@sys@tonumber{\pgf@y}}%
		\pgfmathreciprocal@{\pgfmathresult}%
		\pgf@x\radius\relax%
		\pgf@x\pgfmathresult\pgf@x%
		\edef\polygonradius{\the\pgf@x}%
		%
		% ...and add the outer sep to the corner radius...
		%
		\pgf@xa\minimumsize\relax%
    	\ifdim\pgf@x<.5\pgf@xa%
    		\pgf@x.5\pgf@xa%
    	\fi%
    	\pgf@xa\outersep\relax%
    	\advance\pgf@x\pgf@xa%
		\edef\radius{\the\pgf@x}%
		%
		% ...and calculate the point on the intersection of
		% the line from the external point to \centerpoint and
		% the segment of the star border.
		%
		\pgfpointintersectionoflines{\centerpoint}{\pgfpoint{+\externalx}{+\externaly}}%
			{%
				\pgfpointadd{\centerpoint}%
					{\pgfpointpolar{+\firstangle}{+\radius}}%
				}%	
				{%
					\pgfpointadd{\centerpoint}%
						{\pgfpointpolar{+\secondangle}{+\radius}}%
				}%	
	}%
}%


% \pgf@lib@shapespolygoncorneranchor
%
% Used internally to calculate corner anchor positions.  
%
\def\pgf@lib@shapescorneranchor#1{%
	%
	% Redefine stuff for ease of use.
	%
	\pgf@y\sides%
	\c@pgf@counta\pgf@y%
	\divide\c@pgf@counta65536\relax%
	\edef\sides{\the\c@pgf@counta}%
	\pgf@y\rotate%
	\edef\rotate{\pgf@sys@tonumber{\pgf@y}}%
	%
	% Get the inner angle.
	%
	\pgf@y360pt\relax%
	\divide\pgf@y\sides\relax%
	\edef\polgonanglestep{\pgf@sys@tonumber{\pgf@y}}%
	%
	% Recalculate the polygon corner radius...
	%
	\pgf@y.5\pgf@y%
	\pgfmathcos@{\pgf@sys@tonumber{\pgf@y}}%
	\pgfmathreciprocal@{\pgfmathresult}%
	\pgf@x\radius\relax%
	\pgf@x\pgfmathresult\pgf@x%
	\edef\polygonradius{\the\pgf@x}%
	%
	% ...adjust for minimum size...
	%
	\pgf@xa\minimumsize\relax%
   \ifdim\pgf@x<.5\pgf@xa%
   	\pgf@x.5\pgf@xa%
   \fi%
	%
	% ...and add the outer sep.
	%	
	\pgf@xa\outersep\relax%
   \advance\pgf@x\pgf@xa%		
	\edef\polygonradius{\the\pgf@x}%
	%
	% Calculate the angle.
	%	
	\c@pgf@counta#1\relax%
	\advance\c@pgf@counta-1\relax%
	\pgf@x2.0\pgf@y%
	\multiply\pgf@x\c@pgf@counta%
	\pgf@xa\rotate pt\relax%
	\advance\pgf@x\pgf@xa%
	\advance\pgf@x90pt\relax%
	\ifodd\sides%
	\else%
		\advance\pgf@x-\pgf@y%
	\fi%
	\edef\polygonangle{\the\pgf@x}%
	\pgfpointadd{\centerpoint}%
			{\pgfpointpolar{\polygonangle}{\polygonradius}}%
}%

% \pgf@lib@shapespolygonsideanchor
%
% Used internally to calculate side anchor positions.  
%
\def\pgf@lib@shapessideanchor#1{%
	%
	% Redefine stuff for ease of use.
	%
	\pgf@y\sides%
	\c@pgf@counta\pgf@y%
	\divide\c@pgf@counta65536\relax%
	\edef\sides{\the\c@pgf@counta}%
	\pgf@y\rotate%
	\edef\rotate{\pgf@sys@tonumber{\pgf@y}}%
	%
	% Get the inner angle.
	%
	\pgf@y360pt\relax%
	\divide\pgf@y\sides\relax%
	\edef\polygonanglestep{\the\pgf@y}%
	%
	% Recalculate the polygon corner radius...
	%
	\pgf@y.5\pgf@y%
	\pgfmathcos@{\pgf@sys@tonumber{\pgf@y}}%
	\pgfmathreciprocal@{\pgfmathresult}%
	\pgf@x\radius\relax%
	\pgf@x\pgfmathresult\pgf@x%
	%
	% ...adjust for minimum size...
	%
	\pgf@xa\minimumsize\relax%
  	\ifdim\pgf@x<.5\pgf@xa%
   	\pgf@x.5\pgf@xa%
   \fi%
	%
	% and add the outer sep.
	%	
	\pgf@xa\outersep\relax%
   \advance\pgf@x\pgf@xa%		
	\edef\polygonradius{\the\pgf@x}%
	%
	% Calculate the angle.
	%
	\pgf@y\polygonanglestep%
	\c@pgf@counta#1\relax%
	\advance\c@pgf@counta-1\relax%
	\pgf@x\polygonanglestep pt\relax%
	\multiply\pgf@x\c@pgf@counta%
	\pgf@xa\rotate pt\relax%
	\advance\pgf@x\pgf@xa%
	\advance\pgf@x90pt\relax%
	\ifodd\sides%
	\else%
		\advance\pgf@x-.5\pgf@y%
	\fi%
	\edef\firstangle{\the\pgf@x}%
	\advance\pgf@x\pgf@y%
	\edef\secondangle{\the\pgf@x}%
	\pgfpointlineattime{0.5}{%
		\pgfpointadd{\centerpoint}{\pgfpointpolar{+\firstangle}{+\polygonradius}}%
		}{%
			\pgfpointadd{\centerpoint}{\pgfpointpolar{+\secondangle}{+\polygonradius}}%
	}%
}%



\endinput
