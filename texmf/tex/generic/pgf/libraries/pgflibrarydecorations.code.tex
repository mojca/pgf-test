% Copyright 2008 by Mark Wibrow
%
% This file may be distributed and/or modified
%
% 1. under the LaTeX Project Public License and/or
% 2. under the GNU Public License.
%
% See the file doc/generic/pgf/licenses/LICENSE for more details.

\usepgfmodule{decorations}

\newdimen\pgfdecorationsegmentamplitude
\newdimen\pgfdecorationsegmentlength
\def\pgfdecorationsegmentangle{45}
\def\pgfdecorationsegmentaspect{0.5}
\def\pgfmetadecorationsegmentamplitude{2.5pt}  
\def\pgfmetadecorationsegmentlength{1cm}

\pgfdecorationsegmentamplitude2.5pt
\pgfdecorationsegmentlength10pt

\pgfkeys{%
  /pgf/decoration options/.code={\pgfkeys{/pgf/decoration options/.cd,#1}},
  /pgf/decoration options/.cd,
  amplitude/.code={\pgfmathsetlength\pgfdecorationsegmentamplitude{#1}},
  meta-amplitude/.store in=\pgfmetadecorationsegmentamplitude,
  segment length/.code={\pgfmathsetlength\pgfdecorationsegmentlength{#1}},
  meta-segment length/.store in=\pgfmetadecorationsegmentlength,
  angle/.code={\pgfmathparse{#1}\let\pgfdecorationsegmentangle\pgfmathresult},
  aspect/.code={\pgfmathparse{#1}\let\pgfdecorationsegmentaspect\pgfmathresult},
  text/.store in=\pgfdecorationtext,%                                                 
  text color/.store in=\pgf@lib@decorationtextcolor%
  shape/.initial=circle,
  anchor/.initial=circle,
} 






%
%
% Kind 1: Path deforming decorations
%
% These decorations "deform" paths. That means that the
% orginal characteristic of the path is kept and the number of
% subpaths remains the same -- only, the lines are slightly offset or
% changed by the decoration. For instance a line might be turned into
% a squiggly line or a snaking line or a bumping line.
%

%
%
% Kind 1.1: Path deforming straight line decorations
%
%


% zigzag decoration.
%
\pgfdeclaredecoration{zigzag}{up from center}{
  \state{up from center}[width=+.5\pgfdecorationsegmentlength, next state=big down]
  {
    \pgfpathlineto{\pgfqpoint{.25\pgfdecorationsegmentlength}{\pgfdecorationsegmentamplitude}}
  }
  \state{big down}[switch if less than=+.5\pgfdecorationsegmentlength to center finish,
                   width=+.5\pgfdecorationsegmentlength,
                   next state=big up]
  {
    \pgfpathlineto{\pgfqpoint{.25\pgfdecorationsegmentlength}{-\pgfdecorationsegmentamplitude}}
  }
  \state{big up}[switch if less than=+.5\pgfdecorationsegmentlength to center finish,
                 width=+.5\pgfdecorationsegmentlength,
                 next state=big down]
  {
    \pgfpathlineto{\pgfqpoint{.25\pgfdecorationsegmentlength}{\pgfdecorationsegmentamplitude}}
  }
  \state{center finish}[width=0pt, next state=final]{
    \pgfpathlineto{\pgfpointorigin}
  }
  \state{final}
  {
    \pgfpathlineto{\pgfpointdecoratedpathlast}
  }
}




% saw decoration
%
% Parameters: \pgfdecorationsegmentamplitude, \pgfdecorationsegmentlength

\pgfdeclaredecoration{saw}{initial}
{
  \state{initial}[width=+\pgfdecorationsegmentlength]
  {
    \pgfpathlineto{\pgfqpoint{\pgfdecorationsegmentlength}{\pgfdecorationsegmentamplitude}}
    \pgfpathlineto{\pgfqpoint{\pgfdecorationsegmentlength}{0pt}}
  }
  \state{final}
  {
    \pgfpathlineto{\pgfpointdecoratedpathlast}
  }
}




% random steps decoration
%
% A decoration that consists of random steps heading towards the target
%
% Parameters: \pgfdecorationsegmentamplitude, \pgfdecorationsegmentlength

\pgfdeclaredecoration{random steps}{step}
{
  \state{step}[width=+\pgfdecorationsegmentlength,next state=step]
  {
    \pgfpathlineto{
      \pgfpointadd
      {\pgfqpoint{\pgfdecorationsegmentlength}{0pt}}
      {\pgfpoint{rand*\pgfdecorationsegmentamplitude}{rand*\pgfdecorationsegmentamplitude}}
    }
  }

  \state{final}
  {
    \pgfpathlineto{\pgfpointdecoratedpathlast}
  }
}


% Fractal decorations. They should be replied repeatedly

\pgfdeclaredecoration{Koch curve type 1}{init}
{
  \state{init}[width=\pgfdecoratedsubpathremainingdistance]
  {
    \pgfpathlineto{\pgfpoint{.33333\pgfdecoratedsubpathremainingdistance}{0pt}}
    \pgfpathlineto{\pgfpoint{.33333\pgfdecoratedsubpathremainingdistance}{.33333\pgfdecoratedsubpathremainingdistance}}
    \pgfpathlineto{\pgfpoint{.66666\pgfdecoratedsubpathremainingdistance}{.33333\pgfdecoratedsubpathremainingdistance}}
    \pgfpathlineto{\pgfpoint{.66666\pgfdecoratedsubpathremainingdistance}{0pt}}
    \pgfpathlineto{\pgfpoint{\pgfdecoratedsubpathremainingdistance}{0pt}}
  }
}

\pgfdeclaredecoration{Koch curve type 2}{init}
{
  \state{init}[width=\pgfdecoratedsubpathremainingdistance]
  {
    \pgfpathlineto{\pgfpoint{.25\pgfdecoratedsubpathremainingdistance}{0pt}}
    \pgfpathlineto{\pgfpoint{.25\pgfdecoratedsubpathremainingdistance}{.25\pgfdecoratedsubpathremainingdistance}}
    \pgfpathlineto{\pgfpoint{.5\pgfdecoratedsubpathremainingdistance}{.25\pgfdecoratedsubpathremainingdistance}}
    \pgfpathlineto{\pgfpoint{.5\pgfdecoratedsubpathremainingdistance}{0pt}}
    \pgfpathlineto{\pgfpoint{.5\pgfdecoratedsubpathremainingdistance}{-.25\pgfdecoratedsubpathremainingdistance}}
    \pgfpathlineto{\pgfpoint{.75\pgfdecoratedsubpathremainingdistance}{-.25\pgfdecoratedsubpathremainingdistance}}
    \pgfpathlineto{\pgfpoint{.75\pgfdecoratedsubpathremainingdistance}{0pt}}
    \pgfpathlineto{\pgfpoint{\pgfdecoratedsubpathremainingdistance}{0pt}}
  }
}

\pgfdeclaredecoration{Koch snowflake}{init}
{
  \state{init}[width=\pgfdecoratedsubpathremainingdistance]
  {
    \pgfpathlineto{\pgfpoint{.3333\pgfdecoratedsubpathremainingdistance}{0pt}}
    \pgfpathlineto{\pgfpoint{.5\pgfdecoratedsubpathremainingdistance}{0.2886751347\pgfdecoratedsubpathremainingdistance}}
    \pgfpathlineto{\pgfpoint{.6666\pgfdecoratedsubpathremainingdistance}{0pt}}
    \pgfpathlineto{\pgfpoint{\pgfdecoratedsubpathremainingdistance}{0pt}}
  }
}



% Meta-decoration line zigzag

\pgfdeclaremetadecoration{line zigzag}{line to}{
  \state{line to}[width=\pgfmetadecorationsegmentlength, next state=zigzag]
  {
    \decoration{line along}
  }
  \state{zigzag}[width=\pgfmetadecorationsegmentlength, next state=line to]
  {
    \decoration{zigzag}	
  }
  \state{final}
  {
    \decoration{line along}
  }
}





%
%
% Kind 1.2: Path deforming curved decorations
%
%



% bent decoration
%
% A decoration that looks like someone bent the line a bit.
%
% Parameters: \pgfdecorationsegmentamplitude, \pgfdecorationsegmentaspect

\pgfdeclaredecoration{bent}{bent}
{
  \state{bent}[width=+\pgfdecoratedsubpathremainingdistance]
  {
    \pgfpathcurveto
    {\pgfqpoint{\pgfdecorationsegmentaspect\pgfdecoratedsubpathremainingdistance}{\pgfdecorationsegmentamplitude}}
    {\pgfpointadd{\pgfqpoint{\pgfdecoratedsubpathremainingdistance}{0pt}}
       {\pgfqpoint{-\pgfdecorationsegmentaspect\pgfdecoratedsubpathremainingdistance}{\pgfdecorationsegmentamplitude}}}
    {\pgfqpoint{\pgfdecoratedsubpathremainingdistance}{0pt}}
  }
  \state{final}
  {}
}



% decoration snake
%
% This decoration produces a hopefully optically pleasing squiggly snake. 
%
% Parameters: \pgfdecorationsegmentamplitude, \pgfdecorationsegmentlength

\pgfdeclaredecoration{snake}{initial}
{
  \state{initial}[switch if less than=+.625\pgfdecorationsegmentlength to final,
                  width=+.3125\pgfdecorationsegmentlength,
                  next state=down]
  { 
    \pgfpathcurveto
    {\pgfqpoint{.125\pgfdecorationsegmentlength}{0pt}}
    {\pgfqpoint{.1875\pgfdecorationsegmentlength}{\pgfdecorationsegmentamplitude}}
    {\pgfqpoint{.3125\pgfdecorationsegmentlength}{\pgfdecorationsegmentamplitude}}
  }
  \state{down}[switch if less than=+.8125\pgfdecorationsegmentlength to end down,
               width=+.5\pgfdecorationsegmentlength,
               next state=up]
  {
    \pgfpathcosine{\pgfqpoint{.25\pgfdecorationsegmentlength}{-1\pgfdecorationsegmentamplitude}}
    \pgfpathsine{\pgfqpoint{.25\pgfdecorationsegmentlength}{-1\pgfdecorationsegmentamplitude}}
  }               
  \state{up}[switch if less than=+.8125\pgfdecorationsegmentlength to end up,
             width=+.5\pgfdecorationsegmentlength,
             next state=down]
  {
    \pgfpathcosine{\pgfqpoint{.25\pgfdecorationsegmentlength}{\pgfdecorationsegmentamplitude}}
    \pgfpathsine{\pgfqpoint{.25\pgfdecorationsegmentlength}{\pgfdecorationsegmentamplitude}}
  }               
  \state{end down}[width=+.3125\pgfdecorationsegmentlength,
                   next state=final]
  {
    \pgfpathcurveto
    {\pgfqpoint{.125\pgfdecorationsegmentlength}{\pgfdecorationsegmentamplitude}}
    {\pgfqpoint{.1875\pgfdecorationsegmentlength}{0pt}}
    {\pgfqpoint{.3125\pgfdecorationsegmentlength}{0pt}}
  }  
  \state{end up}[width=+.3125\pgfdecorationsegmentlength,
                 next state=final]
  {
    \pgfpathcurveto
    {\pgfqpoint{.125\pgfdecorationsegmentlength}{-\pgfdecorationsegmentamplitude}}
    {\pgfqpoint{.1875\pgfdecorationsegmentlength}{0pt}}
    {\pgfqpoint{.3125\pgfdecorationsegmentlength}{0pt}}
  }  
  \state{final}
  {
    \pgfpathlineto{\pgfpointdecoratedpathlast}
  }
}


% coil decoration
%
% Parameters: \pgfdecorationsegmentamplitude, \pgfdecorationsegmentlength,

\pgfdeclaredecoration{coil}{coil}
{
  \state{coil}[switch if less than=%
    1.5\pgfdecorationsegmentlength+%
    \pgfdecorationsegmentaspect\pgfdecorationsegmentamplitude+%
    \pgfdecorationsegmentaspect\pgfdecorationsegmentamplitude to last,
               width=+\pgfdecorationsegmentlength]
  {
    \pgfpathcurveto
    {\pgfpoint@oncoil{0    }{ 0.555}{1}}
    {\pgfpoint@oncoil{0.445}{ 1    }{2}}
    {\pgfpoint@oncoil{1    }{ 1    }{3}}
    \pgfpathcurveto
    {\pgfpoint@oncoil{1.555}{ 1    }{4}}
    {\pgfpoint@oncoil{2    }{ 0.555}{5}}
    {\pgfpoint@oncoil{2    }{ 0    }{6}}
    \pgfpathcurveto
    {\pgfpoint@oncoil{2    }{-0.555}{7}}
    {\pgfpoint@oncoil{1.555}{-1    }{8}}
    {\pgfpoint@oncoil{1    }{-1    }{9}}
    \pgfpathcurveto
    {\pgfpoint@oncoil{0.445}{-1    }{10}}
    {\pgfpoint@oncoil{0    }{-0.555}{11}}
    {\pgfpoint@oncoil{0    }{ 0    }{12}}
  }
  \state{last}[width=.5\pgfdecorationsegmentlength+%
    \pgfdecorationsegmentaspect\pgfdecorationsegmentamplitude+%
    \pgfdecorationsegmentaspect\pgfdecorationsegmentamplitude,next state=final]
  {
    \pgfpathcurveto
    {\pgfpoint@oncoil{0    }{ 0.555}{1}}
    {\pgfpoint@oncoil{0.445}{ 1    }{2}}
    {\pgfpoint@oncoil{1    }{ 1    }{3}}
    \pgfpathcurveto
    {\pgfpoint@oncoil{1.555}{ 1    }{4}}
    {\pgfpoint@oncoil{2    }{ 0.555}{5}}
    {\pgfpoint@oncoil{2    }{ 0    }{6}}
  }
  \state{final}
  {
    \pgfpathlineto{\pgfpointdecoratedpathlast}
  }
}

\def\pgfpoint@oncoil#1#2#3{%
  \pgf@x=#1\pgfdecorationsegmentamplitude%
  \pgf@x=\pgfdecorationsegmentaspect\pgf@x%
  \pgf@y=#2\pgfdecorationsegmentamplitude%
  \pgf@xa=0.083333333333\pgfdecorationsegmentlength%
  \advance\pgf@x by#3\pgf@xa%
}


% bumps decoration
%
% Parameters: \pgfdecorationsegmentamplitude, \pgfdecorationsegmentlength

\pgfdeclaredecoration{bumps}{initial}
{
  \state{initial}[width=+.5\pgfdecorationsegmentlength]
  {
    \pgfpathcurveto
    {\pgfqpoint{0pt}{.555\pgfdecorationsegmentamplitude}}
    {\pgfqpoint{0.11125\pgfdecorationsegmentlength}{\pgfdecorationsegmentamplitude}}
    {\pgfqpoint{.25\pgfdecorationsegmentlength}{\pgfdecorationsegmentamplitude}}
    \pgfpathcurveto
    {\pgfqpoint{.38875\pgfdecorationsegmentlength}{\pgfdecorationsegmentamplitude}}
    {\pgfqpoint{.5\pgfdecorationsegmentlength}{.5\pgfdecorationsegmentamplitude}}
    {\pgfqpoint{.5\pgfdecorationsegmentlength}{0\pgfdecorationsegmentamplitude}}
  }
  \state{final}
  {
    \pgfpathlineto{\pgfpointdecoratedpathlast}
  }
}








%
% Kind 2: Path chopping decorations
%
% These decorations change the path by chopping it into smaller
% parts. For instance, a line in the path might be replaced by small
% ticks or unconnected curves or crosses. Applying a chopping
% decoration to a path means that the path can no longer be used for
% filling in the original manner.
%



%
%
% Kind 2.1: Path chopping with open subpaths
%
%


% ticks decoration
%
% Parameters: \pgfdecorationsegmentlength, \pgfdecorationsegmentamplitude

\pgfdeclaredecoration{ticks}{ticks}
{
  \state{ticks}[width=+\pgfdecorationsegmentlength]
  {
    \pgfpathmoveto{\pgfqpoint{0pt}{\pgfdecorationsegmentamplitude}}
    \pgfpathlineto{\pgfqpoint{0pt}{-\pgfdecorationsegmentamplitude}}
  }
  \state{final}
  {
    \pgfpathmoveto{\pgfqpoint{0pt}{\pgfdecorationsegmentamplitude}}
    \pgfpathlineto{\pgfqpoint{0pt}{-\pgfdecorationsegmentamplitude}}
    \pgfpathmoveto{\pgfpointdecoratedpathlast}
  }
}



% expanding waves decoration
%
% Parameters: \pgfdecorationsegmentangle, \pgfdecorationsegmentlength

\pgfdeclaredecoration{expanding waves}{initial}
{
  \state{initial}[width=+\pgfdecorationsegmentlength,next state=wave]
  {}

  \state{wave}[switch if less than=+\pgfdecorationsegmentlength to last,
               width=+\pgfdecorationsegmentlength]
  {
    \pgfpathmoveto{
      \pgfpointadd
      {\pgfqpoint{-\pgfdecoratedcompleteddistance}{0pt}}%
      {\pgfpointpolar{\pgfdecorationsegmentangle}{+\pgfdecoratedcompleteddistance}}}%
    \pgfpatharc{\pgfdecorationsegmentangle}{-\pgfdecorationsegmentangle}{+\pgfdecoratedcompleteddistance}%
  }
  \state{last}[width=+0pt,next state=final]
  {
    \pgfpathmoveto{
      \pgfpointadd
      {\pgfqpoint{-\pgfdecoratedcompleteddistance}{0pt}}%
      {\pgfpointpolar{\pgfdecorationsegmentangle}{+\pgfdecoratedcompleteddistance}}}%
    \pgfpatharc{\pgfdecorationsegmentangle}{-\pgfdecorationsegmentangle}{+\pgfdecoratedcompleteddistance}%
  }
  \state{final}
  {
    \pgfpathmoveto{\pgfpointdecoratedpathlast}
  }
}



% waves decoration
%
% Parameters: \pgfdecorationsegmentangle, \pgfdecorationsegmentlength

\pgfdeclaredecoration{waves}{init}
{
  \state{init}[width=+0pt,next state=wave,persistent precomputation={
    \pgfmathparse{\pgfkeysvalueof{/pgf/decoration options/shape start width}}
    \edef\pgf@lib@dec@ol{\pgfmathresult pt}
  }]{}  
  \state{wave}[width=\pgfdecorationsegmentlength]
  {
    \pgftransformxshift{+\pgfdecorationsegmentlength}
    \pgfpathmoveto{
      \pgfpointadd
      {\pgfqpoint{-\pgf@lib@dec@ol}{0pt}}%
      {\pgfpointpolar{\pgfdecorationsegmentangle}{+\pgf@lib@dec@ol}}}%
    \pgfpatharc{\pgfdecorationsegmentangle}{-\pgfdecorationsegmentangle}{+\pgf@lib@dec@ol}%
  }
  \state{final}
  {
    \pgfpathmoveto{\pgfpointdecoratedpathlast}
  }
}



% border decoration
%
% Parameters: \pgfdecorationsegmentlength, \pgfdecorationsegmentamplitude, \pgfdecorationsegmentangle

\pgfdeclaredecoration{border}{init}
{
  \state{init}[width=+0pt,next state=tick,persistent precomputation={
    \pgfmathparse{\pgfkeysvalueof{/pgf/decoration options/shape start width}}
    \edef\pgf@lib@dec@ol{\pgfmathresult pt}
  }]{}  
  \state{tick}[switch if less than=+\pgfdecorationsegmentlength to last,
               width=+\pgfdecorationsegmentlength]
  {
    \pgfpathmoveto{\pgfpointorigin}
    \pgfpathlineto{\pgfpointpolar{\pgfdecorationsegmentangle}{+\pgfdecorationsegmentamplitude}}
  }
  \state{last}[width=+\pgfdecorationsegmentamplitude,next state=final]
  {
    \pgfpathmoveto{\pgfpointorigin}
    \pgfpathlineto{\pgfpointpolar{\pgfdecorationsegmentangle}{+\pgfdecorationsegmentamplitude}}
  }  
  \state{final}
  {
    \pgfpathmoveto{\pgfpointdecoratedpathlast}
  }
}




% crosses decoration
%
% Parameters: \pgfdecorationsegmentlength, \pgfdecorationobjectsize, \pgfdecorationsegmentamplitude

\pgfdeclaredecoration{crosses}{init}
{
  \state{init}[width=+0pt,next state=crosses,persistent precomputation={
    \pgfmathparse{\pgfkeysvalueof{/pgf/decoration options/shape start width}}
    \edef\pgf@lib@dec@ol{\pgfmathresult pt}
  }]{}  
  \state{crosses}[switch if less than=+\pgfdecorationsegmentlength to last,
                   width=+\pgfdecorationsegmentlength]
  {
    \pgfpathmoveto{\pgfqpoint{0pt}{\pgfdecorationsegmentamplitude}}
    \pgfpathlineto{\pgfqpoint{\pgf@lib@dec@ol}{-\pgfdecorationsegmentamplitude}}
    \pgfpathmoveto{\pgfqpoint{0pt}{-\pgfdecorationsegmentamplitude}}
    \pgfpathlineto{\pgfqpoint{\pgf@lib@dec@ol}{\pgfdecorationsegmentamplitude}}
  }
  \state{last}[width=\pgf@lib@dec@ol,next state=final]
  {
    \pgfpathmoveto{\pgfqpoint{0pt}{\pgfdecorationsegmentamplitude}}
    \pgfpathlineto{\pgfqpoint{\pgf@lib@dec@ol}{-1\pgfdecorationsegmentamplitude}}
    \pgfpathmoveto{\pgfqpoint{0pt}{-\pgfdecorationsegmentamplitude}}
    \pgfpathlineto{\pgfqpoint{\pgf@lib@dec@ol}{\pgfdecorationsegmentamplitude}}
  }
  \state{final}
  {
    \pgfpathmoveto{\pgfpointdecoratedpathlast}
  }
}






% brace decorations
%
% Parameters: \pgfdecorationsegmentamplitude

\pgfdeclaredecoration{brace}{brace}
{
  \state{brace}[width=+\pgfdecoratedremainingdistance,next state=final]
  {
    \pgfpathmoveto{\pgfpointorigin}
    \pgfpathcurveto
    {\pgfqpoint{.15\pgfdecorationsegmentamplitude}{.3\pgfdecorationsegmentamplitude}}
    {\pgfqpoint{.5\pgfdecorationsegmentamplitude}{.5\pgfdecorationsegmentamplitude}}
    {\pgfqpoint{\pgfdecorationsegmentamplitude}{.5\pgfdecorationsegmentamplitude}}
    {
      \pgftransformxshift{+\pgfdecorationsegmentaspect\pgfdecoratedremainingdistance}
      \pgfpathlineto{\pgfqpoint{-\pgfdecorationsegmentamplitude}{.5\pgfdecorationsegmentamplitude}}
      \pgfpathcurveto
      {\pgfqpoint{-.5\pgfdecorationsegmentamplitude}{.5\pgfdecorationsegmentamplitude}}
      {\pgfqpoint{-.15\pgfdecorationsegmentamplitude}{.7\pgfdecorationsegmentamplitude}}
      {\pgfqpoint{0\pgfdecorationsegmentamplitude}{1\pgfdecorationsegmentamplitude}}
      \pgfpathcurveto
      {\pgfqpoint{.15\pgfdecorationsegmentamplitude}{.7\pgfdecorationsegmentamplitude}}
      {\pgfqpoint{.5\pgfdecorationsegmentamplitude}{.5\pgfdecorationsegmentamplitude}}
      {\pgfqpoint{\pgfdecorationsegmentamplitude}{.5\pgfdecorationsegmentamplitude}}
    }
    {
      \pgftransformxshift{+\pgfdecoratedremainingdistance}
      \pgfpathlineto{\pgfqpoint{-\pgfdecorationsegmentamplitude}{.5\pgfdecorationsegmentamplitude}}
      \pgfpathcurveto
      {\pgfqpoint{-.5\pgfdecorationsegmentamplitude}{.5\pgfdecorationsegmentamplitude}}
      {\pgfqpoint{-.15\pgfdecorationsegmentamplitude}{.3\pgfdecorationsegmentamplitude}}
      {\pgfqpoint{0pt}{0pt}}
    }
  }
  \state{final}
  {}
}


% Fractal decoration, should be applied repeatedly

\pgfdeclaredecoration{Cantor set}{init}
{
  \state{init}[width=\pgfdecoratedsubpathremainingdistance]
  {
    \pgfpathlineto{\pgfpoint{.3333\pgfdecoratedsubpathremainingdistance}{0pt}}
    \pgfpathmoveto{\pgfpoint{.6666\pgfdecoratedsubpathremainingdistance}{0pt}}
    \pgfpathlineto{\pgfpoint{\pgfdecoratedsubpathremainingdistance}{0pt}}
  }
}



%
%
% Kind 2.2: Path chopping with closed subpaths
%
%




% triangle decoration
%
% Parameters: \pgfdecorationsegmentlength, \pgfdecorationobjectsize, \pgfdecorationsegmentamplitude

\pgfdeclaredecoration{triangles}{init}
{
  \state{init}[width=+0pt,next state=triangle,persistent precomputation={
    \pgfmathparse{\pgfkeysvalueof{/pgf/decoration options/shape start width}}
    \edef\pgf@lib@dec@ol{\pgfmathresult pt}
  }]{}  
  \state{triangle}[switch if less than=+\pgfdecorationsegmentlength to last,
                   width=+\pgfdecorationsegmentlength]
  {
    \pgfpathmoveto{\pgfqpoint{0pt}{\pgfdecorationsegmentamplitude}}
    \pgfpathlineto{\pgfqpoint{\pgf@lib@dec@ol}{0pt}}
    \pgfpathlineto{\pgfqpoint{0pt}{-\pgfdecorationsegmentamplitude}}
    \pgfpathclose
  }
  \state{last}[width=+\pgf@lib@dec@ol,next state=final]
  {
    \pgfpathmoveto{\pgfqpoint{0pt}{\pgfdecorationsegmentamplitude}}
    \pgfpathlineto{\pgfqpoint{\pgf@lib@dec@ol}{0pt}}
    \pgfpathlineto{\pgfqpoint{0pt}{-\pgfdecorationsegmentamplitude}}
    \pgfpathclose
  }
  \state{final}
  {
    \pgfpathmoveto{\pgfpointdecoratedpathlast}
  }
}





%
%
% Kind 2.3: Path chopping with other subpaths
%
%


% The shape background decoration
%	
% The shape background decoration adds repeated instances of 
% the background path of a specified shape along the path. The shape
% must have been declared  by \pgfdeclareshape. If a shape has
% specialized keys (e.g. the number of points on a star, or the apex
% angle the isosceles triangle), these can be specified in the usual manner. 
%	
% The sepatation between shapes in the path can be specified and can 
% be between the center of the shape or the border of the shape. 
%	
% The height and width of the shape can be independently or
% simultaneously scaled (linearly) along the path. It is also
% possible to prevent the shapes being sloped parallel to the
% path.

% Keys for snake: shape backgrounds
%
% shape              : the shape used in the snake.
% shape sep          : the distance between shape borders.
% shape scaled       : scale the shapes in the snake along the path.
% shape start width  : recommended starting width.
% shape start height : recommended starting height.
% shape start size
% shape end width    : recommended ending width.
% shape end height   : recommended ending height.
% shape end size
% shape sloped       : make the shapes slope parallel to the path.

\pgfkeys{/pgf/decoration options/.cd,
  % 
  shape start width/.initial=2.5pt,
  shape start height/.initial=2.5pt,
  shape end width/.initial=2.5pt,
  shape end height/.initial=2.5pt,
  shape sep/.store in=\pgf@lib@shapesnake@sep,
  shape sloped/.is if=pgfshapesnakesloped,
  shape scaled/.is if=pgfshapesnakescaled,
  shape evenly spread/.store in=\pgf@lib@shapesnake@spread,
  shape start size/.style={%
    shape start width=#1,
    shape start height=#1%
  },%
  shape end size/.style={%
    shape end width=#1,
    shape end height=#1%
  },%
  shape size/.style={%
    shape start size=#1,
    shape end size=#1%
  },%
  shape width/.style={%
    shape start width=#1,
    shape end width=#1
  },
  shape height/.style={%
    shape start height=#1,
    shape end eight=#1
  }
}%

\def\pgf@lib@shapesnake@sep{.25cm, between centers}

\newif\ifpgfshapesnakesloped
\pgfshapesnakeslopedtrue
\newif\ifpgfshapesnakescaled

\let\pgf@lib@shapesnake@spread\pgfutil@empty%

% internal if
\newif\ifpgf@lib@shapesnake@betweenborders

\edef\pgf@lib@shapesnake@initialise{0pt}%

\pgfdeclaresnake{shape backgrounds}{initialise}
{
  \state{initialise}
  [
    width=+\pgf@lib@shapesnake@initialise,
    next state=shape,
    persistent precomputation=
    {
      % 
      % \egroup ends the group started by the automaton before executing
      % a snake state. This prevents the need for (most) \global variables. 
      % 
      % 
      % Check the shape exists.
      % 
      \pgfutil@ifundefined{pgf@sh@bg@\pgfkeysvalueof{/pgf/decoration options/shape}}{%
        \PackageError{PGF}{I do not know the shape `\pgfkeysvalueof{/pgf/decoration options/shape}',
          so I cannot use it in a snake. Check if its library been loaded or if you 
          simply mistyped the name}{}}{}%
      % 
      % Calculate a `default' path size.
      % 
      \pgfinterruptpath%
        \pgfinterruptboundingbox%
          \pgftransformreset%
          \pgf@relevantforpicturesizetrue%
          % 
          % This size of this shape is unimportant, but it should
          % be just large/small enough to avoid huge errors when
          % calculting the scaling factors later on.
          % 
          \pgfkeys{/pgf/inner sep=50pt, /pgf/minimum size=1pt}% Arbitrary lengths.
          \setbox\pgfnodeparttextbox\hbox{}% Assume shape does nothing special if box is empty.
          \let\pgf@sh@savedmacros\pgfutil@empty% 
          \let\pgf@sh@savedpoints\pgfutil@empty%
          \csname pgf@sh@s@\pgfkeysvalueof{/pgf/decoration options/shape}\endcsname%
          \pgf@sh@savedpoints%
          \pgf@sh@savedmacros%
          % 
          % Save the macros and pionts.
          % 
          \expandafter\gdef\expandafter\pgf@lib@shapesnake@points\expandafter{\pgf@sh@savedpoints}%
          \expandafter\gdef\expandafter\pgf@lib@shapesnake@macros\expandafter{\pgf@sh@savedmacros}%
          \csname pgf@sh@bg@\pgfkeysvalueof{/pgf/decoration options/shape}\endcsname% 
          % 
          % Save the dimensions of the shape path.
          % 
          \pgf@x\pgf@picmaxx%
          \pgf@y\pgf@picmaxy%
          \advance\pgf@x-\pgf@picminx%
          \advance\pgf@y-\pgf@picminy%
          \xdef\pgf@lib@shapesnake@shapepathsize{%
            \noexpand\pgf@x\the\pgf@x%
            \noexpand\pgf@y\the\pgf@y%
          }%
        \endpgfinterruptboundingbox%
      \endpgfinterruptpath%
      % 
      \edef\pgf@lib@shapesnake@beforeshape{0pt}%
      \edef\pgf@lib@shapesnake@aftershape{0pt}%
      % 
      \pgfmathsetlength\pgf@x{\pgfkeysvalueof{/pgf/decoration options/shape start width}}%
      \edef\pgf@lib@shapesnake@startwidth{\the\pgf@x}%
      \edef\pgf@lib@shapesnake@width{\the\pgf@x}%
      \pgf@x-\pgf@x%
      \pgfmathaddtolength\pgf@x{\pgfkeysvalueof{/pgf/decoration options/shape end width}}%
      \edef\pgf@lib@shapesnake@widthchange{\the\pgf@x}%
      % 
      \pgfmathsetlength\pgf@y{\pgfkeysvalueof{/pgf/decoration options/shape start height}}%
      \edef\pgf@lib@shapesnake@initialheight{\the\pgf@y}%
      \edef\pgf@lib@shapesnake@height{\the\pgf@y}%
      \pgf@y-\pgf@y%
      \pgfmathaddtolength\pgf@y{\pgfkeysvalueof{/pgf/decoration options/shape end height}}%
      \edef\pgf@lib@shapesnake@heightchange{\the\pgf@y}%		
      % 
      % Calculate the sep.
      % 
      \ifx\pgf@lib@shapesnake@spread\pgfutil@empty%
        % 
        % Not spreading, so easy:
        % 
        \def\pgf@lib@shapesnake@borderstext{between borders}%
        \afterassignment\pgf@lib@shapesnake@setkeyword%
        \expandafter\pgf@x\pgf@lib@shapesnake@sep,\pgf@stop%
        \edef\pgf@lib@shapesnake@sep{\the\pgf@x}%
      \else%
        % 
        % Spreading (a bit of a nuiscence actually).
        % 
        \def\pgf@lib@shapesnake@borderstext{by borders}%
        \afterassignment\pgf@lib@shapesnake@setkeyword%
        \expandafter\c@pgf@counta\pgf@lib@shapesnake@spread,\pgf@stop%
        \ifpgf@lib@shapesnake@betweenborders%
          %  
          % Ok. The required sep between borders is:
          % 
          % (r -(n-1)((a+b)/2))/(n-1)
          % 
          % r = snake length (here, the remaining distance)
          % a = initial width
          % b = end width
          % n = the number of shapes
          % 
          \ifnum\c@pgf@counta>1\relax%
            \advance\c@pgf@counta-1\relax%
            \pgfmathsetlength\pgf@x{\pgfkeysvalueof{/pgf/decoration options/shape start width}}%
            \ifpgfshapesnakescaled%
              \pgfmathaddtolength\pgf@x{\pgfkeysvalueof{/pgf/decoration options/shape end width}}%
            \else%
              \advance\pgf@x\pgf@x%
            \fi%
            \pgf@x.5\pgf@x% (a+b)/2
            \multiply\pgf@x-\c@pgf@counta% -(n-1)((a+b)/2)
            \advance\pgf@x\pgfsnakeremainingdistance%
            \divide\pgf@x\c@pgf@counta%
            \pgf@x.9999\pgf@x% Hackery to control some native TeX inaccuracies.
            % 
            % Unfortunately if the shape is scaled, and evenly spread by borders,
            % it is necessary to do something a bit different to control for 
            % (most) inaccuracies.
            % 
            \ifpgfshapesnakescaled%
              \pgf@xa\pgf@lib@shapesnake@widthchange\relax%
              \divide\pgf@xa\c@pgf@counta%
              \edef\pgf@lib@shapesnake@specialwidth{\the\pgf@xa}%		
            \fi%
          \else%
            \pgf@lib@shapesnake@betweenbordersfalse%
            \pgf@x\pgfsnakeremainingdistance%
            \ifnum\c@pgf@counta=1\relax%
              \pgf@y.5\pgf@x%
              \edef\pgf@lib@shapesnake@initialise{\the\pgf@y}%
            \else%
              \advance\pgf@x5pt\relax% An arbitrary value >0pt.
              \edef\pgf@lib@shapesnake@initialise{\the\pgf@x}%
            \fi%
          \fi%	
        \else%
          %  
          % Between centers.
          % 
          \pgf@x\pgfsnakeremainingdistance%
          \ifnum\c@pgf@counta>1\relax%
            \advance\c@pgf@counta-1\relax%
            \divide\pgf@x\c@pgf@counta\relax%
          \else%
            \ifnum\c@pgf@counta=1\relax%
              \pgf@y.5\pgf@x%
              \edef\pgf@lib@shapesnake@initialise{\the\pgf@y}%
            \else%
              \advance\pgf@x5pt\relax% An arbitrary value >0pt.
              \edef\pgf@lib@shapesnake@initialise{\the\pgf@x}%
            \fi%
          \fi%
        \fi%
        \edef\pgf@lib@shapesnake@sep{\the\pgf@x}%
      \fi%
    }]
  {}
  \state{before shape}
  [
    width=+\pgf@lib@shapesnake@beforeshape,
    next state=shape,
    persistent precomputation=
    {
      \ifpgfshapesnakescaled%
        \ifpgf@lib@shapesnake@betweenborders%
          \ifx\pgf@lib@shapesnake@spread\pgfutil@empty%
            % 
            % Not so straightforward. The required ratio is given by
            % 
            % R = (c+W/2)/(c+r-.5*w)
            % 
            % c = completed distance
            % r = remaining distance
            % W = initial width
            % w = the change in width (i.e., end - start)
            % 
            \pgf@x\pgfsnakecompleteddistance%
            \advance\pgf@x\pgfsnakeremainingdistance%
            \pgf@xa\pgf@lib@shapesnake@startwidth\relax%
            \pgf@xa.5\pgf@xa%
            \advance\pgf@xa\pgfsnakecompleteddistance% c+W/2
            % 
            \pgf@xb\pgf@lib@shapesnake@widthchange\relax%
            \pgf@xb-.5\pgf@xb%
            \advance\pgf@xb\pgf@x% c+r-.5*w
            % 
            \pgfmathdivide@{\pgfmath@tonumber{\pgf@xa}}{\pgfmath@tonumber{\pgf@xb}}%
          \fi%
        \else%
          % 
          % Easy peasy. The required ratio is 
          % 
          % R = c / (c+r)
          % 
          \pgf@x\pgfsnakecompleteddistance%
          \advance\pgf@x\pgfsnakeremainingdistance%
          \pgfmathdivide@{\pgfmath@tonumber{\pgfsnakecompleteddistance}}{\pgfmath@tonumber{\pgf@x}}%			
        \fi%
        % 
        % Get the new width.
        % 
        \ifx\pgf@lib@shapesnake@spread\pgfutil@empty%
          \pgf@x\pgf@lib@shapesnake@widthchange\relax%
          \pgf@x\pgfmathresult\pgf@x%
          \advance\pgf@x\pgf@lib@shapesnake@startwidth\relax%
        \else%
          \ifpgf@lib@shapesnake@betweenborders%
            % 
            % Specical case when snake is scaled, and evenly spread by borders.
            % 
            \pgf@x\pgf@lib@shapesnake@width\relax%
            \advance\pgf@x\pgf@lib@shapesnake@specialwidth\relax%
            \pgf@xa\pgf@x%
            \advance\pgf@xa-\pgf@lib@shapesnake@startwidth\relax%
            \pgf@xb\pgf@lib@shapesnake@widthchange\relax%
            \pgfmathdivide@{\pgfmath@tonumber{\pgf@xa}}{\pgfmath@tonumber{\pgf@xb}}%
          \else%
            \pgf@x\pgf@lib@shapesnake@widthchange\relax%
            \pgf@x\pgfmathresult\pgf@x%
            \advance\pgf@x\pgf@lib@shapesnake@startwidth\relax%
          \fi%
        \fi%		
        \edef\pgf@lib@shapesnake@width{\the\pgf@x}%		
        % 
        % New height = R*h + H
        % 
        \pgf@y\pgf@lib@shapesnake@heightchange\relax%
        \pgf@y\pgfmathresult\pgf@y%
        \advance\pgf@y\pgf@lib@shapesnake@initialheight\relax%
        \edef\pgf@lib@shapesnake@height{\the\pgf@y}%
      \fi%
      % 
      \ifpgf@lib@shapesnake@betweenborders%
        \pgf@x\pgf@lib@shapesnake@width\relax%
        \pgf@x.5\pgf@x%
        \edef\pgf@lib@shapesnake@beforeshape{\the\pgf@x}%
      \else%
        \def\pgf@lib@shapesnake@beforeshape{0pt}%
      \fi%	
    }]
  {}
  \state{shape}[width=+1sp,next state=after shape]
  {
    \ifpgfshapesnakesloped%
    \else%
      \pgftransformrotate{-\pgfsnakeangle}%
    \fi%
    % 
    % Scale the path when it is actually drawn.
    % 
    \pgf@lib@shapesnake@shapepathsize%
    \pgfutil@tempdima\pgf@x%
    \pgfutil@tempdimb\pgf@y%
    \pgf@xa\pgf@lib@shapesnake@width\relax%
    \pgf@xb\pgfutil@tempdima%
    \pgfmathdivide@{\pgfmath@tonumber{\pgf@xa}}{\pgfmath@tonumber{\pgf@xb}}%
    \expandafter\pgftransformxscale\expandafter{\pgfmathresult}%
    % 
    \pgf@ya\pgf@lib@shapesnake@height\relax%
    \pgf@yb\pgfutil@tempdimb%
    \pgfmathdivide@{\pgfmath@tonumber{\pgf@ya}}{\pgfmath@tonumber{\pgf@yb}}%
    \expandafter\pgftransformyscale\expandafter{\pgfmathresult}%
    % 
    % Move to the center anchor.
    % 
    \pgf@lib@shapesnake@points%
    \pgf@lib@shapesnake@macros%
    \pgftransformshift{%
      \pgf@sh@reanchor{\pgfkeysvalueof{/pgf/decoration options/shape}}{\pgfkeysvalueof{/pgf/decoration options/anchor}%
      \pgf@x-\pgf@x%
      \pgf@y-\pgf@y%
    }%
    % 
    % And draw the shape path.
    % 
    \csname pgf@sh@bg@\pgfkeysvalueof{/pgf/decoration options/shape}\endcsname%	
  }
  \state{after shape}
  [
    width=+\pgf@lib@shapesnake@aftershape,
    next state=sep,
    persistent precomputation=
    {
      \ifpgf@lib@shapesnake@betweenborders%
        \pgf@x\pgf@lib@shapesnake@width\relax%
        \pgf@x.5\pgf@x%
        \edef\pgf@lib@shapesnake@aftershape{\the\pgf@x}%
      \else%
        \edef\pgf@lib@shapesnake@aftershape{0pt}%
      \fi%
    }
  ]
  {}
  \state{sep}[width=\pgf@lib@shapesnake@sep,next state=before shape,
              persistent precomputation=\def\pgf@lib@shapesnake@beforeshape{0pt}]
  {}
  \state{final}
  {
    \pgfpathmoveto{\pgfpointdecoratedpathlast}%
  }
}

\def\pgf@lib@shapesnake@setkeyword,{%
  \pgfutil@ifnextchar\pgf@stop{\def\pgf@temp{}\pgf@lib@@@shapesnake@setkeyword}{\pgf@lib@@shapesnake@setkeyword}%
}
\def\pgf@lib@@shapesnake@setkeyword#1,{\def\pgf@temp{#1}\pgf@lib@@@shapesnake@setkeyword}
\def\pgf@lib@@@shapesnake@setkeyword\pgf@stop{%
  \ifx\pgf@temp\pgf@lib@shapesnake@borderstext%
    \pgf@lib@shapesnake@betweenborderstrue%
  \else%
    \pgf@lib@shapesnake@betweenbordersfalse%
  \fi%
}





%
% Kind 3: Path removing decorations
%
% These decorations remove the path. After the decoration has been
% used, the path will be empty. The effect of the decoration is
% to draw/fill/whatever other paths or to draw some text.
%





% Decorates a path with a text. The path is removed during this
% process

\pgfdeclaredecoration{text}{initial}{
  \state{initial}[width=+0pt,
                  next state=scan,
                  persistent precomputation=\let\pgfdecorationrestoftext\pgfdecorationtext]
  {}
  \state{scan}[width=+0pt,
               next state=before typeset,
               persistent precomputation=
               {
                 \pgf@lib@decorations@text@scanchar%
                 \ifvoid\pgf@lib@decorations@text@box%
                   \setbox\pgf@lib@decorations@text@box\hbox{}%
                   \wd\pgf@lib@decorations@text@box16383pt\relax%
                 \fi%
               }]
  {}
  \state{before typeset}[width=+.5\wd\pgf@lib@decorations@text@box, next state=typeset]{}
  \state{typeset}[width=+0pt, next state=after typeset]
  {
    \pgftransformxshift{+-.5\wd\pgf@lib@decorations@text@box}%
    \setbox\pgf@hbox\hbox{\copy\pgf@lib@decorations@text@box}%
    \pgfqboxsynced\pgf@hbox%
  }
  \state{after typeset}[width=+.5\wd\pgf@lib@decorations@text@box, next state=scan]{}
  \state{final}{}
}

% Keys for text decoration:
%
\pgfkeys{/pgf/decoration options/text format delimiters/.code={\expandafter\pgfsetdecoratetextformatdelimiters#1}}}
\def\pgf@lib@decorationtextcolor{black}

% \pgfsetdecoratetextformatdelimiters
% 
% Set the delimiters for formatting in the text decoration.
% NB: Catcodes for delimiters should be 11 or 12.
%
% Examples:
%
% \pgfsetdecoratetextformatdelimiters{|}{}% 2nd argument can be empty.
%
% \def\pgfdecoratetext{A big |\color{red}|red|| apple.}
%
% \pgfsetdecoratetextformatdelimiters{[}{]}
%
% \def\pgfdecoratetext{The [\it]very[+\color{green}]green[] sprouts.}
%
\def\pgfsetdecoratetextformatdelimiters#1#2{%
	\def\pgf@lib@decorations@text@formatchar{#1}%
	\def\pgf@test{#2}%
	\ifx\pgf@test\pgfutil@empty%
		\def\pgf@lib@decorations@text@collectformat##1#1{%
		\pgf@lib@decorations@text@@collectformat##1\pgf@stop}%
	\else%
		\def\pgf@lib@decorations@text@collectformat##1#2{%
			\pgf@lib@decorations@text@@collectformat##1\pgf@stop}%
	\fi%
}

\pgfsetdecoratetextformatdelimiters{|}{}

\newbox\pgf@lib@decorations@text@box
\newif\ifpgf@lib@decorate@textmathmode

\let\pgfdecorationtext\pgfutil@empty
\let\pgfdecorationrestoftext\pgfutil@empty%
\let\pgf@lib@decorations@text@format\pgfutil@empty

\def\pgf@lib@decorations@text@scanchar{%
	\ifx\pgfdecorationrestoftext\pgfutil@empty%
		\let\pgf@lib@decorations@text@char\pgfutil@empty%
		\setbox\pgf@lib@decorations@text@box\box\voidb@x%
		\let\pgf@next\relax%
	\else%
		\let\pgf@next\pgf@lib@decorations@text@@scanchar%
	\fi%
	\pgf@next}

\def\pgf@lib@decorations@text@@scanchar{%
	\expandafter\pgf@lib@decorations@text@@@scanchar\pgfdecorationrestoftext\pgf@stop}

\def\pgf@lib@decorations@text@@@scanchar{%
	\futurelet\pgf@lib@decorations@lettoken%
	\pgf@lib@decorations@text@@@@scanchar}
	
\def\pgf@lib@decorations@text@@@@scanchar{%
	\ifx\pgf@lib@decorations@lettoken\pgfutil@sptoken%
		\let\pgf@next\pgf@lib@decorations@text@insertspace%
	\else%
		\let\pgf@next\pgf@lib@decorations@text@@@@@scanchar%
	\fi%
	\pgf@next}

\def\pgf@lib@decorations@text@@@@@scanchar{%
	\pgfutil@ifnextchar\bgroup{\pgf@lib@decorations@text@collectgroup}%
		{\pgf@lib@decorations@text@@@@@@scanchar}}
		
\def\pgf@lib@decorations@text@collectgroup#1{%
	\def\pgf@lib@decorations@text@char{#1}%
	\pgf@lib@decorations@text@collectrestoftext}
	
\def\pgf@lib@decorations@text@@@@@@scanchar#1{%
	\ifx#1\pgf@stop%
		\pgf@lib@decorations@text@box\box\voidb@x%
		\let\pgf@next\pgf@lib@decorations@text@endoftext%
	\else%
		\def\pgf@lib@decorations@text@char{#1}%
		\ifx#1\space%
			\let\pgf@next\pgf@lib@decorations@text@collectrestoftext%
		\else%
			\ifx#1\ %
				\let\pgf@next\pgf@lib@decorations@text@collectrestoftext%
			\else%
				\ifx\pgf@lib@decorations@text@char\pgf@lib@decorations@text@formatchar%
					\let\pgf@next\pgf@lib@decorations@text@collectformat%
				\else%
					\expandafter\ifcat\noexpand#1\relax%
						\let\pgf@next\pgf@lib@decorations@text@expandcs%
					\else%
						\ifnum\catcode`#1=3\relax%
							\ifpgf@lib@decorate@textmathmode%
								\pgf@lib@decorate@textmathmodefalse%
							\else%
								\pgf@lib@decorate@textmathmodetrue%
							\fi%
							\let\pgf@next\pgf@lib@decorations@text@@@scanchar%
						\else%
							\let\pgf@next\pgf@lib@decorations@text@collectrestoftext%
						\fi%
					\fi%
				\fi%
			\fi%
		\fi%
	\fi%
	\pgf@next%
}

\def\pgf@lib@decorations@text@@collectformat{%
	\pgfutil@ifnextchar+{\pgf@lib@decorations@text@addtoformat}{\pgf@lib@decorations@text@setformat}}
	
\def\pgf@lib@decorations@text@setformat#1\pgf@stop{%
	\def\pgf@lib@decorations@text@format{#1}%
	\pgf@lib@decorations@text@@@scanchar%
}

\def\pgf@lib@decorations@text@addtoformat+#1\pgf@stop{%
	\expandafter\def\expandafter\pgf@lib@decorations@text@format\expandafter{\pgf@lib@decorations@text@format#1}%
	\pgf@lib@decorations@text@@@scanchar%
}

\def\pgf@lib@decorations@text@insertspace{%
	\pgfutil@ifnextchar\bgroup{\pgf@lib@decorations@text@@insertspacegrp}%
		{\pgf@lib@decorations@text@@insertspace}}
		
\def\pgf@lib@decorations@text@@insertspacegrp#1{%
	\pgf@lib@decorations@text@@@@@@scanchar\space{#1}}
	
\def\pgf@lib@decorations@text@@insertspace#1{%
	\pgf@lib@decorations@text@@@@@@scanchar\space#1}
	
\def\pgf@lib@decorations@text@expandcs{%
	\expandafter\expandafter\expandafter\pgf@lib@decorations@text@@@@@scanchar%
		\pgf@lib@decorations@text@char}

\def\pgf@lib@decorations@text@endoftext{%
	\let\pgfdecoraterestoftext\pgfutil@empty%
	\let\pgf@lib@decorations@text@char\pgfutil@empty%
}
\def\pgf@lib@decorations@text@collectrestoftext{%
	\pgf@lib@decorations@text@dobox%
	\futurelet\pgf@lib@decorations@text@lettoken%
	\pgf@lib@decorations@text@@collectrestoftext}

\def\pgf@lib@decorations@text@@collectrestoftext{%
	\ifx\bgroup\pgf@lib@decorations@text@lettoken%
		\let\pgf@next\pgf@lib@decorations@text@@@collectrestoftextgrp%
	\else%
		\let\pgf@next\pgf@lib@decorations@text@@@collectrestoftext%
	\fi%
	\pgf@next}
	
\def\pgf@lib@decorations@text@@@collectrestoftextgrp#1#2\pgf@stop{\def\pgfdecorationrestoftext{{#1}#2}%
}

\def\pgf@lib@decorations@text@@@collectrestoftext#1\pgf@stop{\def\pgfdecorationrestoftext{#1}}

{%
	\catcode`\$3 %
	\gdef\pgf@lib@decorations@mathshift{$}%
	\catcode`\$9 $% For editors with annoying syntax highlighting.
}%

\def\pgf@lib@decorations@text@dobox{%
	\setbox\pgf@lib@decorations@text@box\hbox{%
		\pgfinterruptpicture%
		\begingroup%
			\pgfsetcolor{\pgf@lib@decorationtextcolor}%
			\ifpgf@lib@decorate@textmathmode\pgf@lib@decorations@mathshift\fi%
				\pgf@lib@decorations@text@format\relax%
				\pgf@lib@decorations@text@char%
			\ifpgf@lib@decorate@textmathmode\pgf@lib@decorations@mathshift\fi%
		\endgroup%
		\endpgfinterruptpicture%
	}%
}















%
% Old snakes code:
%

% Alias for old snakes:

\let\pgfsnakesegmentamplitude=\pgfdecorationsegmentamplitude
\let\pgfsnakesegmentlength=\pgfdecorationsegmentlength
\def\pgfsnakesegmentangle{\pgfdecorationsegmentangle}
\def\pgfsnakesegmentobjectlength{\pgfkeysvalueof{/pgf/decoration options/shape start width}}
\def\pgfsnakesegmentaspect{\pgfdecorationsegmentaspect}

\pgfset{%
  /pgf/segment amplitude/.style={/pgf/decoration options={amplitude=#1}},
  /pgf/segment length/.style={/pgf/decoration options={segment length=#1}},
  /pgf/segment angle/.style={/pgf/decoration options={angle=#1}},
  /pgf/segment aspect/.style={pgf/decoration options={aspect=#1}},
  /pgf/segment object length/.style={pgf/decoration options={shape width=#1}}}





\endinput