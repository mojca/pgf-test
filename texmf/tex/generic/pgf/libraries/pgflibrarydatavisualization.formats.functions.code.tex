% Copyright 2008 by Till Tantau
%
% This file may be distributed and/or modified
%
% 1. under the LaTeX Project Public License and/or
% 2. under the GNU Public License.
%
% See the file doc/generic/pgf/licenses/LICENSE for more details.

\ProvidesFileRCS[v\pgfversion] $Header: /home/mojca/cron/mojca/github/cvs/pgf/pgf/generic/pgf/libraries/Attic/pgflibrarydatavisualization.formats.functions.code.tex,v 1.2 2008/06/09 15:53:49 tantau Exp $

\usepgfmodule{datavisualization}




% Function evaluator
%
% (This uses TeX evaluation and is slow. *You have been warned*.)
%
% Description:
%
% The data contains the description of variables, followed by
% functions.
%
% The format for the variables is
%
%   var <name> = "["<start>:<end>"]" [ step <steps> | samples <samples>];
%
% (The first two [ and ] are "real", the second pair means "optional",
% see the examples.)
%
% The format for the functions is
%
%   func <name> = <body>;
%
% Additionally, the variables and functions stored in the keys
% /pgf/data visualization/vars and /pgf/data visualization/funcs will
% also be used, the format for them is the same as above.
%
% The <name> is always the name of a (data point) attribute. Inside
% the <body> it can be accessed using the \value macro, which is
% defined locally, here.
%
% Newlines are not importants, everything is gathered into a single
% line, the semicolons are used to detect the beginnings and ends of
% declarations.
%
% With everything setup in this way, for each variable, the variable
% ranges throught the values between start and end, either with the
% given step or the given number of samples. If neither is given, the
% current value of /pgf/data visualization/samples is used. If there
% are multiple variables, they all range over their respective
% intervals independently. Thus, "var x = [0,1]; var y = [0:1];", 
% assuming samples=25, gives you 625 data points.
%
% As the variables range over their respective values, the functions
% are evaluated using the function stored in /pgf/data
% visualization/parser. By default this is set to \pgfmathparse, but
% you can use a different parser, if you like.
%
% Example:
%
% \pgfdatavisualizationrender[format=function]
% \data{
%
%   % This is a parametric plot with the parameter called t.
%   var t = [0:360] samples 60;
%
%   % Let's draw a circle:
%   func x = sin(\value t);
%   func y = cos(\value t);
% }
%
% \pgfdatavisualizationrender[format=function]
% \data{
%
%   % This a surface plot data...
%   var x = [0:1];
%   var y = [0:1];
%
%   % Let's draw a circle:
%   func height = \value x*\value x + \value y*\value y;
% }

\pgfdeclaredataformat{function}
{ % no catcode code
}
{ % Startup
  % Init collectors
  \let\value=\relax% avoid expansion
  \pgfkeysgetvalue{/pgf/data visualization/vars}\pgf@lib@dv@collector
}
{% Line arguments
  #1%
}
{ % Line action
  \edef\pgf@lib@dv@collector{\pgf@lib@dv@collector\space #1}%
}
{% Ignore empty lines
}
{%
  \edef\pgf@lib@dv@collector{\pgf@lib@dv@collector\space \pgfkeysvalueof{/pgf/data visualization/funcs}}%
  \let\pgf@lib@dv@vars\pgfutil@empty%
  \let\pgf@lib@dv@funcs\pgfutil@empty%
  % launch parser on the result
  \expandafter\pgf@lib@dv@func@parse \pgf@lib@dv@collector\pgf@stop%
  \let\value\pgf@lib@dv@getval
  \pgf@lib@dv@vars\pgfdatapoint\pgf@stop%
}

\pgfkeys{
  /pgf/data visualization/vars/.initial=,
  /pgf/data visualization/funcs/.initial=,
  /pgf/data visualization/samples/.initial=25}

\def\pgf@lib@dv@getval#1{\pgfkeysvalueof{/data point/#1}}

\def\pgf@lib@dv@func@parse{%
  \pgfutil@ifnextchar\pgf@stop{% done
    \pgfutil@gobble%
  }
  {%
    \pgfutil@ifnextchar v{% variable
      \pgf@lib@dv@var%
    }{%
      \pgf@lib@dv@func%
    }%
  }%
}

\def\pgf@lib@dv@func func #1=#2;{%
  \expandafter\def\expandafter\pgf@lib@dv@vars\expandafter{\pgf@lib@dv@vars\pgf@lib@dv@eval@func{#1}{#2}}
  \pgf@lib@dv@func@parse%
}
\def\pgf@lib@dv@var var #1=[#2:#3]{%
  \pgfutil@ifnextchar s{%
    \pgf@lib@dv@samplesorsteps{#1}{#2}{#3}%
  }{%
    \expandafter\def\expandafter\pgf@lib@dv@vars\expandafter{%
      \pgf@lib@dv@vars\pgf@lib@dv@samples@var{#1}{#2}{#3}{\pgfkeysvalueof{/pgf/data visualization/samples}}}%
    \expandafter\pgf@lib@dv@func@parse\pgfutil@gobble%
  }
}
\def\pgf@lib@dv@samplesorsteps#1#2#3s{%
  \pgfutil@ifnextchar t{% step
    \pgf@lib@dv@step{#1}{#2}{#3}%
  }{% sample
    \pgf@lib@dv@samples{#1}{#2}{#3}%
  }%
}
\def\pgf@lib@dv@step#1#2#3tep#4;{%
  \expandafter\def\expandafter\pgf@lib@dv@vars\expandafter{\pgf@lib@dv@vars\pgf@lib@dv@step@var{#1}{#2}{#3}{#4}}%
  \pgf@lib@dv@func@parse%
}
\def\pgf@lib@dv@samples#1#2#3amples#4;{%
  \expandafter\def\expandafter\pgf@lib@dv@vars\expandafter{\pgf@lib@dv@vars\pgf@lib@dv@samples@var{#1}{#2}{#3}{#4}}%
  \pgf@lib@dv@func@parse%
}

%
% Executer...
%
\def\pgf@lib@dv@samples@var#1#2#3#4{%
  \pgfmathparse{#3}%
  \let\pgf@lib@dv@temp=\pgfmathresult%
  \pgfmathparse{#2}%
  \pgfmathparse{\pgf@lib@dv@temp-\pgfmathresult}%
  \pgfmathparse{\pgfmathresult/#4}%
  \let\pgf@lib@dv@temp=\pgfmathresult%
  \pgf@lib@dv@step@var{#1}{#2}{#3}{\pgf@lib@dv@temp}%
}

\def\pgf@lib@dv@step@var#1#2#3#4#5\pgf@stop{%
  {%
    \def\pgf@loop@body{#5}%
    \pgfmathparse{#2}%
    \pgfkeyslet{/data point/#1}{\pgfmathresult}%
    \pgfmathparse{#3}%
    \let\pgf@loop@end\pgfmathresult%
    \pgfmathparse{#4}%
    \let\pgf@loop@step\pgfmathresult%
    \pgfmathloop%
      \pgfmathlessthan{\pgf@loop@end}{\pgfkeysvalueof{/data point/#1}}
    \ifdim\pgfmathresult pt<1pt\relax%
      \pgf@loop@body%
      \pgfmathparse{\pgfkeysvalueof{/data point/#1}+\pgf@loop@step}%
      \pgfkeyslet{/data point/#1}{\pgfmathresult}%
    \repeatpgfmathloop%%
  }%
}

\def\pgf@lib@dv@eval@func#1#2{%
  \pgfmathparse{#2}%
  \pgfkeyslet{/data point/#1}{\pgfmathresult}%  
}


\endinput
