% Copyright 2006 by Till Tantau
%
% This file may be distributed and/or modified
%
% 1. under the LaTeX Project Public License and/or
% 2. under the GNU Public License.
%
% See the file doc/generic/pgf/licenses/LICENSE for more details.

\ProvidesFileRCS $Header: /home/mojca/cron/mojca/github/cvs/pgf/pgf/generic/pgf/basiclayer/pgfcoreexternal.code.tex,v 1.16 2010/08/31 10:12:15 ludewich Exp $



\newif\ifpgfexternal@aux@in@dpth
\newif\ifpgfexternal@info

\pgfkeys{
    /pgf/images/include external/.code={\pgfimage{#1}},
    /pgf/images/aux in dpth/.is if=pgfexternal@aux@in@dpth,
    /pgf/images/aux in dpth/.default=true,
    /pgf/images/trim external/.code n args={4}{%
        \def\pgf@external@kern@left{#1}%
        \def\pgf@external@kern@bottom{#2}%
        \def\pgf@external@kern@right{#3}%
        \def\pgf@external@kern@top{#4}%
        \ifx\pgf@external@kern@left\pgfutil@empty\def\pgf@external@kern@left{0sp}\fi%
        \ifx\pgf@external@kern@bottom\pgfutil@empty\def\pgf@external@kern@bottom{0sp}\fi%
        \ifx\pgf@external@kern@right\pgfutil@empty\def\pgf@external@kern@right{0sp}\fi%
        \ifx\pgf@external@kern@top\pgfutil@empty\def\pgf@external@kern@top{0sp}\fi%
    },
    /pgf/images/trim external={1truein}{1truein}{1truein}{1truein},
    /pgf/images/external info/.is if=pgfexternal@info,
    /pgf/images/external info/.default=true,
}

% The external read/write mechanism is used as follows:
%
% In your document, you put a number or all of your pgfpicture (or
% tikzpicture) environments inside a pair of \beginpgfexternal and
% \endpgfexternal. The \beginpgfexternal gets one parameter called the
% graphics name.
%
% Once this is done, the next step is to declare the names of your
% graphics in the document at the beginning using the
% \pgfdeclaregraphicname command.
%
% Now, when you typeset your document, each time a \beginpgfexternal
% command is encountered, it will be checked whether the corresponding
% graphics file exists. If this is the case, it will be included and
% the code between \beginpgfexternal and \endpgfexternal is
% ignored. Otherwise, the code is executed normally.
%
% Things change when you run tex on your file with the \jobname set to
% the name of a decalred graphics name. In this case, only your
% graphic will be typeset, the rest of the document will be "thrown away". 
% To be more precise, everything is gobbled up to the beginning
% of the first \beginpgfexternal with the given graphics name. Then,
% the content of the "environment" is put in a box (this "environment"
% need not contain a graphics, it may contain anything that can be put
% in a box). Then a single page is output whose size is exactly the
% size of the picture.
%
% So, once you have run tex repeatedly, each time with the jobname set
% to each of the graphics, you can then run tex on the whole document
% and this will cause all graphics to be included rather than begin
% typeset.
%
% Finally, once all the graphics have been created, you can also say
% \input pgfexternal.tex instead of including pgf/TikZ.


\newif\ifpgfexternalreadmainaux
\pgfexternalreadmainauxtrue

% Tells pgf which jobname is the name of the real file
%
% #1 = name
%
% Description:
%
% This command is used to tell pgf that the file named #1 should be
% typeset normally. If \jobname is not equal to #1, only
% the graphic called \jobname will be typeset. 
%
% Example:
%
% \pgfrealjobname{survey}
%
% Note that afterwards, \jobname will be set to the argument. In our
% example, \jobname will be 'survey'. This is to accomplish
% compatibility with aux-file generation.

\def\pgfrealjobname#1{%
  \global\let\pgfactualjobname=\jobname
  \edef\pgf@tempa{\expandafter\string\csname #1\endcsname}%
  \edef\pgf@tempb{\expandafter\string\csname\jobname\endcsname}%
  \ifx\pgf@tempa\pgf@tempb%
  \else%
    \pgf@external@grabshipouttrue%
    \pgfexternal@nofiles% see also \pgf@external@init@aux@in@dpth
    \ifpgfexternalreadmainaux
        % and reset the jobname. This should allow to handle any
        % \label/\ref constructions which are stored in \jobname.aux (and
        % which won't be found otherwise)
        \gdef\jobname{#1}%
    \fi
    \ifpgfexternal@aux@in@dpth
        \csname newwrite\endcsname\w@pgfexternal@auxout
    \fi
  \fi%
  \gdef\pgfrealjobname##1{}% avoid multiple calls.
}

\def\pgfexternal@nofiles{%
    % replace \relax. The \nofiles macros does
    % \let\makeglossary=\relax
    % but the glossary.sty calls \renewcommand\makeglossary (which
    % will fail if \makeglossary=\relax). Stupid, but it works.
    \let\pgfexternal@nofiles@=\relax
    \def\relax{\relax}%
    %
    % suppress generation of LaTeX .aux, .toc etc files. 
    % generation of these files is not thread-safe.
    % the \csname \endcsname yields \relax if \nofiles doesn't exist.
    \csname nofiles\endcsname
    %
    \let\relax=\pgfexternal@nofiles@
}%

\newif\ifpgf@external@grabshipout
\newbox\pgfexternal@startupbox

\pgfutil@ifundefined{AtBeginDocument}{}{%
    \AtBeginDocument{%
      \ifpgf@external@grabshipout%
        \global\let\pgfexternal@originalshipout=\shipout%
        \global\def\shipout{\setbox\pgfpic=}%
        \maxdeadcycles=10000%
        % Ok, gather everything we have seen up to now in a box. This box
        % will contain any specials that have been used.
        {%
          \output{\global\setbox\pgfexternal@startupbox=\vbox{\csname @begindvi\endcsname\unvbox255}}%
          \hbox{}\eject%
        }%
      \fi%
    }
}%

\pgfutil@ifundefined{AtEndDocument}{}{%
    \AtEndDocument{%
      \ifpgf@external@grabshipout
        \pgfutil@ifundefined{pgfexternal@did@a@shipout}{%
          \pgfexternal@error@no@shipout
        }{}%
      \fi
    }%
}

% This will be overwritten by the tikz external lib if it is loaded.
\def\pgfexternal@error@no@shipout{%
    \PackageError{pgf}{Sorry, image externalization failed: the resulting image was EMPTY. I tried to externalize '\pgfactualjobname'. Perhaps there was a typo somewhere? Please check that your document contains '\string\beginpgfgraphicnamed{\pgfactualjobname} ... \string\endpgfgraphicnamed'}%
}%


% "Environment" for an external graphic.
%
% #1 = graphic name
%
% Example:
%
% \beginpgfgraphicnamed{main-graphic2}
%   \begin{tikzpicture}
%     ...
%   \end{tikzpicture}
% \endpgfgraphicnamed

\def\pgf@externalbegin#1{%
  % First, check whether we wish to grab this graphic:
  \let\pgf@next=\pgf@external@normal%
  \ifpgf@external@grabshipout%
    \edef\pgf@tempa{\expandafter\string\csname #1\endcsname}%
    \edef\pgf@tempb{\expandafter\string\csname\pgfactualjobname\endcsname}%
    \ifx\pgf@tempa\pgf@tempb%
      \let\pgf@next=\pgf@external@grab%
    \fi%
  \fi%
  \pgf@next{#1}%
}


\AtBeginDocument{
  \let\beginpgfgraphicnamed=\pgf@externalbegin% overwrite definition of pgfexternal.tex if necessary   
  \let\endpgfgraphicnamed=\unskip
}


% Normal operation: Include an external graphic instead of the
% picture, if such an external picture exists.
\def\pgf@external@normal#1{%
  \let\pgf@filename=\pgfutil@empty%
  \expandafter\pgf@findfile\pgfsys@imagesuffixlist:+{#1}%
  \ifx\pgf@filename\pgfutil@empty%
    % Ok, no such image... Just typeset the picture normally.
    \let\pgf@next=\relax%
  \else%
    \let\pgf@next=\pgf@replacepicturebygraphic%
  \fi%
  \pgf@next%  
}

\long\def\pgf@replacepicturebygraphic#1\endpgfgraphicnamed{%
    \expandafter\pgfincludeexternalgraphics\expandafter{\pgf@filename}\unskip
}

% This is almost the same as \includegraphics{#1}, but it checks
% whether '#1.dpth' exists. In such case, it restores the boxes depth
% stored in #1.dpth.
%
% Furthermore, the .dpth file may contain .aux related information
% collected for the external graphics (references). They start after
% the first line (if any).
\def\pgfincludeexternalgraphics#1{%
    \begingroup
    \pgfexternalreaddpth{#1}%
    \pgfkeysifdefined{/pgf/images/include external/#1/.@cmd}{%
        \pgfkeysgetvalue{/pgf/images/include external/#1/.@cmd}\pgf@tempa
        \pgfkeyslet{/pgf/images/include external/.@cmd}\pgf@tempa
    }{}%
    \setbox1=\hbox{\pgfkeysvalueof{/pgf/images/include external/.@cmd}{#1}\pgfeov}%
    \ifdim\pgfexternaltrimleft=0sp \else \kern\pgfexternaltrimleft\relax \fi
    \ifdim\pgfexternaldepth=0pt
        \box1
    \else
        \dimen0=\pgfexternaldepth\relax
        \hbox{\lower\dimen0 \box1 }%
    \fi
    \ifdim\pgfexternaltrimright=0sp \else \kern\pgfexternaltrimright\relax \fi
    \endgroup
}

% Defines \pgfexternaldepth to be the depth of the external picture
% '#1' and \pgfexternaltrimleft / \pgfexternaltrimright to be the 'hbaseline' shift.
%
%
% Furthermore, it handles any auxiliary information stored in the
% .dpth file (it appends it to the main aux file).
%
\def\pgfexternalreaddpth#1{%
    % no \begingroup. Handle that manually:
    \edef\pgfexternalreaddpth@restore{%
        \noexpand\endlinechar=\the\endlinechar\space
        \noexpand\catcode`\noexpand\@=\the\catcode`\@\space
    }%
    %
    \def\pgfexternaldepth{0sp}%
    \def\pgfexternaltrimleft{0sp}%
    \def\pgfexternaltrimright{0sp}%
    \endlinechar=-1 % suppress white space at end
    \catcode`\@=11
    \openin\r@pgf@reada=#1.dpth
    \pgfincludeexternalgraphics@read@dpth
    %
    \pgfexternalreaddpth@restore
}%
%
% The .dpth consists of 0-N lines, where each is either a single
% dimension (the box' depth) or something which belongs to the .aux
% file (such lines will always start with a macro).
\def\pgfincludeexternalgraphics@read@dpth{%
    \ifeof\r@pgf@reada
        \closein\r@pgf@reada
    \else
        \read\r@pgf@reada to\pgfincludeexternalgraphics@auxline
        \ifx\pgfincludeexternalgraphics@auxline\pgfutil@empty
        \else
            \expandafter\pgfincludeexternalgraphics@read@dpth@line\pgfincludeexternalgraphics@auxline\pgfincludeexternalgraphics@read@dpth@line@EOI
        \fi
        \expandafter\pgfincludeexternalgraphics@read@dpth
    \fi
}%

\def\pgfexternal@restore#1{}%

\long\def\pgfincludeexternalgraphics@read@dpth@line@getfirst#1#2\pgfincludeexternalgraphics@read@dpth@line@EOI{%
    \def\pgf@first{#1}%
    \def\pgf@second{#2}%
}%
\long\def\pgfincludeexternalgraphics@read@dpth@line#1\pgfincludeexternalgraphics@read@dpth@line@EOI{%
    \pgfincludeexternalgraphics@read@dpth@line@getfirst#1\pgfincludeexternalgraphics@read@dpth@line@EOI
    \expandafter\ifcat\expandafter\noexpand\pgf@first\relax
        % Ah -- the first token is a control sequence. It belongs to
        % the .aux file.
        %
        \expandafter\ifx\pgf@first\pgfexternal@restore
            \pgf@second%
        \else
            % do NOT execute #1! many LaTeX commands don't support it (\label for example)
            \pgfutil@ifundefined{if@filesw}{%
                % sorry, .aux file support only for latex
            }{%
                % append to main .aux file (for forward references)
                \if@filesw
                    {%
                    \toks0={#1}%
                    % believe it or not, but the
                    % \def\dpthimport{...}\dpthimport *makes* a
                    % difference! In ensures any occuring `##' characters are properly expanded to `#'.
                    \immediate\write\@auxout{%
                        \noexpand\def\noexpand\dpthimport{\the\toks0}\noexpand\dpthimport}%
                    }%
                \fi             
            }%
        \fi
    \else% it is the depth (which is simply a number for backwards compatibility)
        \def\pgfexternaldepth{#1}%
    \fi
}%

% Stores '#1' (expanded!) into the .dpth file of the currently
% exported image. 
%
% This command has only an effect if an image is being exported.
%
% The stored contents will be read in the context where
% /pgf/images/include external is active (i.e. by
% \pgfexternalreaddpth).
%
\def\pgfexternalstorecommand#1{%
}%
\def\pgfexternalstorecommand@isexporting#1{%
    \immediate\write\w@pgfexternal@auxout{\noexpand\pgfexternal@restore{#1}}%
}%



% Grab operation: If jobname matches the graphic name, typeset this
% picture normall.
% REMARK:
%   this method is also invoked from within the tikz external library.
\def\pgf@external@grab#1{%
  \def\pgf@filename{#1}%
  \ifpgfexternal@aux@in@dpth    
    \begingroup
    \pgf@external@init@aux@in@dpth
    \let\G@refundefinedtrue=\pgf@external@grab@refundefinedtrue
  \fi
  \gdef\pgf@trimleft@final{0sp}%
  \gdef\pgf@trimright@final{0sp}%
  \setbox\pgfpic=\hbox\bgroup\bgroup%
    \let\endpgfgraphicnamed=\pgf@externalend%
}

{\catcode`\^=12 \gdef\pgfexternal@hat{^}}

\expandafter\let\expandafter\pgf@external@grab@refundefinedtrue@orig\csname G@refundefinedtrue\endcsname

\def\pgf@external@grab@refundefinedtrue{%
    \pgf@external@grab@refundefinedtrue@orig
    \begingroup
    \def\n{\pgfexternal@hat\pgfexternal@hat J}%
    \pgfexternalstorecommand{%
      \noexpand\immediate\noexpand\write16{\pgf@external@grab@refundefinedtrue@warning}%
      \noexpand\G@refundefinedtrue
    }%
    \endgroup
}%
\def\pgf@external@grab@refundefinedtrue@warning{%
    LaTeX Warning: External picture `\pgfactualjobname' contains undefined references\noexpand\on@line.\n
}%

\let\pgf@external@@protected@write@orig=\protected@write
\long\def\pgf@external@@protected@write@immediate#1#2#3{%
  \begingroup
    \let\pgf@write@=\write
    \def\write{\noexpand\immediate\pgf@write@}%
    \pgf@external@@protected@write@orig{#1}{#2}{#3}%
  \endgroup
}%

\def\pgf@external@init@aux@in@dpth{%
  \let\pgfexternalstorecommand=\pgfexternalstorecommand@isexporting
  %
  % tell LaTeX to write aux files...
  \csname @fileswtrue\endcsname
  % ... but redirect output to the .dpth file!
  \immediate\openout\w@pgfexternal@auxout=\pgf@filename.dpth
  \let\@auxout=\w@pgfexternal@auxout
  % ... and disable the correct page numbers. I can't get that
  % (because the correct page number is only available in the
  % shipout routine). Use immediate output:
  \let\protected@write=\pgf@external@@protected@write@immediate
}%

\def\pgf@externalend@storeshifts#1{%
  \immediate\write#1{\the\dp\pgfpic}% store the depth simply as number for backw. compat.
  \ifx\pgf@trimleft@final\pgfutil@empty\else
    \immediate\write#1{\noexpand\pgfexternal@restore{\noexpand\def\noexpand\pgfexternaltrimleft{\pgf@trimleft@final}}}%
  \fi
  \ifx\pgf@trimright@final\pgfutil@empty\else
    \immediate\write#1{\noexpand\pgfexternal@restore{\noexpand\def\noexpand\pgfexternaltrimright{\pgf@trimright@final}}}%
  \fi
  \ifpgfexternal@info
    \immediate\write#1{\noexpand\pgfexternal@restore{\noexpand\def\noexpand\pgfexternalwidth{\the\wd\pgfpic}}}%
    \immediate\write#1{\noexpand\pgfexternal@restore{\noexpand\def\noexpand\pgfexternalheight{\the\dimen0}}}%
  \fi
}
% REMARK:
%   this method is also invoked from within the tikz external library.
\def\pgf@externalend{%
  \unskip\egroup\egroup%
  {%
    \def\pgf@external@trim{0}%
    \def\pgf@external@store@dpth{0}%
    %
    \ifpgf@trim@lowlevel
      \gdef\pgf@trimleft@final{}%
      \gdef\pgf@trimright@final{}%
    \else
      \ifdim\pgf@trimleft@final=0sp  \gdef\pgf@trimleft@final{}  \else\def\pgf@external@trim{1}\fi
      \ifdim\pgf@trimright@final=0sp \gdef\pgf@trimright@final{} \else\def\pgf@external@trim{1}\fi
      \if1\pgf@external@trim
          % UNDO the trimming! export to pdf doesn't supported trimmed
          % bounding boxes (has to do with the mediabox/trimbox etc).
          % I'll keep the bounding box intact and store the trim information 
          % into the .dpth file.
          \setbox\pgfpic=\hbox{%
              \ifx\pgf@trimleft@final\pgfutil@empty\else\kern-\pgf@trimleft@final\fi
              \box\pgfpic
              \ifx\pgf@trimright@final\pgfutil@empty\else\kern-\pgf@trimright@final\fi
          }%
          \def\pgf@external@store@dpth{1}%
      \fi
    \fi
    %
    \parindent0pt % leave the space
%    \leftmargin0pt%
%    \rightmargin0pt%
    \hoffset=0pt % reset to default
    \voffset=0pt
    \dimen0\ht\pgfpic%
    \advance\dimen0\dp\pgfpic%
    \ifdim\dp\pgfpic=0pt\relax
    \else% store the picture's depth. Otherwise, it would be lost.
      \def\pgf@external@store@dpth{1}%
    \fi
    \ifpgfexternal@info
      \def\pgf@external@store@dpth{1}%
    \fi
    %
    \if1\pgf@external@store@dpth
      \ifpgfexternal@aux@in@dpth
        \pgf@externalend@storeshifts\@auxout
        \immediate\closeout\@auxout
      \else
        \immediate\openout\pgf@plotwrite=\pgf@filename.dpth
        \pgf@externalend@storeshifts\pgf@plotwrite
        \immediate\closeout\pgf@plotwrite
      \fi
    \fi
    \pgfsys@papersize{\the\wd\pgfpic}{\the\dimen0}%
    %
    \setbox0=\vbox{%
      % I admit that I don't know why there is this kerning:
      \kern -\pgf@external@kern@top\relax%
      \hbox{%
        \kern -\pgf@external@kern@left\relax%
        \hbox to0pt{%
          \wd\pgfexternal@startupbox=0pt %
          \ht\pgfexternal@startupbox=0pt %
          \dp\pgfexternal@startupbox=0pt %
          \box\pgfexternal@startupbox%
          \pgfsys@atbegindocument\hss}%
        \box\pgfpic%
        \kern\pgf@external@kern@right\relax%
      }%
      \kern\pgf@external@kern@bottom\relax%
    }%
    %
    % compatibility with eso-pic package: \nointerlineskip is not
    % allowed here, but the eso-pic package uses \nointerlineskip in
    % its shipout routine.
    \let\pgfexternal@nointerlineskip@orig=\nointerlineskip
    \let\nointerlineskip=\pgfexternal@nointerlineskip
    %
    \pgfexternal@originalshipout\box0 %
    \gdef\pgfexternal@did@a@shipout{1}%
  }%
  \ifpgfexternal@aux@in@dpth
    \endgroup
  \fi
}

\def\pgfexternal@nointerlineskip{%
    \ifvmode
      \pgfexternal@nointerlineskip@orig
    \else
      \relax% \nointerlineskip in horizontal mode not allowed. Silently ignore the error message.
    \fi
}%

\endinput
% vi: tabstop=4 shiftwidth=2 expandtab
