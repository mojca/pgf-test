% Copyright 2006 by Till Tantau
%
% This file may be distributed and/or modified
%
% 1. under the LaTeX Project Public License and/or
% 2. under the GNU Public License.
%
% See the file doc/generic/pgf/licenses/LICENSE for more details.

\ProvidesFileRCS $Header: /home/mojca/cron/mojca/github/cvs/pgf/pgf/generic/pgf/basiclayer/pgfcorepathconstruct.code.tex,v 1.20 2009/01/21 13:10:20 tantau Exp $


\newdimen\pgf@path@lastx
\newdimen\pgf@path@lasty

\let\pgfgetpath=\pgfsyssoftpath@getcurrentpath
\let\pgfsetpath=\pgfsyssoftpath@setcurrentpath

% Replace corners by arcs.
%
% #1 = in-size of arc
% #2 = out-size of arc
% 
% Description:
% 
% This command influences path construction command like
% \pgfpathlineto or \pgfpatharc. It will cause the corners at the end
% of these commands to be replaced by little arcs. If the
% corner is a 90 degrees corner and if #1=#2, a quarter-circle of
% radius #1 is put in place of the corner. If #1 and #2 are different,
% the quarter circle will instead by a quarter ellipse. If the angle
% is different from 90 degrees, a deformed quarter circle will
% result, which may or may not be desirable. For a ``perfect'' arc you
% must use the \pgfpatharc command. 
% 
% 
% Example: One rounded corner.
%
% \pgfpathmoveto{\pgfpointxy{0}{0}}
% \pgfsetcornersarced{4pt}{4pt}
% \pgfpathlineto{\pgfpointxy{0}{1}}
% \pgfpathlineto{\pgfpointxy{1}{1}}
% \pgfstroke
%
% Example: A rounded rectangle
%
% \pgfsetcornersarced{4pt}{4pt}
% \pgfpathrectangle{\pgfpointorigin}{\pgfpoint{1cm}{1cm}}
% \pgfstroke
%
% Example: A rounded triangles
%
% \pgfsetcornersarced{4pt}{4pt}
% \pgfpathmoveto{\pgfpointorigin}
% \pgfpathlineto{\pgfpoint{1cm}{0cm}}
% \pgfpathlineto{\pgfpoint{1cm}{1cm}}
% \pgfpathclose
% \pgfstroke

\newif\ifpgf@arccorners

\def\pgfsetcornersarced#1{%
  \pgf@process{#1}%
  \edef\pgf@corner@arc{{\the\pgf@x}{\the\pgf@y}}%
  \pgf@arccornerstrue%
  \ifdim\pgf@x=0pt%
    \ifdim\pgf@y=0pt\relax%
      \pgf@arccornersfalse%
    \fi%
  \fi%
}

\def\pgf@roundcornerifneeded{%
  \ifpgf@arccorners\expandafter\pgfsyssoftpath@specialround\pgf@corner@arc\fi%
}


% Move current point to #1.
%
% #1 = new current point
% 
% Example:
%
% \pgfpathmoveto{\pgfxy(0,0)}
% \pgfpathlineto{\pgfxy(0,1)}
% \pgfstroke

\def\pgfpathmoveto#1{%
  \pgfpointtransformed{#1}%
  \pgf@protocolsizes{\pgf@x}{\pgf@y}%
  \pgfsyssoftpath@moveto{\the\pgf@x}{\the\pgf@y}%
  \global\pgf@path@lastx=\pgf@x%
  \global\pgf@path@lasty=\pgf@y%
}

\def\pgf@protocolsizes#1#2{%
  \ifpgf@relevantforpicturesize%
    \ifdim#1<\pgf@picminx\global\pgf@picminx#1\fi%
    \ifdim#1>\pgf@picmaxx\global\pgf@picmaxx#1\fi%
    \ifdim#2<\pgf@picminy\global\pgf@picminy#2\fi%
    \ifdim#2>\pgf@picmaxy\global\pgf@picmaxy#2\fi%
    \ifpgf@size@hooked%
      \let\pgf@size@hook@x#1\let\pgf@size@hook@y#2\pgf@path@size@hook%
    \fi%
  \fi%
  \ifdim#1<\pgf@pathminx\global\pgf@pathminx#1\fi%
  \ifdim#1>\pgf@pathmaxx\global\pgf@pathmaxx#1\fi%
  \ifdim#2<\pgf@pathminy\global\pgf@pathminy#2\fi%
  \ifdim#2>\pgf@pathmaxy\global\pgf@pathmaxy#2\fi%
}
\newif\ifpgf@size@hooked
\let\pgf@path@size@hook=\pgfutil@empty%

\def\pgf@resetpathsizes{%
  \global\pgf@pathmaxx=-16000pt\relax%
  \global\pgf@pathminx=16000pt\relax%
  \global\pgf@pathmaxy=-16000pt\relax%
  \global\pgf@pathminy=16000pt\relax%
}

\def\pgf@getpathsizes#1{%
  \edef#1{{\the\pgf@pathmaxx}{\the\pgf@pathminx}{\the\pgf@pathmaxy}{\the\pgf@pathminy}}%
}
\def\pgf@setpathsizes#1{%
  \expandafter\pgf@@setpathsizes#1%
}
\def\pgf@@setpathsizes#1#2#3#4{%
  \global\pgf@pathmaxx=#1\relax%
  \global\pgf@pathminx=#2\relax%
  \global\pgf@pathmaxy=#3\relax%
  \global\pgf@pathminy=#4\relax%
}





% Append a line from the current point to #1 to the current path.
%
% #1 = end of line
% 
% Example:
%
% \pgfpathmoveto{\pgfxy(0,0)}
% \pgfpathlineto{\pgfxy(0,1)}
% \pgfstroke

\def\pgfpathlineto#1{%
  \pgfpointtransformed{#1}%
  \pgf@protocolsizes{\pgf@x}{\pgf@y}%
  \pgf@roundcornerifneeded%
  \pgfsyssoftpath@lineto{\the\pgf@x}{\the\pgf@y}%
  \global\pgf@path@lastx=\pgf@x%
  \global\pgf@path@lasty=\pgf@y%
}



% Close the current path.
%
% Example:
%
% % Draws two triangles
% \pgfpathmoveto{\pgfxy(0,0)}
% \pgfpathlineto{\pgfxy(0,1)}
% \pgfpathlineto{\pgfxy(1,0)}
% \pgfclosepath
% \pgfpathmoveto{\pgfxy(2,0)}
% \pgfpathlineto{\pgfxy(2,1)}
% \pgfpathlineto{\pgfxy(3,0)}
% \pgfpathclose
% \pgfstroke

\def\pgfpathclose{%
  \pgf@roundcornerifneeded%
  \pgfsyssoftpath@closepath%
}


% Append a cubic bezier spline from the current point to #3 with control
% points #1 and #2 to the current path.
%
% #1 = first control point
% #2 = second control point
% #3 = end point
% 
% Example:
%
% \pgfpathmoveto{\pgfpointxy{0}{0}}
% \pgfpathcurveto{\pgfpointxy{0}{1}}{\pgfpointxy{1}{1}}{\pgfpointxy{1}{2}}
% \pgfstroke

\def\pgfpathcurveto#1#2#3{%
  \pgfpointtransformed{#3}%
  \pgf@protocolsizes{\pgf@x}{\pgf@y}%
  \pgf@xb=\pgf@x%
  \pgf@yb=\pgf@y%
  \pgfpointtransformed{#2}%
  \pgf@protocolsizes{\pgf@x}{\pgf@y}%
  \pgf@xa=\pgf@x%
  \pgf@ya=\pgf@y%
  \pgfpointtransformed{#1}%
  \pgf@protocolsizes{\pgf@x}{\pgf@y}%
  \pgf@roundcornerifneeded%
  \pgfsyssoftpath@curveto{\the\pgf@x}{\the\pgf@y}{\the\pgf@xa}{\the\pgf@ya}{\the\pgf@xb}{\the\pgf@yb}%
  \global\pgf@path@lastx=\pgf@xb%
  \global\pgf@path@lasty=\pgf@yb%
}

% Append a quadratic bezier spline from the current point to #2 with
% control point #1 to the current path.
%
% #1 = control point
% #2 = end point
% 
% Example:
%
% \pgfpathmoveto{\pgfpointxy{0}{0}}
% \pgfpathquadraticcurveto{\pgfpointxy{1}{1}}{\pgfpointxy{2}{0}}
% \pgfstroke

\def\pgfpathquadraticcurveto#1#2{%
  \pgfpointtransformed{#2}%
  \pgf@protocolsizes{\pgf@x}{\pgf@y}%
  \pgf@xb=\pgf@x%
  \pgf@yb=\pgf@y%
  \pgfpointtransformed{#1}%
  \pgf@xc=.6666666\pgf@x%
  \pgf@yc=.6666666\pgf@y%
  % compute second control point:
  \pgf@xa=.33333333\pgf@xb%
  \pgf@ya=.33333333\pgf@yb%
  \advance\pgf@xa by\pgf@xc%
  \advance\pgf@ya by\pgf@yc%
  \pgf@protocolsizes{\pgf@xa}{\pgf@ya}%
  % compute first control point
  \advance\pgf@xc by.3333333\pgf@path@lastx%
  \advance\pgf@yc by.3333333\pgf@path@lasty%
  \pgf@protocolsizes{\pgf@xc}{\pgf@yc}%
  \pgf@roundcornerifneeded%
  \pgfsyssoftpath@curveto{\the\pgf@xc}{\the\pgf@yc}{\the\pgf@xa}{\the\pgf@ya}{\the\pgf@xb}{\the\pgf@yb}%
  \global\pgf@path@lastx=\pgf@xb%
  \global\pgf@path@lasty=\pgf@yb%
}


% Append an arc to the current point, where the current point is at
% angle #1 and the end is at angle #2. If #2 > #1, the arc is drawn
% counter-clockwise, otherwise it is clockwise.
%
% #1 = angle of first point
% #2 = angle of second point
% #3 = radius or x-radius/y-radius
% 
% Example:
%
% \pgfpathmoveto{\pgfxy(0,0)}
% \pgfpatharc{0}{90}{2cm}
% \pgfstroke

\def\pgfpatharc#1#2#3{%
  {%
    \pgfmathparse{#1}\let\pgf@temp@a=\pgfmathresult%
    \pgfmathparse{#2}\let\pgf@temp@b=\pgfmathresult%
    \pgfutil@in@{and }{#3}%
    \ifpgfutil@in@%
      \pgf@arc@get@radii#3\pgf@arc@stop%
    \else
      \pgf@arc@get@radii#3and #3\pgf@arc@stop%
    \fi%
    \pgf@arc@local@angle@a=\pgf@temp@a pt%
    \pgf@arc@local@angle@b=\pgf@temp@b pt%
%     \relax%
%     \ifdim\pgf@arc@local@angle@a>360pt\relax%
%       \advance\pgf@arc@local@angle@a by-360pt\relax%
%     \fi%
%     \ifdim\pgf@arc@local@angle@a<-360pt\relax%
%       \advance\pgf@arc@local@angle@a by360pt\relax%
%     \fi%
%     \ifdim\pgf@arc@local@angle@b>360pt\relax%
%       \advance\pgf@arc@local@angle@b by-360pt\relax%
%     \fi%
%     \ifdim\pgf@arc@local@angle@b<-360pt\relax%
%       \advance\pgf@arc@local@angle@b by360pt\relax%
%     \fi%
    \loop%
      \pgfutil@tempdima=\pgf@arc@local@angle@a%
      \advance\pgfutil@tempdima by-\pgf@arc@local@angle@b\relax%
      \ifdim\pgfutil@tempdima<0pt\relax%
        \pgfutil@tempdima=-\pgfutil@tempdima\relax%
      \fi%
    \ifdim\pgfutil@tempdima>90pt\relax%
      \ifdim\pgfutil@tempdima>115pt\relax%
        \pgf@arc@temp=90pt% big skip
      \else%
        \pgf@arc@temp=60pt% smaller skip to ensure wide segments
                          % (important shortened end segments because
                          % of arrow tips)
      \fi%
      \ifdim\pgf@arc@local@angle@b>\pgf@arc@local@angle@a\relax%
        {%
          \pgf@arc@local@angle@b=\pgf@arc@local@angle@a\relax%
          \advance\pgf@arc@local@angle@b by\pgf@arc@temp\relax%
          \pgf@arc%
        }
        \advance\pgf@arc@local@angle@a by\pgf@arc@temp\relax%
      \else
        {%
          \pgf@arc@local@angle@b=\pgf@arc@local@angle@a\relax%
          \advance\pgf@arc@local@angle@b by-\pgf@arc@temp\relax%
          \pgf@arc%
        }%
        \advance\pgf@arc@local@angle@a by-\pgf@arc@temp\relax%
      \fi%
    \repeat%
    \pgf@roundcornerifneeded%
    \pgf@arc%
  }%
}
\dimendef\pgf@arc@local@angle@a=0
\dimendef\pgf@arc@local@angle@b=1
\dimendef\pgf@arc@temp=2

\def\pgf@arc@get@radii#1and #2\pgf@arc@stop{%
  \pgfmathparse{#1}\let\pgf@arc@radius@a=\pgfmathresult%
  \pgfmathparse{#2}\let\pgf@arc@radius@b=\pgfmathresult%
}


\def\pgf@arc{%
  {%
  % \pgfmathsetlength{\pgfutil@tempdima}{#3}%
  % \pgfmathsetlength{\pgfutil@tempdimb}{#4}%
  \pgfutil@tempdima=\pgf@arc@radius@a pt%
  \pgfutil@tempdimb=\pgf@arc@radius@b pt%
  %
  \pgf@xa=\pgf@arc@local@angle@a\relax% 
  \pgf@xb=\pgf@arc@local@angle@b\relax%
  % \pgfutil@tempcnta=#1\relax%
  %    \pgfutil@tempcntb=#2\relax%
  \advance\pgf@xb by-\pgf@xa\relax%
  \ifdim\pgf@xb<0pt\relax%
    \pgf@xb=-\pgf@xb\relax%
  \fi%
  \ifdim\pgf@xb>85pt\relax% hackery to correct the control points
    \pgfutil@tempdima=0.05522847\pgfutil@tempdima\relax%
    \pgfutil@tempdimb=0.05522847\pgfutil@tempdimb\relax%
  \else%
    \ifdim\pgf@xb>75pt\relax%  
      \pgfutil@tempdima=0.0546\pgfutil@tempdima\relax%
      \pgfutil@tempdimb=0.0546\pgfutil@tempdimb\relax%
    \else%
      \ifdim\pgf@xb>60pt\relax%  
        \pgfutil@tempdima=0.054025\pgfutil@tempdima\relax%
        \pgfutil@tempdimb=0.054025\pgfutil@tempdimb\relax%
      \else%
        \pgfutil@tempdima=0.054\pgfutil@tempdima\relax%
        \pgfutil@tempdimb=0.054\pgfutil@tempdimb\relax%
      \fi%
    \fi%
  \fi
  \pgfutil@tempdima=\pgf@sys@tonumber{\pgf@xb}\pgfutil@tempdima\relax%
  \divide\pgfutil@tempdima by 9\relax%
  \pgfutil@tempdimb=\pgf@sys@tonumber{\pgf@xb}\pgfutil@tempdimb\relax%
  \divide\pgfutil@tempdimb by 9\relax%
  %.. controls +(\pgf@xa+90:\pgfutil@tempdima) and +(\pgf@xb-90:\pgfutil@tempdima) .. +(-(#1:#3)+(#2:#3))%
  % store first support vector in xa/ya:
  \pgf@xa=\pgf@arc@local@angle@a\relax%
  \ifdim\pgf@arc@local@angle@b>\pgf@arc@local@angle@a\relax%
    \advance\pgf@xa by 90pt\relax%
  \else%
    \advance\pgf@xa by -90pt\relax%
  \fi%
  \edef\pgf@arc@angle{\pgf@sys@tonumber{\pgf@xa}}%  
  \pgfpointtransformed{\pgfpointpolar{\pgf@arc@angle}{\pgfutil@tempdima and \pgfutil@tempdimb}}%
  \advance\pgf@x by-\pgf@pt@x%
  \advance\pgf@y by-\pgf@pt@y%
  \pgf@xa=\pgf@path@lastx%
  \pgf@ya=\pgf@path@lasty%
  \advance\pgf@xa by \pgf@x%
  \advance\pgf@ya by \pgf@y%
  % store target in xb/yb:
  \pgfpointtransformed{\pgfpointpolar{\pgf@sys@tonumber{\pgf@arc@local@angle@a}}{\pgf@arc@radius@a pt and \pgf@arc@radius@b pt}}%
  \pgf@xb=\pgf@path@lastx%
  \pgf@yb=\pgf@path@lasty%
  \advance\pgf@xb by -\pgf@x%
  \advance\pgf@yb by -\pgf@y%
  \pgfpointtransformed{\pgfpointpolar{\pgf@sys@tonumber{\pgf@arc@local@angle@b}}{\pgf@arc@radius@a pt and \pgf@arc@radius@b pt}}%
  \advance\pgf@xb by \pgf@x%
  \advance\pgf@yb by \pgf@y%
  % store second support xc/yc:
  \ifdim\pgf@arc@local@angle@b>\pgf@arc@local@angle@a\relax%
    \advance\pgf@arc@local@angle@b by -90pt\relax%
  \else%
    \advance\pgf@arc@local@angle@b by 90pt\relax%
  \fi%
  \pgfpointtransformed{\pgfpointpolar{\pgf@sys@tonumber{\pgf@arc@local@angle@b}}{\pgfutil@tempdima and \pgfutil@tempdimb}}%
  \advance\pgf@x by-\pgf@pt@x%
  \advance\pgf@y by-\pgf@pt@y%
  \pgf@xc=\pgf@xb\relax%
  \pgf@yc=\pgf@yb\relax%
  \advance \pgf@xc by \pgf@x\relax%
  \advance \pgf@yc by \pgf@y\relax%
  \pgfsyssoftpath@curveto{\the\pgf@xa}{\the\pgf@ya}{\the\pgf@xc}{\the\pgf@yc}{\the\pgf@xb}{\the\pgf@yb}%
  \global\pgf@path@lastx=\pgf@xb%
  \global\pgf@path@lasty=\pgf@yb%
  \pgf@protocolsizes{\pgf@xa}{\pgf@ya}%
  \pgf@protocolsizes{\pgf@xb}{\pgf@yb}%
  \pgf@protocolsizes{\pgf@xc}{\pgf@yc}%
  }%
}

% Append an arc to the current point, where the arc is on an ellipse
% given by two axis vectors.
%
% #1 = angle of first point
% #2 = angle of second point
% #3 = first axis
% #4 = second axis
% 
% Example:
%
% \pgfpathmoveto{\pgfxy(0,0)}
% \pgfpatharcaxes{0}{90}{\pgfpointxy{2}{0}}{\pgfpointxy{0}{2}}
% \pgfstroke

\def\pgfpatharcaxes#1#2#3#4{%
  {%
    \pgftransformtriangle{\pgfpointorigin}{#3}{#4}%
    \pgfpatharc{#1}{#2}{1pt}%
  }%
}




% Append  an ellipse to the current path.
%
% #1 = center
% #2 = first axis
% #3 = second axis
%  
% Example:
%
% % Add a circle of radius 3cm around the origin
% \pgfpathellipse{\pgforigin}{\pgfxy(2,0)}{\pgfxy(0,1)}
%
% % Draw a non-filled circle of radius 1cm around the point (1,1)
% \pgfpathellipse{\pgfxy(1,1)}{\pgfxy(1,1)}{\pgfxy(-2,2)}
% \pgfstroke

\def\pgfpathellipse#1#2#3{%
  \pgfpointtransformed{#1}% store center in xc/yc
  \pgf@xc=\pgf@x%
  \pgf@yc=\pgf@y%
  \pgfpointtransformed{#2}%
  \pgf@xa=\pgf@x% store first axis in xa/ya
  \pgf@ya=\pgf@y%
  \advance\pgf@xa by-\pgf@pt@x%
  \advance\pgf@ya by-\pgf@pt@y%
  \pgfpointtransformed{#3}%
  \pgf@xb=\pgf@x% store second axis in xb/yb
  \pgf@yb=\pgf@y%
  \advance\pgf@xb by-\pgf@pt@x%
  \advance\pgf@yb by-\pgf@pt@y%
  {%
    \advance\pgf@xa by\pgf@xc%
    \advance\pgf@ya by\pgf@yc%
    \pgfsyssoftpath@moveto{\the\pgf@xa}{\the\pgf@ya}%
    \pgf@protocolsizes{\pgf@xa}{\pgf@ya}%
  }%
  \pgf@x=0.5522847\pgf@xb% first arc
  \pgf@y=0.5522847\pgf@yb%
  \advance\pgf@x by\pgf@xa%
  \advance\pgf@y by\pgf@ya%
  \advance\pgf@x by\pgf@xc%
  \advance\pgf@y by\pgf@yc%
  \edef\pgf@temp{{\the\pgf@x}{\the\pgf@y}}%
  \pgf@protocolsizes{\pgf@x}{\pgf@y}%
  \pgf@x=0.5522847\pgf@xa%
  \pgf@y=0.5522847\pgf@ya%
  \advance\pgf@x by\pgf@xb%
  \advance\pgf@y by\pgf@yb%
  {%
    \advance\pgf@x by\pgf@xc%
    \advance\pgf@y by\pgf@yc%
    \advance\pgf@xb by\pgf@xc%
    \advance\pgf@yb by\pgf@yc%    
    \expandafter\pgfsyssoftpath@curveto\pgf@temp{\the\pgf@x}{\the\pgf@y}{\the\pgf@xb}{\the\pgf@yb}%
    \pgf@protocolsizes{\pgf@x}{\pgf@y}%
    \pgf@protocolsizes{\pgf@xb}{\pgf@yb}%
  }%
  \pgf@xa=-\pgf@xa% flip first axis
  \pgf@ya=-\pgf@ya%
  \pgf@x=0.5522847\pgf@xa% second arc
  \pgf@y=0.5522847\pgf@ya%
  \advance\pgf@x by\pgf@xb%
  \advance\pgf@y by\pgf@yb%
  \advance\pgf@x by\pgf@xc%
  \advance\pgf@y by\pgf@yc%
  \edef\pgf@temp{{\the\pgf@x}{\the\pgf@y}}%
  \pgf@protocolsizes{\pgf@x}{\pgf@y}%
  \pgf@x=0.5522847\pgf@xb%
  \pgf@y=0.5522847\pgf@yb%
  \advance\pgf@x by\pgf@xa%
  \advance\pgf@y by\pgf@ya%
  {%
    \advance\pgf@x by\pgf@xc%
    \advance\pgf@y by\pgf@yc%
    \advance\pgf@xa by\pgf@xc%
    \advance\pgf@ya by\pgf@yc%    
    \expandafter\pgfsyssoftpath@curveto\pgf@temp{\the\pgf@x}{\the\pgf@y}{\the\pgf@xa}{\the\pgf@ya}%  
    \pgf@protocolsizes{\pgf@x}{\pgf@y}%
    \pgf@protocolsizes{\pgf@xa}{\pgf@ya}%
  }%
  \pgf@xb=-\pgf@xb% flip second axis
  \pgf@yb=-\pgf@yb%
  \pgf@x=0.5522847\pgf@xb% third arc
  \pgf@y=0.5522847\pgf@yb%
  \advance\pgf@x by\pgf@xa%
  \advance\pgf@y by\pgf@ya%
  \advance\pgf@x by\pgf@xc%
  \advance\pgf@y by\pgf@yc%
  \edef\pgf@temp{{\the\pgf@x}{\the\pgf@y}}%
  \pgf@protocolsizes{\pgf@x}{\pgf@y}%
  \pgf@x=0.5522847\pgf@xa%
  \pgf@y=0.5522847\pgf@ya%
  \advance\pgf@x by\pgf@xb%
  \advance\pgf@y by\pgf@yb%
  {%
    \advance\pgf@x by\pgf@xc%
    \advance\pgf@y by\pgf@yc%
    \advance\pgf@xb by\pgf@xc%
    \advance\pgf@yb by\pgf@yc%    
    \expandafter\pgfsyssoftpath@curveto\pgf@temp{\the\pgf@x}{\the\pgf@y}{\the\pgf@xb}{\the\pgf@yb}%  
    \pgf@protocolsizes{\pgf@x}{\pgf@y}%
    \pgf@protocolsizes{\pgf@xb}{\pgf@yb}%
  }%
  \pgf@xa=-\pgf@xa% flip first axis once more
  \pgf@ya=-\pgf@ya%
  \pgf@x=0.5522847\pgf@xa% fourth arc
  \pgf@y=0.5522847\pgf@ya%
  \advance\pgf@x by\pgf@xb%
  \advance\pgf@y by\pgf@yb%
  \advance\pgf@x by\pgf@xc%
  \advance\pgf@y by\pgf@yc%
  \edef\pgf@temp{{\the\pgf@x}{\the\pgf@y}}%
  \pgf@protocolsizes{\pgf@x}{\pgf@y}%
  \pgf@x=0.5522847\pgf@xb%
  \pgf@y=0.5522847\pgf@yb%
  \advance\pgf@x by\pgf@xa%
  \advance\pgf@y by\pgf@ya%
  {%
    \advance\pgf@x by\pgf@xc%
    \advance\pgf@y by\pgf@yc%
    \advance\pgf@xa by\pgf@xc%
    \advance\pgf@ya by\pgf@yc%    
    \expandafter\pgfsyssoftpath@curveto\pgf@temp{\the\pgf@x}{\the\pgf@y}{\the\pgf@xa}{\the\pgf@ya}%
    \pgf@protocolsizes{\pgf@x}{\pgf@y}%
    \pgf@protocolsizes{\pgf@xa}{\pgf@ya}%
  }%
  \pgfsyssoftpath@closepath%
  \pgfsyssoftpath@moveto{\the\pgf@xc}{\the\pgf@yc}%
}



% Append a circle to the current path
%
% #1 = center
% #2 = radius
%  
% Example:
%
% % Append a circle of radius 3cm around the the point (1,1)
% \pgfpathcircle{\pgxy(1,1)}{3cm}

\def\pgfpathcircle#1#2{\pgfpathellipse{#1}{\pgfpoint{#2}{0pt}}{\pgfpoint{0pt}{#2}}}




% Append a rectangle to the current path
%
% #1 = lower left corner point of rectangle
% #2 = width and height vector
%  
% Example:
%
% % A rectangle with corners (2,2) and (3,3)
% \pgfpathrectangle{\pgfpointxy{2}{2}}{\pgfpointxy{1}{1}}

\def\pgfpathrectangle{%
  \let\pgfrect@next=\pgf@specialrect%
  \ifpgf@pt@identity%
    \ifpgf@arccorners%
    \else%
      \let\pgfrect@next=\pgf@normalrect%
    \fi%
  \fi%
  \pgfrect@next%
}

\def\pgf@normalrect#1#2{%
  \pgf@process{#2}%
  \pgf@xa=\pgf@x%
  \pgf@ya=\pgf@y%
  \pgfpointtransformed{#1}%
  \pgfsyssoftpath@rect{\the\pgf@x}{\the\pgf@y}{\the\pgf@xa}{\the\pgf@ya}%
  \pgf@protocolsizes{\pgf@x}{\pgf@y}%
  \advance\pgf@x by\pgf@xa\relax%
  \advance\pgf@y by\pgf@ya\relax%
  \pgf@protocolsizes{\pgf@x}{\pgf@y}%
}

\def\pgf@specialrect#1#2{%
  \pgf@process{#2}%
  \pgf@xa=\pgf@x%
  \pgf@ya=\pgf@y%
  \pgf@process{#1}%
  \pgf@xb=\pgf@x%
  \pgf@yb=\pgf@y%
  \advance\pgf@xa by\pgf@xb%
  \advance\pgf@ya by\pgf@yb%
  \pgfpathmoveto{\pgfqpoint{\pgf@xa}{\pgf@ya}}%
  \pgfpathlineto{\pgfqpoint{\pgf@xb}{\pgf@ya}}%
  \pgfpathlineto{\pgfqpoint{\pgf@xb}{\pgf@yb}}%
  \pgfpathlineto{\pgfqpoint{\pgf@xa}{\pgf@yb}}%
  \pgfpathclose%
  \pgfpathmoveto{\pgfqpoint{\pgf@xb}{\pgf@yb}}%
}

% Append a rectangle to the current path
%
% #1 = one corner of the rectangle
% #2 = opposite corner of the rectangle
%  
% Example:
%
% % A rectangle with corners (2,2) and (3,3)
% \pgfpathrectanglecorners{\pgfpointxy{2}{2}}{\pgfpointxy{3}{3}}

\def\pgfpathrectanglecorners#1#2{%
  \pgf@process{#2}%
  \pgf@xc=\pgf@x%
  \pgf@yc=\pgf@y%
  \pgf@process{#1}%
  \advance\pgf@xc by-\pgf@x%
  \advance\pgf@yc by-\pgf@y%
  \pgfpathrectangle{#1}{\pgfqpoint{\pgf@xc}{\pgf@yc}}%
}


% Append a grid to the current path.
%
% #1 = lower left point of grid
% #2 = upper right point of grid
%  
% Options: 
%  
% stepx = x-step dimension (default 1cm)
% stepy = y-step dimension (default 1cm)
% step  = dimesion vector
%  
% Example:
%
% \pgfsetlinewidth{0.8pt}
% \pgfgrid{\pgfxy(0,0)}{\pgfxy(3,2)}
% \pgfsetlinewidth{0.4pt}
% \pgfgrid[stepx=1cm,stepy=1cm]{\pgfxy(0,0)}{\pgfxy(3,2)}

\pgfkeys{
  /pgf/stepx/.initial=1cm,
  /pgf/stepy/.initial=1cm,
  /pgf/step/.code={\pgf@process{#1}\pgfkeysalso{/pgf/stepx/.expanded=\the\pgf@x,/pgf/stepy/.expanded=\the\pgf@y}},
  /pgf/step/.value required
}

\def\pgfpathgrid{\pgfutil@ifnextchar[{\pgf@pathgrid}{\pgf@pathgrid[]}}
\def\pgf@pathgrid[#1]#2#3{%
  \pgfset{#1}%
  \pgfmathsetlength\pgf@xc{\pgfkeysvalueof{/pgf/stepx}}%
  \pgfmathsetlength\pgf@yc{\pgfkeysvalueof{/pgf/stepy}}%
  \pgf@process{#3}%
  \pgf@xb=\pgf@x%
  \pgf@yb=\pgf@y%
  \pgf@process{#2}%
  \pgf@xa=\pgf@x\relax%
  \pgf@ya=\pgf@y\relax%
  {%
    % compute bounding box
    % first corner
    \pgf@x=\pgf@xb%
    \pgf@y=\pgf@yb%
    \pgf@pos@transform{\pgf@x}{\pgf@y}%
    \pgf@protocolsizes{\pgf@x}{\pgf@y}%
    % second corner
    \pgf@x=\pgf@xb%
    \pgf@y=\pgf@ya%
    \pgf@pos@transform{\pgf@x}{\pgf@y}%
    \pgf@protocolsizes{\pgf@x}{\pgf@y}%
    % third corner
    \pgf@x=\pgf@xa%
    \pgf@y=\pgf@yb%
    \pgf@pos@transform{\pgf@x}{\pgf@y}%
    \pgf@protocolsizes{\pgf@x}{\pgf@y}%
    % fourth corner
    \pgf@x=\pgf@xa%
    \pgf@y=\pgf@ya%
    \pgf@pos@transform{\pgf@x}{\pgf@y}%
    \pgf@protocolsizes{\pgf@x}{\pgf@y}%
  }%
  \c@pgf@counta=\pgf@y\relax%
  \c@pgf@countb=\pgf@yc\relax%
  \divide\c@pgf@counta by\c@pgf@countb\relax%
  \pgf@y=\c@pgf@counta\pgf@yc\relax%
  \ifdim\pgf@y<\pgf@ya%
    \advance\pgf@y by\pgf@yc%
  \fi%
  \loop% horizontal lines
    {%
      \pgf@xa=\pgf@x%
      \pgf@ya=\pgf@y%
      \pgf@pos@transform{\pgf@xa}{\pgf@ya}
      \pgfsyssoftpath@moveto{\the\pgf@xa}{\the\pgf@ya}%
      \pgf@xa=\pgf@xb%
      \pgf@ya=\pgf@y%
      \pgf@pos@transform{\pgf@xa}{\pgf@ya}
      \pgfsyssoftpath@lineto{\the\pgf@xa}{\the\pgf@ya}%
    }%
    \advance\pgf@y by\pgf@yc%
  \ifdim\pgf@y<\pgf@yb%
  \repeat%
  \advance\pgf@y by-0.01pt\relax%
  \ifdim\pgf@y<\pgf@yb%
    {%
      \pgf@xa=\pgf@x%
      \pgf@ya=\pgf@y%
      \pgf@pos@transform{\pgf@xa}{\pgf@ya}
      \pgfsyssoftpath@moveto{\the\pgf@xa}{\the\pgf@ya}%
      \pgf@xa=\pgf@xb%
      \pgf@ya=\pgf@y%
      \pgf@pos@transform{\pgf@xa}{\pgf@ya}
      \pgfsyssoftpath@lineto{\the\pgf@xa}{\the\pgf@ya}%
    }%
  \fi%
  \c@pgf@counta=\pgf@x\relax%
  \c@pgf@countb=\pgf@xc\relax%
  \divide\c@pgf@counta by\c@pgf@countb\relax%
  \pgf@x=\c@pgf@counta\pgf@xc\relax%
  \ifdim\pgf@x<\pgf@xa%
    \advance\pgf@x by\pgf@xc%
  \fi%
  \loop% vertical lines
    {%
      \pgf@xc=\pgf@x%
      \pgf@yc=\pgf@ya%
      \pgf@pos@transform{\pgf@xc}{\pgf@yc}
      \pgfsyssoftpath@moveto{\the\pgf@xc}{\the\pgf@yc}%
      \pgf@xc=\pgf@x%
      \pgf@yc=\pgf@yb%
      \pgf@pos@transform{\pgf@xc}{\pgf@yc}
      \pgfsyssoftpath@lineto{\the\pgf@xc}{\the\pgf@yc}%
    }%
    \advance\pgf@x by\pgf@xc%
  \ifdim\pgf@x<\pgf@xb%
  \repeat%
  \advance\pgf@x by-0.01pt\relax%
  \ifdim\pgf@x<\pgf@xb%
    {%
      \pgf@xc=\pgf@x%
      \pgf@yc=\pgf@ya%
      \pgf@pos@transform{\pgf@xc}{\pgf@yc}
      \pgfsyssoftpath@moveto{\the\pgf@xc}{\the\pgf@yc}%
      \pgf@xc=\pgf@x%
      \pgf@yc=\pgf@yb%
      \pgf@pos@transform{\pgf@xc}{\pgf@yc}
      \pgfsyssoftpath@lineto{\the\pgf@xc}{\the\pgf@yc}%
    }%
  \fi%
}



% Append two half-parabolas to the path
%
% #1 = bend (relative to current point)
% #2 = end point (relative to bend point)
%  
% Description:
%  
% This command appends a half-parabola that starts at the current point
% and has its bend at #1+current point. Then, a second parabola is
% appended that starts at #1+current point, where it also has its
% minimum/maximum, and ends at #1+current point+#2, which becomes the
% new current point. 
%  
% By setting #2 = (0,0) you draw only a half parabola that goes from the
% current point to the bend; by setting #1 = (0,0)
% you draw a half parabola that going to current point + #2 and has its
% bend at the current point. 
%  
% Examples:
%
% % Half-parabola going ``up and right''
% \pgfpathmoveto{\pgfpointorigin}
% \pgfpathparabola{\pgfpointorigin}{\pgfpoint{2cm}{4cm}}
%
% % Half-parabola going ``down and right''
% \pgfpathmoveto{\pgfpointorigin}
% \pgfpathparabola{\pgfpoint{-2cm}{4cm}}}{\pgfpointorigin}
%
% % Full parabola
% \pgfpathmoveto{\pgfpointorigin}
% \pgfpathparabola{\pgfpoint{-2cm}{4cm}}{\pgfpoint{2cm}{4cm}}

\def\pgfpathparabola#1#2{%
  {%
    \pgf@process{#2}% untransformed
    \pgf@xb=\pgf@x%
    \pgf@yb=\pgf@y%
    \pgf@process{#1}% untransformed
    \pgf@xc=\pgf@x%
    \pgf@yc=\pgf@y%
    \pgfutil@tempswatrue%
    \ifdim\pgf@xb=0pt\relax%
      \ifdim\pgf@yb=0pt\relax%
        \pgfutil@tempswafalse%
      \fi%
    \fi%  
    {%
      \ifpgfutil@tempswa%
        \pgf@arccornersfalse
      \else%
      \fi%
      \pgfutil@tempswatrue%
      \ifdim\pgf@xc=0pt\relax%
        \ifdim\pgf@yc=0pt\relax%
          \pgfutil@tempswafalse%
        \fi%
      \fi%
      \ifpgfutil@tempswa  
      {%
        \pgf@pt@x=\pgf@path@lastx%
        \pgf@pt@y=\pgf@path@lasty%
        \pgfpathcurveto%
          {\pgfqpoint{.1125\pgf@xc}{.225\pgf@yc}}% found by trial and error
          {\pgfqpoint{.5\pgf@xc}{\pgf@yc}}% found by trial and error
          {\pgfqpoint{\pgf@xc}{\pgf@yc}}%
      }%
      \fi%
    }%
    \ifpgfutil@tempswa%
      \pgf@xc=\pgf@xb%
      \pgf@yc=\pgf@yb%
      {%
        \pgf@pt@x=\pgf@path@lastx%
        \pgf@pt@y=\pgf@path@lasty%
        \pgfpathcurveto%
          {\pgfqpoint{.5\pgf@xc}{0\pgf@yc}}% found by trial and error
          {\pgfqpoint{.8875\pgf@xc}{.775\pgf@yc}}% found by trial and error
          {\pgfqpoint{\pgf@xc}{\pgf@yc}}%
      }%
    \fi%
  }%  
}  




% Append a sine curve between 0 and \pi/2 to the path.
%
% #1 = vector, describing the width and height of the curve
%  
% Description:
%  
% This command appends a sine curve in the interval 0 and \pi/2 to the
% current path. The sine curve ends at currentpoint+#1.
%  
% Examples:
%
% % One complete sine in the interval [0,\pi]
% \pgfpathmoveto{\pgfpointorigin}
% \pgfpathsine{\pgfpoint{1.57cm}{1cm}}
% \pgfpathcosine{\pgfpoint{3.141cm}{0cm}}

\def\pgfpathsine#1{%
  {%
    \pgf@process{#1}% untransformed
    \pgf@xc=\pgf@x%
    \pgf@yc=\pgf@y%
    \pgf@pt@x=\pgf@path@lastx% evil trickery to transform to the last point
    \pgf@pt@y=\pgf@path@lasty%
    \pgfpathcurveto%
      {\pgfqpoint{.31831\pgf@xc}{.5\pgf@yc}}% found by trial and error
      {\pgfqpoint{.63503\pgf@xc}{\pgf@yc}}% found by trial and error
      {\pgfqpoint{\pgf@xc}{\pgf@yc}}%
  }%  
}  

% Append a cosine curve between 0 and \pi/2 to the path.
%
% #1 = vector, describing the width and height of the curve
%  
% Examples:
%
% % One complete sine in the interval [0,\pi]
% \pgfpathmoveto{\pgfpointorigin}
% \pgfpathsine{\pgfpoint{1.57cm}{1cm}}
% \pgfpathcosine{\pgfpoint{3.141cm}{0cm}}

\def\pgfpathcosine#1{%
  {%
    \pgf@process{#1}% untransformed
    \pgf@xc=\pgf@x%
    \pgf@yc=\pgf@y%
    \pgf@pt@x=\pgf@path@lastx% evil trickery to transform to the last point
    \pgf@pt@y=\pgf@path@lasty%
    \pgfpathcurveto%
      {\pgfqpoint{.36497\pgf@xc}{0pt}}% found by trial and error
      {\pgfqpoint{.68169\pgf@xc}{.5\pgf@yc}}% found by trial and error
      {\pgfqpoint{\pgf@xc}{\pgf@yc}}%
  }%  
}  



%	Draw part of a curve between two specified times s and t.
%	
%	#1 - s.
%	#2 - t.
%	#3,#4,#5,#6 - the curve.
%	
%	There are two versions, \pgfpathcurvebetweentime and 
%	\pgfpathcurvebetweentime*. The latter does not insert a moveto
%	to the first point.
%
\newif\ifpgf@ignoremoveto
\def\pgfpathcurvebetweentime{\futurelet\pgf@lettoken\pgf@pathcurvebetweentime}
\def\pgf@pathcurvebetweentime{%
	\ifx\pgf@lettoken*%
		\pgf@ignoremovetotrue%
		\def\pgf@next{\afterassignment\pgf@@pathcurvebetweentime\let\pgf@lettoken=}%
	\else%
		\pgf@ignoremovetofalse%
		\let\pgf@next=\pgf@@pathcurvebetweentime%
	\fi%
	\pgf@next}

\def\pgf@@pathcurvebetweentime#1#2#3#4#5#6{%
	\pgfmathparse{#1}%
	\let\pgf@time@s=\pgfmathresult%
	\pgfmathparse{#2}%
	\let\pgf@time@t=\pgfmathresult%
	\ifdim\pgf@time@s pt>\pgf@time@t pt\relax%
		\pgfmathsetmacro\pgf@time@s{1-#1}%
		\pgfmathsetmacro\pgf@time@t{1-#2}%
		\pgf@@@pathcurvebetweentime{\pgf@time@t}{#6}{#5}{#4}{#3}%
	\else%
		\pgf@@@pathcurvebetweentime{\pgf@time@t}{#3}{#4}{#5}{#6}%
	\fi%
}

\def\pgf@@@pathcurvebetweentime#1#2#3#4#5{%
	% Q1 = P1.
	\pgf@process{#2}% 
	\pgf@xc=\pgf@x%
	\pgf@yc=\pgf@y%
	% Q2 = P1 + t*(P2-P1).
	\pgf@process{%
		\pgf@process{#3}%
		\pgf@xa=#1\pgf@x%
		\pgf@ya=#1\pgf@y%
		\pgf@process{#2}%
		\pgf@xb=\pgf@x%
		\pgf@yb=\pgf@y%
		\advance\pgf@x by-#1\pgf@xb%
		\advance\pgf@y by-#1\pgf@yb%
		\advance\pgf@x by\pgf@xa%
		\advance\pgf@y by\pgf@ya%
	}%
	\pgf@xb=\pgf@x%
	\pgf@yb=\pgf@y%
	% Q3 = Q2 + t*((P2 + t*(P3-P2)) - Q2).
	\pgf@process{%
		\pgf@process{#4}%
		\pgf@xa=#1\pgf@x%
		\pgf@ya=#1\pgf@y%	
		%
		\pgf@process{#3}%
		\pgf@xc=\pgf@x%
		\pgf@yc=\pgf@y%
		\advance\pgf@xc by-#1\pgf@x%
		\advance\pgf@yc by-#1\pgf@y%
		%
		\pgf@x=\pgf@xb%
		\pgf@y=\pgf@yb%
		\advance\pgf@x by#1\pgf@xa%
		\advance\pgf@y by#1\pgf@ya%		
		\advance\pgf@x by-#1\pgf@xb%
		\advance\pgf@y by-#1\pgf@yb%
		\advance\pgf@x by#1\pgf@xc%
		\advance\pgf@y by#1\pgf@yc%
	}%
	\pgf@xa=\pgf@x%
	\pgf@ya=\pgf@y%
	% Q4 = (1-t)^3*P1 + 3*t(1-t)^2*P2 + 3*t^2(1-t)*P3 + t^3*P4.
	\pgf@process{\pgfpointcurveattime{#1}{#2}{#3}{#4}{#5}}% 
	\ifx#1\pgf@time@t%
		% First time round...
		\pgfmathdivide@{\pgf@time@s}{\pgf@time@t}%
		\pgfmathadd@{-\pgfmathresult}{1}%
		\let\pgf@time@s=\pgfmathresult%
		\edef\pgf@marshal{%
			\noexpand\pgf@@@pathcurvebetweentime{\noexpand\pgf@time@s}%
				{\noexpand\pgfqpoint{\the\pgf@x}{\the\pgf@y}}{\noexpand\pgfqpoint{\the\pgf@xa}{\the\pgf@ya}}%
				{\noexpand\pgfqpoint{\the\pgf@xb}{\the\pgf@yb}}{\noexpand\pgfqpoint{\the\pgf@xc}{\the\pgf@yc}}%
		}%
	\else%
	  % ...second time round.
		\ifpgf@ignoremoveto%
			\edef\pgf@marshal{%
				\noexpand\pgfpathcurveto{\noexpand\pgfqpoint{\the\pgf@xa}{\the\pgf@ya}}%
					{\noexpand\pgfqpoint{\the\pgf@xb}{\the\pgf@yb}}{\noexpand\pgfqpoint{\the\pgf@xc}{\the\pgf@yc}}%
			}%
		\else%
			\edef\pgf@marshal{%
				\noexpand\pgfpathmoveto{\noexpand\pgfqpoint{\the\pgf@x}{\the\pgf@y}}%
				\noexpand\pgfpathcurveto{\noexpand\pgfqpoint{\the\pgf@xa}{\the\pgf@ya}}%
					{\noexpand\pgfqpoint{\the\pgf@xb}{\the\pgf@yb}}{\noexpand\pgfqpoint{\the\pgf@xc}{\the\pgf@yc}}%
			}%
		\fi%
	\fi%
	\pgf@marshal%
}


\endinput
